\documentclass[12pt,a4paper]{article}

\makeatletter
	\usepackage[utf8]{inputenc}
\usepackage[T1]{fontenc}
\usepackage{ucs}

\usepackage[french]{babel,varioref}

\usepackage[top=2cm, bottom=2cm, left=1.5cm, right=1.5cm]{geometry}
\usepackage{enumitem}

\usepackage{multicol}

\usepackage{color}
\usepackage{hyperref}
\hypersetup{
    colorlinks,
    citecolor=black,
    filecolor=black,
    linkcolor=black,
    urlcolor=black
}

\usepackage{amsthm}

\usepackage{tcolorbox}
\tcbuselibrary{listingsutf8}

\usepackage{ifplatform}

\usepackage{ifthen}

\usepackage{cbdevtool}


% MISC

\newtcblisting{latexex}{%
	sharp corners,%
	left=1mm, right=1mm,%
	bottom=1mm, top=1mm,%
	colupper=red!75!blue,%
	listing side text
}

\newtcblisting{latexex-flat}{%
	sharp corners,%
	left=1mm, right=1mm,%
	bottom=1mm, top=1mm,%
	colupper=red!75!blue,%
}

\newtcblisting{latexex-alone}{%
	sharp corners,%
	left=1mm, right=1mm,%
	bottom=1mm, top=1mm,%
	colupper=red!75!blue,%
	listing only
}


\newcommand\env[1]{\texttt{#1}}
\newcommand\macro[1]{\env{\textbackslash{}#1}}



\setlength{\parindent}{0cm}
\setlist{noitemsep}

\theoremstyle{definition}
\newtheorem*{remark}{Remarque}

\usepackage[raggedright]{titlesec}

\titleformat{\paragraph}[hang]{\normalfont\normalsize\bfseries}{\theparagraph}{1em}{}
\titlespacing*{\paragraph}{0pt}{3.25ex plus 1ex minus .2ex}{0.5em}


\newcommand\separation{
	\medskip
	\hfill\rule{0.5\textwidth}{0.75pt}\hfill
	\medskip
}


\newcommand\extraspace{
	\vspace{0.25em}
}


\newcommand\ascii{\texttt{ASCII}}


	\usepackage{01-conditional}
\makeatother


% == EXTRAS == %

\usepackage[top=2cm, bottom=2cm, left=1.5cm, right=1.5cm]{geometry}


\begin{document}

\section{Probabilité}

    \subsection{Probabilité conditionnelle}

			\paragraph{Un exemple type}

Dans l'exemple ci-dessous, les options \texttt{x} et \texttt{X} font référence au mot anglais \emph{\og expand \fg} qui peut signifier \emph{\og développer \fg} comme on l'entend en mathématiques.

\begin{tcblisting}{}
Écrire des probabilités conditionnelles :
$\probacond{p}{A}{B} = \probacond*{p}{A}{B} 
                     = \probacond[x]{p}{A}{B} 
                     = \probacond[X]{p}{A}{B}$.
\end{tcblisting}


            \paragraph{Fiche technique}

\IDmacro{probacond}{1}{3}

\IDmacro{probacond*}{1}{3}

\IDoption{} si utilisée avec la valeur \verb+x+, cette option donne l'écriture sous forme de fraction via \verb+\frac+ tandis qu'avec \verb+X+ ce sera \verb+\dfrac+ qui sera utilisée.

\IDarg{1} le nom de la probabilité.

\IDarg{2} l'ensemble dont on veut calculer la probabilité.

\IDarg{3} l'ensemble qui donne la condition.

\end{document}
