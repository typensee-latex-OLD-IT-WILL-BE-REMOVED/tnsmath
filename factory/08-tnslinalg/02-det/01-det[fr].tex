\documentclass[12pt,a4paper]{article}

\makeatletter
    \usepackage[utf8]{inputenc}
\usepackage[T1]{fontenc}
\usepackage{ucs}

\usepackage[french]{babel,varioref}

\usepackage[top=2cm, bottom=2cm, left=1.5cm, right=1.5cm]{geometry}
\usepackage{enumitem}

\usepackage{multicol}

\usepackage{makecell}

\usepackage{color}
\usepackage{hyperref}
\hypersetup{
    colorlinks,
    citecolor=black,
    filecolor=black,
    linkcolor=black,
    urlcolor=black
}

\usepackage{amsthm}

\usepackage{tcolorbox}
\tcbuselibrary{listingsutf8}

\usepackage{ifplatform}

\usepackage{ifthen}

\usepackage{cbdevtool}


% MISC

\newtcblisting{latexex}{%
	sharp corners,%
	left=1mm, right=1mm,%
	bottom=1mm, top=1mm,%
	colupper=red!75!blue,%
	listing side text
}

\newtcblisting{latexex-flat}{%
	sharp corners,%
	left=1mm, right=1mm,%
	bottom=1mm, top=1mm,%
	colupper=red!75!blue,%
}

\newtcblisting{latexex-alone}{%
	sharp corners,%
	left=1mm, right=1mm,%
	bottom=1mm, top=1mm,%
	colupper=red!75!blue,%
	listing only
}


\newcommand\env[1]{\texttt{#1}}
\newcommand\macro[1]{\env{\textbackslash{}#1}}



\setlength{\parindent}{0cm}
\setlist{noitemsep}

\theoremstyle{definition}
\newtheorem*{remark}{Remarque}

\usepackage[raggedright]{titlesec}

\titleformat{\paragraph}[hang]{\normalfont\normalsize\bfseries}{\theparagraph}{1em}{}
\titlespacing*{\paragraph}{0pt}{3.25ex plus 1ex minus .2ex}{0.5em}


\newcommand\separation{
	\medskip
	\hfill\rule{0.5\textwidth}{0.75pt}\hfill
	\medskip
}


\newcommand\extraspace{
	\vspace{0.25em}
}


\newcommand\whyprefix[2]{%
	\textbf{\prefix{#1}}-#2%
}

\newcommand\mwhyprefix[2]{%
	\texttt{#1 = #1-#2}%
}

\newcommand\prefix[1]{%
	\texttt{#1}%
}


\newcommand\inenglish{\@ifstar{\@inenglish@star}{\@inenglish@no@star}}

\newcommand\@inenglish@star[1]{%
	\emph{\og #1 \fg}%
}

\newcommand\@inenglish@no@star[1]{%
	\@inenglish@star{#1} en anglais%
}


\newcommand\ascii{\texttt{ASCII}}


% Example
\newcounter{paraexample}[subsubsection]

\newcommand\@newexample@abstract[2]{%
	\paragraph{%
		#1%
		\if\relax\detokenize{#2}\relax\else {} -- #2\fi%
	}%
}



\newcommand\newparaexample{\@ifstar{\@newparaexample@star}{\@newparaexample@no@star}}

\newcommand\@newparaexample@no@star[1]{%
	\refstepcounter{paraexample}%
	\@newexample@abstract{Exemple \theparaexample}{#1}%
}

\newcommand\@newparaexample@star[1]{%
	\@newexample@abstract{Exemple}{#1}%
}


% Change log
\newcommand\topic{\@ifstar{\@topic@star}{\@topic@no@star}}

\newcommand\@topic@no@star[1]{%
	\textbf{\textsc{#1}.}%
}

\newcommand\@topic@star[1]{%
	\textbf{\textsc{#1} :}%
}







    \usepackage{01-det}
\makeatother



\begin{document}

%\section{Matrices}

\subsection{\texorpdfstring{Calculs expliqués des déterminants $2 \times 2$}%
                           {Calculs expliqués des déterminants 2x2}} \label{tnslinalg-2D-det}

\newparaexample*{}

\begin{latexex}
 $\calcdettwo*    {a}{c}%
                  {b}{d}
= \calcdettwo     {a}{c}%
                  {b}{d}
= \calcdettwo[exp]{a}{c}%
                  {b}{d}$
\end{latexex}


\begin{remark}
	Il existe deux autres types de développement.
	\begin{enumerate}
		\item $\calcdettwo[cexp]{a}{c}{b}{d}$ s'obtient via l'option \verb+cexp+.
		
		\item $\calcdettwo[texp]{a}{c}{b}{d}$ s'obtient via l'option \verb+texp+
	\end{enumerate}
	\prefix{exp} est pour \whyprefix{exp}{and} soit \inenglish{développer}, \prefix{c} pour \macro{cdot} et enfin \prefix{t} pour \macro{times}.
\end{remark}


% ---------------------- %


\subsection{Fiches techniques}

\IDmacro{calcdettwo }{1}{4}  où \quad \mwhyprefix{c}{alculate}

\IDmacro{calcdettwo*}{1}{4}  où \quad \mwhyprefix{c}{alculate}

\IDoption{} la valeur par défaut est \verb+std+ pour \verb+standard+. Voici les différentes valeurs possibles.
\begin{enumerate}
	\item \verb+std + : on utilise l'écriture matricielle.

	\item \verb+exp + : ceci demande d'afficher une formule développée en utilisant $\times$ pour les produits.

	\item \verb+cexp+ : comme \verb+exp+ mais avec le symbole $\cdot$ obtenu via \macro{cdot}.

	\item \verb+sexp+ : comme \verb+exp+ mais avec un espace pour séparer les facteurs de chaque produit.
\end{enumerate}

\IDarg{1} l'entrée à la position $(1, 1)$

\IDarg{2} l'entrée à la position $(1, 2)$

\extraspace

\IDarg{3} l'entrée à la position $(2, 1)$

\IDarg{4} l'entrée à la position $(2, 2)$                   








%% ---------------------- %
%
%
%\subsection{\texorpdfstring{Calculs expliqués des déterminants $3 \times 3$}%
%                              {Calculs expliqués des déterminants 3x3}}
%
%Dans les exemples suivants, le préfixe \prefix{c} reste pour \whyprefix{c}{alculer} et \prefix{three} est pour \inenglish{trois}.
%
%
%% ---------------------- %
%
%
%\paragraph{Exemple 1 -- Bien pour les débutants}
%
%\begin{latexex}
%$\calcdetthree{a}{c}{c}%
%           {x}{y}{z}%
%           {r}{s}{t}$
%\end{latexex}         
%
%
%% ---------------------- %
%
%
%\paragraph{Exemple 2 -- Plus efficace à l'usage}
%
%\begin{latexex}
%$\calcdetthree*{a}{c}{c}%
%            {x}{y}{z}%
%            {r}{s}{t}$
%\end{latexex}                      
%
%
%% ---------------------- %
%
%
%\subsection{Fi    ches techniques}
%
%\paragraph{\texorpdfstring{Calculs expliqués des déterminants $3 \times 3$}%
%                          {Calculs expliqués des déterminants 3x3}}
%
%\IDmacro*{calcdetthree}{9}  où \quad \mwhyprefix{c}{alculate}
%
%\IDmacro*{calcdetthree*}{9}  où \quad \mwhyprefix{c}{alculate}
%
%\IDarg{1} l'entrée à la position $(1, 1)$
%
%\IDarg{2} l'entrée à la position $(1, 2)$
%
%\IDarg{3} l'entrée à la position $(1, 3)$
%
%\extraspace
%
%\IDarg{4} l'entrée à la position $(2, 1)$
%
%\IDarg{5} l'entrée à la position $(2, 2)$
%
%\IDarg{6} l'entrée à la position $(2, 3)$
%
%\extraspace
%
%\IDarg{7} l'entrée à la position $(3, 1)$
%
%\IDarg{8} l'entrée à la position $(3, 2)$
%
%\IDarg{9} l'entrée à la position $(3, 3)$
%

\end{document}
