\documentclass[12pt,a4paper]{article}

\makeatletter
    \usepackage[utf8]{inputenc}
\usepackage[T1]{fontenc}
\usepackage{ucs}

\usepackage[french]{babel,varioref}

\usepackage[top=2cm, bottom=2cm, left=1.5cm, right=1.5cm]{geometry}
\usepackage{enumitem}

\usepackage{multicol}

\usepackage{color}
\usepackage{hyperref}
\hypersetup{
    colorlinks,
    citecolor=black,
    filecolor=black,
    linkcolor=black,
    urlcolor=black
}

\usepackage{amsthm}

\usepackage{tcolorbox}
\tcbuselibrary{listingsutf8}

\usepackage{ifplatform}

\usepackage{ifthen}

\usepackage{cbdevtool}


% MISC

\newtcblisting{latexex}{%
	sharp corners,%
	left=1mm, right=1mm,%
	bottom=1mm, top=1mm,%
	colupper=red!75!blue,%
	listing side text
}

\newtcblisting{latexex-flat}{%
	sharp corners,%
	left=1mm, right=1mm,%
	bottom=1mm, top=1mm,%
	colupper=red!75!blue,%
}

\newtcblisting{latexex-alone}{%
	sharp corners,%
	left=1mm, right=1mm,%
	bottom=1mm, top=1mm,%
	colupper=red!75!blue,%
	listing only
}


\newcommand\env[1]{\texttt{#1}}
\newcommand\macro[1]{\env{\textbackslash{}#1}}



\setlength{\parindent}{0cm}
\setlist{noitemsep}

\theoremstyle{definition}
\newtheorem*{remark}{Remarque}

\usepackage[raggedright]{titlesec}

\titleformat{\paragraph}[hang]{\normalfont\normalsize\bfseries}{\theparagraph}{1em}{}
\titlespacing*{\paragraph}{0pt}{3.25ex plus 1ex minus .2ex}{0.5em}


\newcommand\separation{
	\medskip
	\hfill\rule{0.5\textwidth}{0.75pt}\hfill
	\medskip
}


\newcommand\extraspace{
	\vspace{0.25em}
}


\newcommand\ascii{\texttt{ASCII}}


    \usepackage{01-matrices}
\makeatother



\begin{document}

\chapter{Algèbre linéaire}

\section{Matrices via \texttt{nicematrix}}

Le gros du boulot est fait par l'excellent package \verb+nicematrix+
\footnote{
	On impose l'option \texttt{transparent}.
}.
\verb+tnslinalg+ propose en plus une macro
%de deux macros
à but pédagogique : voir la section \ref{tnslinalg-2D-det} page \pageref{tnslinalg-2D-det}.
Veuillez vous reporter à la documentation de \verb+nicematrix+ pour savoir comment s'y prendre en général.


% ---------------------- %


\subsection{Quelques exemples pour bien démarrer}

\newparaexample{Vu dans la documentation de \texttt{nicematrix}}

\begin{latexex}
$\begin{pmatrix}
    1      & \cdots & \cdots & 1      \\
    0      & \ddots &        & \vdots \\
    \vdots & \ddots & \ddots & \vdots \\
    0      & \cdots & 0      & 1
\end{pmatrix}$
\end{latexex}


% ---------------------- %


\newparaexample{}

\begin{latexex}
$\begin{vmatrix}
    1      & \cdots & \cdots & 1      \\
    0      & \ddots &        & \vdots \\
    \vdots & \ddots & \ddots & \vdots \\
    0      & \cdots & 0      & 1
\end{vmatrix}$
\end{latexex}


% ---------------------- %


\newparaexample{}

\begin{latexex}
$\begin{bmatrix}
    1      & \cdots & \cdots & 1      \\
    0      & \ddots &        & \vdots \\
    \vdots & \ddots & \ddots & \vdots \\
    0      & \cdots & 0      & 1
\end{bmatrix}$
\end{latexex}


% ---------------------- %


\newparaexample{Vu dans la documentation de \texttt{nicematrix}}

\begin{latexex}
$\begin{pNiceMatrix}[name = mymatrix]
     1 & 2 & 3 \\
     4 & 5 & 6 \\
     7 & 8 & 9
 \end{pNiceMatrix}$

 \tikz[remember picture,
       overlay]
 \draw[red]
     (mymatrix-2-2) circle (2.5mm);
\end{latexex}
%

% ---------------------- %


\newparaexample{Vu dans la documentation de \texttt{nicematrix}}

\begin{latexex}
$\left(
     \begin{NiceArray}{cccc:c}
         1  & 2  & 3  & 4  & 5  \\
         6  & 7  & 8  & 9  & 10 \\
         11 & 12 & 13 & 14 & 15
     \end{NiceArray}
 \right)$
\end{latexex}$


% ---------------------- %


\newparaexample{Proposition de l'auteur de \texttt{nicematrix} suite à une discussion par mail}

\begin{latexex}
% Besoin du package ``ifthen``.
\newcommand\aij{%
  a_{\arabic{iRow}\arabic{jCol}}%
}

$\begin{bNiceArray}{*{5}{>{%
     \ifthenelse{\value{iRow}>0}{\aij}{}%
 }c}}[
     first-col,
     first-row,
     code-for-first-row
       = \text{\textbf{\arabic{jCol}}},
     code-for-first-col
       = \text{\textbf{\arabic{iRow}}}
   ]
       & & & & & \\
       & & & & & \\
       & & & & &
 \end{bNiceArray}$
\end{latexex}$


% ---------------------- %


\newparaexample{Avec des calculs automatiques}

\begin{latexex}
\newcounter{cntaij}
\newcommand\aij{%
    \setcounter{cntaij}{\value{iRow}}%
    \addtocounter{cntaij}{\value{jCol}}%
    \addtocounter{cntaij}{-1}%
    \arabic{cntaij}%
}

Si $a_{ij} = i + j - 1$ alors

$(a_{ij})_{1 \leq i \leq 3 ,
           1 \leq j \leq 5}
 =
 \begin{bNiceArray}{*{5}{>{\aij}c}}
     & & & & \\
     & & & & \\
     & & & &
 \end{bNiceArray}$
\end{latexex}

\end{document}
