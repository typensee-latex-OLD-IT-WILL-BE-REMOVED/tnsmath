\documentclass[12pt,a4paper]{article}

\makeatletter
    \usepackage[utf8]{inputenc}
\usepackage[T1]{fontenc}
\usepackage{ucs}

\usepackage[french]{babel,varioref}

\usepackage[top=2cm, bottom=2cm, left=1.5cm, right=1.5cm]{geometry}
\usepackage{enumitem}

\usepackage{multicol}

\usepackage{makecell}

\usepackage{color}
\usepackage{hyperref}
\hypersetup{
    colorlinks,
    citecolor=black,
    filecolor=black,
    linkcolor=black,
    urlcolor=black
}

\usepackage{amsthm}

\usepackage{tcolorbox}
\tcbuselibrary{listingsutf8}

\usepackage{ifplatform}

\usepackage{ifthen}

\usepackage{cbdevtool}


% MISC

\newtcblisting{latexex}{%
	sharp corners,%
	left=1mm, right=1mm,%
	bottom=1mm, top=1mm,%
	colupper=red!75!blue,%
	listing side text
}

\newtcblisting{latexex-flat}{%
	sharp corners,%
	left=1mm, right=1mm,%
	bottom=1mm, top=1mm,%
	colupper=red!75!blue,%
}

\newtcblisting{latexex-alone}{%
	sharp corners,%
	left=1mm, right=1mm,%
	bottom=1mm, top=1mm,%
	colupper=red!75!blue,%
	listing only
}


\newcommand\env[1]{\texttt{#1}}
\newcommand\macro[1]{\env{\textbackslash{}#1}}



\setlength{\parindent}{0cm}
\setlist{noitemsep}

\theoremstyle{definition}
\newtheorem*{remark}{Remarque}

\usepackage[raggedright]{titlesec}

\titleformat{\paragraph}[hang]{\normalfont\normalsize\bfseries}{\theparagraph}{1em}{}
\titlespacing*{\paragraph}{0pt}{3.25ex plus 1ex minus .2ex}{0.5em}


\newcommand\separation{
	\medskip
	\hfill\rule{0.5\textwidth}{0.75pt}\hfill
	\medskip
}


\newcommand\extraspace{
	\vspace{0.25em}
}


\newcommand\whyprefix[2]{%
	\textbf{\prefix{#1}}-#2%
}

\newcommand\mwhyprefix[2]{%
	\texttt{#1 = #1-#2}%
}

\newcommand\prefix[1]{%
	\texttt{#1}%
}


\newcommand\inenglish{\@ifstar{\@inenglish@star}{\@inenglish@no@star}}

\newcommand\@inenglish@star[1]{%
	\emph{\og #1 \fg}%
}

\newcommand\@inenglish@no@star[1]{%
	\@inenglish@star{#1} en anglais%
}


\newcommand\ascii{\texttt{ASCII}}


% Example
\newcounter{paraexample}[subsubsection]

\newcommand\@newexample@abstract[2]{%
	\paragraph{%
		#1%
		\if\relax\detokenize{#2}\relax\else {} -- #2\fi%
	}%
}



\newcommand\newparaexample{\@ifstar{\@newparaexample@star}{\@newparaexample@no@star}}

\newcommand\@newparaexample@no@star[1]{%
	\refstepcounter{paraexample}%
	\@newexample@abstract{Exemple \theparaexample}{#1}%
}

\newcommand\@newparaexample@star[1]{%
	\@newexample@abstract{Exemple}{#1}%
}


% Change log
\newcommand\topic{\@ifstar{\@topic@star}{\@topic@no@star}}

\newcommand\@topic@no@star[1]{%
	\textbf{\textsc{#1}.}%
}

\newcommand\@topic@star[1]{%
	\textbf{\textsc{#1} :}%
}






    % == PACKAGES USED == %

\RequirePackage{amsmath}
\RequirePackage{relsize}
\RequirePackage{xparse}


% == DEFINITIONS == %

% Settable texts
\@ifpackagewith{babel}{french}{
    \newcommand\lymathsep{;}
    \newcommand\lymathsubsep{,}

    \newcommand\textopchoice{choix}
    \newcommand\textopcond{cond}
    \newcommand\textopdef{déf}
    \newcommand\textophyp{hyp}
    \newcommand\textopid{id}
    \newcommand\textoptest{?}
}{
    \newcommand\lymathsep{,}
    \newcommand\lymathsubsep{;}

    \newcommand\textopchoice{choice}
    \newcommand\textopcond{cond}
    \newcommand\textopdef{def}
    \newcommand\textophyp{hyp}
    \newcommand\textopid{id}
    \newcommand\textoptest{?}
}


\newcommand\textexplainleft{\{}
\newcommand\textexplainright{\}}
\newcommand\textexplainspacein{2em}


% Tools - Apply same macro to all arguments

% #1        : main macro
% #2        : macro to apply to arguments
% #3 and #4 : the two arguments
\newcommand\@apply@macro@two@args[4]{%
    #1{#2{#3}}{#2{#4}}%
}


% Tools - Deco over a math symbol

\newcommand\@over@math@symbol[2]{%
	\mathrel{\overset{\mathrm{\text{\raisebox{.5ex}{#1}}}}{#2}}%
}


% Tools - Intervals

\newcommand\@extra@phantom{%
    \vphantom{\relsize{1.25}{\text{$\displaystyle F_1^2$}}}%
}

\newcommand\@interval@tool@star[5]{%
    \ensuremath{ \left#1 \@extra@phantom \right. \!\! #2 #3 #4 \left. \@extra@phantom \!\! \right#5}%
}

\newcommand\@interval@tool@no@star[5]{\ensuremath{ \left#1 #2 #3 #4 \right#5}}


% Tools - Multi-arguments
%
% Source : the following lines come directly for the following post
%
%    * https://tex.stackexchange.com/a/475291/6880

\ExplSyntaxOn
% General purpose macro for defining other macros
    \NewDocumentCommand{\makemultiargument}{mmmmmo}{
        \lymath_multiarg:nnnnnn{#1}{#2}{#3}{#4}{#5}{#6}
    }
 
% Allocate a private variable
    \seq_new:N \l__lymath_generic_seq

% The internal version of the general purpose macro
    \cs_new_protected:Nn \lymath_multiarg:nnnnnn{
        % #1 = separator
        % #2 = multiargument
        % #3 = code before
          % #4 = code between
          % #5 = code after
          % #6 = ornament to items

        % A group allows nesting
        \group_begin:
         % Split the multiargument into parts
        \seq_set_split:Nnn \l__lymath_generic_seq { #1 } { #2 }
        % Apply the ornament to the items
          \tl_if_novalue:nF { #6 }{
            \seq_set_eq:NN \l__lymath_temp_seq \l__lymath_generic_seq
            \seq_set_map:NNn \l__lymath_generic_seq \l__lymath_generic_seq { #6 }
           }
        % Execute the <code before>
          #3
        % Deliver the items, with the chosen material between them
          \seq_use:Nn \l__lymath_generic_seq { #4 }
          % Execute the <code after>
         #5
          % End the group started at the beginning
          \group_end:
    }    
\ExplSyntaxOff


    \usepackage{02-matrices}
\makeatother



\begin{document}

%\section{Algèbre}

\subsection{Matrices}

Tout le boulot ou presque est fait par l'excellent package \verb+nicematrix+ auquel on impose l'option \verb+transparent+ avec l'ajout 
d'une macro
%de deux macros
\emph{\og maison \fg} à but pédagogique.
Veuillez vous reporter à la documentation de \verb+nicematrix+ pour savoir comment s'y prendre en général.


\subsubsection{\texorpdfstring{Calculs expliqués des déterminants $2 \times 2$}%
                              {Calculs expliqués des déterminants 2x2}}

Dans l'exemple suivant, le préfixe \prefix{c} est pour \whyprefix{c}{alculer} et \prefix{two} est pour \inenglish{deux}.

\begin{latexex}
$\cdettwo{a}{b}%
         {c}{d}$
\end{latexex}                    


% ---------------------- %


\subsubsection{Fiches techniques}

\paragraph{\texorpdfstring{Calculs expliqués des déterminants $2 \times 2$}%
                          {Calculs expliqués des déterminants 3x3}}

\IDmacro*{cdettwo}{4}  où \quad \mwhyprefix{c}{alculate}

\IDarg{1} l'entrée à la position $(1, 1)$

\IDarg{2} l'entrée à la position $(1, 2)$

\extraspace

\IDarg{3} l'entrée à la position $(2, 1)$

\IDarg{4} l'entrée à la position $(2, 2)$                   








%% ---------------------- %
%
%
%\subsubsection{\texorpdfstring{Calculs expliqués des déterminants $3 \times 3$}%
%                              {Calculs expliqués des déterminants 3x3}}
%
%Dans les exemples suivants, le préfixe \prefix{c} reste pour \whyprefix{c}{alculer} et \prefix{three} est pour \inenglish{trois}.
%
%
%% ---------------------- %
%
%
%\paragraph{Exemple 1 -- Bien pour les débutants}
%
%\begin{latexex}
%$\cdetthree{a}{b}{c}%
%           {x}{y}{z}%
%           {r}{s}{t}$
%\end{latexex}         
%
%
%% ---------------------- %
%
%
%\paragraph{Exemple 2 -- Plus efficace à l'usage}
%
%\begin{latexex}
%$\cdetthree*{a}{b}{c}%
%            {x}{y}{z}%
%            {r}{s}{t}$
%\end{latexex}                      
%
%
%% ---------------------- %
%
%
%\subsubsection{Fi    ches techniques}
%
%\paragraph{\texorpdfstring{Calculs expliqués des déterminants $3 \times 3$}%
%                          {Calculs expliqués des déterminants 3x3}}
%
%\IDmacro*{cdetthree}{9}  où \quad \mwhyprefix{c}{alculate}
%
%\IDmacro*{cdetthree*}{9}  où \quad \mwhyprefix{c}{alculate}
%
%\IDarg{1} l'entrée à la position $(1, 1)$
%
%\IDarg{2} l'entrée à la position $(1, 2)$
%
%\IDarg{3} l'entrée à la position $(1, 3)$
%
%\extraspace
%
%\IDarg{4} l'entrée à la position $(2, 1)$
%
%\IDarg{5} l'entrée à la position $(2, 2)$
%
%\IDarg{6} l'entrée à la position $(2, 3)$
%
%\extraspace
%
%\IDarg{7} l'entrée à la position $(3, 1)$
%
%\IDarg{8} l'entrée à la position $(3, 2)$
%
%\IDarg{9} l'entrée à la position $(3, 3)$
%

% ---------------------- %


\subsubsection{Quelques exemples pour bien démarrer}

\paragraph{Exemple 1 (cf. la documentation de \texttt{nicematrix})}

\begin{latexex}
$\begin{pmatrix}
    1      & \cdots & \cdots & 1      \\
    0      & \ddots &        & \vdots \\
    \vdots & \ddots & \ddots & \vdots \\
    0      & \cdots & 0      & 1
\end{pmatrix}$
\end{latexex}


% ---------------------- %


\paragraph{Exemple 2}

\begin{latexex}
$\begin{vmatrix}
    1      & \cdots & \cdots & 1      \\
    0      & \ddots &        & \vdots \\
    \vdots & \ddots & \ddots & \vdots \\
    0      & \cdots & 0      & 1
\end{vmatrix}$
\end{latexex}


% ---------------------- %


\paragraph{Exemple 2}

\begin{latexex}
$\begin{bmatrix}
    1      & \cdots & \cdots & 1      \\
    0      & \ddots &        & \vdots \\
    \vdots & \ddots & \ddots & \vdots \\
    0      & \cdots & 0      & 1
\end{bmatrix}$
\end{latexex}


% ---------------------- %


\paragraph{Exemple 3 (cf. la documentation de \texttt{nicematrix})}

\begin{latexex}
$\begin{pNiceMatrix}[name = mymatrix]
    1 & 2 & 3 \\
    4 & 5 & 6 \\
    7 & 8 & 9
\end{pNiceMatrix}$

\tikz[remember picture,
      overlay]
\draw[red] (mymatrix-2-2) circle (2.5mm);
\end{latexex}


% ---------------------- %


\paragraph{Exemple 4 (cf. la documentation de \texttt{nicematrix})}

\begin{latexex}
$\left(
    \begin{NiceArray}{CCCC:C}
        1  & 2  & 3  & 4  & 5  \\
        6  & 7  & 8  & 9  & 10 \\
        11 & 12 & 13 & 14 & 15
    \end{NiceArray}
\right)$
\end{latexex}$


% ---------------------- %


\paragraph{Exemple 5 proposé par l'auteur de \texttt{nicematrix} suite à une discussion par mail}

\begin{latexex}
% Besoin du package ``ifthen``.
\newcommand\aij{%
  a_{\arabic{iRow}\arabic{jCol}}%
}

$\begin{bNiceArray}%
  {*{5}{>{%
    \ifthenelse{\value{iRow}>0}{\aij}{}%
}C}}[
    first-col,
    first-row,
    code-for-first-row
      = \mathbf{\arabic{jCol}},
    code-for-first-col
      = \mathbf{\arabic{iRow}}
  ]
      & & & & & \\
      & & & & & \\
      & & & & &
\end{bNiceArray}$
\end{latexex}$


% ---------------------- %


\paragraph{Exemple 6 avec des calculs automatiques}

\begin{latexex}
\newcounter{cntaij}
\newcommand\aij{%
    \setcounter{cntaij}{\value{iRow}}%
    \addtocounter{cntaij}{\value{jCol}}%
    \addtocounter{cntaij}{-1}%
    \arabic{cntaij}%
}
Si $a_{ij} = i + j - 1$ alors

$(a_{ij})_{1 \leq i \leq 3 ,
           1 \leq j \leq 5}
=
 \begin{bNiceArray}{*{5}{>{\aij}C}}
    & & & & \\
    & & & & \\
    & & & &
 \end{bNiceArray}$
\end{latexex}

\end{document}
