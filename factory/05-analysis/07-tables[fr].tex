\documentclass[12pt,a4paper]{article}

\makeatletter
    \usepackage[utf8]{inputenc}
\usepackage[T1]{fontenc}
\usepackage{ucs}

\usepackage[french]{babel,varioref}

\usepackage[top=2cm, bottom=2cm, left=1.5cm, right=1.5cm]{geometry}
\usepackage{enumitem}

\usepackage{multicol}

\usepackage{color}
\usepackage{hyperref}
\hypersetup{
    colorlinks,
    citecolor=black,
    filecolor=black,
    linkcolor=black,
    urlcolor=black
}

\usepackage{amsthm}

\usepackage{tcolorbox}
\tcbuselibrary{listingsutf8}

\usepackage{ifplatform}

\usepackage{ifthen}

\usepackage{cbdevtool}


% MISC

\newtcblisting{latexex}{%
	sharp corners,%
	left=1mm, right=1mm,%
	bottom=1mm, top=1mm,%
	colupper=red!75!blue,%
	listing side text
}

\newtcblisting{latexex-flat}{%
	sharp corners,%
	left=1mm, right=1mm,%
	bottom=1mm, top=1mm,%
	colupper=red!75!blue,%
}

\newtcblisting{latexex-alone}{%
	sharp corners,%
	left=1mm, right=1mm,%
	bottom=1mm, top=1mm,%
	colupper=red!75!blue,%
	listing only
}


\newcommand\env[1]{\texttt{#1}}
\newcommand\macro[1]{\env{\textbackslash{}#1}}



\setlength{\parindent}{0cm}
\setlist{noitemsep}

\theoremstyle{definition}
\newtheorem*{remark}{Remarque}

\usepackage[raggedright]{titlesec}

\titleformat{\paragraph}[hang]{\normalfont\normalsize\bfseries}{\theparagraph}{1em}{}
\titlespacing*{\paragraph}{0pt}{3.25ex plus 1ex minus .2ex}{0.5em}


\newcommand\separation{
	\medskip
	\hfill\rule{0.5\textwidth}{0.75pt}\hfill
	\medskip
}


\newcommand\extraspace{
	\vspace{0.25em}
}


\newcommand\ascii{\texttt{ASCII}}


    \usepackage{07-tables}
\makeatother


\begin{document}

% \section{Analysis}

\subsection{Tableaux de variation et de signe}

\paragraph{Comment ça marche ?}

Tout le boulot est fait par le package \verb+tkz-tab+ auquel on impose le choix d'une pointe de flèche plus visible. Nous vous demandons donc de vous reporter à la documentation de \verb+tkz-tab+ pour savoir comment s'y prendre.


% ---------------------- %


\paragraph{Un exemple de tableaux de signes}

\begin{latexex-flat}
\begin{tikzpicture}
    \tkzTabInit{$x$ / 1 , $\cos(x)$ / 1}%
               {$0$, $\frac{\pi}{2}$, $\pi$}
    
    \tkzTabLine{ , + , z , - , }
\end{tikzpicture}
\end{latexex-flat}


% ---------------------- %


\paragraph{Un exemple de tableaux de variation}

\begin{latexex-flat}
\begin{tikzpicture}
    \tkzTabInit{$x$ / 1 , $f(x)$ / 1.5}%
               {$-\infty$, $p$, $+\infty$}
    
    \tkzTabVar{+/ , -/ $f(p)$ , +/ }
\end{tikzpicture}
\end{latexex-flat}


% ---------------------- %


\paragraph{Un exemple de tableaux de variation avec une dérivée}

\begin{latexex-flat}
\begin{tikzpicture}
    \tkzTabInit{$x$ / 1 , $\cos(x)$ / 1, $\sin(x)$ / 1.5}%
               {$0$, $\frac{\pi}{2}$, $\pi$}
    
    \tkzTabLine{ , + , z , - , }
    \tkzTabVar{-/ 0 , +/ 1 , -/ 0}
\end{tikzpicture}
\end{latexex-flat}

\end{document}
