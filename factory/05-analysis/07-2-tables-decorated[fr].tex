\documentclass[12pt,a4paper]{article}

\makeatletter
    \usepackage[utf8]{inputenc}
\usepackage[T1]{fontenc}
\usepackage{ucs}

\usepackage[french]{babel,varioref}

\usepackage[top=2cm, bottom=2cm, left=1.5cm, right=1.5cm]{geometry}
\usepackage{enumitem}

\usepackage{multicol}

\usepackage{makecell}

\usepackage{color}
\usepackage{hyperref}
\hypersetup{
    colorlinks,
    citecolor=black,
    filecolor=black,
    linkcolor=black,
    urlcolor=black
}

\usepackage{amsthm}

\usepackage{tcolorbox}
\tcbuselibrary{listingsutf8}

\usepackage{ifplatform}

\usepackage{ifthen}

\usepackage{cbdevtool}


% MISC

\newtcblisting{latexex}{%
	sharp corners,%
	left=1mm, right=1mm,%
	bottom=1mm, top=1mm,%
	colupper=red!75!blue,%
	listing side text
}

\newtcblisting{latexex-flat}{%
	sharp corners,%
	left=1mm, right=1mm,%
	bottom=1mm, top=1mm,%
	colupper=red!75!blue,%
}

\newtcblisting{latexex-alone}{%
	sharp corners,%
	left=1mm, right=1mm,%
	bottom=1mm, top=1mm,%
	colupper=red!75!blue,%
	listing only
}


\newcommand\env[1]{\texttt{#1}}
\newcommand\macro[1]{\env{\textbackslash{}#1}}



\setlength{\parindent}{0cm}
\setlist{noitemsep}

\theoremstyle{definition}
\newtheorem*{remark}{Remarque}

\usepackage[raggedright]{titlesec}

\titleformat{\paragraph}[hang]{\normalfont\normalsize\bfseries}{\theparagraph}{1em}{}
\titlespacing*{\paragraph}{0pt}{3.25ex plus 1ex minus .2ex}{0.5em}


\newcommand\separation{
	\medskip
	\hfill\rule{0.5\textwidth}{0.75pt}\hfill
	\medskip
}


\newcommand\extraspace{
	\vspace{0.25em}
}


\newcommand\whyprefix[2]{%
	\textbf{\prefix{#1}}-#2%
}

\newcommand\mwhyprefix[2]{%
	\texttt{#1 = #1-#2}%
}

\newcommand\prefix[1]{%
	\texttt{#1}%
}


\newcommand\inenglish{\@ifstar{\@inenglish@star}{\@inenglish@no@star}}

\newcommand\@inenglish@star[1]{%
	\emph{\og #1 \fg}%
}

\newcommand\@inenglish@no@star[1]{%
	\@inenglish@star{#1} en anglais%
}


\newcommand\ascii{\texttt{ASCII}}


% Example
\newcounter{paraexample}[subsubsection]

\newcommand\@newexample@abstract[2]{%
	\paragraph{%
		#1%
		\if\relax\detokenize{#2}\relax\else {} -- #2\fi%
	}%
}



\newcommand\newparaexample{\@ifstar{\@newparaexample@star}{\@newparaexample@no@star}}

\newcommand\@newparaexample@no@star[1]{%
	\refstepcounter{paraexample}%
	\@newexample@abstract{Exemple \theparaexample}{#1}%
}

\newcommand\@newparaexample@star[1]{%
	\@newexample@abstract{Exemple}{#1}%
}


% Change log
\newcommand\topic{\@ifstar{\@topic@star}{\@topic@no@star}}

\newcommand\@topic@no@star[1]{%
	\textbf{\textsc{#1}.}%
}

\newcommand\@topic@star[1]{%
	\textbf{\textsc{#1} :}%
}







    \usepackage{07-tables}
\makeatother


\usepackage{amsmath}

\newcommand\ee{e}


\begin{document}

% \section{Analysis}

%\subsection{Tableaux de variation et de signe}

\subsubsection{Décorer un tableau de signe - TODO}

\paragraph{Motivation}

Considérons le tableau suivant et imaginons que l'on souhaite le justifier tout en l'expliquer à des débutants.

\begin{center}
    \begin{tikzpicture}
        \tkzTabInit[
            lgt   = 3.5, % Il faut de la place pour le dernier produit !
            espcl = 2.5  % On réduit la largeur des colonnes pour les signes.
        ]{
            $x$                             / 1   ,
            Signe de \\ $2 x - 3$           / 1.5 ,
            Signe de \\ $-x + 5$            / 1.5 ,
            Signe de \\ $(2 x - 3)(-x + 5)$ / 1.5 
        }{
                    $-\infty$     , $\dfrac{3}{2}$     , $5$     , $+\infty$
        }
        \tkzTabLine{          , - , z              , + , t   , + ,          }
        \tkzTabLine{          , + , t              , + , z   , - ,          }
        \tkzTabLine{          , - , z              , + , z   , - ,          }
    \end{tikzpicture}
\end{center}
Deux options s'offrent à nous pour justifier comment est rempli le tableau.
\begin{enumerate}
    \item Classiquement, on résout les deux inéquations $2 x - 3 > 0$ et $-x + 5 > 0$ par exemple puis on complète les deux premières lignes 
    \footnote{
        Notons que cette approche est un peu scandaleuse car il faudrait en toute rigueur aussi résoudre
        $2 x - 3 < 0$ , $-x + 5 < 0$ , $2 x - 3 = 0$ et $-x + 5 = 0$.
        Personne ne le fait car l'on pense aux variations d'une fonction affine. Dans ce cas, pourquoi ne pas juste utiliser ce dernier argument. C'est ce que propose la 2\ieme{} méthode. 
    }
    pour en déduire la dernière avec la règle du signe d'un produit.

    \item Tout collégien 
\end{enumerate}

\end{document}
