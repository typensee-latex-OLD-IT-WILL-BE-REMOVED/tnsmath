\documentclass[12pt,a4paper]{article}

\makeatletter
	\usepackage[utf8]{inputenc}
\usepackage[T1]{fontenc}
\usepackage{ucs}

\usepackage[french]{babel,varioref}

\usepackage[top=2cm, bottom=2cm, left=1.5cm, right=1.5cm]{geometry}
\usepackage{enumitem}

\usepackage{multicol}

\usepackage{color}
\usepackage{hyperref}
\hypersetup{
    colorlinks,
    citecolor=black,
    filecolor=black,
    linkcolor=black,
    urlcolor=black
}

\usepackage{amsthm}

\usepackage{tcolorbox}
\tcbuselibrary{listingsutf8}

\usepackage{ifplatform}

\usepackage{ifthen}

\usepackage{cbdevtool}


% MISC

\newtcblisting{latexex}{%
	sharp corners,%
	left=1mm, right=1mm,%
	bottom=1mm, top=1mm,%
	colupper=red!75!blue,%
	listing side text
}

\newtcblisting{latexex-flat}{%
	sharp corners,%
	left=1mm, right=1mm,%
	bottom=1mm, top=1mm,%
	colupper=red!75!blue,%
}

\newtcblisting{latexex-alone}{%
	sharp corners,%
	left=1mm, right=1mm,%
	bottom=1mm, top=1mm,%
	colupper=red!75!blue,%
	listing only
}


\newcommand\env[1]{\texttt{#1}}
\newcommand\macro[1]{\env{\textbackslash{}#1}}



\setlength{\parindent}{0cm}
\setlist{noitemsep}

\theoremstyle{definition}
\newtheorem*{remark}{Remarque}

\usepackage[raggedright]{titlesec}

\titleformat{\paragraph}[hang]{\normalfont\normalsize\bfseries}{\theparagraph}{1em}{}
\titlespacing*{\paragraph}{0pt}{3.25ex plus 1ex minus .2ex}{0.5em}


\newcommand\separation{
	\medskip
	\hfill\rule{0.5\textwidth}{0.75pt}\hfill
	\medskip
}


\newcommand\extraspace{
	\vspace{0.25em}
}


\newcommand\ascii{\texttt{ASCII}}


	\usepackage{06-integral-calculus}
\makeatother


% == EXTRAS == %

\makeatletter
    \newcommand\@interval@tool@star[4]{%
    	\ensuremath{ \left#1 \vphantom{\relsize{1.25}{\text{$\displaystyle F_1^2$}}} \right. \!\! #2 ; #3 \left. \vphantom{\relsize{1.25}{\text{$\displaystyle F_1^2$}}} \!\! \right#4}%
	}

    \newcommand\@interval@tool@no@star[4]{\ensuremath{ \left#1 #2 \, ; #3 \right#4}}
\makeatother

\newcommand\dd[1]{d#1}



\begin{document}

% \section{Analysis}

\subsection{Calcul intégral}

\subsubsection{L'opérateur crochet -- 1\textsuperscript{ère} version}

\paragraph{Exemple d'utilisation 1}

Il faut savoir que \verb+hook+ signifie \inenglish{crochet} mais la bonne traduction est en fait \emph{\og square bracket \fg}.
\begin{latexex}
$\displaystyle
 \int_{a}^{b} f(x) \dd{x}
 = \hook{F(x)}{a}{b}$

$\hook{F(x)}{a}{b}
 = F(b) - F(a)$
\end{latexex}


% ---------------------- %


\paragraph{Exemple d'utilisation 2}

Par défaut, les crochets s'étirent verticalement si besoin, mais si cela vous dérange, vous pouvez faire appel à la version étoilée de la macro comme dans l'exemple suivant

\begin{latexex}
$\displaystyle
 \hook{\frac{x - 1}{5 + x^2}}{a}{b}
 = \hook*{\frac{x - 1}{5 + x^2}}{a}{b}$
\end{latexex}


% ---------------------- %


\subsubsection{Fiches techniques}

\paragraph{L'opérateur crochet -- 1\textsuperscript{ère} version}

\IDmacro*{hook}{3}

\IDmacro*{hook*}{3}

\IDarg{1} le contenu entre les crochets.

\IDarg{2} la borne inférieure affichée en indice.

\IDarg{3} la borne supérieure affichée en exposant.


% ---------------------- %


\subsubsection{L'opérateur crochet -- 2\textsuperscript{nde} version}

\paragraph{Exemple d'utilisation 1}

Ci-dessous \prefix{v} est pour \whyprefix{v}{ertical}.

\begin{latexex}
$\vhook{F(x)}{a}{b}
 = \hook{F(x)}{a}{b}$.
\end{latexex}


% ---------------------- %


\paragraph{Exemple d'utilisation 2}

Tout comme avec la première version de l'opérateur crochet, vous pouvez utiliser une version étoilée pour empêcher l'étirement verticalement du trait vertical. Voici un exemple.

\begin{latexex}
$\displaystyle
 \vhook{\frac{x - 1}{5 + x^2}}{a}{b}
 = \vhook*{\frac{x - 1}{5 + x^2}}{a}{b}$
\end{latexex}


% ---------------------- %


\subsubsection{Fiches techniques}

\paragraph{L'opérateur crochet -- 2\textsuperscript{nde} version}

\IDmacro*{vhook}{3}   où \quad \mwhyprefix{v}{ertical}

\IDmacro*{vhook*}{3}  où \quad \mwhyprefix{v}{ertical}

\IDarg{1} le contenu avant le trait vertical.

\IDarg{2} la borne inférieure affichée en indice.

\IDarg{3} la borne supérieure affichée en exposant.


% ---------------------- %


\subsubsection{Intégrales multiples}

Le package réduit les espacements entres des symboles $\int$ successifs. Voici un exemple.

\begin{latexex-flat}
$\displaystyle
 \int \int \int F(x;y;z) \dd{x} \dd{y} \dd{z}
 = \int_{a}^{b} \int_{c}^{d} \int_{e}^{f} F(x;y;z) \dd{x} \dd{y} \dd{z}$
\end{latexex-flat}


\begin{remark}
	Par défaut, \LaTeX{} affiche
	$\displaystyle
	 \stdint \stdint \stdint F(x;y;z) \dd{x} \dd{y} \dd{z}
     =
     \stdint_{a}^{b} \stdint_{c}^{d} \stdint_{e}^{f}
     F(x;y;z) \dd{x} \dd{y} \dd{z}$.
     Nous avons obtenu ce résultat en utilisant \macro{stdint} qui est l'opérateur proposé de façon standard par \LaTeX.
\end{remark}


% ---------------------- %


\subsubsection{Fiches techniques}

\paragraph{L'opérateur d'intégration standard}

\IDmacro*{stdint}{0}

\end{document}
