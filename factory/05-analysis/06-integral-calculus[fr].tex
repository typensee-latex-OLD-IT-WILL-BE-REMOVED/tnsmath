\documentclass[12pt,a4paper]{article}

\makeatletter
	\usepackage[utf8]{inputenc}
\usepackage[T1]{fontenc}
\usepackage{ucs}

\usepackage[french]{babel,varioref}

\usepackage[top=2cm, bottom=2cm, left=1.5cm, right=1.5cm]{geometry}
\usepackage{enumitem}

\usepackage{multicol}

\usepackage{makecell}

\usepackage{color}
\usepackage{hyperref}
\hypersetup{
    colorlinks,
    citecolor=black,
    filecolor=black,
    linkcolor=black,
    urlcolor=black
}

\usepackage{amsthm}

\usepackage{tcolorbox}
\tcbuselibrary{listingsutf8}

\usepackage{ifplatform}

\usepackage{ifthen}

\usepackage{cbdevtool}


% MISC

\newtcblisting{latexex}{%
	sharp corners,%
	left=1mm, right=1mm,%
	bottom=1mm, top=1mm,%
	colupper=red!75!blue,%
	listing side text
}

\newtcblisting{latexex-flat}{%
	sharp corners,%
	left=1mm, right=1mm,%
	bottom=1mm, top=1mm,%
	colupper=red!75!blue,%
}

\newtcblisting{latexex-alone}{%
	sharp corners,%
	left=1mm, right=1mm,%
	bottom=1mm, top=1mm,%
	colupper=red!75!blue,%
	listing only
}


\newcommand\env[1]{\texttt{#1}}
\newcommand\macro[1]{\env{\textbackslash{}#1}}



\setlength{\parindent}{0cm}
\setlist{noitemsep}

\theoremstyle{definition}
\newtheorem*{remark}{Remarque}

\usepackage[raggedright]{titlesec}

\titleformat{\paragraph}[hang]{\normalfont\normalsize\bfseries}{\theparagraph}{1em}{}
\titlespacing*{\paragraph}{0pt}{3.25ex plus 1ex minus .2ex}{0.5em}


\newcommand\separation{
	\medskip
	\hfill\rule{0.5\textwidth}{0.75pt}\hfill
	\medskip
}


\newcommand\extraspace{
	\vspace{0.25em}
}


\newcommand\whyprefix[2]{%
	\textbf{\prefix{#1}}-#2%
}

\newcommand\mwhyprefix[2]{%
	\texttt{#1 = #1-#2}%
}

\newcommand\prefix[1]{%
	\texttt{#1}%
}


\newcommand\inenglish{\@ifstar{\@inenglish@star}{\@inenglish@no@star}}

\newcommand\@inenglish@star[1]{%
	\emph{\og #1 \fg}%
}

\newcommand\@inenglish@no@star[1]{%
	\@inenglish@star{#1} en anglais%
}


\newcommand\ascii{\texttt{ASCII}}


% Example
\newcounter{paraexample}[subsubsection]

\newcommand\@newexample@abstract[2]{%
	\paragraph{%
		#1%
		\if\relax\detokenize{#2}\relax\else {} -- #2\fi%
	}%
}



\newcommand\newparaexample{\@ifstar{\@newparaexample@star}{\@newparaexample@no@star}}

\newcommand\@newparaexample@no@star[1]{%
	\refstepcounter{paraexample}%
	\@newexample@abstract{Exemple \theparaexample}{#1}%
}

\newcommand\@newparaexample@star[1]{%
	\@newexample@abstract{Exemple}{#1}%
}


% Change log
\newcommand\topic{\@ifstar{\@topic@star}{\@topic@no@star}}

\newcommand\@topic@no@star[1]{%
	\textbf{\textsc{#1}.}%
}

\newcommand\@topic@star[1]{%
	\textbf{\textsc{#1} :}%
}







	\usepackage{06-integral-calculus}
\makeatother


% == EXTRAS == %

\makeatletter
    \newcommand\@interval@tool@star[4]{%
    	\ensuremath{ \left#1 \vphantom{\relsize{1.25}{\text{$\displaystyle F_1^2$}}} \right. \!\! #2 ; #3 \left. \vphantom{\relsize{1.25}{\text{$\displaystyle F_1^2$}}} \!\! \right#4}%
	}

    \newcommand\@interval@tool@no@star[4]{\ensuremath{ \left#1 #2 \, ; #3 \right#4}}
\makeatother

\newcommand\dd[1]{d#1}

\newcommand\eqdef{:=}



\begin{document}

% \section{Analysis}

\subsection{Calcul intégral}

\subsubsection{Intégrales multiples}

Commençons par un point important : le package réduit les espacements entres des symboles $\int$ successifs. Voici un exemple.

\begin{latexex}
$\displaystyle
 \int \int \int 
 F(x;y;z) \dd{x} \dd{y} \dd{z}$

$\displaystyle
 \int_{a}^{b} \int_{c}^{d} \int_{e}^{f} 
 F(x;y;z) \dd{x} \dd{y} \dd{z}$
\end{latexex}


\begin{remark}
	Par défaut, \LaTeX{} affiche
	$\displaystyle
	 \stdint \stdint \stdint
	 F(x;y;z) \dd{x} \dd{y} \dd{z}$
    et
    $\displaystyle
	 \stdint_{a}^{b} \stdint_{c}^{d} \stdint_{e}^{f}
     F(x;y;z) \dd{x} \dd{y} \dd{z}$.
    Nous avons obtenu ce résultat en utilisant \macro{stdint} qui est l'opérateur proposé de façon standard par \LaTeX.
\end{remark}


% ---------------------- %


\subsubsection{Fiches techniques}

\paragraph{L'opérateur d'intégration standard}

\IDmacro*{stdint}{0}


% ---------------------- %


\subsubsection{Une fonctionnelle d'intégration clés en main}

\paragraph{Exemple 1 - À quoi bon ?}

Le 1\ier{} exemple qui suit semblera être une hérésie pour les habitués de \LaTeX{} mais rappelons que le but de \verb+lymath+ est de rendre les documents facilement modifiables globalement ou localement comme le montre le 2\ieme{} exemple.

\begin{latexex}
 $\displaystyle
  \integrate{a}{b}{f(x)}{x}
= \int_{x=a}^{x=b} f(x) \dd{x}$

 $\displaystyle
  \integrate*{a}{b}{f(x)}{x}
  \eqdef
  \integrate{a}{b}{f(x)}{x}$
\end{latexex}


\paragraph{Exemple 2 - Le mode \texttt{displaystyle}}

La macro \macro{dintegrate*} présentée ci-dessous possède aussi une version non étoilée \macro{dintegrate}.

\begin{latexex}
 $\dintegrate*{a}{b}{f(x)}{x}
= \integrate*{a}{b}{f(x)}{x}$
\end{latexex}


% ---------------------- %


\subsubsection{Fiches techniques}

\paragraph{Fonctionnelle d'intégration}

\IDmacro*{integrate}{4}

\IDmacro*{integrate*}{4}

\extraspace

\IDmacro*{dintegrate}{4}   où \quad \mwhyprefix{d}{isplaystyle}

\IDmacro*{dintegrate*}{4}  où \quad \mwhyprefix{d}{isplaystyle}

\IDarg{1} ce qui est en bas du symbole $\int_{\bullet}$ .

\IDarg{2} ce qui est en haut du symbole $\int^{\bullet}$ .

\IDarg{3} la fonction intégrée.

\IDarg{4} suivant quoi on intègre.









% ---------------------- %


\subsubsection{L'opérateur crochet}


\paragraph{Exemple 1}

Dans l'exemple suivant, il peut sembler un peu lourd de taper \verb+\hook*{a}{b}{F(x)}{x}+ alors que $x$ ne sera pas utilisé mais à l'usage ceci permet des copier-coller très efficaces vis à vis de la macro \macro{integrate}.

\begin{latexex}
 $\hook{a}{b}{F(x)}{x}
  \eqdef F(b) - F(a)$

 $\dintegrate*{a}{b}{f(x)}{x}
= \hook*{a}{b}{F(x)}{x}$

\end{latexex}


\begin{remark}
	Il faut savoir que \macro{hook} signifie \inenglish{crochet} mais la bonne traduction du terme mathématique est en fait \emph{\og square bracket \fg}. Ceci étant dit l'auteur de \verb+lymath+ trouve plus efficace d'utiliser \macro{hook} comme nom de macro.
\end{remark}


% ---------------------- %


\paragraph{Exemple 2 - Des crochets non extensibles}

Dans l'exemple suivant, on utilise l'option \prefix{sb} pour \whyprefix{s}{mall} \whyprefix{b}{rackets} soit \inenglish{petits crochets}. Les options sont disponibles à la fois pour \macro{hook} et \macro{hook*}.


\begin{latexex}
 $\displaystyle
  \hook*{a}{b}%
        {\frac{x - 1}{5 + x^2}}{x}
= \hook*[sb]{a}{b}%
            {\frac{x - 1}{5 + x^2}}{x}$
\end{latexex}


% ---------------------- %


\paragraph{Exemple 3 - Un trait vertical épuré}

Via les options \prefix{r} et \prefix{sr} pour \whyprefix{s}{mall} et \whyprefix{r}{ull} soit \inenglish*{petit} et \inenglish{trait}, on obtient ce qui suit.

\begin{latexex}
 $\displaystyle
  \hook[r]{a}{b}%
          {\frac{x - 1}{5 + x^2}}{x}
= \hook[sr]{a}{b}%
           {\frac{x - 1}{5 + x^2}}{x}$
\end{latexex}


% ---------------------- %


\subsubsection{Fiches techniques}

\paragraph{L'opérateur crochet}

\IDmacro{hook}{1}{4}

\IDmacro{hook*}{1}{4}

\IDoption{} la valeur par défaut est \verb+b+. Voici les différentes valeurs possibles.
\begin{enumerate}
	\item \verb+b+ : des crochets extensibles sont utilisés.

	\item \verb+sb+ : des crochets non extensibles sont utilisés.

	\item \verb+r+ : un unique trait vertical extensible est utilisé à droite.

	\item \verb+sr+ : un unique trait vertical non extensible est utilisé à droite.
\end{enumerate}

\IDarg{1} ce qui est en bas du crochet fermant.

\IDarg{2} ce qui est en haut du crochet fermant.

\IDarg{3} la fonction sur laquelle effectuer le calcul.

\IDarg{4} la variable pour les calculs.

\end{document}
