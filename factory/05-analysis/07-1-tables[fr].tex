\documentclass[12pt,a4paper]{article}

\makeatletter
    \usepackage[utf8]{inputenc}
\usepackage[T1]{fontenc}
\usepackage{ucs}

\usepackage[french]{babel,varioref}

\usepackage[top=2cm, bottom=2cm, left=1.5cm, right=1.5cm]{geometry}
\usepackage{enumitem}

\usepackage{multicol}

\usepackage{makecell}

\usepackage{color}
\usepackage{hyperref}
\hypersetup{
    colorlinks,
    citecolor=black,
    filecolor=black,
    linkcolor=black,
    urlcolor=black
}

\usepackage{amsthm}

\usepackage{tcolorbox}
\tcbuselibrary{listingsutf8}

\usepackage{ifplatform}

\usepackage{ifthen}

\usepackage{cbdevtool}


% MISC

\newtcblisting{latexex}{%
	sharp corners,%
	left=1mm, right=1mm,%
	bottom=1mm, top=1mm,%
	colupper=red!75!blue,%
	listing side text
}

\newtcblisting{latexex-flat}{%
	sharp corners,%
	left=1mm, right=1mm,%
	bottom=1mm, top=1mm,%
	colupper=red!75!blue,%
}

\newtcblisting{latexex-alone}{%
	sharp corners,%
	left=1mm, right=1mm,%
	bottom=1mm, top=1mm,%
	colupper=red!75!blue,%
	listing only
}


\newcommand\env[1]{\texttt{#1}}
\newcommand\macro[1]{\env{\textbackslash{}#1}}



\setlength{\parindent}{0cm}
\setlist{noitemsep}

\theoremstyle{definition}
\newtheorem*{remark}{Remarque}

\usepackage[raggedright]{titlesec}

\titleformat{\paragraph}[hang]{\normalfont\normalsize\bfseries}{\theparagraph}{1em}{}
\titlespacing*{\paragraph}{0pt}{3.25ex plus 1ex minus .2ex}{0.5em}


\newcommand\separation{
	\medskip
	\hfill\rule{0.5\textwidth}{0.75pt}\hfill
	\medskip
}


\newcommand\extraspace{
	\vspace{0.25em}
}


\newcommand\whyprefix[2]{%
	\textbf{\prefix{#1}}-#2%
}

\newcommand\mwhyprefix[2]{%
	\texttt{#1 = #1-#2}%
}

\newcommand\prefix[1]{%
	\texttt{#1}%
}


\newcommand\inenglish{\@ifstar{\@inenglish@star}{\@inenglish@no@star}}

\newcommand\@inenglish@star[1]{%
	\emph{\og #1 \fg}%
}

\newcommand\@inenglish@no@star[1]{%
	\@inenglish@star{#1} en anglais%
}


\newcommand\ascii{\texttt{ASCII}}


% Example
\newcounter{paraexample}[subsubsection]

\newcommand\@newexample@abstract[2]{%
	\paragraph{%
		#1%
		\if\relax\detokenize{#2}\relax\else {} -- #2\fi%
	}%
}



\newcommand\newparaexample{\@ifstar{\@newparaexample@star}{\@newparaexample@no@star}}

\newcommand\@newparaexample@no@star[1]{%
	\refstepcounter{paraexample}%
	\@newexample@abstract{Exemple \theparaexample}{#1}%
}

\newcommand\@newparaexample@star[1]{%
	\@newexample@abstract{Exemple}{#1}%
}


% Change log
\newcommand\topic{\@ifstar{\@topic@star}{\@topic@no@star}}

\newcommand\@topic@no@star[1]{%
	\textbf{\textsc{#1}.}%
}

\newcommand\@topic@star[1]{%
	\textbf{\textsc{#1} :}%
}







    \usepackage{07-tables}
\makeatother


\usepackage{amsmath}

\newcommand\ee{e}


\begin{document}

% \section{Analysis}

\subsection{Tableaux de variation et de signe}

\subsubsection{Les bases}

\paragraph{Comment ça marche ?}

Tout le boulot est fait par le package \verb+tkz-tab+ auquel on impose le choix d'une pointe de flèche plus visible via le réglage \verb+\tkzTabSetup[arrowstyle = triangle 60]+.

\medskip

Nous donnons quelques exemples classiques d'utilisation \emph{(les codes ont été mis en forme pour faciliter la compréhension de la syntaxe à suivre)}.
Si besoin reportez vous à la documentation de \verb+tkz-tab+ pour obtenir des compléments d'information. 


% ---------------------- %


\newparaexample{Avec des signes}

\begin{latexex-flat}
\begin{tikzpicture}
    \tkzTabInit{
        $x$       / 1 ,
        $\cos(x)$ / 1
    }{
                $0$    , $\frac{\pi}{2}$     , $\pi$
    }
    \tkzTabLine{   , + , z               , - ,      }
\end{tikzpicture}
\end{latexex-flat}


% ---------------------- %


\newparaexample{ Avec des variations}

\begin{latexex-flat}
\begin{tikzpicture}
    \tkzTabInit{
        $x$    / 1   ,
        $f(x)$ / 1.5
    }{
               $-\infty$ , $p$       , $+\infty$
    }
    \tkzTabVar{+/        , -/ $f(p)$ , +/       }
\end{tikzpicture}
\end{latexex-flat}


% ---------------------- %


\newparaexample{Variations via une dérivée}

\begin{latexex-flat}
\begin{tikzpicture}
    \tkzTabInit{
        $x$       / 1   ,
        $\cos(x)$ / 1   ,
        $\sin(x)$ / 1.5
    }{
                $0$      , $\frac{\pi}{2}$     , $\pi$
    }
    \tkzTabLine{     , + , z               , - ,      }
    \tkzTabVar {-/ 0     , +/ 1                , -/ 0 }
\end{tikzpicture}
\end{latexex-flat}


% ---------------------- %


\newparaexample{Une image intermédiaire avec une seule flèche}

\begin{latexex-flat}
\begin{tikzpicture}
    \tkzTabInit{
        $x$     / 1   ,
        $3 x^2$ / 1   ,
        $x^3$   / 1.5
    }{
               $-\infty$         , $0$   , $+\infty$
    }
    \tkzTabLine{             , + , 0 , + ,             }
    \tkzTabVar {-/ $-\infty$     , R     , +/ $+\infty$}
    %
    \tkzTabIma{1}{3}{2} % Position entre 1ière et 3ième valeur puis rang entierrelatif.
              {$0$}     % Valeur de l'image.
\end{tikzpicture}
\end{latexex-flat}


% ---------------------- %


\newparaexample{Valeurs intermédiaires ou interdites}

\begin{latexex-flat}
\begin{tikzpicture}
    \tkzTabInit[espcl = 6]{ % Largeur entre les valeurs du tableau.
        $x$            /1   ,
        $\dfrac{1}{x}$ /1.25 ,
        $\ln$          /1.75
    }{
                $0$                , $+\infty$
    }
    \tkzTabLine{d              , + ,              }
    \tkzTabVar {D- / $-\infty$     , + / $+\infty$}
    %
    \tkzTabVal{1}{2}{0.35} % Position entre 1ière et 2ième valeur puis en proportion.
              {1}{0}       % x_1 et f(x_1)
    \tkzTabVal{1}{2}{0.65} % Position entre 1ière et 2ième valeur puis en proportion.
              {$\ee$}{1}   % x_2 et f(x_2)
\end{tikzpicture}
\end{latexex-flat}

Voici un autre exemple pour comprendre comment utiliser \macro{tkzTabVal} avec en plus l'option \verb+draw+ qui peut rendre service.

\begin{latexex-flat}
\begin{tikzpicture}
    \tkzTabInit[espcl = 4]{
        $x$     / 1   ,
        $f'(x)$ / 1   , 
        $f(x)$  / 1.5
    }{
               $0$                 , $e$           , $+\infty$}
    \tkzTabLine{d              , + , 0         , - ,          }
    \tkzTabVar {D- / $-\infty$     , + / $\ee$     , - / $0$  }
    %
    \tkzTabVal[draw]{1}{2}{0.5} % Position entre 1ière et 2ième valeur au milieu.
                    {$1$}{$\dfrac{1}{\ee}$}
    \tkzTabVal[draw]{2}{3}{0.5} % Position entre 2ième et 3ième valeur au milieu.
                    {$\ee^2$}{$1$}
\end{tikzpicture}
\end{latexex-flat}

\end{document}
