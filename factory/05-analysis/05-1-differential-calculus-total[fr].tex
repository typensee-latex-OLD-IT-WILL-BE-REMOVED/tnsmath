\documentclass[12pt,a4paper]{article}

\makeatletter
    \usepackage[utf8]{inputenc}
\usepackage[T1]{fontenc}
\usepackage{ucs}

\usepackage[french]{babel,varioref}

\usepackage[top=2cm, bottom=2cm, left=1.5cm, right=1.5cm]{geometry}
\usepackage{enumitem}

\usepackage{multicol}

\usepackage{color}
\usepackage{hyperref}
\hypersetup{
    colorlinks,
    citecolor=black,
    filecolor=black,
    linkcolor=black,
    urlcolor=black
}

\usepackage{amsthm}

\usepackage{tcolorbox}
\tcbuselibrary{listingsutf8}

\usepackage{ifplatform}

\usepackage{ifthen}

\usepackage{cbdevtool}


% MISC

\newtcblisting{latexex}{%
	sharp corners,%
	left=1mm, right=1mm,%
	bottom=1mm, top=1mm,%
	colupper=red!75!blue,%
	listing side text
}

\newtcblisting{latexex-flat}{%
	sharp corners,%
	left=1mm, right=1mm,%
	bottom=1mm, top=1mm,%
	colupper=red!75!blue,%
}

\newtcblisting{latexex-alone}{%
	sharp corners,%
	left=1mm, right=1mm,%
	bottom=1mm, top=1mm,%
	colupper=red!75!blue,%
	listing only
}


\newcommand\env[1]{\texttt{#1}}
\newcommand\macro[1]{\env{\textbackslash{}#1}}



\setlength{\parindent}{0cm}
\setlist{noitemsep}

\theoremstyle{definition}
\newtheorem*{remark}{Remarque}

\usepackage[raggedright]{titlesec}

\titleformat{\paragraph}[hang]{\normalfont\normalsize\bfseries}{\theparagraph}{1em}{}
\titlespacing*{\paragraph}{0pt}{3.25ex plus 1ex minus .2ex}{0.5em}


\newcommand\separation{
	\medskip
	\hfill\rule{0.5\textwidth}{0.75pt}\hfill
	\medskip
}


\newcommand\extraspace{
	\vspace{0.25em}
}


\newcommand\ascii{\texttt{ASCII}}

    % == PACKAGES USED == %

\RequirePackage{amsmath}
\RequirePackage{relsize}
\RequirePackage{xparse}


% == DEFINITIONS == %

% Settable texts
\@ifpackagewith{babel}{french}{
    \newcommand\lymathsep{;}
    \newcommand\lymathsubsep{,}
    \newcommand\textopchoice{choix}
    \newcommand\textopcond{cond}
    \newcommand\textopdef{déf}
    \newcommand\textophyp{hyp}
    \newcommand\textopid{id}
}{
	\newcommand\lymathsep{,}
	\newcommand\lymathsubsep{;}
    \newcommand\textopchoice{choice}
    \newcommand\textopcond{cond}
    \newcommand\textopdef{def}
    \newcommand\textophyp{hyp}
    \newcommand\textopid{id}
}


\newcommand\textexplainleft{\{}
\newcommand\textexplainright{\}}
\newcommand\textexplainspacebefore{\qquad}
\newcommand\textexplainspacein{\qquad}


% Tools - Apply same macro to all arguments

% #1        : main macro
% #2        : macro to apply to arguments
% #3 and #4 : the two arguments
\newcommand\@apply@macro@two@args[4]{%
	#1{#2{#3}}{#2{#4}}%
}




% Tools - Intervals

\newcommand\@extra@phantom{%
	\vphantom{\relsize{1.25}{\text{$\displaystyle F_1^2$}}}%
}

\newcommand\@interval@tool@star[5]{%
	\ensuremath{ \left#1 \@extra@phantom \right. \!\! #2 #3 #4 \left. \@extra@phantom \!\! \right#5}%
}

\newcommand\@interval@tool@no@star[5]{\ensuremath{ \left#1 #2 #3 #4 \right#5}}


% Tools - Multi-arguments
%
% Source : the following lines come directly for the following post
%
%    * https://tex.stackexchange.com/a/475291/6880

\ExplSyntaxOn
% General purpose macro for defining other macros
	\NewDocumentCommand{\makemultiargument}{mmmmmo}{
		\lymath_multiarg:nnnnnn{#1}{#2}{#3}{#4}{#5}{#6}
	}
 
% Allocate a private variable
	\seq_new:N \l__lymath_generic_seq

% The internal version of the general purpose macro
	\cs_new_protected:Nn \lymath_multiarg:nnnnnn{
		% #1 = separator
		% #2 = multiargument
		% #3 = code before
	  	% #4 = code between
	  	% #5 = code after
	  	% #6 = ornament to items

		% A group allows nesting
		\group_begin:
	 	% Split the multiargument into parts
		\seq_set_split:Nnn \l__lymath_generic_seq { #1 } { #2 }
		% Apply the ornament to the items
	  	\tl_if_novalue:nF { #6 }{
	    	\seq_set_eq:NN \l__lymath_temp_seq \l__lymath_generic_seq
	    	\seq_set_map:NNn \l__lymath_generic_seq \l__lymath_generic_seq { #6 }
	   	}
		% Execute the <code before>
	  	#3
		% Deliver the items, with the chosen material between them
	  	\seq_use:Nn \l__lymath_generic_seq { #4 }
  		% Execute the <code after>
	 	#5
  		% End the group started at the beginning
	  	\group_end:
	}	
\ExplSyntaxOff


    \usepackage{05-differential-calculus}
\makeatother


\begin{document}

% \section{Analysis}

\subsection{Calcul différentiel}

\subsubsection{\texorpdfstring{Les opérateurs $\pp{}$ et $\dd{}$}%
                               {Les opérateurs "d rond" et "d droit"}}

\newparaexample{}

\begin{latexex}
$\dd{t} = \dd[1]{t}$ ou $\dd[n]{x}$

$\pp{t} = \pp[1]{t}$ ou $\pp[n]{x}$
\end{latexex}


% ---------------------- %


\subsubsection{Fiches techniques}

\paragraph{Calcul différentiel}

\IDmacro{dd}{1}{1}

\IDmacro{pp}{1}{1}

\IDoption{} utilisée, cette option sera mise en exposant du symbole $\pp{}$ ou $\dd{}$.

\IDarg{} la variable de différentiation à droite du symbole $\pp{}$ ou $\dd{}$.


% ---------------------- %


\subsubsection{Dérivations totales d'une fonction -- Version longue mais polymorphe}

\newparaexample{Différentes écritures possibles}

La macro \macro{der} est stricte du point de vue sémantique car on doit lui fournir la fonction, l'ordre de dérivation et la variable de dérivation
\emph{(voir la section \ref{short-der} qui présente la macro \macro{sder} permettant une rédaction efficace pour obtenir $\sder[e]{f}{1}$ ou $\sder{f}{1}$)}.
Voici plusieurs mises en forme faciles à taper via l'option de \macro{der}.
Attention bien entendu à n'utiliser l'option par défaut \prefix{u} qu'avec un ordre de dérivation de valeur naturelle connue !

\begin{latexex}
 $\der   {f}{3}{x}
= \der[e]{f}{3}{x}$

 $\der[i] {u}{k}{x}
= \der[f] {u}{k}{x}
= \der[sf]{u}{k}{x}$
\end{latexex}


On peut aussi ajouter autour de la fonction des parenthèses extensibles ou non.
Ci-dessous on montre aussi une écriture du type \emph{\og opérateur fonctionnel \fg}.

\begin{latexex}
 $\der[osf,sp]{\frac{1}{2} uv}{k}{x}
= \der[of,p]  {\dfrac{1}{2} uv}{k}{x}$
\end{latexex}


\begin{remark}
	Expliquons les valeurs des options.
	\begin{enumerate}
		\item \prefix{u}, la valeur par défaut, est pour \whyprefix{u}{suel} soit l'écriture avec les primes. Cette option ne marchera pas avec un nombre symbolique de dérivations. 

		\item \prefix{e} est pour \whyprefix{e}{xposant}.

		\item \prefix{i} est pour \whyprefix{i}{ndice}.

		\item \prefix{f} est pour \whyprefix{f}{raction} avec aussi \prefix{sf} pour une écriture réduite où \prefix{s} est pour \whyprefix{s}{mall} soit \inenglish{petit}.

		\item \prefix{of} et \prefix{osf} utilisent le préfixe \prefix{o} pour \whyprefix{o}{pérateur}.
		
		\smallskip
		\item \prefix{p} est pour \whyprefix{p}{arenthèse} : dans ce cas les parenthèses seront extensibles.

		\item \prefix{sp} est pour des parenthèses non extensibles.
	\end{enumerate}
\end{remark}


% ---------------------- %


\newparaexample{Pas de uns inutiles}

\begin{latexex}
 $\der[i ]{u}{1}{x}
= \der[f ]{u}{1}{x}
= \der[sf]{u}{1}{x}
= \der[of]{u}{1}{x}$
\end{latexex}


\begin{remark}
	Voici comment forcer les exposants $1$ si besoin.

	\begin{latexex}
 $\der[i ]{u}{\,\!1}{x}
= \der[f ]{u}{\,\!1}{x}
= \der[sf]{u}{\,\!1}{x}
= \der[of]{u}{\,\!1}{x}$
\end{latexex}
\end{remark}


% ---------------------- %


\subsubsection{Dérivations totales d'une fonction -- Version courte pour les écritures standard} \label{short-der}

Dans l'exemple suivant le code manque de sémantique car on n'indique pas la variable de dérivation.
Ceci étant dit à l'usage la macro \macro{sder} rend de grands services.
Ici le préfixe \prefix{s} est pour \whyprefix{s}{imple} voire \whyprefix{s}{impliste}...
Voici des exemples où de nouveau l'option par défaut \prefix{u} ne sera fonctionnelle qu'avec un ordre de dérivation de valeur naturelle connue !

\begin{latexex}
 $\sder{f}{1} = \der{f}{1}{x}$

 $\sder{f}{1}
= \sder[e]{f}{1}$

 $\sder[sp ]{\dfrac{1}{2} uv}{2}
= \sder[e,p]{\dfrac{1}{2} uv}{2}$
\end{latexex}


\begin{remark}
	Ici les seules options disponibles sont \prefix{u}, \prefix{e}, \prefix{p} et \prefix{sp}.
\end{remark}


% ---------------------- %


\subsubsection{L'opérateur de dérivation totale}

Ce qui suit peut rendre service au niveau universitaire.
Les options possibles sont \verb+f+, valeur par défaut, \verb+sf+ et \verb+i+ avec les mêmes significations que pour la macro \macro{der}.

\begin{latexex}
 $\derope    {k}{x}
= \derope[sf]{k}{x}
= \derope[i] {k}{x}$

 $\derope    {1}{x}
= \derope[sf]{1}{x}
= \derope[i] {1}{x}$
\end{latexex}


% ---------------------- %


\subsubsection{Fiches techniques}

\paragraph{Dérivations totales}

\IDmacro{der}{1}{3}

\IDoption{} la valeur par défaut est \verb+u+. 
\begin{enumerate}
	\item \verb+u+ : écriture usuelle avec des primes \emph{(ceci nécessite d'avoir une valeur entière naturelle connue du nombre de dérivations successives)}.

	\item \verb+e+ : écriture via un exposant entre des parenthèses.
	
	\item \verb+i+ : écriture via un indice.

	\item \verb+f+ : écriture via une fraction en mode display.

	\item \verb+sf+ : écriture via une fraction en mode non display.

	\item \verb+of+ : écriture via une fraction en mode display sous la forme d'un opérateur \emph{(la fonction est à côté de la fraction)}.

	\item \verb+osf+ : écriture via une fraction en mode non display sous la forme d'un opérateur \emph{(la fonction est à côté de la fraction)}.

	\smallskip
	\item \verb+p+ : ajout de parenthèses extensibles autour de la fonction.

	\item \verb+sp+ : ajout de parenthèses non extensibles autour de la fonction.
\end{enumerate}


\IDarg{1} la fonction à dériver.

\IDarg{2} l'ordre de dérivation.

\IDarg{3} la variable de dérivation.


\separation


\IDmacro{sder}{1}{2} où \quad \mwhyprefix{s}{imple}

\IDoption{} la valeur par défaut est \verb+u+. Les options disponibles sont \verb+u+, \verb+e+, \verb+p+ et \verb+sp+ : voir la fiche technique de \macro{sder} ci-dessus.

\IDarg{1} la fonction à dériver.

\IDarg{2} l'ordre de dérivation.


% ---------------------- %


\paragraph{Dérivation totale -- Opérateur fonctionnel}

\IDmacro{derope}{1}{2} où \quad \mwhyprefix{ope}{rator}

\IDoption{} la valeur par défaut est \verb+f+. Les options disponibles sont \verb+f+, \verb+sf+ et \verb+i+ : voir la fiche technique de \macro{der} donnée un peu plus haut.

\IDarg{1} la fonction à dériver.

\IDarg{2} l'ordre de dérivation.

\end{document}
