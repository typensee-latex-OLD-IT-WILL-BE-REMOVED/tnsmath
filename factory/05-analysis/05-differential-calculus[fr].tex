\documentclass[12pt,a4paper]{article}

\makeatletter
	\usepackage[utf8]{inputenc}
\usepackage[T1]{fontenc}
\usepackage{ucs}

\usepackage[french]{babel,varioref}

\usepackage[top=2cm, bottom=2cm, left=1.5cm, right=1.5cm]{geometry}
\usepackage{enumitem}

\usepackage{multicol}

\usepackage{color}
\usepackage{hyperref}
\hypersetup{
    colorlinks,
    citecolor=black,
    filecolor=black,
    linkcolor=black,
    urlcolor=black
}

\usepackage{amsthm}

\usepackage{tcolorbox}
\tcbuselibrary{listingsutf8}

\usepackage{ifplatform}

\usepackage{ifthen}

\usepackage{cbdevtool}


% MISC

\newtcblisting{latexex}{%
	sharp corners,%
	left=1mm, right=1mm,%
	bottom=1mm, top=1mm,%
	colupper=red!75!blue,%
	listing side text
}

\newtcblisting{latexex-flat}{%
	sharp corners,%
	left=1mm, right=1mm,%
	bottom=1mm, top=1mm,%
	colupper=red!75!blue,%
}

\newtcblisting{latexex-alone}{%
	sharp corners,%
	left=1mm, right=1mm,%
	bottom=1mm, top=1mm,%
	colupper=red!75!blue,%
	listing only
}


\newcommand\env[1]{\texttt{#1}}
\newcommand\macro[1]{\env{\textbackslash{}#1}}



\setlength{\parindent}{0cm}
\setlist{noitemsep}

\theoremstyle{definition}
\newtheorem*{remark}{Remarque}

\usepackage[raggedright]{titlesec}

\titleformat{\paragraph}[hang]{\normalfont\normalsize\bfseries}{\theparagraph}{1em}{}
\titlespacing*{\paragraph}{0pt}{3.25ex plus 1ex minus .2ex}{0.5em}


\newcommand\separation{
	\medskip
	\hfill\rule{0.5\textwidth}{0.75pt}\hfill
	\medskip
}


\newcommand\extraspace{
	\vspace{0.25em}
}


\newcommand\ascii{\texttt{ASCII}}

	% == PACKAGES USED == %

\RequirePackage{amsmath}
\RequirePackage{relsize}
\RequirePackage{xparse}


% == DEFINITIONS == %

% Settable texts
\@ifpackagewith{babel}{french}{
    \newcommand\lymathsep{;}
    \newcommand\lymathsubsep{,}
    \newcommand\textopchoice{choix}
    \newcommand\textopcond{cond}
    \newcommand\textopdef{déf}
    \newcommand\textophyp{hyp}
    \newcommand\textopid{id}
}{
	\newcommand\lymathsep{,}
	\newcommand\lymathsubsep{;}
    \newcommand\textopchoice{choice}
    \newcommand\textopcond{cond}
    \newcommand\textopdef{def}
    \newcommand\textophyp{hyp}
    \newcommand\textopid{id}
}


\newcommand\textexplainleft{\{}
\newcommand\textexplainright{\}}
\newcommand\textexplainspacebefore{\qquad}
\newcommand\textexplainspacein{\qquad}


% Tools - Apply same macro to all arguments

% #1        : main macro
% #2        : macro to apply to arguments
% #3 and #4 : the two arguments
\newcommand\@apply@macro@two@args[4]{%
	#1{#2{#3}}{#2{#4}}%
}




% Tools - Intervals

\newcommand\@extra@phantom{%
	\vphantom{\relsize{1.25}{\text{$\displaystyle F_1^2$}}}%
}

\newcommand\@interval@tool@star[5]{%
	\ensuremath{ \left#1 \@extra@phantom \right. \!\! #2 #3 #4 \left. \@extra@phantom \!\! \right#5}%
}

\newcommand\@interval@tool@no@star[5]{\ensuremath{ \left#1 #2 #3 #4 \right#5}}


% Tools - Multi-arguments
%
% Source : the following lines come directly for the following post
%
%    * https://tex.stackexchange.com/a/475291/6880

\ExplSyntaxOn
% General purpose macro for defining other macros
	\NewDocumentCommand{\makemultiargument}{mmmmmo}{
		\lymath_multiarg:nnnnnn{#1}{#2}{#3}{#4}{#5}{#6}
	}
 
% Allocate a private variable
	\seq_new:N \l__lymath_generic_seq

% The internal version of the general purpose macro
	\cs_new_protected:Nn \lymath_multiarg:nnnnnn{
		% #1 = separator
		% #2 = multiargument
		% #3 = code before
	  	% #4 = code between
	  	% #5 = code after
	  	% #6 = ornament to items

		% A group allows nesting
		\group_begin:
	 	% Split the multiargument into parts
		\seq_set_split:Nnn \l__lymath_generic_seq { #1 } { #2 }
		% Apply the ornament to the items
	  	\tl_if_novalue:nF { #6 }{
	    	\seq_set_eq:NN \l__lymath_temp_seq \l__lymath_generic_seq
	    	\seq_set_map:NNn \l__lymath_generic_seq \l__lymath_generic_seq { #6 }
	   	}
		% Execute the <code before>
	  	#3
		% Deliver the items, with the chosen material between them
	  	\seq_use:Nn \l__lymath_generic_seq { #4 }
  		% Execute the <code after>
	 	#5
  		% End the group started at the beginning
	  	\group_end:
	}	
\ExplSyntaxOff


	\usepackage{05-differential-calculus}
\makeatother


\begin{document}

% \section{Analysis}

\subsection{Calcul différentiel}

\subsubsection{\texorpdfstring{Les opérateurs $\pp{}$ et $\dd{}$}%
                               {Les opérateurs "d rond" et "d droit"}}

\paragraph{Exemple d'utilisation}

\begin{latexex}
$\dd{f}$ , $\pp{t}$ ,
$\dd[5]{x}$ ou $\pp[n]{x}$
\end{latexex}


% ---------------------- %


\subsubsection{Fiches techniques}

\paragraph{Calcul différentiel}

\IDmacro{dd}{1}{1}

\IDmacro{pp}{1}{1}

\IDoption{} utilisée, cette option sera mise en exposant du symbole $\pp{}$ ou $\dd{}$.

\IDarg{} la variable de différentiation à droite du symbole $\pp{}$ ou $\dd{}$.


% ---------------------- %


\subsubsection{Dérivation totale}

\paragraph{Exemple d'utilisation 1}

\begin{latexex}
$\displaystyle
 \derpow*{f} (a) 
 = \derpow{f} (a)
 = \derfrac{f}{x} (a)
 = \dersub{f}{x} (a)$
\end{latexex}


% ---------------------- %


\paragraph{Exemple d'utilisation 2}

\begin{latexex}
$\displaystyle
 \derpow*[3]{f}(a) 
 = \derpow[3]{f} (a)
 = \derfrac[3]{f}{x} (a)
 = \dersub[3]{f}{x} (a)$

$\displaystyle
 \derpow*[10]{\cos} a 
 = \derfrac[10]{\cos}{x} (a)$
\end{latexex}


% ---------------------- %


\paragraph{Exemple d'utilisation 3}

\begin{latexex}
$\displaystyle
 \derpow[3]{f} (a)
 = \derfrac*[3]{\left( \frac{1}{x} \right)}{x} (a)$
\end{latexex}


% ---------------------- %


\paragraph{Exemple d'utilisation 4}

\begin{latexex}
$\displaystyle
 \derpar[3]{\dfrac{1}{2} uv}
 = \derpar*[3]{\dfrac{1}{2} uv}$

$\displaystyle
 \sderpar[3]{\dfrac{1}{2} uv}
 = \sderpar*[3]{\dfrac{1}{2} uv}$
\end{latexex}


% ---------------------- %


\subsubsection{Fiches techniques}

\paragraph{Dérivation totale}

\IDmacro{derpar}{1}{1}

\IDmacro{derpar*}{1}{1}

\extraspace

\IDmacro{sderpar}{1}{1}

\IDmacro{sderpar*}{1}{1}

\extraspace

\IDmacro{derpow}{1}{1}

\IDmacro{derpow*}{1}{1}

\IDoption{} utilisée, cette option sera l'exposant de dérivation mis entre des parenthèses pour la version non étoilée, et le nombre de primes pour la version étoilée.

\IDarg{} la fonction à différencier.


\separation


\IDmacro{derfrac}{1}{2}

\IDmacro{derfrac*}{1}{2}

\IDmacro{dersub}{1}{2}

\IDoption{} utilisée, cette option sera l'exposant de dérivation.

\IDarg{1} la fonction à dériver.

\IDarg{2} la variable.


% ---------------------- %


\subsubsection{Dérivation partielle}

\paragraph{Exemple d'utilisation 1}

\begin{latexex}
$\displaystyle
 \partialfrac{f}{x} (a;b)
 = \partialsub{f}{x} (a;b)
 = \partialprime{f}{x} (a;b)$
\end{latexex}


% ---------------------- %


\paragraph{Exemple d'utilisation 2}

\begin{latexex}
$\displaystyle
 \partialfrac[3]{G}{f^2 | v} (a;b)
 = \partialfrac{G}{f^2 | v} (a;b)
 = \dots$
 
$\displaystyle
 \dots
 =\partialsub{G}{f^2 | v} (a;b)
 = \partialprime{G}{f^2 | v} (a;b)$
\end{latexex}


% ---------------------- %


\paragraph{Exemple d'utilisation 3}

\begin{latexex}
$\displaystyle
 \partialfrac[2]{f}{x | y}
 = \partialfrac*[2]{%
      \left( \frac{1}{cos(x y)} \right)%
   }{x | y}$
\end{latexex}


% ---------------------- %


\subsubsection{Fiches techniques}

\paragraph{Dérivation partielle}


\IDmacro{partialfrac}{1}{2}

\IDmacro{partialfrac*}{1}{2}

\IDoption{} utilisée, cette option sera l'exposant total de dérivation mis en exposant de $\pp$.

\IDarg{1} la fonction à dériver partiellement.

\IDarg{2} les variables utilisées pour la dérivation partielle en utilisant la syntaxe suivante : par exemple, \verb+x | y^3 | ...+ indique de dériver suivant $x$ une fois, puis suivant $y$ trois fois... etc.


\separation


\IDmacro*{partialsub}{2}

\IDmacro*{partialprime}{2}

\IDarg{1} la fonction à dériver partiellement.

\IDarg{2} les variables utilisées pour la dérivation partielle en utilisant la syntaxe suivante : par exemple, \verb+x | y^3 | ...+ indique de dériver suivant $x$ une fois, puis suivant $y$ trois fois... etc.

\end{document}
