\documentclass[12pt,a4paper]{article}

\makeatletter
	\usepackage[utf8]{inputenc}
\usepackage[T1]{fontenc}
\usepackage{ucs}

\usepackage[french]{babel,varioref}

\usepackage[top=2cm, bottom=2cm, left=1.5cm, right=1.5cm]{geometry}
\usepackage{enumitem}

\usepackage{multicol}

\usepackage{makecell}

\usepackage{color}
\usepackage{hyperref}
\hypersetup{
    colorlinks,
    citecolor=black,
    filecolor=black,
    linkcolor=black,
    urlcolor=black
}

\usepackage{amsthm}

\usepackage{tcolorbox}
\tcbuselibrary{listingsutf8}

\usepackage{ifplatform}

\usepackage{ifthen}

\usepackage{cbdevtool}


% MISC

\newtcblisting{latexex}{%
	sharp corners,%
	left=1mm, right=1mm,%
	bottom=1mm, top=1mm,%
	colupper=red!75!blue,%
	listing side text
}

\newtcblisting{latexex-flat}{%
	sharp corners,%
	left=1mm, right=1mm,%
	bottom=1mm, top=1mm,%
	colupper=red!75!blue,%
}

\newtcblisting{latexex-alone}{%
	sharp corners,%
	left=1mm, right=1mm,%
	bottom=1mm, top=1mm,%
	colupper=red!75!blue,%
	listing only
}


\newcommand\env[1]{\texttt{#1}}
\newcommand\macro[1]{\env{\textbackslash{}#1}}



\setlength{\parindent}{0cm}
\setlist{noitemsep}

\theoremstyle{definition}
\newtheorem*{remark}{Remarque}

\usepackage[raggedright]{titlesec}

\titleformat{\paragraph}[hang]{\normalfont\normalsize\bfseries}{\theparagraph}{1em}{}
\titlespacing*{\paragraph}{0pt}{3.25ex plus 1ex minus .2ex}{0.5em}


\newcommand\separation{
	\medskip
	\hfill\rule{0.5\textwidth}{0.75pt}\hfill
	\medskip
}


\newcommand\extraspace{
	\vspace{0.25em}
}


\newcommand\whyprefix[2]{%
	\textbf{\prefix{#1}}-#2%
}

\newcommand\mwhyprefix[2]{%
	\texttt{#1 = #1-#2}%
}

\newcommand\prefix[1]{%
	\texttt{#1}%
}


\newcommand\inenglish{\@ifstar{\@inenglish@star}{\@inenglish@no@star}}

\newcommand\@inenglish@star[1]{%
	\emph{\og #1 \fg}%
}

\newcommand\@inenglish@no@star[1]{%
	\@inenglish@star{#1} en anglais%
}


\newcommand\ascii{\texttt{ASCII}}


% Example
\newcounter{paraexample}[subsubsection]

\newcommand\@newexample@abstract[2]{%
	\paragraph{%
		#1%
		\if\relax\detokenize{#2}\relax\else {} -- #2\fi%
	}%
}



\newcommand\newparaexample{\@ifstar{\@newparaexample@star}{\@newparaexample@no@star}}

\newcommand\@newparaexample@no@star[1]{%
	\refstepcounter{paraexample}%
	\@newexample@abstract{Exemple \theparaexample}{#1}%
}

\newcommand\@newparaexample@star[1]{%
	\@newexample@abstract{Exemple}{#1}%
}


% Change log
\newcommand\topic{\@ifstar{\@topic@star}{\@topic@no@star}}

\newcommand\@topic@no@star[1]{%
	\textbf{\textsc{#1}.}%
}

\newcommand\@topic@star[1]{%
	\textbf{\textsc{#1} :}%
}







	\usepackage{01-equal-signs-n-co}
\makeatother


\begin{document}

\section{Différents types de comparaisons \og standard \fg}

D'un point de vue pédagogique, il peut être intéressant de disposer de différentes façon d'écrire une égalité, une non égalité ou une inégalité.
Bien entendu on tord les règles de typographie avec ce type de pratique mais c'est pour le bien de la communauté éducative.


\subsection{Définir quelque chose}

L'exemple suivant montre deux façons de rédiger une égalité signifiant une définition \emph{(la section \ref{tnslog-texts-for-opes} explique comment est défini le texte \emph{\og \txtopdef \fg})}.

\begin{latexex}
$f(x) \eqdef x^3 + 1$

$f(x) \eqdef* x^3 + 1$
\end{latexex}


% ---------------------- %


\subsection{Indiquer une identité}

L'exemple suivant montre deux façons de rédiger des identités avec une notation symbolique non standard \emph{(la section \ref{tnslog-texts-for-opes} explique comment est défini le texte \emph{\og \txtopid \fg})}.

\begin{latexex}
$(a + b)^2 \eqid a^2 + b^2 + 2 a b$

$(a + b)^2 \eqid* a^2 + b^2 + 2 a b$
\end{latexex}


% ---------------------- %


\subsection{Une égalité à vérifier ou non, une hypothèse, une condition}

Se reporter à la section \ref{tnslog-texts-for-opes} pour savoir comment sont définis les textes \emph{\og \txtopcons \fg} , \emph{\og \txtopcond \fg} et \emph{\og \txtophyp \fg}.

\begin{latexex}
$(a + b)^3 \eqtest a^3 + b^3 + 3 a b$

$(a + b)^3 \neqid a^3 + b^3 + 3 a b$

$x \neqhyp 0$  ou
$x \neqcond 0$ ou
$x \eqcons 0$
\end{latexex}


% ---------------------- %


\subsection{Une égalité indiquant le choix d'une valeur ou l'application d'une relation}

La section \ref{tnslog-texts-for-opes} permet de savoir comment les textes \emph{\og \txtopchoice \fg} et \emph{\og \txtopappli \fg} sont définis.

\begin{latexex}
$x \geqcond 4$ implique
$x^2 \geqcons 16$.

Donc $x \eqchoice 123$ donne
$123^2 \geqappli 16$.
\end{latexex}


% ---------------------- %


\subsection{Une égalité indiquant l'équation d'une courbe}

La section \ref{tnslog-texts-for-opes} permet de savoir comment les texte \emph{\og \txtopplot \fg} est défini.

\begin{latexex}
$M \in C: y \eqplot x^2 + 3$
donne
$y_M \eqappli x_M^2 + 3$.
\end{latexex}


% ---------------------- %


\subsection{Différents types d'inéquations}

Le principe reste le même pour les symboles d'équations excepté qu'il n'y a ici aucune écriture purement symbolique. Voici un code \og fourre-tout \fg{} montrant quelques exemples.

\begin{latexex}
$x \leqtest x^2$ ou $x \lesscons x^2$ ou
$x \geqhyp 1$    ou $x \gtrcond 2$.
\end{latexex}


% ---------------------- %


\subsection{Des formes négatives aussi pour les inéquations}

Tous les opérateurs de comparaison ont une forme négative qui s'obtient en préfixant le nom de l'opérateur par \verb+n+.
Voici quelques exemples d'utilisation.

\begin{latexex}
$x \nlesshyp 3$ ou
$y \nleqtest 4$ ou
$z \ngeqcons 5$
\end{latexex}


% ---------------------- %


\subsection{Une table récapitulative}

La table \ref{tnslog-table:deco-opes} \vpageref{tnslog-table:deco-opes} fournit toutes les associations autorisées entre opérateurs de comparaison et décorations.


% ---------------------- %


\subsection{Textes utilisés} \label{tnslog-texts-for-opes}

Voici les macros définissant les textes utilisés qui tiennent compte de l'utilisation ou non de l'option \verb+french+ de \verb+babel+. Nous ne donnons que les versions françaises.

\vspace{-.5em}

\begin{center}
	\begin{tabular}{l@{\kern1ex}l@{\kern2cm}l@{\kern1ex}l}
% == All texts - START == %
        \macro{txtopappli\{\}} & donne \emph{\og \txtopappli \fg}
        & \macro{txtopchoice\{\}} & donne \emph{\og \txtopchoice \fg} \\
        \macro{txtopcond\{\}} & donne \emph{\og \txtopcond \fg}
        & \macro{txtopcons\{\}} & donne \emph{\og \txtopcons \fg} \\
        \macro{txtopdef\{\}} & donne \emph{\og \txtopdef \fg}
        & \macro{txtophyp\{\}} & donne \emph{\og \txtophyp \fg} \\
        \macro{txtopid\{\}} & donne \emph{\og \txtopid \fg}
        & \macro{txtopplot\{\}} & donne \emph{\og \txtopplot \fg} \\
        \macro{txtoptest\{\}} & donne \emph{\og \txtoptest \fg}
        & & \\
% == All texts - END == %
	\end{tabular}
\end{center}

% ---------------------- %


\section{Fiches techniques}

\subsection{Opérateurs décorés -- Les textes}

\begin{multicols}{4}
% == Technical infos - Texts - START == %

\foreach \k in {txtopappli, txtopchoice, txtopcond, txtopcons, txtopdef, txtophyp, txtopid, txtopplot, txtoptest}{

	\IDmacro[n]{\k}

}

% == Technical infos - Texts - END == %

\end{multicols}

% ---------------------- %


\subsection{Opérateurs de comparaison supplémentaires}

% == Technical infos - Operators - START == %

\begin{multicols}{6}
    \foreach \k in {eqdef, eqdef*, eqid, eqid*, eqplot, eqappli, eqchoice, eqcond, eqcons, eqhyp, eqtest}{
        \IDope{\k}

    }
\end{multicols}

\separation

\begin{multicols}{6}
    \foreach \k in {neqid, neqplot, neqappli, neqchoice, neqcond, neqcons, neqhyp, neqtest}{
        \IDope{\k}

    }
\end{multicols}

\separation

\begin{multicols}{6}
    \foreach \k in {lessplot, lessappli, lesschoice, lesscond, lesscons, lesshyp, lesstest}{
        \IDope{\k}

    }
\end{multicols}

\separation

\begin{multicols}{6}
    \foreach \k in {nlessplot, nlessappli, nlesschoice, nlesscond, nlesscons, nlesshyp, nlesstest}{
        \IDope{\k}

    }
\end{multicols}

\separation

\begin{multicols}{6}
    \foreach \k in {leqplot, leqappli, leqchoice, leqcond, leqcons, leqhyp, leqtest}{
        \IDope{\k}

    }
\end{multicols}

\separation

\begin{multicols}{6}
    \foreach \k in {nleqplot, nleqappli, nleqchoice, nleqcond, nleqcons, nleqhyp, nleqtest}{
        \IDope{\k}

    }
\end{multicols}

\separation

\begin{multicols}{6}
    \foreach \k in {gtrplot, gtrappli, gtrchoice, gtrcond, gtrcons, gtrhyp, gtrtest}{
        \IDope{\k}

    }
\end{multicols}

\separation
% == Technical infos - Operators - END == %

\end{document}

