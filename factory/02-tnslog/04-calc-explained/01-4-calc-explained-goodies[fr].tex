\documentclass[12pt,a4paper]{article}

\makeatletter
    \usepackage[utf8]{inputenc}
\usepackage[T1]{fontenc}
\usepackage{ucs}

\usepackage[french]{babel,varioref}

\usepackage[top=2cm, bottom=2cm, left=1.5cm, right=1.5cm]{geometry}
\usepackage{enumitem}

\usepackage{multicol}

\usepackage{makecell}

\usepackage{color}
\usepackage{hyperref}
\hypersetup{
    colorlinks,
    citecolor=black,
    filecolor=black,
    linkcolor=black,
    urlcolor=black
}

\usepackage{amsthm}

\usepackage{tcolorbox}
\tcbuselibrary{listingsutf8}

\usepackage{ifplatform}

\usepackage{ifthen}

\usepackage{cbdevtool}


% MISC

\newtcblisting{latexex}{%
	sharp corners,%
	left=1mm, right=1mm,%
	bottom=1mm, top=1mm,%
	colupper=red!75!blue,%
	listing side text
}

\newtcblisting{latexex-flat}{%
	sharp corners,%
	left=1mm, right=1mm,%
	bottom=1mm, top=1mm,%
	colupper=red!75!blue,%
}

\newtcblisting{latexex-alone}{%
	sharp corners,%
	left=1mm, right=1mm,%
	bottom=1mm, top=1mm,%
	colupper=red!75!blue,%
	listing only
}


\newcommand\env[1]{\texttt{#1}}
\newcommand\macro[1]{\env{\textbackslash{}#1}}



\setlength{\parindent}{0cm}
\setlist{noitemsep}

\theoremstyle{definition}
\newtheorem*{remark}{Remarque}

\usepackage[raggedright]{titlesec}

\titleformat{\paragraph}[hang]{\normalfont\normalsize\bfseries}{\theparagraph}{1em}{}
\titlespacing*{\paragraph}{0pt}{3.25ex plus 1ex minus .2ex}{0.5em}


\newcommand\separation{
	\medskip
	\hfill\rule{0.5\textwidth}{0.75pt}\hfill
	\medskip
}


\newcommand\extraspace{
	\vspace{0.25em}
}


\newcommand\whyprefix[2]{%
	\textbf{\prefix{#1}}-#2%
}

\newcommand\mwhyprefix[2]{%
	\texttt{#1 = #1-#2}%
}

\newcommand\prefix[1]{%
	\texttt{#1}%
}


\newcommand\inenglish{\@ifstar{\@inenglish@star}{\@inenglish@no@star}}

\newcommand\@inenglish@star[1]{%
	\emph{\og #1 \fg}%
}

\newcommand\@inenglish@no@star[1]{%
	\@inenglish@star{#1} en anglais%
}


\newcommand\ascii{\texttt{ASCII}}


% Example
\newcounter{paraexample}[subsubsection]

\newcommand\@newexample@abstract[2]{%
	\paragraph{%
		#1%
		\if\relax\detokenize{#2}\relax\else {} -- #2\fi%
	}%
}



\newcommand\newparaexample{\@ifstar{\@newparaexample@star}{\@newparaexample@no@star}}

\newcommand\@newparaexample@no@star[1]{%
	\refstepcounter{paraexample}%
	\@newexample@abstract{Exemple \theparaexample}{#1}%
}

\newcommand\@newparaexample@star[1]{%
	\@newexample@abstract{Exemple}{#1}%
}


% Change log
\newcommand\topic{\@ifstar{\@topic@star}{\@topic@no@star}}

\newcommand\@topic@no@star[1]{%
	\textbf{\textsc{#1}.}%
}

\newcommand\@topic@star[1]{%
	\textbf{\textsc{#1} :}%
}







    \usepackage{01-stepcalc}
\makeatother



\begin{document}

%\section{Détailler un raisonnement simple}

\subsection{Un mini hack très utile pour des \emph{\og étapes alignées \fg}}

Vous pouvez écrire très facilement des calculs ou raisonnement simples alignés comme suit sans trop vous fatiguez.

\begin{latexex}
\begin{stepcalc}[style = sar]
    (a + b) (a + b)
        \explnext{}
    (a + b)^2
        \explnext{}
    a^2 + b^2 + 2 a b
        \comthis{Pourquoi ?}
        \explnext{}
    a^2 + 2 a b + b^2
\end{stepcalc}
\end{latexex}

On a accès à une autre mise en forme \emph{(ceci peut rendre aussi service)}. 

\begin{latexex}
\begin{stepcalc}[style = ar]
    (a + b) (a + b)
        \explnext{}
    (a + b)^2
        \explnext{Pourquoi ?}
    a^2 + b^2 + 2 a b
        \explnext{}
    a^2 + 2 a b + b^2
\end{stepcalc}
\end{latexex}

Enfin dans le cadre de calculs à faire expliquer par des élèves, ce qui suit peut être utile.

\begin{latexex}
Donner les justifications J1, J2 et J3.

\medskip
\begin{stepcalc}
    (a + b) (a + b)
        \explnext{J1}
    (a + b)^2
        \explnext{J2}
    a^2 + b^2 + 2 a b
        \explnext{J3}
    a^2 + 2 a b + b^2
\end{stepcalc}
\end{latexex}


% ---------------------- %


\subsection{Un conseil de mise en forme}

Voici un style de codage que nous trouvons très facile à relire et maintenir.

\begin{latexex}
\begin{stepcalc}[com = al]
    (a + b) (a + b)
    	\comthis{Forme facto.}
    	\explnext{Via $x^2 = x \cdot x$.}
    %
    (a + b)^2
    	\comthis*{Au passage...}
    	\explnext*{Id.Rq - Dév.}%
                  {Id.Rq - Facto.}
    %
    a^2 + 2 a b + b^2
    	\comthis{Forme dév.}
\end{stepcalc}
\end{latexex}


\end{document}