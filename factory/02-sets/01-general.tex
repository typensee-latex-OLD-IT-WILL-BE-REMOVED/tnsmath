\documentclass[12pt,a4paper]{article}

% == FOR DOC AND TESTS - START == %

\usepackage[utf8]{inputenc}
\usepackage{ucs}
\usepackage[top=2cm, bottom=2cm, left=1.5cm, right=1.5cm]{geometry}

\usepackage{color}
\usepackage{hyperref}
\hypersetup{
    colorlinks,
    citecolor=black,
    filecolor=black,
    linkcolor=black,
    urlcolor=black
}

\usepackage{enumitem}

\usepackage{amsthm}

\usepackage{tcolorbox}
\tcbuselibrary{listings}

\usepackage{pgffor}
\usepackage{xstring}


% MISC

\setlength{\parindent}{0cm}
\setlist{noitemsep}

\theoremstyle{definition}
\newtheorem*{remark}{Remark}


% Technical IDs

\newwrite\tempfile

\immediate\openout\tempfile=x-\jobname.macros-x.txt

\AtEndDocument{\immediate\closeout\tempfile}

\newcommand\IDconstant[1]{%
    \immediate\write\tempfile{constant@#1}%
}

\makeatletter
	\newcommand\IDmacro{\@ifstar{\@IDmacroStar}{\@IDmacroNoStar}}
	
    \newcommand\@IDmacroNoStar[3]{%
        \texttt{%
        	\textbackslash#1%
        	\IfStrEq{#2}{0}{}{%
        		\,\,[#2 Option%
				\IfStrEq{#2}{1}{}{s}]%
			}%
    	    \IfStrEq{#3}{}{}{%
	    		\,\,(#3 Argument%
				\IfStrEq{#3}{1}{}{s})%
			}
	   	}
        \immediate\write\tempfile{macro@#1@#2@#3}%
    }

    \newcommand\@IDmacroStar[2]{%
        \@IDmacroNoStar{#1}{0}{#2}%
    }

	\newcommand\@IDoptarg{\@ifstar{\@IDoptargStar}{\@IDoptargNoStar}}
	
	\newcommand\@IDoptargStar[2]{%
    	\vspace{0.5em}
		--- \texttt{#1%
			\IfStrEq{#2}{}{:}{\,#2:}%
		}%
	}

	\newcommand\@IDoptargNoStar[2]{%
    	\IfStrEq{#2}{}{%
			\@IDoptargStar{#1}{}%
		}{%
			\@IDoptargStar{#1}{\##2}%
		}%
	}

	\newcommand\IDkey[1]{%
    	\@IDoptarg*{Option}{{\itshape "#1"}}%
	}

	\newcommand\IDoption[1]{%
    	\@IDoptarg{Option}{#1}%
	}

	\newcommand\IDarg[1]{%
    	\@IDoptarg{Argument}{#1}%
	}
\makeatother

% == FOR DOC AND TESTS - END == %


% == EXTRAS == %

\newcommand\ascii{\texttt{ASCII}}


% == PACKAGES USED == %

\usepackage{amssymb}
\usepackage{dsfont}
\usepackage{mathrsfs}
\usepackage{yhmath}

\usepackage{graphicx}
\usepackage{xstring}


% == DEFINITIONS == %

% Source for mathpzc : page 68 de ''The Comprehensive LATEX Symbol List''


% Only the upper case letters

\makeatletter

	\newcommand\probaset[1]{
    	\IfSubStr{,A,B,C,D,E,F,G,H,I,J,K,L,M,N,O,P,Q,R,S,T,U,V,W,X,Y,Z,}{,#1,}{
        	\mathcal{#1}
    	}{%
        	\PackageError{lymath}{Command << probaset >> only accepts single upper case letters}%
    	}
	}

% With ''mathscr'', we only have the upper case letters.
% So we use the following trick for lower case letters.

	\DeclareMathAlphabet{\mathpzc}{OT1}{pzc}{m}{it}

	\newcommand\geoset[1]{
	    \IfSubStr{,A,B,C,D,E,F,G,H,I,J,K,L,M,N,O,P,Q,R,S,T,U,V,W,X,Y,Z,}{,#1,}{
        	\mathscr{#1}
    	}{%
	        \IfSubStr{,a,b,c,d,e,f,g,h,i,j,k,l,m,n,o,p,q,r,s,t,u,v,w,x,y,z,}{,#1,}{
        	    \mathpzc{#1}
    	    }{%
	            \PackageError{lymath}{Command << fieldset >> only accepts single h, k or upper case letter}%
        	}
    	}
	}


% Only the upper case letters and also h and k.
%
% With ''\mathbb'', there are only the upper case letters.

	\newcommand\fieldset[1]{
    	\IfSubStr{,A,B,C,D,E,F,G,H,I,J,K,L,M,N,O,P,Q,R,S,T,U,V,W,X,Y,Z,h,k,}{,#1,}{
    	    \mathds{#1}
    	}{%
        	\PackageError{lymath}{Command << fieldset >> only accepts letters h or k alone, and upper case letters}%
    	}
	}


	\newcommand\setpos{\text{\bf{\tiny+}}}
	\newcommand\setneg{\vphantom{\setpos}-}
	\newcommand\setstar{\text{*}}


    \newcommand\specialset{\@ifstar{\@specialsetStar}{\@specialsetNoStar}}

    \newcommand\@specialsetNoStar[2]{
        \IfEqCase{#2}{
            {s}{ #1^{\setstar} }
            {n}{ #1^{\phantom{\setstar}}_{\setneg} }
            {p}{ #1^{\phantom{\setstar}}_{\setpos} }
            {sn}{ #1^{\setstar}_{\setneg} }
            {sp}{ #1^{\setstar}_{\setpos} }
        }
    }

    \newcommand\@specialsetStar[2]{
        \IfEqCase{#2}{
            {s}{ #1_{\neq 0} }
            {n}{ #1_{\le 0} }
            {p}{ #1_{\ge 0} }
            {sn}{ #1_{< 0} }
            {sp}{ #1_{> 0} }
        }
    }
\makeatother

% The empty set

\newcommand\nullset{\ensuremath{\text{\scalebox{1.2}{$\varnothing$}}}}

% List of classical sets

\newcommand\NN{\ensuremath{\fieldset{N}}}
\newcommand\NNs{\ensuremath{\specialset{\NN}{s}}}

\newcommand\ZZ{\ensuremath{\fieldset{Z}}}
\newcommand\ZZp{\ensuremath{\specialset{\ZZ}{p}}}
\newcommand\ZZn{\ensuremath{\specialset{\ZZ}{n}}}
\newcommand\ZZs{\ensuremath{\specialset{\ZZ}{s}}}
\newcommand\ZZsn{\ensuremath{\specialset{\ZZ}{sn}}}
\newcommand\ZZsp{\ensuremath{\specialset{\ZZ}{sp}}}

\newcommand\DD{\ensuremath{\fieldset{D}}}
\newcommand\DDp{\ensuremath{\specialset{\DD}{p}}}
\newcommand\DDn{\ensuremath{\specialset{\DD}{n}}}
\newcommand\DDs{\ensuremath{\specialset{\DD}{s}}}
\newcommand\DDsn{\ensuremath{\specialset{\DD}{sn}}}
\newcommand\DDsp{\ensuremath{\specialset{\DD}{sp}}}

\newcommand\QQ{\ensuremath{\fieldset{Q}}}
\newcommand\QQp{\ensuremath{\specialset{\QQ}{p}}}
\newcommand\QQn{\ensuremath{\specialset{\QQ}{n}}}
\newcommand\QQs{\ensuremath{\specialset{\QQ}{s}}}
\newcommand\QQsn{\ensuremath{\specialset{\QQ}{sn}}}
\newcommand\QQsp{\ensuremath{\specialset{\QQ}{sp}}}

\newcommand\RR{\ensuremath{\fieldset{R}}}
\newcommand\RRp{\ensuremath{\specialset{\RR}{p}}}
\newcommand\RRn{\ensuremath{\specialset{\RR}{n}}}
\newcommand\RRs{\ensuremath{\specialset{\RR}{s}}}
\newcommand\RRsn{\ensuremath{\specialset{\RR}{sn}}}
\newcommand\RRsp{\ensuremath{\specialset{\RR}{sp}}}

\newcommand\CC{\ensuremath{\fieldset{C}}}
\newcommand\CCs{\ensuremath{\specialset{\CC}{s}}}

\newcommand\HH{\ensuremath{\fieldset{H}}}
\newcommand\HHs{\ensuremath{\specialset{\HH}{s}}}

\newcommand\OO{\ensuremath{\fieldset{O}}}
\newcommand\OOs{\ensuremath{\specialset{\OO}{s}}}



\begin{document}

\section{Different kind of sets}

    \subsection{Sets for geometry}

        \subsubsection{Example of use}

\begin{tcblisting}{}
You can semantically write $\geoset{C}$, $\geoset{D}$ and $\geoset{d}$ but you can
not write things like \verb+$\geoset{ABC}$+.
\end{tcblisting}


        \subsubsection{Technical ID}

\IDmacro*{geoset}{1}

\IDarg{} one single \ascii{} letter indicating a geometrical set.



    \subsection{Sets for probability}

        \subsubsection{Example of use}

\begin{tcblisting}{}
You can semantically write $\probaset{E}$ and $\probaset{G}$ but you can not write
things like \verb+$\probaset{ABC}$+.
\end{tcblisting}


        \subsubsection{Technical ID}

\IDmacro*{probaset}{1}

\IDarg{} one single upper \ascii{} upper letter indicating a probabilistic set.



    \subsection{Sets for rings and fields theory}

        \subsubsection{Example of use}

\begin{tcblisting}{}
You can semantically write $\fieldset{A}$, $\fieldset{K}$, $\fieldset{h}$ and
$\fieldset{k}$, but you can't write things like \verb+$\fieldset{ABC}$+.
\end{tcblisting}


        \subsubsection{Technical ID}

\IDmacro*{fieldset}{1}

\IDarg{} either one of the letters \texttt{h} and \texttt{k}, or one single upper \ascii{} upper letter indicating a field or ring like set.



    \subsection{Classical sets}

\begin{tcblisting}{}
You can directly use $\nullset$, $\NN$, $\ZZ$, $\DD$, $\QQ$, $\RR$, $\CC$, $\HH$
and $\OO$.
\end{tcblisting}



    \subsection{Classical sets with suffixes}

\begin{tcblisting}{}
It is easy to type $\RRn$, $\RRp$, $\RRs$, $\RRsn$ and $\RRsp$.
\end{tcblisting}


We have used suffixes \verb+n+ for \verb+Negative+, \verb+p+ for \verb+Positive+, and \verb+s+ for \verb+Star+ with the additional composite suffixes \verb+sn+ et \verb+sp+.

\medskip

Note that you can't use \verb+$\CCn$+ for $\specialset{\CC}{n}$ because the set $\CC$ doesn't have any standard powerful ordered structure. Take a look at the next section to see how to write $\specialset{\CC}{n}$ if you need it.

\medskip

The following table shows when you can add one of the suffixes \verb+n+, \verb+p+, \verb+s+, \verb+sn+ and \verb+sp+.

% == Table of suffixes - START == %
\newcommand\xx{\phantom{$\times$}}
\begin{table}[h]
    \caption{Suffixes}
    \begin{center}
        \begin{tabular}{c|c|c|c|c|c}
  & \verb+n+ & \verb+p+ & \verb+s+ & \verb+sn+ & \verb+sp+ \\
\hline \verb+N+ & \xx & \xx & $\times$ & \xx & \xx \\
\hline \verb+Z+ & $\times$ & $\times$ & $\times$ & $\times$ & $\times$ \\
\hline \verb+D+ & $\times$ & $\times$ & $\times$ & $\times$ & $\times$ \\
\hline \verb+Q+ & $\times$ & $\times$ & $\times$ & $\times$ & $\times$ \\
\hline \verb+R+ & $\times$ & $\times$ & $\times$ & $\times$ & $\times$ \\
\hline \verb+C+ & \xx & \xx & $\times$ & \xx & \xx \\
\hline \verb+H+ & \xx & \xx & $\times$ & \xx & \xx \\
\hline \verb+O+ & \xx & \xx & $\times$ & \xx & \xx \\
        \end{tabular}
    \end{center}
    \label{default}
\end{table}
% == Table of suffixes - END == %



    \subsection{Suffixes on demand}

        \subsubsection{Example of use}

\begin{tcblisting}{}
You can indeed write things like $\specialset{\CC}{n}$ or $\specialset{\HH}{sp}$. 
There is also $\specialset*{\probaset{P}}{n}$ with another formatting.
\end{tcblisting}


        \subsubsection{Technical IDs}

\IDmacro*{specialset}{2}

\IDmacro*{specialset*}{2}

\IDarg{1} the set to be "suffixed".

\IDarg{2} one of the suffixes \verb+n+, \verb+p+, \verb+s+, \verb+sn+ and \verb+sp+.

\end{document}
