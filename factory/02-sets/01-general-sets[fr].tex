\documentclass[12pt,a4paper]{article}

\makeatletter
	\usepackage[utf8]{inputenc}
\usepackage[T1]{fontenc}
\usepackage{ucs}

\usepackage[french]{babel,varioref}

\usepackage[top=2cm, bottom=2cm, left=1.5cm, right=1.5cm]{geometry}
\usepackage{enumitem}

\usepackage{multicol}

\usepackage{makecell}

\usepackage{color}
\usepackage{hyperref}
\hypersetup{
    colorlinks,
    citecolor=black,
    filecolor=black,
    linkcolor=black,
    urlcolor=black
}

\usepackage{amsthm}

\usepackage{tcolorbox}
\tcbuselibrary{listingsutf8}

\usepackage{ifplatform}

\usepackage{ifthen}

\usepackage{cbdevtool}


% MISC

\newtcblisting{latexex}{%
	sharp corners,%
	left=1mm, right=1mm,%
	bottom=1mm, top=1mm,%
	colupper=red!75!blue,%
	listing side text
}

\newtcblisting{latexex-flat}{%
	sharp corners,%
	left=1mm, right=1mm,%
	bottom=1mm, top=1mm,%
	colupper=red!75!blue,%
}

\newtcblisting{latexex-alone}{%
	sharp corners,%
	left=1mm, right=1mm,%
	bottom=1mm, top=1mm,%
	colupper=red!75!blue,%
	listing only
}


\newcommand\env[1]{\texttt{#1}}
\newcommand\macro[1]{\env{\textbackslash{}#1}}



\setlength{\parindent}{0cm}
\setlist{noitemsep}

\theoremstyle{definition}
\newtheorem*{remark}{Remarque}

\usepackage[raggedright]{titlesec}

\titleformat{\paragraph}[hang]{\normalfont\normalsize\bfseries}{\theparagraph}{1em}{}
\titlespacing*{\paragraph}{0pt}{3.25ex plus 1ex minus .2ex}{0.5em}


\newcommand\separation{
	\medskip
	\hfill\rule{0.5\textwidth}{0.75pt}\hfill
	\medskip
}


\newcommand\extraspace{
	\vspace{0.25em}
}


\newcommand\whyprefix[2]{%
	\textbf{\prefix{#1}}-#2%
}

\newcommand\mwhyprefix[2]{%
	\texttt{#1 = #1-#2}%
}

\newcommand\prefix[1]{%
	\texttt{#1}%
}


\newcommand\inenglish{\@ifstar{\@inenglish@star}{\@inenglish@no@star}}

\newcommand\@inenglish@star[1]{%
	\emph{\og #1 \fg}%
}

\newcommand\@inenglish@no@star[1]{%
	\@inenglish@star{#1} en anglais%
}


\newcommand\ascii{\texttt{ASCII}}


% Example
\newcounter{paraexample}[subsubsection]

\newcommand\@newexample@abstract[2]{%
	\paragraph{%
		#1%
		\if\relax\detokenize{#2}\relax\else {} -- #2\fi%
	}%
}



\newcommand\newparaexample{\@ifstar{\@newparaexample@star}{\@newparaexample@no@star}}

\newcommand\@newparaexample@no@star[1]{%
	\refstepcounter{paraexample}%
	\@newexample@abstract{Exemple \theparaexample}{#1}%
}

\newcommand\@newparaexample@star[1]{%
	\@newexample@abstract{Exemple}{#1}%
}


% Change log
\newcommand\topic{\@ifstar{\@topic@star}{\@topic@no@star}}

\newcommand\@topic@no@star[1]{%
	\textbf{\textsc{#1}.}%
}

\newcommand\@topic@star[1]{%
	\textbf{\textsc{#1} :}%
}






	% == PACKAGES USED == %

\RequirePackage{amsmath}
\RequirePackage{relsize}
\RequirePackage{xparse}


% == DEFINITIONS == %

% Settable texts
\@ifpackagewith{babel}{french}{
    \newcommand\lymathsep{;}
    \newcommand\lymathsubsep{,}

    \newcommand\textopchoice{choix}
    \newcommand\textopcond{cond}
    \newcommand\textopdef{déf}
    \newcommand\textophyp{hyp}
    \newcommand\textopid{id}
    \newcommand\textoptest{?}
}{
    \newcommand\lymathsep{,}
    \newcommand\lymathsubsep{;}

    \newcommand\textopchoice{choice}
    \newcommand\textopcond{cond}
    \newcommand\textopdef{def}
    \newcommand\textophyp{hyp}
    \newcommand\textopid{id}
    \newcommand\textoptest{?}
}


\newcommand\textexplainleft{\{}
\newcommand\textexplainright{\}}
\newcommand\textexplainspacein{2em}


% Tools - Apply same macro to all arguments

% #1        : main macro
% #2        : macro to apply to arguments
% #3 and #4 : the two arguments
\newcommand\@apply@macro@two@args[4]{%
    #1{#2{#3}}{#2{#4}}%
}


% Tools - Deco over a math symbol

\newcommand\@over@math@symbol[2]{%
	\mathrel{\overset{\mathrm{\text{\raisebox{.5ex}{#1}}}}{#2}}%
}


% Tools - Intervals

\newcommand\@extra@phantom{%
    \vphantom{\relsize{1.25}{\text{$\displaystyle F_1^2$}}}%
}

\newcommand\@interval@tool@star[5]{%
    \ensuremath{ \left#1 \@extra@phantom \right. \!\! #2 #3 #4 \left. \@extra@phantom \!\! \right#5}%
}

\newcommand\@interval@tool@no@star[5]{\ensuremath{ \left#1 #2 #3 #4 \right#5}}


% Tools - Multi-arguments
%
% Source : the following lines come directly for the following post
%
%    * https://tex.stackexchange.com/a/475291/6880

\ExplSyntaxOn
% General purpose macro for defining other macros
    \NewDocumentCommand{\makemultiargument}{mmmmmo}{
        \lymath_multiarg:nnnnnn{#1}{#2}{#3}{#4}{#5}{#6}
    }
 
% Allocate a private variable
    \seq_new:N \l__lymath_generic_seq

% The internal version of the general purpose macro
    \cs_new_protected:Nn \lymath_multiarg:nnnnnn{
        % #1 = separator
        % #2 = multiargument
        % #3 = code before
          % #4 = code between
          % #5 = code after
          % #6 = ornament to items

        % A group allows nesting
        \group_begin:
         % Split the multiargument into parts
        \seq_set_split:Nnn \l__lymath_generic_seq { #1 } { #2 }
        % Apply the ornament to the items
          \tl_if_novalue:nF { #6 }{
            \seq_set_eq:NN \l__lymath_temp_seq \l__lymath_generic_seq
            \seq_set_map:NNn \l__lymath_generic_seq \l__lymath_generic_seq { #6 }
           }
        % Execute the <code before>
          #3
        % Deliver the items, with the chosen material between them
          \seq_use:Nn \l__lymath_generic_seq { #4 }
          % Execute the <code after>
         #5
          % End the group started at the beginning
          \group_end:
    }    
\ExplSyntaxOff


	\usepackage{01-general-sets}
\makeatother


\usepackage{relsize}



\begin{document}

\section{Ensembles}

    \subsection{Différents types d'ensembles}

        \subsubsection{Ensembles versus accolades}

            \paragraph{Exemple d'utilisation 1}

\begin{tcblisting}{}
Un ensemble de beaux nombres : $\geneset{1 ; 3 ; 5}$ .
\end{tcblisting}


            \paragraph{Exemple d'utilisation 2}

\begin{tcblisting}{}
Choisissez votre camp :
$\displaystyle \geneset{\frac{1}{3} ; \frac{5}{7} ; \frac{9}{11}}$
ou 
$\displaystyle \geneset*{\frac{1}{3} ; \frac{5}{7} ; \frac{9}{11}}$ .
\end{tcblisting}


            \paragraph{Fiches techniques}

\IDmacro*{geneset}{1}

\IDmacro*{geneset*}{1}

\IDarg{} la définition de l'ensemble.


        \subsubsection{Ensembles pour la géométrie}

            \paragraph{Exemple d'utilisation 1}

\begin{tcblisting}{}
Vous pouvez écrire sémantiquement $\geoset{C}$, $\geoset{D}$ et $\geoset{d}$ mais pas
taper \verb+$\geoset{ABC}$+.
\end{tcblisting}


            \paragraph{Exemple d'utilisation 2}

\begin{tcblisting}{}
Pour les indices, utilisez $\geoset*{C}{1}$, $\geoset*{C}{2}$ \dots
\end{tcblisting}


            \paragraph{Fiches techniques}

\IDmacro*{geoset}{1}

\IDarg{} un seul caractère \ascii{} indiquant un ensemble géométrique.


\bigskip


\IDmacro*{geoset*}{2}

\IDarg{1} un seul caractère \ascii{} indiquant $\geoset{U}$ dans le nom $\geoset*{U}{d}$ d'un ensemble géométrique.

\IDarg{2} un texte donnant $d$ dans le nom $\geoset*{U}{d}$ d'un ensemble géométrique.


        \subsubsection{Ensembles probabilistes}

            \paragraph{Exemple d'utilisation 1}

\begin{tcblisting}{}
Vous pouvez écrire sémantiquement $\probaset{E}$ et $\probaset{G}$ mais pas taper
\verb+$\probaset{ABC}$+.
\end{tcblisting}


            \paragraph{Exemple d'utilisation 2}

\begin{tcblisting}{}
Pour les indices, utilisez $\probaset*{E}{1}$, $\probaset*{E}{2}$ \dots
\end{tcblisting}


            \paragraph{Fiches techniques}

\IDmacro*{probaset}{1}

\IDarg{} un seul caractère \ascii{} majuscule indiquant un ensemble probabiliste.


\bigskip


\IDmacro*{probaset*}{2}

\IDarg{1} un seul caractère \ascii{} majuscule indiquant $\probaset{U}$ dans le nom $\probaset*{U}{d}$ d'un ensemble probabiliste.

\IDarg{2} un texte donnant $d$ dans le nom $\probaset*{U}{d}$ d'un ensemble probabiliste.



        \subsubsection{Ensembles pour l'algèbre générale}

            \paragraph{Exemple d'utilisation 1}

\begin{tcblisting}{}
Vous pouvez écrire sémantiquement $\algeset{A}$, $\algeset{K}$, $\algeset{h}$ et
$\algeset{k}$ mais pas taper \verb+$\algeset{ABC}$+.
\end{tcblisting}


            \paragraph{Exemple d'utilisation 2}

\begin{tcblisting}{}
Pour les indices, utilisez $\algeset*{k}{1}$, $\algeset*{k}{2}$ \dots
\end{tcblisting}


            \paragraph{Fiches techniques}

\IDmacro*{algeset}{1}

\IDarg{} soit l'une des lettres  \texttt{h} et \texttt{k}, soit un seul caractère \ascii{} majuscule indiquant un ensemble de type anneau ou corps.


\bigskip


\IDmacro*{algeset*}{2}

\IDarg{1} un seul caractère \ascii{} indiquant $\algeset{U}$ dans le nom $\algeset*{U}{d}$ d'un ensemble de type anneau ou corps.

\IDarg{2} un texte donnant $d$ dans le nom $\algeset*{U}{d}$ d'un ensemble de type anneau ou corps.



        \subsubsection{Ensembles classiques}

\begin{tcblisting}{}
Vous pouvez utiliser directement $\nullset$, $\NN$, $\ZZ$, $\DD$, $\QQ$, $\RR$, $\CC$,
mais aussi $\PP$ pour l'ensemble des nombres premiers, $\HH$ pour les quaternions et
enfin $\OO$ pour les octonions.
\end{tcblisting}



        \subsubsection{Ensembles classiques suffixés}

\begin{tcblisting}{}
Il est facile de taper $\RRn$, $\RRp$, $\RRs$, $\RRsn$ et $\RRsp$.
\end{tcblisting}


Nous avons utilisé les suffixes \verb+n+ pour \verb+Negatif+, \verb+p+ pour \verb+Positif+, et \verb+s+ pour \verb+star+, soit "étoile" en anglais. Il y a aussi les suffixes composites \verb+sn+ et \verb+sp+.

\medskip

Notez qu'il est interdit d'utiliser \verb+$\CCn$+ pour $\specialset{\CC}{n}$ car l'ensemble $\CC$ ne possède pas de structure ordonnée standard. Jetez un oeil à la section suivante pour apprendre à taper $\specialset{\CC}{n}$ si vous en avez besoin. L'interdiction est ici purement sémantique !

\medskip

\begin{remark}
	La table \ref{table:suffixes-sets} \vpageref{table:suffixes-sets} montre les associations autorisées entre ensembles classiques et suffixes.
\end{remark}

% == Table of suffixes - START == %

\newcommand\xx{\phantom{$\times$}}
\begin{table}[h]
    \caption{Suffixes}
    \begin{center}
        \begin{tabular}{c|c|c|c|c|c}
              & \verb+n+ & \verb+p+ & \verb+s+ & \verb+sn+ & \verb+sp+ \\
            \hline \verb+N+ & \xx & \xx & $\times$ & \xx & \xx \\
            \hline \verb+P+ & \xx & \xx & \xx & \xx & \xx \\
            \hline \verb+Z+ & $\times$ & $\times$ & $\times$ & $\times$ & $\times$ \\
            \hline \verb+D+ & $\times$ & $\times$ & $\times$ & $\times$ & $\times$ \\
            \hline \verb+Q+ & $\times$ & $\times$ & $\times$ & $\times$ & $\times$ \\
            \hline \verb+R+ & $\times$ & $\times$ & $\times$ & $\times$ & $\times$ \\
            \hline \verb+C+ & \xx & \xx & $\times$ & \xx & \xx \\
            \hline \verb+H+ & \xx & \xx & $\times$ & \xx & \xx \\
            \hline \verb+O+ & \xx & \xx & $\times$ & \xx & \xx \\
        \end{tabular}
    \end{center}
    \label{table:suffixes-sets}
\end{table}

% == Table of suffixes - END == %



        \subsubsection{Des suffixes à la carte}

            \paragraph{Exemple d'utilisation}

\begin{tcblisting}{}
Il est tout de même possible d'écrire $\specialset{\CC}{n}$ ou $\specialset{\HH}{sp}$.
Il y a aussi $\specialset*{\probaset{P}}{n}$ avec une autre mise en forme.
\end{tcblisting}


            \paragraph{Fiches techniques}

\IDmacro*{specialset}{2}

\IDmacro*{specialset*}{2}

\IDarg{1} l'ensemble à "suffixer".

\IDarg{2} l'un des suffixes \verb+n+, \verb+p+, \verb+s+, \verb+sn+ ou \verb+sp+.

\end{document}