\documentclass[12pt,a4paper]{article}

\makeatletter
    \usepackage[utf8]{inputenc}
\usepackage[T1]{fontenc}
\usepackage{ucs}

\usepackage[french]{babel,varioref}

\usepackage[top=2cm, bottom=2cm, left=1.5cm, right=1.5cm]{geometry}
\usepackage{enumitem}

\usepackage{multicol}

\usepackage{makecell}

\usepackage{color}
\usepackage{hyperref}
\hypersetup{
    colorlinks,
    citecolor=black,
    filecolor=black,
    linkcolor=black,
    urlcolor=black
}

\usepackage{amsthm}

\usepackage{tcolorbox}
\tcbuselibrary{listingsutf8}

\usepackage{ifplatform}

\usepackage{ifthen}

\usepackage{cbdevtool}


% MISC

\newtcblisting{latexex}{%
	sharp corners,%
	left=1mm, right=1mm,%
	bottom=1mm, top=1mm,%
	colupper=red!75!blue,%
	listing side text
}

\newtcblisting{latexex-flat}{%
	sharp corners,%
	left=1mm, right=1mm,%
	bottom=1mm, top=1mm,%
	colupper=red!75!blue,%
}

\newtcblisting{latexex-alone}{%
	sharp corners,%
	left=1mm, right=1mm,%
	bottom=1mm, top=1mm,%
	colupper=red!75!blue,%
	listing only
}


\newcommand\env[1]{\texttt{#1}}
\newcommand\macro[1]{\env{\textbackslash{}#1}}



\setlength{\parindent}{0cm}
\setlist{noitemsep}

\theoremstyle{definition}
\newtheorem*{remark}{Remarque}

\usepackage[raggedright]{titlesec}

\titleformat{\paragraph}[hang]{\normalfont\normalsize\bfseries}{\theparagraph}{1em}{}
\titlespacing*{\paragraph}{0pt}{3.25ex plus 1ex minus .2ex}{0.5em}


\newcommand\separation{
	\medskip
	\hfill\rule{0.5\textwidth}{0.75pt}\hfill
	\medskip
}


\newcommand\extraspace{
	\vspace{0.25em}
}


\newcommand\whyprefix[2]{%
	\textbf{\prefix{#1}}-#2%
}

\newcommand\mwhyprefix[2]{%
	\texttt{#1 = #1-#2}%
}

\newcommand\prefix[1]{%
	\texttt{#1}%
}


\newcommand\inenglish{\@ifstar{\@inenglish@star}{\@inenglish@no@star}}

\newcommand\@inenglish@star[1]{%
	\emph{\og #1 \fg}%
}

\newcommand\@inenglish@no@star[1]{%
	\@inenglish@star{#1} en anglais%
}


\newcommand\ascii{\texttt{ASCII}}


% Example
\newcounter{paraexample}[subsubsection]

\newcommand\@newexample@abstract[2]{%
	\paragraph{%
		#1%
		\if\relax\detokenize{#2}\relax\else {} -- #2\fi%
	}%
}



\newcommand\newparaexample{\@ifstar{\@newparaexample@star}{\@newparaexample@no@star}}

\newcommand\@newparaexample@no@star[1]{%
	\refstepcounter{paraexample}%
	\@newexample@abstract{Exemple \theparaexample}{#1}%
}

\newcommand\@newparaexample@star[1]{%
	\@newexample@abstract{Exemple}{#1}%
}


% Change log
\newcommand\topic{\@ifstar{\@topic@star}{\@topic@no@star}}

\newcommand\@topic@no@star[1]{%
	\textbf{\textsc{#1}.}%
}

\newcommand\@topic@star[1]{%
	\textbf{\textsc{#1} :}%
}







    \usepackage{02-tree}
\makeatother


% == EXTRAS == %


\begin{document}

%\section{Probabilité}

\subsection{Arbres pondérés}

\paragraph{Que se passe-t-il en coulisse ?}

Le gros du travail est fait par le package \verb+forest+ qui utilise \verb+TiKz+. Ceci permet de faire des choses sympathiques comme dans le 2\ieme{} exemple ci-dessous.


% ---------------------- %


\paragraph{Exemple 1 -- Le cas type}

Dans le code suivant l'environnement \verb+probatree+ utilise en coulisse celui nommé \verb+forest+ du package \verb+forest+. Des réglages spécifiques sont faits pour obtenir le résultat ci-après.
À cela s'ajoutent les styles spéciaux \verb+pweight+, \verb+apweight+ et \verb+bpweight+ qui facilitent l'écriture des pondérations sur les branches
\footnote{
    \texttt{pweight} vient de \emph{\og probability \fg} et \emph{\og weight\fg} soit \emph{\og probabilité \fg} et \emph{\og poids\fg} en anglais.
    Quant à \texttt{a} et \texttt{b} au début de \texttt{apweight} et \texttt{bpweight} respectivement, ils viennent de \emph{\og above \fg} et \emph{\og below\fg} soit \emph{\og dessus \fg} et \emph{\og dessous\fg} en anglais.
}.

\begin{latexex}
\begin{probatree}
    [
        [$A$, pweight = $a$
            [$B$, pweight = $b$]
            [$C$, pweight = $c$]
        ]
        [$D$, bpweight = $d$
            [$E$, apweight = $e$]
            [$F$, bpweight = $f$]
        ]
    ]
\end{probatree}
\end{latexex}


% ---------------------- %


\paragraph{Exemple 2 -- Des poids cachés partout}

On peut cacher tous les poids via l'environnement étoilé \verb+probatree*+ sans avoir à retaper un arbre où les pondérations ont déjà été indiquées.

\begin{latexex}
\begin{probatree*}
    [$A$, pweight = $a$
        [$B$, pweight = $b$]
        [$C$, pweight = $c$]
    ]
\end{probatree*}
\end{latexex}


% ---------------------- %


\paragraph{Exemple 3 -- Des poids cachés localement}

Pour ne cacher que certains poids, il faudra utiliser, à la main, le style \verb+pweight*+ comme dans l'exemple ci-dessous.

\begin{latexex}
\begin{probatree}
    [
        [$A$, pweight = $a$
            [$B$, pweight* = $b$]
            [$C$, pweight  = $c$]
        ]
        [$D$, pweight* = $d$]
    ]
\end{probatree}
\end{latexex}


% ---------------------- %


\paragraph{Exemple 4 -- Un signe $=$ et/ou une virgule dans les étiquettes}

Vous ne pouvez pas utiliser directement un signe $=$ ou une virgule dans les étiquettes des branches. L'astuce pour contourner cette limitation consiste juste à mettre le contenu de l'étiquette dans des accolades.

\begin{latexex}
\begin{probatree}
    [
        [$A$, apweight = {$a = 0,1$}]
        [$B$, bpweight = $b$]
    ]
\end{probatree}
\end{latexex}


% ---------------------- %


\paragraph{Exemple 5 -- Des cadres facilement}

Via la clé \verb+frame+, il est très aisé d'encadrer un sous-arbre comme le montre l'exemple suivant. Dans l'exemple ci-après nous utilisons la bidouille \verb+{},s sep = 1.3cm+ qui évite que les cadres se superposent.

\begin{latexex}
\begin{probatree}
    [{}, s sep = 1.3cm 
     % Astuce pour espacer les cadres.
        [$A$, pweight = $a$,
              frame   = red
            [$B$, pweight = $b$]
            [$C$, pweight = $c$]
        ]
        [$D$, pweight = $d$,
              frame   = blue
            [$E$, pweight = $e$
                [$F$, pweight = $f$]
                [$G$, pweight = $g$]
            ]
            [$H$, pweight = $h$
                [$I$, pweight = $i$]
                [$J$, pweight = $j$]
            ]
        ]
    ]
\end{probatree}
\end{latexex}


% ---------------------- %


\paragraph{Exemple 6 -- Des cadres faits à la main}

En utilisant la machinerie de \verb+TiKz+ il est facile de décorer un arbre de probabilité comme ci-dessous où le cadre s'appuie sur trois noeuds nommés. Notons que cet exemple est tout simplement infaisable avec la clé \verb+frame+.

\begin{latexex}
\begin{probatree}
    [
        [$A$, pweight = $a$,
              name    = nA
            [$B$, pweight = $b$,
                  name    = nB
                [$C$, pweight = $c$]
                [$D$, pweight = $d$]
               ]
            [$F$, pweight = $f$,
                  name    = nF]
        ]
        [$G$, pweight = $g$]
    ]
    \node[draw = orange,
          thick,
          rounded corners,
          fit = (nA)(nB)(nF)] {};
\end{probatree}
\end{latexex}







% ---------------------- %


%\paragraph{Exemple 7 -- Indiquer un chemin}
%
%L'exemple suivant demande un peu plus d'implication mais de nouveau tout est obtenu via la machinerie de \verb+TiKz+.
%
%\begin{latexex}
%\begin{probatree}
%    [{}, name = start
%        [$A$, pweight = $a$, 
%              name    = nA
%            [$B$, pweight = $b$]
%            [$C$, pweight = $c$,
%                  name    = nC]
%        ]
%        [$D$, pweight = $d$,
%            [$E$, pweight = $e$]    
%            [$F$, pweight = $f$]
%        ]
%    ]
%    \draw[blue,
%          rounded corners,
%          dashed,
%          line width=0.7pt]          
%    (start.90) --
%    (nA.135) --
%    (nA.45) --
%    (nC.45) --
%    (nC.315) --
%    (nC.225) --
%    (nA.315) --
%    (nA.225) --
%    (start.315) --
%    (start.180) --
%    (start.90);
%\end{probatree}
%\end{latexex}


% ---------------------- %


\subsubsection{Fiches techniques}

\paragraph{Arbres pondérés}

\IDenv*{probatree}

\IDenv*{probatree*}

\Content{} un arbre codé en utilisant la syntaxe supportée par le package \verb+forest+.

\IDkey{pweight} pour écrire un poids sur le milieu d'une branche.

\IDkey{apweight} pour écrire un poids au-dessus le milieu d'une branche.

\IDkey{bpweight} pour écrire un poids en-dessous du milieu d'une branche.

\IDkey{frame} pour encadrer un sous-arbre depuis un noeud vers toutes les feuilles de celui-ci.

\end{document}
