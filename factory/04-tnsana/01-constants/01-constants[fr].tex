\documentclass[12pt,a4paper]{article}

\makeatletter
	\usepackage[utf8]{inputenc}
\usepackage[T1]{fontenc}
\usepackage{ucs}

\usepackage[french]{babel,varioref}

\usepackage[top=2cm, bottom=2cm, left=1.5cm, right=1.5cm]{geometry}
\usepackage{enumitem}

\usepackage{multicol}

\usepackage{color}
\usepackage{hyperref}
\hypersetup{
    colorlinks,
    citecolor=black,
    filecolor=black,
    linkcolor=black,
    urlcolor=black
}

\usepackage{amsthm}

\usepackage{tcolorbox}
\tcbuselibrary{listingsutf8}

\usepackage{ifplatform}

\usepackage{ifthen}

\usepackage{cbdevtool}


% MISC

\newtcblisting{latexex}{%
	sharp corners,%
	left=1mm, right=1mm,%
	bottom=1mm, top=1mm,%
	colupper=red!75!blue,%
	listing side text
}

\newtcblisting{latexex-flat}{%
	sharp corners,%
	left=1mm, right=1mm,%
	bottom=1mm, top=1mm,%
	colupper=red!75!blue,%
}

\newtcblisting{latexex-alone}{%
	sharp corners,%
	left=1mm, right=1mm,%
	bottom=1mm, top=1mm,%
	colupper=red!75!blue,%
	listing only
}


\newcommand\env[1]{\texttt{#1}}
\newcommand\macro[1]{\env{\textbackslash{}#1}}



\setlength{\parindent}{0cm}
\setlist{noitemsep}

\theoremstyle{definition}
\newtheorem*{remark}{Remarque}

\usepackage[raggedright]{titlesec}

\titleformat{\paragraph}[hang]{\normalfont\normalsize\bfseries}{\theparagraph}{1em}{}
\titlespacing*{\paragraph}{0pt}{3.25ex plus 1ex minus .2ex}{0.5em}


\newcommand\separation{
	\medskip
	\hfill\rule{0.5\textwidth}{0.75pt}\hfill
	\medskip
}


\newcommand\extraspace{
	\vspace{0.25em}
}


\newcommand\ascii{\texttt{ASCII}}


	\usepackage{01-constants}
\makeatother

\usepackage{amsmath}


\begin{document}

\chapter{Analyse}

\section{Constantes et paramètres}

\subsection{Constantes classiques}

\paragraph{La liste complète}

% List of classical constants - START

\begin{latexex}
$\ggamma$ , $\ppi$ , $\ttau$ ,
$\ee$ , $\ii$ , $\jj$ 
et $\kk$ où $\ttau = 2 \ppi$
\end{latexex}

% List of classical constants - END


\begin{remark}
	Faites attention car \verb+{\Large $\ppi \neq \pi$}+ produit {\Large $\ppi \neq \pi$}. Comme vous le constatez, les symboles ne sont pas identiques. Ceci est vraie pour toutes les constantes grecques.
\end{remark}


% ---------------------- %


\subsection{Fiches techniques}

\vspace{-1em}
\begin{multicols}{2}
% == Docs for contants - START == %

\foreach \k in {ggamma, ppi, ttau, ee, ii, jj, kk}{

	\IDmacro*{\k}{0}

}

% == Docs for contants - END == %
\vfill\null
\end{multicols}


% ---------------------- %


\subsection{Constantes latines personnelles}

La macro \macro{param} est surtout là pour une utilisation pédagogique.

\begin{latexex}
$\param{a} x^2 + \param{b} x + \param{c}$
ou
$a x^2 + b x + c$
\end{latexex}


% ---------------------- %


\subsection{Fiches techniques}

\IDmacro*{param}{1}

\IDarg{} un texte utilisant l'alphabet latin.

\end{document}
