\documentclass[12pt,a4paper]{article}

\makeatletter
    \usepackage[utf8]{inputenc}
\usepackage[T1]{fontenc}
\usepackage{ucs}

\usepackage[french]{babel,varioref}

\usepackage[top=2cm, bottom=2cm, left=1.5cm, right=1.5cm]{geometry}
\usepackage{enumitem}

\usepackage{multicol}

\usepackage{makecell}

\usepackage{color}
\usepackage{hyperref}
\hypersetup{
    colorlinks,
    citecolor=black,
    filecolor=black,
    linkcolor=black,
    urlcolor=black
}

\usepackage{amsthm}

\usepackage{tcolorbox}
\tcbuselibrary{listingsutf8}

\usepackage{ifplatform}

\usepackage{ifthen}

\usepackage{cbdevtool}


% MISC

\newtcblisting{latexex}{%
	sharp corners,%
	left=1mm, right=1mm,%
	bottom=1mm, top=1mm,%
	colupper=red!75!blue,%
	listing side text
}

\newtcblisting{latexex-flat}{%
	sharp corners,%
	left=1mm, right=1mm,%
	bottom=1mm, top=1mm,%
	colupper=red!75!blue,%
}

\newtcblisting{latexex-alone}{%
	sharp corners,%
	left=1mm, right=1mm,%
	bottom=1mm, top=1mm,%
	colupper=red!75!blue,%
	listing only
}


\newcommand\env[1]{\texttt{#1}}
\newcommand\macro[1]{\env{\textbackslash{}#1}}



\setlength{\parindent}{0cm}
\setlist{noitemsep}

\theoremstyle{definition}
\newtheorem*{remark}{Remarque}

\usepackage[raggedright]{titlesec}

\titleformat{\paragraph}[hang]{\normalfont\normalsize\bfseries}{\theparagraph}{1em}{}
\titlespacing*{\paragraph}{0pt}{3.25ex plus 1ex minus .2ex}{0.5em}


\newcommand\separation{
	\medskip
	\hfill\rule{0.5\textwidth}{0.75pt}\hfill
	\medskip
}


\newcommand\extraspace{
	\vspace{0.25em}
}


\newcommand\whyprefix[2]{%
	\textbf{\prefix{#1}}-#2%
}

\newcommand\mwhyprefix[2]{%
	\texttt{#1 = #1-#2}%
}

\newcommand\prefix[1]{%
	\texttt{#1}%
}


\newcommand\inenglish{\@ifstar{\@inenglish@star}{\@inenglish@no@star}}

\newcommand\@inenglish@star[1]{%
	\emph{\og #1 \fg}%
}

\newcommand\@inenglish@no@star[1]{%
	\@inenglish@star{#1} en anglais%
}


\newcommand\ascii{\texttt{ASCII}}


% Example
\newcounter{paraexample}[subsubsection]

\newcommand\@newexample@abstract[2]{%
	\paragraph{%
		#1%
		\if\relax\detokenize{#2}\relax\else {} -- #2\fi%
	}%
}



\newcommand\newparaexample{\@ifstar{\@newparaexample@star}{\@newparaexample@no@star}}

\newcommand\@newparaexample@no@star[1]{%
	\refstepcounter{paraexample}%
	\@newexample@abstract{Exemple \theparaexample}{#1}%
}

\newcommand\@newparaexample@star[1]{%
	\@newexample@abstract{Exemple}{#1}%
}


% Change log
\newcommand\topic{\@ifstar{\@topic@star}{\@topic@no@star}}

\newcommand\@topic@no@star[1]{%
	\textbf{\textsc{#1}.}%
}

\newcommand\@topic@star[1]{%
	\textbf{\textsc{#1} :}%
}






    % == PACKAGES USED == %

\RequirePackage[Symbolsmallscale]{upgreek}
\RequirePackage{xstring}


% == DEFINITIONS == %

% Constants - START

% User's constants

\newcommand\param[1]{%
    \mathop{{}%
        \IfStrEqCase{#1}{%
            {gamma}{\upgamma}%
        {pi}{\uppi}%
        {tau}{\uptau}%
        }[\text{\textbf{#1}}]%
    }%
}

% Classical constants
    
\newcommand\ggamma{\param{gamma}}
\newcommand\ppi{\param{pi}}
\newcommand\ttau{\param{tau}}
\newcommand\ee{\param{e}}
\newcommand\ii{\param{i}}
\newcommand\jj{\param{j}}
\newcommand\kk{\param{k}}

% Constants - END

    % == PACKAGES USED == %

\RequirePackage{mathtools}


% == DEFINITIONS == %

% Source :
%    * https://tex.stackexchange.com/a/43009/6880
%
\DeclarePairedDelimiter\abs{\lvert}{\rvert}%

\let\tnsana@old@abs\abs
\def\abs{\@ifstar{\tnsana@old@abs}{\tnsana@old@abs*}}

	\usepackage{amsmath}

    \usepackage{01-tables}
\makeatother


\begin{document}

%% \section{Analysis}
%
%\section{Tableaux de variation et de signe}

\subsection{Décorer facilement un tableau}

\subsubsection{Motivation}

Considérons le tableau suivant et imaginons que nous voulions l'expliquer à un débutant.

\begin{center}
	\input{tikz/tabsignvar-deco-basic-1.tkz}
\end{center}

Deux options s'offrent à nous pour justifier comment a été rempli le tableau.

\begin{enumerate}
    \item Classiquement on résout par exemple juste les deux inéquations $2 x - 3 > 0$ et $-x + 5 > 0$ puis on complète les deux premières lignes
    \footnote{
        Notons que cette approche est un peu scandaleuse car il faudrait en toute rigueur aussi résoudre
        $2 x - 3 < 0$ , $-x + 5 < 0$ , $2 x - 3 = 0$ et $-x + 5 = 0$.
        Personne ne le fait car l'on pense aux variations d'une fonction affine. Dans ce cas pourquoi ne pas juste utiliser ce dernier argument?
        C'est ce que propose la 2\ieme{} méthode.
    }
    pour en déduire la dernière via la règle des signes d'un produit.

    \item On peut proposer une méthode moins sujette à la critique qui s'appuie sur la représentation graphique d'une fonction affine en produisant le tableau suivant.
\end{enumerate}

\begin{center}
	\input{tikz/tabsignvar-deco-basic-2.tkz}
\end{center}


Pour produire le 2\ieme{} tableau, en plus du code \verb#tkz-tab# pour le tableau de signe
\footnote{
	Nous avons utilisé les réglages optionnels
	\texttt{lgt = 3.5} et \texttt{espcl = 2.5} de \macro{tkzTabInit}
	pour avoir de la place dans la 1\iere{} colonne pour le dernier produit
	et aussi réduire la largeur des colonnes pour les signes.
},
il a fallu ajouter les lignes données ci-dessous où sont utilisées les macros     \macro{backLine}, \macro{graphSign} et \macro{comLine} proposées par \verb+tnsana+ \emph{(la syntaxe simple à suivre sera expliquée dans les trois sections suivantes)}.
	Indiquons que les lignes pour les signes doivent utiliser un coefficient minimal de \texttt{1.5} pour la hauteur afin d'éviter la superposition des graphiques.

\medskip

\inputlatexexalone{tikz/tabsignvar-deco-basic-2-short.tkz}


\begin{remark}
	Il est aussi possible de décorer une ligne de variation comme cela sera montré dans l'exemple \ref{tnsana-graphsign-com-two-lines} page \pageref{tnsana-graphsign-com-two-lines}. 
\end{remark}


% ---------------------- %


\subsubsection{Ajouter une couleur de fond à une ligne}

La modification de la couleur de fond d'une ligne se fait via la macro \macro{backLine}
\footnote{
    L'auteur de \prefix{tnsana} n'est absolument pas un fan de la casse en bosses de chameau mais par souci de cohérence avec ce que propose \prefix{tkz-tab} le nom \macro{backLine} a été proposé à la place de \macro{backline}.
}
pour \whyprefix{back}{ground} \prefix{of the line} soit \inenglish{fond de la ligne}.
Cette macro possède un argument optionnel et un obligatoire.

\begin{itemize}[label=\small\textbullet, itemsep=.25em]
    \item \textit{L'argument optionnel : choix de la couleur de fond.}
          
          \smallskip
          
          Ci-dessus nous avons utilisé la couleur par défaut qui est  \verb#gray!30#.


    \medskip
    \item \textit{L'argument obligatoire : les numéros de ligne séparés par des virgules.}
          
          \smallskip
          
          La numérotation des lignes commence à $0$ comme en informatique. Ainsi \verb#\backLine{0,3}# ajoute une couleur de fond à la ligne des valeurs utiles de $x$ et à la 3\ieme{} ligne de signes, ou moins intuitivement à la (3+1)\ieme{} ligne du tableau.
\end{itemize}



% ---------------------- %


\subsubsection{Commenter une ligne}

L'ajout de commentaires courts se fait via la macro \macro{comLine} pour \whyprefix{com}{ment a} \prefix{line} soit \inenglish{commenter une ligne}.
Cette macro possède un argument optionnel et deux obligatoires.

\begin{itemize}[label=\small\textbullet, itemsep=.25em]
    \item \textit{L'argument optionnel : choix de la couleur du texte.}
          
          \smallskip
          
          Ci-dessus nous avons utilisé \verb#\comLine[gray]{0}{...}# pour avoir un texte en gris.


    \item \textit{Le 1\ier{} argument : le numéro de ligne \emph{(comme pour \macro{backLine})}.}


    \item \textit{Le 2\ieme{} argument : texte du commentaire.}
          
          \smallskip
          
          Par défaut aucun retour à la ligne n'est possible.
          Si besoin se reporter à l'exemple \ref{tnsana-graphsign-com-two-lines},  page \pageref{tnsana-graphsign-com-two-lines} section \ref{tnsana-graphsign-examples}, où est montré comment écrire sur plusieurs lignes.
\end{itemize}


% ---------------------- %


\subsubsection{Graphiques pour expliquer des signes}

Pour le moment, la macro \macro{graphSign} propose différents types de graphiques de fonctions dites de référence.
Avant de voir ce qui est proposé rappelons que la convention est de prendre $0$ pour numéro de la toute 1\iere{} ligne contenant les valeurs utiles de la variable.

\begin{enumerate}
    \item \textbf{\itshape Fonctions affines non constantes avec une contrainte.}
          
          \smallskip

          Pour les fonctions du type $f(x) = a x + b$ avec $a \neq 0$, nous devons connaître le signe de $a$ et la racine $r$ de $f$.
          Le codage est simple : considérons par exemple \verb#\graphSign{2}{ax+b, an}{$5$}#.
		  %
          \begin{itemize}[label=\small\textbullet, itemsep=.25em]
          		\item \textit{1\ier{} argument $2$.}

		              \smallskip
		              Ceci indique d'ajouter le graphique dans la 2\ieme{} ligne de signes.


          		\item \textit{\texttt{ax+b} dans le 2\ieme{} argument.}

		              \smallskip
		              Ce code sans espace indique une fonction affine.
		              

          		\item \textit{\texttt{an} dans le 2\ieme{} argument.}

		              \smallskip
		              Ce code venant de \prefix{a négatif} ajoute la condition $a < 0$.

		
				\item \textit{3\ieme{} argument $5$.}

		              \smallskip
		              Ceci donne la racine.
          \end{itemize}

          Donc pour ajouter dans la 4\ieme{} ligne de signe le graphique de $f(x) = 3x$, on utilisera dans ce cas \verb#\graphSign{4}{ax+b, ap}{$0$}# où \prefix{ap} pour \prefix{a positif} code la condition $a > 0$.


    % ==================== %


    \bigskip
    \item \textbf{\itshape Fonctions trinômiales du 2\ieme{} degré avec deux contraintes.}
          
          \smallskip

          Pour les fonctions du type $f(x) = a x^2 + b x + c$ avec $a \neq 0$, en plus du signe de $a$ nous devons connaître celui du discriminant $\Delta = b^2 - 4ac$, ce dernier pouvant être nul, sans oublier les racines réelles éventuelles.
		  Voyons comment coder ce genre de chose via par exemple \verb#\graphSign{5}{ax2+bx+c, an, dp}{$r_1$}{$r_2$}#.
		  %
          \begin{itemize}[label=\small\textbullet, itemsep=.25em]
          		\item \textit{1\ier{} argument $5$.}

		              \smallskip
		              On indique la 5\ieme{} ligne de signes.

          		\item \textit{\texttt{ax2+bx+c} dans le 2\ieme{} argument.}

		              \smallskip
		              Ce code sans espace indique un trinôme du 2\ieme{} degré.


          		\item \textit{\texttt{an} dans le 2\ieme{} argument \emph{(comme avant).}}


          		\item \textit{\texttt{dp} dans le 2\ieme{} argument.}

		              \smallskip
		              Ce code venant de \prefix{discriminant positif} ajoute la condition $\Delta > 0$.
		              En plus de \prefix{dn} et \prefix{dp} il y a aussi \prefix{dz} pour \prefix{discriminant zéro}.
		
		
				\item \textit{3\ieme{} et 4\ieme{} arguments $r_1$ et $r_2$.}

		              \smallskip
		              Ceci donne les deux racines réelles avec obligatoirement $r_1 < r_2$.
          \end{itemize}


          Ainsi pour indiquer dans la 3\ieme{} ligne de signe la courbe relative à $f(x) = - 4 x^2$, on utilisera \verb#\graphSign{3}{ax2+bx+c, an, dz}{$0$}#.


          \smallskip

          Enfin le graphique associé au trinôme $f(x) = 7 x^2 + 3$, qui est sans racine réelle, s'obtiendra dans la 4\ieme{} ligne de signe via \verb#\graphSign{4}{ax2+bx+c, ap, dn}#.


    % ==================== %


    \bigskip
    \item \textbf{\itshape Fonctions sans contrainte.}
          
          \smallskip

          Voici ce qui est disponible via \verb#\graphSign{noligne}{codefonc}# où les valeurs possibles de \verb#codefonc# sont les suivantes.
          
		  \begin{center}
		  	  \begin{tabular}{r|c|c|c|c|c|c}
				  \verb+codefonc+
				      &  \verb#x2#
				      &  \verb#sqrt#
				      &  \verb#1/x#
				      &  \verb#abs#
				      &  \verb#exp#
				      &  \verb#ln#
				  \\
				  \hline
				  $f(x)$
				      &  $x^{2\vphantom{X^{X^X}}}$
				      &  $\sqrt x$
				      &  $\frac1x$
				      &  $\abs x$
				      &  $\exp x$
				      &  $\ln x$ 
		  	  \end{tabular}
		  \end{center}
\end{enumerate}

\end{document}
