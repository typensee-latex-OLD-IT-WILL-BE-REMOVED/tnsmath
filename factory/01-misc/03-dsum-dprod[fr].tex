\documentclass[12pt,a4paper]{article}

\makeatletter
	\usepackage[utf8]{inputenc}
\usepackage[T1]{fontenc}
\usepackage{ucs}

\usepackage[french]{babel,varioref}

\usepackage[top=2cm, bottom=2cm, left=1.5cm, right=1.5cm]{geometry}
\usepackage{enumitem}

\usepackage{multicol}

\usepackage{color}
\usepackage{hyperref}
\hypersetup{
    colorlinks,
    citecolor=black,
    filecolor=black,
    linkcolor=black,
    urlcolor=black
}

\usepackage{amsthm}

\usepackage{tcolorbox}
\tcbuselibrary{listingsutf8}

\usepackage{ifplatform}

\usepackage{ifthen}

\usepackage{cbdevtool}


% MISC

\newtcblisting{latexex}{%
	sharp corners,%
	left=1mm, right=1mm,%
	bottom=1mm, top=1mm,%
	colupper=red!75!blue,%
	listing side text
}

\newtcblisting{latexex-flat}{%
	sharp corners,%
	left=1mm, right=1mm,%
	bottom=1mm, top=1mm,%
	colupper=red!75!blue,%
}

\newtcblisting{latexex-alone}{%
	sharp corners,%
	left=1mm, right=1mm,%
	bottom=1mm, top=1mm,%
	colupper=red!75!blue,%
	listing only
}


\newcommand\env[1]{\texttt{#1}}
\newcommand\macro[1]{\env{\textbackslash{}#1}}



\setlength{\parindent}{0cm}
\setlist{noitemsep}

\theoremstyle{definition}
\newtheorem*{remark}{Remarque}

\usepackage[raggedright]{titlesec}

\titleformat{\paragraph}[hang]{\normalfont\normalsize\bfseries}{\theparagraph}{1em}{}
\titlespacing*{\paragraph}{0pt}{3.25ex plus 1ex minus .2ex}{0.5em}


\newcommand\separation{
	\medskip
	\hfill\rule{0.5\textwidth}{0.75pt}\hfill
	\medskip
}


\newcommand\extraspace{
	\vspace{0.25em}
}


\newcommand\ascii{\texttt{ASCII}}


	\usepackage{03-dsum-dprod}
\makeatother



\begin{document}

%\section{Quelques modifications générales}

\subsection{Sommes et produits en mode ligne}

Pour limiter l'espace, \LaTeX{} affiche $\sum_{k=0}^{n}$ et non $\dsum_{k=0}^{n}$ sauf si l'on utilise la commande \macro{displaystyle}.
Les macros \macro{dsum} et \macro{dprod} permettent de se passer de \macro{displaystyle}.
Voici un exemple.


\begin{latexex}
$\dsum_{k=0}^{n} 2^k =
 \sum_{k=0}^{n} 2^k$

$\dprod_{k=1}^{n} k =
 \prod_{k=1}^{n} k$
\end{latexex}


% ---------------------- %


\subsection{Fiches techniques}

\paragraph{Sommes et produits en mode ligne}

Les macros suivantes sans argument ont un comportement spécifique vis à vis des mises en index et en exposant. 


\separation


\IDmacro*{dprod}{0}

\IDmacro*{dsum}{0}

\end{document}
