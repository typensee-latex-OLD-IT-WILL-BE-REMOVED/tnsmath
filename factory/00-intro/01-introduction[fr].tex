\documentclass[12pt,a4paper]{article}

\makeatletter
	\usepackage[utf8]{inputenc}
\usepackage[T1]{fontenc}
\usepackage{ucs}

\usepackage[french]{babel,varioref}

\usepackage[top=2cm, bottom=2cm, left=1.5cm, right=1.5cm]{geometry}
\usepackage{enumitem}

\usepackage{multicol}

\usepackage{color}
\usepackage{hyperref}
\hypersetup{
    colorlinks,
    citecolor=black,
    filecolor=black,
    linkcolor=black,
    urlcolor=black
}

\usepackage{amsthm}

\usepackage{tcolorbox}
\tcbuselibrary{listingsutf8}

\usepackage{ifplatform}

\usepackage{ifthen}

\usepackage{cbdevtool}


% MISC

\newtcblisting{latexex}{%
	sharp corners,%
	left=1mm, right=1mm,%
	bottom=1mm, top=1mm,%
	colupper=red!75!blue,%
	listing side text
}

\newtcblisting{latexex-flat}{%
	sharp corners,%
	left=1mm, right=1mm,%
	bottom=1mm, top=1mm,%
	colupper=red!75!blue,%
}

\newtcblisting{latexex-alone}{%
	sharp corners,%
	left=1mm, right=1mm,%
	bottom=1mm, top=1mm,%
	colupper=red!75!blue,%
	listing only
}


\newcommand\env[1]{\texttt{#1}}
\newcommand\macro[1]{\env{\textbackslash{}#1}}



\setlength{\parindent}{0cm}
\setlist{noitemsep}

\theoremstyle{definition}
\newtheorem*{remark}{Remarque}

\usepackage[raggedright]{titlesec}

\titleformat{\paragraph}[hang]{\normalfont\normalsize\bfseries}{\theparagraph}{1em}{}
\titlespacing*{\paragraph}{0pt}{3.25ex plus 1ex minus .2ex}{0.5em}


\newcommand\separation{
	\medskip
	\hfill\rule{0.5\textwidth}{0.75pt}\hfill
	\medskip
}


\newcommand\extraspace{
	\vspace{0.25em}
}


\newcommand\ascii{\texttt{ASCII}}

\makeatother


\newcommand\dintegrate[4]{%
  \int_{#1}^{#2} #3 d#4%
}

\newcommand\hook[4]{%
	\left[ #3 \right]_{#4=#1}^{#4=#2}%
}


\begin{document}

\section{Introduction}

\LaTeX{} est un excellent langage, pour ne pas dire le meilleur, pour rédiger des documents contenant des formules mathématiques.
Malheureusement toute la puissance de \LaTeX{} permet d'écrire des codes très peu sémantiques.
Le modeste but du package \verb+lymath+ est de fournir quelques macros sémantiques pour la rédaction de formules mathématiques élémentaires. Considérons le code \LaTeX{} suivant.

\begin{latexex-alone}
Sachant que $f^\prime(x) = cos(x^2)$ sur $[a ; b]$ , nous avons :

$\displaystyle \int_a^b cos(x^2) dx = \left[ f(x) \right]_{x=a}^{x=b}$.
\end{latexex-alone}


Avec \verb+lymath+, vous pouvez écrire le code suivant.

\begin{latexex-alone}
Sachant que $\sder{f}(x) = cos(x^2)$ sur $\intervalC{a}{b}$, nous avons :

$\dintegrate*{a}{b}{cos(x^2)}{x} = \hook{a}{b}{f(x)}{x}$.
\end{latexex-alone}


Même si certaines commandes sont plus longues à écrire que ce que permet \LaTeX{}, il y a des avantages à utiliser des commandes sémantiques.
\begin{enumerate}
	\item La mise en forme dans votre document devient consistante.

	\item Il est facile de changer une mise en forme sur l'ensemble d'un document ou localement via certaines options.

	\item \verb+lymath+ résout certains problèmes "complexes" pour vous.
\end{enumerate}


% ---------------------- %


\section{Comment lire cette documentation ?}

Le choix a été fait de fournir des exemples comme documentation du package suivis de fiches techniques des macros-commandes \emph{(voir la section dédiée \ref{techincal-ids})}. Les exemples se présentent comme ci-dessous et sont généralement très courts.

\begin{latexex}
$\dintegrate*{a}{b}{cos(x^2)}{x}
 =
 \hook*{a}{b}{f(x)}{x}$
\end{latexex}


% ---------------------- %


\section{A propos des macros}

\subsection{Règles de nommage}

\subsubsection{Les macros de même \og type \fg}

Les macros partageant une même fonctionnalité mathématique suivent les règles suivantes.
\begin{enumerate}
	\item Un nom de base explicite est choisi comme par exemple \macro{dotproduct} pour \inenglish{produit scalaire} ou \macro{set} pour \inenglish{ensemble}.

	\item Si besoin, on spécialise du point de vue sémantique avec un préfixe et/ou un suffixe. Voici deux exemples.
	\begin{enumerate}
		\item Dans \macro{vdotproduct}, le préfixe \prefix{v} est pour \whyprefix{v}{ecteur} car cette macro s'utilise avec des noms de vecteurs \verb+u+ et \verb+v+ et non directement des vecteurs \verb+\vect{u}+ et \verb+\vect{v}+.

		\item Dans \macro{setproba}, le suffixe \prefix{proba} est pour \whyprefix{proba}{bilité} car cette macro sert à écrire des ensembles munis d'une probabilité
	      \footnote{
	      	Ce choix est assumé même si on obtient un nom faisant penser à \emph{\og régler ... \fg} au lieu de \emph{\og ensemble de type ... \fg}.
		  }.
	\end{enumerate}

	\item Si l'on propose différentes mises en forme pour une même signification sémantique alors ceci se fera via des versions étoilées et/ou par le biais d'option(s) comme dans \macro{dotproduct[r]} pour obtenir des produits scalaires utilisant des chevrons \emph{(\prefix{r} est pour \whyprefix{r}{after} soit \inenglish{chevron})}.
\end{enumerate}


% ---------------------- %


\subsubsection{Les formes \og négatives \fg{} des macros}

Les formes \og négatives \fg{} des macros auront un nom préfixé par la lettre \prefix{n} en référence à \whyprefix{n}{ot}.


% ---------------------- %


\subsubsection{Les macros en mode \texttt{displaystyle}}

Les macros évitant d'avoir à taper \macro{displaystyle} auront un nom préfixé par la lettre \prefix{d} comme par exemple dans \macro{dintegrate}.


% ---------------------- %


\subsubsection{Les macros standards redéfinies}

Certaines macros comme \verb+\frac+ sont un peu revues par \verb+lymath+.
Dans ce cas, les versions standard restent accessibles en utilisant le préfixe \prefix{std} ce qui donne ici la macro \macro{stdfrac}.


% ---------------------- %


\subsection{Versions étoilées}

Les versions étoilées proposent des mises en forme correspondant aux cas les moins usuels ou les moins pratiques : par exemple, la macro \macro{sder} utilise un exposant entre des parenthèses pour rédiger une dérivée n\ieme{} tandis que la version étoilée, si c'est possible, affichera des primes en exposant.


% ---------------------- %


\subsection{Les arguments : deux conventions à connaître}

\subsubsection{Avec un nombre fixé d'arguments}

Dans ce cas, c'est la syntaxe \LaTeX{} usuelle qui sera à utiliser comme dans \macro{vdotprod\{u\}\{v\}}.


% ---------------------- %


\subsubsection{Avec un nombre variable d'arguments}

Certaines macros offrent la possibilité de fournir un nombre variable d'arguments comme dans \macro{coord\{x | y | z | t\}} et \macro{coord\{x | y\}}.
Ceci se fait en utilisant un seul argument, au sens de \LaTeX{}, dont le contenu est formé de morceaux séparés par des traits verticaux \verb+|+.
Ainsi dans \macro{coord\{x | y | z | t\}}, l'unique argument \verb+x | y | z | t+, au sens de \LaTeX{}, sera analysé par \verb+lymath+ comme étant formé des quatre arguments \verb+x+ , \verb+y+ , \verb+z+ et \verb+t+.


% ---------------------- %


\subsubsection{Important}

Certaines macros pourront demander des arguments non utilisés pour la mise en forme. Pourquoi cela ? Tout simplement pour permettre des copier-coller lors de la rédaction : voir par exemple le calcul du déterminant de deux vecteurs du plan pour vérifier leur colinéarité \emph{(ceci est présenté dans la section \ref{colinearity-criteria})}.


% ---------------------- %


\section{Couleurs}

Certaines macros utilisent de la couleur. Les choix faits sont tels que l'impression en noir et blanc ne soit pas impactée.


\end{document}
