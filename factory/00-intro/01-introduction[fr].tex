\documentclass[12pt,a4paper]{article}

\makeatletter
	\usepackage[utf8]{inputenc}
\usepackage[T1]{fontenc}
\usepackage{ucs}

\usepackage[french]{babel,varioref}

\usepackage[top=2cm, bottom=2cm, left=1.5cm, right=1.5cm]{geometry}
\usepackage{enumitem}

\usepackage{multicol}

\usepackage{makecell}

\usepackage{color}
\usepackage{hyperref}
\hypersetup{
    colorlinks,
    citecolor=black,
    filecolor=black,
    linkcolor=black,
    urlcolor=black
}

\usepackage{amsthm}

\usepackage{tcolorbox}
\tcbuselibrary{listingsutf8}

\usepackage{ifplatform}

\usepackage{ifthen}

\usepackage{cbdevtool}


% MISC

\newtcblisting{latexex}{%
	sharp corners,%
	left=1mm, right=1mm,%
	bottom=1mm, top=1mm,%
	colupper=red!75!blue,%
	listing side text
}

\newtcblisting{latexex-flat}{%
	sharp corners,%
	left=1mm, right=1mm,%
	bottom=1mm, top=1mm,%
	colupper=red!75!blue,%
}

\newtcblisting{latexex-alone}{%
	sharp corners,%
	left=1mm, right=1mm,%
	bottom=1mm, top=1mm,%
	colupper=red!75!blue,%
	listing only
}


\newcommand\env[1]{\texttt{#1}}
\newcommand\macro[1]{\env{\textbackslash{}#1}}



\setlength{\parindent}{0cm}
\setlist{noitemsep}

\theoremstyle{definition}
\newtheorem*{remark}{Remarque}

\usepackage[raggedright]{titlesec}

\titleformat{\paragraph}[hang]{\normalfont\normalsize\bfseries}{\theparagraph}{1em}{}
\titlespacing*{\paragraph}{0pt}{3.25ex plus 1ex minus .2ex}{0.5em}


\newcommand\separation{
	\medskip
	\hfill\rule{0.5\textwidth}{0.75pt}\hfill
	\medskip
}


\newcommand\extraspace{
	\vspace{0.25em}
}


\newcommand\whyprefix[2]{%
	\textbf{\prefix{#1}}-#2%
}

\newcommand\mwhyprefix[2]{%
	\texttt{#1 = #1-#2}%
}

\newcommand\prefix[1]{%
	\texttt{#1}%
}


\newcommand\inenglish{\@ifstar{\@inenglish@star}{\@inenglish@no@star}}

\newcommand\@inenglish@star[1]{%
	\emph{\og #1 \fg}%
}

\newcommand\@inenglish@no@star[1]{%
	\@inenglish@star{#1} en anglais%
}


\newcommand\ascii{\texttt{ASCII}}


% Example
\newcounter{paraexample}[subsubsection]

\newcommand\@newexample@abstract[2]{%
	\paragraph{%
		#1%
		\if\relax\detokenize{#2}\relax\else {} -- #2\fi%
	}%
}



\newcommand\newparaexample{\@ifstar{\@newparaexample@star}{\@newparaexample@no@star}}

\newcommand\@newparaexample@no@star[1]{%
	\refstepcounter{paraexample}%
	\@newexample@abstract{Exemple \theparaexample}{#1}%
}

\newcommand\@newparaexample@star[1]{%
	\@newexample@abstract{Exemple}{#1}%
}


% Change log
\newcommand\topic{\@ifstar{\@topic@star}{\@topic@no@star}}

\newcommand\@topic@no@star[1]{%
	\textbf{\textsc{#1}.}%
}

\newcommand\@topic@star[1]{%
	\textbf{\textsc{#1} :}%
}






\makeatother


\begin{document}

\section{Introduction}

\LaTeX{} est un excellent langage, pour ne pas dire le meilleur, pour rédiger des documents contenant des formules mathématiques.
Malheureusement toute la puissance de \LaTeX{} permet d'écrire des codes très peu sémantiques.
Le modeste but du package \verb+lymath+ est de fournir quelques macros sémantiques pour la rédaction de formules mathématiques élémentaires. Considérons le code \LaTeX{} suivant.

\begin{tcblisting}{listing only}
Sachant que $\frac{df}{dx}(x) = 4 cos(x^2)$ sur $[a ; b]$ , nous avons :

$\int_a^b cos(x^2) dx = \left[ \frac{1}{4} f(x) \right]_a^b$.
\end{tcblisting}


Avec \verb+lymath+, vous pouvez écrire le code suivant.

\begin{tcblisting}{listing only}
Sachant que $\derfrac{f}{x}(x) = 4 cos(x^2)$ sur $\intervalC{a}{b}$, nous avons :

$\int_a^b cos(x^2) \dd{x} = \hook{\frac{1}{4} f(x)}{a}{b}$.
\end{tcblisting}


Même si certaines commandes sont plus longues à écrire que ce que permet \LaTeX{}, il y a trois avantages à utiliser des commandes sémantiques.
\begin{enumerate}
	\item La mise en forme dans votre document sera consistante.

	\item Il est facile de changer une mise en forme sur l'ensemble d'un document.

	\item \verb+lymath+ résout certains problèmes "complexes" pour vous.
\end{enumerate}



\section{Comment lire cette documentation ?}

Le choix a été fait de fournir des exemples comme documentation du package suivis de fiches techniques des macros-commandes. Les exemples se présentent comme ci-dessous \textit{(un code \LaTeX{} suivi de sa mise en forme)}.

\begin{tcblisting}{}
Sachant que $\displaystyle \frac{df}{dx}(x) = 4 cos(x^2)$ sur $[a ; b]$ , nous avons :
$\displaystyle \int_a^b cos(x^2) dx = \left[ \frac{1}{4} f(x) \right]_a^b$.
\end{tcblisting}


\section{Versions étoilées}

Généralement les versions étoilées proposent des mises en forme faisant un peu moins de travail que la version non étoilée \textit{(il y a tout de même quelques exceptions)}. Par exemple une macro utilisant des parenthèses extensibles aura sa version étoilée qui n'utilisera que des parenthèses non extensibles.



\section{A propos des arguments \LaTeX, une convention à connaître}

\verb+lymath+ propose des macros avec un \textbf{nombre fixé d'arguments}. Dans ce cas, la syntaxe \LaTeX{} est gardée comme dans \verb+\derfrac{f}{x}+ .

\medskip

Par contre, pour les macros avec un \textbf{nombre variable d'arguments}, la convention sera toujours d'utiliser un seul argument, au sens \LaTeX{}, dont le contenu sera formé de morceaux séparés par des traits verticaux \verb+|+ comme dans \verb+\coord{3 | -4 | 0}+ où l'unique argument \verb+3 | -4 | 0+ contient les trois morceaux \verb+3+ , \verb+-4+ et \verb+0+ .

\end{document}
