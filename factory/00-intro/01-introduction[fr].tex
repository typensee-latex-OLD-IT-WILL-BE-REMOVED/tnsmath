\documentclass[12pt,a4paper]{article}

\makeatletter
	\usepackage[utf8]{inputenc}
\usepackage[T1]{fontenc}
\usepackage{ucs}

\usepackage[french]{babel,varioref}

\usepackage[top=2cm, bottom=2cm, left=1.5cm, right=1.5cm]{geometry}
\usepackage{enumitem}

\usepackage{multicol}

\usepackage{color}
\usepackage{hyperref}
\hypersetup{
    colorlinks,
    citecolor=black,
    filecolor=black,
    linkcolor=black,
    urlcolor=black
}

\usepackage{amsthm}

\usepackage{tcolorbox}
\tcbuselibrary{listingsutf8}

\usepackage{ifplatform}

\usepackage{ifthen}

\usepackage{cbdevtool}


% MISC

\newtcblisting{latexex}{%
	sharp corners,%
	left=1mm, right=1mm,%
	bottom=1mm, top=1mm,%
	colupper=red!75!blue,%
	listing side text
}

\newtcblisting{latexex-flat}{%
	sharp corners,%
	left=1mm, right=1mm,%
	bottom=1mm, top=1mm,%
	colupper=red!75!blue,%
}

\newtcblisting{latexex-alone}{%
	sharp corners,%
	left=1mm, right=1mm,%
	bottom=1mm, top=1mm,%
	colupper=red!75!blue,%
	listing only
}


\newcommand\env[1]{\texttt{#1}}
\newcommand\macro[1]{\env{\textbackslash{}#1}}



\setlength{\parindent}{0cm}
\setlist{noitemsep}

\theoremstyle{definition}
\newtheorem*{remark}{Remarque}

\usepackage[raggedright]{titlesec}

\titleformat{\paragraph}[hang]{\normalfont\normalsize\bfseries}{\theparagraph}{1em}{}
\titlespacing*{\paragraph}{0pt}{3.25ex plus 1ex minus .2ex}{0.5em}


\newcommand\separation{
	\medskip
	\hfill\rule{0.5\textwidth}{0.75pt}\hfill
	\medskip
}


\newcommand\extraspace{
	\vspace{0.25em}
}


\newcommand\ascii{\texttt{ASCII}}

\makeatother


\begin{document}

\section{Introduction}

\LaTeX{} est un excellent langage, pour ne pas dire le meilleur, pour rédiger des documents contenant des formules mathématiques.
Malheureusement toute la puissance de \LaTeX{} permet d'écrire des codes très peu sémantiques.
Le modeste but du package \verb+lymath+ est de fournir quelques macros sémantiques pour la rédaction de formules mathématiques élémentaires. Considérons le code \LaTeX{} suivant.

\begin{tcblisting}{listing only}
Sachant que $\frac{df}{dx}(x) = 4 cos(x^2)$ sur $[a ; b]$ , nous avons :

$\int_a^b cos(x^2) dx = \left[ \frac{1}{4} f(x) \right]_a^b$.
\end{tcblisting}


Avec \verb+lymath+, vous pouvez écrire le code suivant.

\begin{tcblisting}{listing only}
Sachant que $\derfrac{f}{x}(x) = 4 cos(x^2)$ sur $\intervalC{a}{b}$, nous avons :

$\int_a^b cos(x^2) \dd{x} = \hook{\frac{1}{4} f(x)}{a}{b}$.
\end{tcblisting}


Même si certaines commandes sont plus longues à écrire que ce que permet \LaTeX{}, il y a trois avantages à utiliser des commandes sémantiques.
\begin{enumerate}
	\item La mise en forme dans votre document sera consistante.

	\item Il est facile de changer une mise en forme sur l'ensemble d'un document.

	\item \verb+lymath+ résout certains problèmes "complexes" pour vous.
\end{enumerate}



\section{Comment lire cette documentation ?}

Le choix a été fait de fournir des exemples comme documentation du package suivis de fiches techniques des macros-commandes. Les exemples se présentent comme ci-dessous \textit{(un code \LaTeX{} suivi de sa mise en forme)}.

\begin{tcblisting}{}
Sachant que $\displaystyle \frac{df}{dx}(x) = 4 cos(x^2)$ sur $[a ; b]$ , nous avons :
$\displaystyle \int_a^b cos(x^2) dx = \left[ \frac{1}{4} f(x) \right]_a^b$.
\end{tcblisting}



\section{A propos des macros}

\subsection{Règles de nommage}

\subsubsection{Les macros de même \og type \fg}

Les macros partageant une même fonctionnalité mathématique auront toute le même préfixe comme par exemple pour \verb+\derpow+ , \verb+\derfrac+ \dots{} utilisables pour rédiger des dérivées de fonctions.
Ce choix est assumé même si pour les macros du type \verb+\set...+ on obtient un nom pouvant faire penser à \emph{\og régler ... \fg} au lieu de \emph{\og ensemble du type ... \fg}.



\subsubsection{Les macros en mode \texttt{displaystyle}}

Les macros évitant d'avoir à taper \verb+displaystyle+ auront un nom commençant par la lettre \verb+d+.



\subsection{Versions étoilées}

Les versions étoilées proposent des mises en forme correspondant aux cas les moins usuels : par exemple une macro utilisant des parenthèses rendra ces dernières extensibles sauf dans sa version étoilée.



\subsection{Les arguments, deux conventions à connaître}

\subsubsection{Nombre fixé d'arguments}

Dans ce cas, c'est la syntaxe \LaTeX{} usuelle qui sera à utiliser comme dans \verb+\derfrac{f}{x}+.



\subsubsection{Nombre variable d'arguments}

Certaines macros offrent la possibilité de fournir un nombre variable d'arguments comme dans \verb+\coord{x | y | z | t}+ et \verb+\coord{x | y}+.
Ceci se fait en utilisant un seul argument, au sens de \LaTeX{}, dont le contenu est formé de morceaux séparés par des traits verticaux \verb+|+.
Ainsi dans \verb+\coord{x | y | z | t}+, l'unique argument \verb+x | y | z | t+, au sens de \LaTeX{}, sera analysé par \verb+lyxam+ comme étant formé des quatre arguments \verb+x+ , \verb+y+ , \verb+z+ et \verb+t+.

\end{document}
