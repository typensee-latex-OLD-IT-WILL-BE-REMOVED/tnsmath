\documentclass[12pt,a4paper]{book}

\makeatletter
	\usepackage[utf8]{inputenc}
\usepackage[T1]{fontenc}
\usepackage{ucs}

\usepackage[french]{babel,varioref}

\usepackage[top=2cm, bottom=2cm, left=1.5cm, right=1.5cm]{geometry}
\usepackage{enumitem}

\usepackage{multicol}

\usepackage{makecell}

\usepackage{color}
\usepackage{hyperref}
\hypersetup{
    colorlinks,
    citecolor=black,
    filecolor=black,
    linkcolor=black,
    urlcolor=black
}

\usepackage{amsthm}

\usepackage{tcolorbox}
\tcbuselibrary{listingsutf8}

\usepackage{ifplatform}

\usepackage{ifthen}

\usepackage{cbdevtool}


% MISC

\newtcblisting{latexex}{%
	sharp corners,%
	left=1mm, right=1mm,%
	bottom=1mm, top=1mm,%
	colupper=red!75!blue,%
	listing side text
}

\newtcblisting{latexex-flat}{%
	sharp corners,%
	left=1mm, right=1mm,%
	bottom=1mm, top=1mm,%
	colupper=red!75!blue,%
}

\newtcblisting{latexex-alone}{%
	sharp corners,%
	left=1mm, right=1mm,%
	bottom=1mm, top=1mm,%
	colupper=red!75!blue,%
	listing only
}


\newcommand\env[1]{\texttt{#1}}
\newcommand\macro[1]{\env{\textbackslash{}#1}}



\setlength{\parindent}{0cm}
\setlist{noitemsep}

\theoremstyle{definition}
\newtheorem*{remark}{Remarque}

\usepackage[raggedright]{titlesec}

\titleformat{\paragraph}[hang]{\normalfont\normalsize\bfseries}{\theparagraph}{1em}{}
\titlespacing*{\paragraph}{0pt}{3.25ex plus 1ex minus .2ex}{0.5em}


\newcommand\separation{
	\medskip
	\hfill\rule{0.5\textwidth}{0.75pt}\hfill
	\medskip
}


\newcommand\extraspace{
	\vspace{0.25em}
}


\newcommand\whyprefix[2]{%
	\textbf{\prefix{#1}}-#2%
}

\newcommand\mwhyprefix[2]{%
	\texttt{#1 = #1-#2}%
}

\newcommand\prefix[1]{%
	\texttt{#1}%
}


\newcommand\inenglish{\@ifstar{\@inenglish@star}{\@inenglish@no@star}}

\newcommand\@inenglish@star[1]{%
	\emph{\og #1 \fg}%
}

\newcommand\@inenglish@no@star[1]{%
	\@inenglish@star{#1} en anglais%
}


\newcommand\ascii{\texttt{ASCII}}


% Example
\newcounter{paraexample}[subsubsection]

\newcommand\@newexample@abstract[2]{%
	\paragraph{%
		#1%
		\if\relax\detokenize{#2}\relax\else {} -- #2\fi%
	}%
}



\newcommand\newparaexample{\@ifstar{\@newparaexample@star}{\@newparaexample@no@star}}

\newcommand\@newparaexample@no@star[1]{%
	\refstepcounter{paraexample}%
	\@newexample@abstract{Exemple \theparaexample}{#1}%
}

\newcommand\@newparaexample@star[1]{%
	\@newexample@abstract{Exemple}{#1}%
}


% Change log
\newcommand\topic{\@ifstar{\@topic@star}{\@topic@no@star}}

\newcommand\@topic@no@star[1]{%
	\textbf{\textsc{#1}.}%
}

\newcommand\@topic@star[1]{%
	\textbf{\textsc{#1} :}%
}






\makeatother


\begin{document}

\section{A propos des macros}

\subsection{Règles de nommage}

\subsubsection{Les macros de même \og type \fg}

Les macros partageant une même fonctionnalité mathématique suivent les règles suivantes.

\begin{enumerate}[itemsep = .5em]
	\item Un nom de base explicite est choisi comme par exemple \prefix{dotproduct} pour \inenglish{produit scalaire} ou \prefix{set} pour \inenglish{ensemble}.

	\item Si besoin on spécialise du point de vue sémantique avec un préfixe et/ou un suffixe. Voici deux exemples.
	\begin{enumerate}
		\item Dans \macro{vdotproduct}, le préfixe \prefix{v} est pour \whyprefix{v}{ecteur} car cette macro s'utilise avec des noms de vecteurs \verb+u+ et \verb+v+ et non directement des vecteurs \verb+\vect{u}+ et \verb+\vect{v}+, autrement dit c'est \macro{vdotproduct} qui se charge d'appliquer \verb+\vect+ à \verb+u+ et \verb+v+.

		\item Dans \macro{setproba}, le suffixe \prefix{proba} est pour \whyprefix{proba}{bilité} car cette macro sert à écrire des ensembles munis d'une probabilité
	      \footnote{
	      	Ce choix est assumé même si on obtient un nom faisant penser à \emph{\og régler ... \fg} au lieu de \emph{\og ensemble de type ... \fg}.
		  }.
	\end{enumerate}

	\item Si l'on propose différentes mises en forme pour une même signification sémantique alors ceci se fera via des versions étoilées et/ou par le biais d'option(s) comme dans \macro{dotproduct[r]} pour obtenir des produits scalaires utilisant des chevrons $\langle$ et $\rangle$ \emph{(\prefix{r} est pour \whyprefix{r}{after} soit \inenglish{chevron})}.
\end{enumerate}


% ---------------------- %


\subsubsection{Les formes \og négatives \fg{} des macros}

Les formes \og négatives \fg{} des macros auront un nom préfixé par la lettre \prefix{n} en référence à \whyprefix{n}{ot}. C'est l'usage dans le monde \LaTeX{} comme par exemple pour \macro{neq}. 


% ---------------------- %


\subsubsection{Les macros en mode \texttt{displaystyle}}

Les macros évitant d'avoir à taper \macro{displaystyle} auront un nom préfixé par la lettre \prefix{d} comme par exemple pour \macro{dintegrate}.


% ---------------------- %


\subsubsection{Les macros \og textuelles \fg}

Certains macros produisent du contenu de type texte. Ces dernières seront toutes préfixées par \prefix{txt}.
Par exemple \macro{txtopdef} est un texte utilisé pour décorer le signe égal, ou aussi \macro{texfuncdef} permet de saisir rapidement la définition explicite d'une fonction pour l'avoir directement dans du texte et non dans une formule.


% ---------------------- %


\subsubsection{Les macros standards redéfinies}

Certaines macros comme \verb+\frac+ sont un peu revues par \verb+tnsmath+.
Dans ce cas, les versions standard restent accessibles en utilisant le préfixe \prefix{std} ce qui donne ici la macro \macro{stdfrac}.


% ---------------------- %


\subsubsection{Casse utilisée pour les lettres}

Les macros à usage graphique utiliseront une casse en bosses de chameau comme c'est le cas par exemple
pour \macro{ptreeFrame} qui trace des cadres sur des arbres de probabilité,
ou pour \macro{graphSign} qui ajoute des graphiques de signe près d'une ligne d'un tableau de signe
\footnote{
	La raison qui a poussé à ce choix est expliqué dans la présentation des outils de décoration des tableaux de signe : voir la présentation de la macro \macro{backLine}.
}.

\medskip

Les macros non graphiques n'utiliseront que des minuscules. L'auteur de \verb#tnsmath# préfère cette convention car elle est plus efficace à utiliser lors de la saisie et qu'elle impose aussi aux concepteurs de ne pas proposer des noms de macro à rallonge
\footnote{
	Cette pratique n'est pas inutile en interne pour autant par exemple pour nommer de façon compréhensible des macros privées.
}.

\end{document}
