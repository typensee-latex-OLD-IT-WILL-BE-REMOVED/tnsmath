\documentclass[12pt,a4paper]{book}

\makeatletter
	\usepackage[utf8]{inputenc}
\usepackage[T1]{fontenc}
\usepackage{ucs}

\usepackage[french]{babel,varioref}

\usepackage[top=2cm, bottom=2cm, left=1.5cm, right=1.5cm]{geometry}
\usepackage{enumitem}

\usepackage{multicol}

\usepackage{color}
\usepackage{hyperref}
\hypersetup{
    colorlinks,
    citecolor=black,
    filecolor=black,
    linkcolor=black,
    urlcolor=black
}

\usepackage{amsthm}

\usepackage{tcolorbox}
\tcbuselibrary{listingsutf8}

\usepackage{ifplatform}

\usepackage{ifthen}

\usepackage{cbdevtool}


% MISC

\newtcblisting{latexex}{%
	sharp corners,%
	left=1mm, right=1mm,%
	bottom=1mm, top=1mm,%
	colupper=red!75!blue,%
	listing side text
}

\newtcblisting{latexex-flat}{%
	sharp corners,%
	left=1mm, right=1mm,%
	bottom=1mm, top=1mm,%
	colupper=red!75!blue,%
}

\newtcblisting{latexex-alone}{%
	sharp corners,%
	left=1mm, right=1mm,%
	bottom=1mm, top=1mm,%
	colupper=red!75!blue,%
	listing only
}


\newcommand\env[1]{\texttt{#1}}
\newcommand\macro[1]{\env{\textbackslash{}#1}}



\setlength{\parindent}{0cm}
\setlist{noitemsep}

\theoremstyle{definition}
\newtheorem*{remark}{Remarque}

\usepackage[raggedright]{titlesec}

\titleformat{\paragraph}[hang]{\normalfont\normalsize\bfseries}{\theparagraph}{1em}{}
\titlespacing*{\paragraph}{0pt}{3.25ex plus 1ex minus .2ex}{0.5em}


\newcommand\separation{
	\medskip
	\hfill\rule{0.5\textwidth}{0.75pt}\hfill
	\medskip
}


\newcommand\extraspace{
	\vspace{0.25em}
}


\newcommand\ascii{\texttt{ASCII}}

\makeatother


\begin{document}

\chapter{Généralités}

\section{Introduction}

\LaTeX{} est un excellent langage, pour ne pas dire le meilleur, pour rédiger des documents contenant des formules mathématiques.
Malheureusement toute la puissance de \LaTeX{} permet d'écrire des codes très peu sémantiques.
Le modeste but du package \verb+tnsmath+ est de fournir quelques macros sémantiques
\footnote{
	En fait l'aspect sémantique est l'objectif commun de tous les packages de la suite \texttt{tns}.
}
pour la rédaction de documents mathématiques élémentaires. Considérons le code \LaTeX{} suivant.

\begin{latexex-alone}
Sachant que $f^\prime(x) = cos(x^2)$ sur $[a ; b]$ , nous avons :

$\displaystyle \int_a^b cos(x^2) dx = \left[ f(x) \right]_{x=a}^{x=b}$.
\end{latexex-alone}


Avec \verb+tnsmath+, vous pouvez écrire le code suivant.

\begin{latexex-alone}
Sachant que $\sder{f}(x) = cos(x^2)$ sur $\intervalC{a}{b}$, nous avons :

$\dintegrate*{a}{b}{cos(x^2)}{x} = \hook{a}{b}{f(x)}{x}$.
\end{latexex-alone}


Même si certaines commandes sont plus longues à écrire que ce que permet \LaTeX{}, il y a des avantages à utiliser des commandes sémantiques.
\begin{enumerate}
	\item La mise en forme du document devient consistante.

	\item Il est facile de changer une mise en forme sur l'ensemble d'un document ou localement via certaines options.

	\item \verb+tnsmath+ résout certains problèmes \og complexes \fg{} pour vous.
\end{enumerate}


% ---------------------- %


% tnscom used - START
\section{Beta-dépendance}

\verb#tnscom# qui est disponible sur \url{https://github.com/typensee-latex/tnscom.git} est un package utilisé en coulisse.
% tnscom used - END


\end{document}
