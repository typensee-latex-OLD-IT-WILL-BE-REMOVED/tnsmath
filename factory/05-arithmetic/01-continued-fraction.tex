\documentclass[12pt,a4paper]{article}

% == FOR DOC AND TESTS - START == %

\usepackage[utf8]{inputenc}
\usepackage{ucs}
\usepackage[top=2cm, bottom=2cm, left=1.5cm, right=1.5cm]{geometry}

\usepackage{color}
\usepackage{hyperref}
\hypersetup{
    colorlinks,
    citecolor=black,
    filecolor=black,
    linkcolor=black,
    urlcolor=black
}

\usepackage{enumitem}

\usepackage{amsthm}

\usepackage{tcolorbox}
\tcbuselibrary{listings}

\usepackage{pgffor}
\usepackage{xstring}


% MISC

\setlength{\parindent}{0cm}
\setlist{noitemsep}

\theoremstyle{definition}
\newtheorem*{remark}{Remark}


% Technical IDs

\newwrite\tempfile

\immediate\openout\tempfile=x-\jobname.macros-x.txt

\AtEndDocument{\immediate\closeout\tempfile}

\newcommand\IDconstant[1]{%
    \immediate\write\tempfile{constant@#1}%
}

\makeatletter
	\newcommand\IDmacro{\@ifstar{\@IDmacroStar}{\@IDmacroNoStar}}
	
    \newcommand\@IDmacroNoStar[3]{%
        \texttt{%
        	\textbackslash#1%
        	\IfStrEq{#2}{0}{}{%
        		\,\,[#2 Option%
				\IfStrEq{#2}{1}{}{s}]%
			}%
    	    \IfStrEq{#3}{}{}{%
	    		\,\,(#3 Argument%
				\IfStrEq{#3}{1}{}{s})%
			}
	   	}
        \immediate\write\tempfile{macro@#1@#2@#3}%
    }

    \newcommand\@IDmacroStar[2]{%
        \@IDmacroNoStar{#1}{0}{#2}%
    }

	\newcommand\@IDoptarg{\@ifstar{\@IDoptargStar}{\@IDoptargNoStar}}
	
	\newcommand\@IDoptargStar[2]{%
    	\vspace{0.5em}
		--- \texttt{#1%
			\IfStrEq{#2}{}{:}{\,#2:}%
		}%
	}

	\newcommand\@IDoptargNoStar[2]{%
    	\IfStrEq{#2}{}{%
			\@IDoptargStar{#1}{}%
		}{%
			\@IDoptargStar{#1}{\##2}%
		}%
	}

	\newcommand\IDkey[1]{%
    	\@IDoptarg*{Option}{{\itshape "#1"}}%
	}

	\newcommand\IDoption[1]{%
    	\@IDoptarg{Option}{#1}%
	}

	\newcommand\IDarg[1]{%
    	\@IDoptarg{Argument}{#1}%
	}
\makeatother

% == FOR DOC AND TESTS - END == %


% == PACKAGES USED == %

\usepackage{mathtools}
\usepackage{ifmtarg}


% == DEFINITIONS == %

% Sources :
%    * https://groups.google.com/forum/?hl=fr&fromgroups#!topic/fr.comp.text.tex/UrUZiurKwm0
%    * http://tex.stackexchange.com/questions/68190/continued-fraction-in-inline-equations/68196#68196
%    * http://tex.stackexchange.com/questions/23432/how-to-create-my-own-math-operator-with-limits

\makeatletter
% Operator
	\newcommand\contfracope{%
		\operatornamewithlimits{%
			\mathchoice{% * Display style
				\vcenter{\hbox{\huge $\mathcal{K}$}}%
			}{%           * Text style
				\vcenter{\hbox{\Large $\mathcal{K}$}}%
			}{%           * Script style
				\mathrm{\mathcal{K}}%
			}{%           * Script script style
				\mathrm{\mathcal{K}}%
			}
		}
	}

% Single continued fraction (useful ?)
	\newcommand\singlecontfrac[2]{%
		\begin{array}{@{}c@{}}%
			\multicolumn{1}{c|}{#1}%
			\\%
			\hline%
			\multicolumn{1}{|c}{#2}%
		\end{array}%
	}

% Standard continued fraction
	\newcommand\contfrac{\@ifstar{\@contfracStar}{\@contfracNoStar}}
	
	\newcommand\@contfracNoStar[1]{%
		\@contfracNoStarRecu@#1//\@nil%
	}
	\def\@contfracNoStarRecu@#1//#2\@nil{%
		\@ifmtarg{#2}{%
			#1%
		}{%
			#1 + \cfrac{1}{\@contfracNoStarRecu@#2\@nil}%
		}%
	}	

	\newcommand\@contfracStar[1]{%
		\@contfracStarBefore@#1//\@nil%
	}
	\def\@contfracStarBefore@#1//#2\@nil{%
    	\@ifmtarg{#2}{%
        	#1%
        }{%
        	#1 \@contfracStarRecu@#2\@nil%
        }%
    }
    \def\@contfracStarRecu@#1//#2\@nil{%
    	\@ifmtarg{#2}{%
        	+ \singlecontfrac{1}{#1}%
        }{%
        	+ \singlecontfrac{1}{#1} \@contfracStarRecu@#2\@nil%
        }%
    }

% Generalized continued fraction
	\newcommand\contfracgene{\@ifstar{\@contfracgeneStar}{\@contfracgeneNoStar}}
	
	\newcommand\@contfracgeneNoStar[1]{%
		\@contfracgeneNoStarRecu@#1////\@nil%
	}
	\def\@contfracgeneNoStarRecu@#1//#2//#3\@nil{%
		\@ifmtarg{#2}{%
			#1%
		}{%
			#1 + \cfrac{#2}{\@contfracgeneNoStarRecu@#3\@nil}%
		}%
	}
	

	\newcommand\@contfracgeneStar[1]{%
		\@contfracgeneStarBefore@#1//\@nil%
	}
	\def\@contfracgeneStarBefore@#1//#2\@nil{%
    	\@ifmtarg{#2}{%
        	#1%
        }{%
        	#1 \@contfracgeneStarRecu@#2\@nil%
        }%
    }	
    \def\@contfracgeneStarRecu@#1//#2//#3\@nil{%
        \@ifmtarg{#3}{%
            + \singlecontfrac{#1}{#2}%
        }{%
            + \singlecontfrac{#1}{#2} \@contfracgeneStarRecu@#3\@nil%
        }%
    }
\makeatother


\begin{document}

\section{Continued fractions}

	\subsection{Standard continued fractions}

        \subsubsection{Example of use}

\begin{tcblisting}{}
It is easy to write what is just after where the inline notation seems to have been 
introduced by Alfred Pringsheim (the left notation is always space consuming for a 
better readability).

$ \contfrac{u_0 // u_1 // u_2 // \dots // u_n}
= \contfrac*{u_0 // u_1 // u_2 // \dots // u_n}$
\end{tcblisting}


        \subsubsection{Technical IDs}

\IDmacro*{contfrac}{1}

\IDmacro*{contfrac*}{1}

\IDarg{} all the elements of the continued fraction separated by \verb+//+.




	\subsection{Generalized continued fractions}

        \subsubsection{Example of use}

\begin{tcblisting}{}
You can use similar notations for generalized continued fractions :

$\displaystyle \contfracgene{a // b // c // d // e // f // \dots // y // z}
             = \contfracgene*{a // b // c // d // e // f // \dots // y // z}$
\end{tcblisting}


        \subsubsection{Technical IDs}

\IDmacro*{contfracgene}{1}

\IDmacro*{contfracgene*}{1}

\IDarg{} all the elements of the generalized continued fraction separated by \verb+//+.



	\subsection{Single like continued fraction}

        \subsubsection{Example of use}

\begin{tcblisting}{}
Crazy men really (?) need to write things like $\singlecontfrac{a}{b}$.

% The existence of this macro just comes from its use internally.
\end{tcblisting}


        \subsubsection{Technical ID}

\IDmacro*{singlecontfrac}{2}

\IDarg{1} the pseudo numerator

\IDarg{2} the pseudo denominator



    \subsection{\texorpdfstring{The $\contfracope$ operator}%
                               {The K operator}}

        \subsubsection{Example of use \#1}

\IDconstant{contfracope}

\begin{tcblisting}{}
The following notation is very closed to the one used by Carl Friedrich Gauss:

$\displaystyle
  \contfracope_{k=1}^{n} (b_k:c_k)
= \cfrac{b_1}{\contfracgene{c_1 // b_2 // c_2 // b_3 // \dots // b_n // c_n}}$
\end{tcblisting}


\begin{remark}
	The letter $\contfracope$ comes from "kettenbruch" which means "continued
fraction" in german.
\end{remark}


        \subsubsection{Example of use \#2}

\begin{tcblisting}{}
$\displaystyle
  u_0 + \contfracope_{k=1}^{n} (1:u_k)
= \contfrac{u_0 // u_1 // u_2 // \dots // u_n}$
\end{tcblisting}

\end{document}
