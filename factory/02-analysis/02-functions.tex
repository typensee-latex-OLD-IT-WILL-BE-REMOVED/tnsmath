\documentclass[12pt,a4paper]{article}

% == FOR DOC AND TESTS - START == %

\usepackage[utf8]{inputenc}
\usepackage{ucs}
\usepackage[top=2cm, bottom=2cm, left=1.5cm, right=1.5cm]{geometry}

\usepackage{color}
\usepackage{hyperref}
\hypersetup{
    colorlinks,
    citecolor=black,
    filecolor=black,
    linkcolor=black,
    urlcolor=black
}

\usepackage{enumitem}

\usepackage{tcolorbox}
\tcbuselibrary{listings}

\usepackage{pgffor}
\usepackage{xstring}

\setlength{\parindent}{0cm}
\setlist{noitemsep}

% Technical IDs
\newwrite\tempfile

\immediate\openout\tempfile=x-\jobname.macros-x.txt

\AtEndDocument{\immediate\closeout\tempfile}

\newcommand\IDconstant[1]{%
    \immediate\write\tempfile{constant@#1}%
}

\makeatletter
	\newcommand\IDmacro{\@ifstar{\@IDmacroStar}{\@IDmacroNoStar}}
	
    \newcommand\@IDmacroNoStar[3]{%
        \texttt{%
        	\textbackslash#1%
        	\IfStrEq{#2}{0}{}{%
        		\,\,[#2 Option%
				\IfStrEq{#2}{1}{}{s}]%
			}%
    	    \,\,(#3 Argument%
				\IfStrEq{#3}{1}{}{s})%
	   	}
        \immediate\write\tempfile{macro@#1@#2@#3}%
    }

    \newcommand\@IDmacroStar[2]{%
        \@IDmacroNoStar{#1}{0}{#2}%
    }

	\newcommand\@IDoptarg[2]{%
    	\vspace{0.5em}
		--- \texttt{#1 \##2:}%
	}

	\newcommand\IDoption[1]{%
    	\@IDoptarg{Option}{#1}%
	}

	\newcommand\IDarg[1]{%
    	\@IDoptarg{Argument}{#1}%
	}
\makeatother

% == FOR DOC AND TESTS - END == %


% == PACKAGES USED == %

\usepackage{amsmath}


% == DEFINITIONS == %

% Source for DeclareMathMacro :
%    * http://forum.mathematex.net/latex-f6/bonnes-commandes-de-base-t12278.html

% One useful tool

\makeatletter
    \newcommand\DeclareMathMacro[2]{%
        \expandafter\let\csname original@\expandafter\@gobble\string#1\endcsname=#1
        \expandafter\def\csname\expandafter\@gobble\string#1\endcsname{%
            \relax%
            \ifmmode#2%
            \else%
                \csname original@\expandafter\@gobble\string#1\expandafter\endcsname%
            \fi%
        }%
    }
\makeatother

% Classical functions

    \DeclareMathMacro{\pgcd}{\operatorname{pgcd}}
    \DeclareMathMacro{\ppcm}{\operatorname{ppcm}}
    \DeclareMathMacro{\ch}{\operatorname{ch}}
    \DeclareMathMacro{\sh}{\operatorname{sh}}
    \DeclareMathMacro{\th}{\operatorname{th}}
    \DeclareMathMacro{\ach}{\operatorname{ach}}
    \DeclareMathMacro{\ash}{\operatorname{ash}}
    \DeclareMathMacro{\ath}{\operatorname{ath}}
    \DeclareMathMacro{\arccosh}{\operatorname{arccosh}}
    \DeclareMathMacro{\arcsinh}{\operatorname{arcsinh}}
    \DeclareMathMacro{\arctanh}{\operatorname{arctanh}}
    \DeclareMathMacro{\acos}{\operatorname{acos}}
    \DeclareMathMacro{\asin}{\operatorname{asin}}
    \DeclareMathMacro{\atan}{\operatorname{atan}}

    \newcommand\expb[1]{\exp_{#1}}
    \newcommand\logb[1]{\log_{#1}}



\begin{document}

\section{Special functions}

    \subsection{Some examples of use}

\begin{tcblisting}{}
Some additional special functions :
$\ch x \neq ch x$, $\ppcm(x;y)$, $\lg x =\logb{2} x$ and $\expb{6} y$.
\end{tcblisting}


    \subsection{Functions without parameter}

All the following macros don't have any parameter.

% List of functions without parameter

\foreach \k in {pgcd, ppcm, ch, sh, th, ach, ash, ath, arccosh, arcsinh, arctanh, acos, asin, atan}{\IDconstant{\k}}

\medskip

\begin{tabular*}{\textwidth}%
                {@{\extracolsep{\fill}}*{4}{l}}
    \verb+\pgcd+ & \verb+\ppcm+ & \verb+\ch+ & \verb+\sh+\\
    \verb+\th+ & \verb+\ach+ & \verb+\ash+ & \verb+\ath+\\
    \verb+\arccosh+ & \verb+\arcsinh+ & \verb+\arctanh+ & \verb+\acos+\\
    \verb+\asin+ & \verb+\atan+ &  & \\
\end{tabular*}

















    \subsection{Functions with parameters}

		\subsubsection{The complete list}

All the following macros have at least one parameter.

% List of functions with parameters

\medskip

\begin{tabular*}{\textwidth}%
                {@{\extracolsep{\fill}}*{4}{l}}
    \verb+\expb+ \, (1 parameter) & \verb+\logb+ \, (1 parameter) &  & \\
\end{tabular*}


		\subsubsection{Technical IDs}

\IDmacro*{expb}{1}

\IDmacro*{logb}{1}

\IDarg{1} the base of the exponential or the logarithm

\end{document}
