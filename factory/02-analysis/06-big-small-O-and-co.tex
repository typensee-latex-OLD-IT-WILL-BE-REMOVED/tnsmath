\documentclass[12pt,a4paper]{article}

% == FOR DOC AND TESTS - START == %

\usepackage[utf8]{inputenc}
\usepackage{ucs}
\usepackage[top=2cm, bottom=2cm, left=1.5cm, right=1.5cm]{geometry}

\usepackage{color}
\usepackage{hyperref}
\hypersetup{
    colorlinks,
    citecolor=black,
    filecolor=black,
    linkcolor=black,
    urlcolor=black
}

\usepackage{enumitem}

\usepackage{amsthm}

\usepackage{tcolorbox}
\tcbuselibrary{listings}

\usepackage{pgffor}
\usepackage{xstring}


% MISC

\setlength{\parindent}{0cm}
\setlist{noitemsep}

\theoremstyle{definition}
\newtheorem*{remark}{Remark}


% Technical IDs

\newwrite\tempfile

\immediate\openout\tempfile=x-\jobname.macros-x.txt

\AtEndDocument{\immediate\closeout\tempfile}

\newcommand\IDconstant[1]{%
    \immediate\write\tempfile{constant@#1}%
}

\makeatletter
	\newcommand\IDmacro{\@ifstar{\@IDmacroStar}{\@IDmacroNoStar}}
	
    \newcommand\@IDmacroNoStar[3]{%
        \texttt{%
        	\textbackslash#1%
        	\IfStrEq{#2}{0}{}{%
        		\,\,[#2 Option%
				\IfStrEq{#2}{1}{}{s}]%
			}%
    	    \,\,(#3 Argument%
				\IfStrEq{#3}{1}{}{s})%
	   	}
        \immediate\write\tempfile{macro@#1@#2@#3}%
    }

    \newcommand\@IDmacroStar[2]{%
        \@IDmacroNoStar{#1}{0}{#2}%
    }

	\newcommand\@IDoptarg[2]{%
    	\vspace{0.5em}
		--- \texttt{#1%
			\IfStrEq{#2}{}{:}{\,\##2:}%
		}%
	}

	\newcommand\IDoption[1]{%
    	\@IDoptarg{Option}{#1}%
	}

	\newcommand\IDarg[1]{%
    	\@IDoptarg{Argument}{#1}%
	}
\makeatother

% == FOR DOC AND TESTS - END == %


% == EXTRAS == %

\usepackage{amssymb}
\usepackage{textgreek}

% == PACKAGES USED == %

\usepackage{bm}
\usepackage{graphicx}
\usepackage{yhmath}


% == DEFINITIONS == %

% Sources :
%     1) http://forum.mathematex.net/latex-f6/bonnes-commandes-de-base-t12278.html
%     2) http://tex.stackexchange.com/questions/30944/mathcalo-and-font-size
%     3) https://tex.stackexchange.com/a/53091/6880

\makeatletter
    \newcommand\@bigtAsymptoricOpe[2]{%
        \ensuremath{%
            \if\relax\detokenize{#2}\relax
                #1%
            \else
                \mathop{}\mathopen{}#1\mathopen{}\left( #2 \right)%
            \fi
        }%
    }

    \newcommand\bigomega[1]{%
        \@bigtAsymptoricOpe{\bm{\Omega}}{#1}%
    }

    \newcommand\bigtheta[1]{%
        \@bigtAsymptoricOpe{\bm{\Theta}}{#1}%
    }

    \newcommand\bigO[1]{%
        \@bigtAsymptoricOpe{\mathcal{O}}{#1}%
    }

    \newcommand\smallO[1]{%
        \if\relax\detokenize{#1}\relax
            \mathchoice{% * Display style
                {\scriptstyle\mathcal{O}}%
            }{%           * Text style
                {\scriptstyle\mathcal{O}}%
            }{%           * Script style
                {\scriptscriptstyle\mathcal{O}}%
            }{%           * Script script style
                %\scalebox{0.8}{$\scriptscriptstyle\mathcal{O}$}%
            }
        \else
            \mathchoice{% * Display style
                \operatorname{\scriptstyle\mathcal{O}}\!\left(#1\right)%
            }{%           * Text style
                \operatorname{\scriptstyle\mathcal{O}}\!\left(#1\right)%
            }{%           * Script style
                \operatorname{\scriptscriptstyle\mathcal{O}}\left(#1\right)%
            }{%           * Script script style
                \operatorname{\scalebox{0.8}{$\scriptscriptstyle\mathcal{O}$}}\left(#1\right)%
            }
        \fi
    }
\makeatother



\begin{document}

\section{Asymptotic comparisons of sequences and functions}

    \subsection{\texorpdfstring{The $\bigO{}$ and $\smallO{}$ notations}%
                               {The "big O" and "small O" notations}}

        \subsubsection{Example of use}

\begin{tcblisting}{}
Let's see how to use the symbols $\bigO{}$ and $\smallO{}$ created by Landau.

\medskip

You can write $\bigO{x} \neq \smallO{x}$ and $e^{t + \smallO{t}} = e^{\bigO{t}}$.
\end{tcblisting}


        \subsubsection{Technical IDs}

\IDmacro*{bigO}{1}

\IDmacro*{smallO}{1}

\IDarg{} the content inside the braces after the symbol $\bigO{}$ or $\smallO{}$.



    \subsection{\texorpdfstring{The $\bigomega{}$ notation}%
                               {The "big Omega" notation}}

        \subsubsection{Example of use}

\begin{tcblisting}{}
Let's see how to use the symbol $\bigomega{}$ created by Hardy and Littlewood.

\medskip

$f(n) = \bigomega{g(n)}$ means: $\exists (m, n_0)$ such that
$n \geqslant n_0$ implies $f(n) \geqslant m g(n)$.
\end{tcblisting}


        \subsubsection{Technical IDs}

\IDmacro*{bigomega}{1}

\IDarg{} the content inside the braces after the symbol $\bigomega{}$.



    \subsection{\texorpdfstring{The $\bigtheta{}$ notation}%
                               {The "big Theta" notation}}

        \subsubsection{Example of use}

\begin{tcblisting}{}
Let's see how to use the symbol $\bigtheta{}$.

\medskip

$f(n) = \bigtheta{g(n)}$ means: $\exists (m, M, n_0)$ such that
$n \geqslant n_0$ implies $m g(n) \leqslant f(n) \leqslant M g(n)$.
\end{tcblisting}


        \subsubsection{Technical IDs}

\IDmacro*{bigtheta}{1}

\IDarg{} the content inside the braces after the symbol $\bigtheta{}$.

\end{document}
