\documentclass[12pt,a4paper]{article}

% == FOR DOC AND TESTS - START == %

\usepackage[utf8]{inputenc}
\usepackage{ucs}
\usepackage[top=2cm, bottom=2cm, left=1.5cm, right=1.5cm]{geometry}

\usepackage{color}
\usepackage{hyperref}
\hypersetup{
    colorlinks,
    citecolor=black,
    filecolor=black,
    linkcolor=black,
    urlcolor=black
}

\usepackage{enumitem}

\usepackage{amsthm}

\usepackage{tcolorbox}
\tcbuselibrary{listings}

\usepackage{pgffor}
\usepackage{xstring}


% MISC

\setlength{\parindent}{0cm}
\setlist{noitemsep}

\theoremstyle{definition}
\newtheorem*{remark}{Remark}


% Technical IDs

\newwrite\tempfile

\immediate\openout\tempfile=x-\jobname.macros-x.txt

\AtEndDocument{\immediate\closeout\tempfile}

\newcommand\IDconstant[1]{%
    \immediate\write\tempfile{constant@#1}%
}

\makeatletter
	\newcommand\IDmacro{\@ifstar{\@IDmacroStar}{\@IDmacroNoStar}}
	
    \newcommand\@IDmacroNoStar[3]{%
        \texttt{%
        	\textbackslash#1%
        	\IfStrEq{#2}{0}{}{%
        		\,\,[#2 Option%
				\IfStrEq{#2}{1}{}{s}]%
			}%
    	    \IfStrEq{#3}{}{}{%
	    		\,\,(#3 Argument%
				\IfStrEq{#3}{1}{}{s})%
			}
	   	}
        \immediate\write\tempfile{macro@#1@#2@#3}%
    }

    \newcommand\@IDmacroStar[2]{%
        \@IDmacroNoStar{#1}{0}{#2}%
    }

	\newcommand\@IDoptarg{\@ifstar{\@IDoptargStar}{\@IDoptargNoStar}}
	
	\newcommand\@IDoptargStar[2]{%
    	\vspace{0.5em}
		--- \texttt{#1%
			\IfStrEq{#2}{}{:}{\,#2:}%
		}%
	}

	\newcommand\@IDoptargNoStar[2]{%
    	\IfStrEq{#2}{}{%
			\@IDoptargStar{#1}{}%
		}{%
			\@IDoptargStar{#1}{\##2}%
		}%
	}

	\newcommand\IDkey[1]{%
    	\@IDoptarg*{Option}{{\itshape "#1"}}%
	}

	\newcommand\IDoption[1]{%
    	\@IDoptarg{Option}{#1}%
	}

	\newcommand\IDarg[1]{%
    	\@IDoptarg{Argument}{#1}%
	}
\makeatother

% == FOR DOC AND TESTS - END == %


% == PACKAGES USED == %

\usepackage{yhmath}
\usepackage{xstring}


% == DEFINITIONS == %

% Sources :
%    * http://forum.mathematex.net/latex-f6/en-tete-de-ds-t12933.html#p124908
%    * http://forum.mathematex.net/latex-f6/derivee-avec-un-d-droit-et-espace-t12932.html#p124930
%    * http://forum.mathematex.net/latex-f6/remplacer-des-espaces-par-autre-chose-t12952.html#p125062
%    * http://forum.mathematex.net/latex-f6/probleme-de-remplacement-de-cdots-t13047.html#p125782

\noexpandarg % This is necessary so as to '' \derfrac[3]{\cos}{x} ''  works.

\makeatletter
% dd, partial and pp usefull operators
    \newcommand{\@diffOpe}[3]{%
        #3%
        \IfStrEq{#1}{0}{}{^{#1}\!}%
        \IfBeginWith{#2}{f}{\!}{}%
        \hspace{0.07em}#2%
    }

    \DeclareRobustCommand\@dder{
        \mathop{}\mathopen{}\mathrm{d}
    }

    \newcommand\dd[2][0]{\@diffOpe{#1}{#2}{\@dder}}

    \let\original@partial\partial
    \renewcommand{\partial}{%
        \original@partial\mathopen{}%
    }

    \newcommand\pp[2][0]{\@diffOpe{#1}{#2}{\partial}}

% Power writing of total derivate
    \newcommand\derpow[2][\relax]{
        \IfStrEq{#1}{\relax}{
            #2^{\left( 1 \right)}
        }{
            #2^{\left( #1 \right)}
        }
    }

% Fractional writing of total derivate
	\newcommand\derfrac{\@ifstar{\@derfracStar}{\@derfracNoStar}}

    \newcommand\@derfracNoStar[3][\relax]{
        \IfStrEq{#1}{\relax}{
            \frac{\dd{#2}}{\dd{#3}}
        }{
            \frac{\dd[#1]{#2}}{\dd{#3}^{#1}}
        }
    }

    \newcommand\@derfracStar[3][\relax]{
    	\@derfracNoStar[#1]{}{#3} #2
    }

% Subscript writing of total derivate
    \newcommand\dersub[3][\relax]{
% The following command works because xstring traits {...} like a single character.
        \dd%
        \IfStrEq{#1}{\relax}{%
        	^{\vphantom{\prime}}%
        }{%
	        ^{#1}%
	    }%
	    _{#3} #2%
    }

% Subscript writing of partial derivate
    \newcommand\@bracketIt[1]{(#1)}

    \newcommand\partialsub[2]{
% The following command works because xstring traits {...} like a single character.
        \StrSubstitute{#2}{^}{\@bracketIt}[\@index]
        \partial%
        ^{\vphantom{\prime}}%
        _{\expandafter\StrSubstitute\expandafter{\@index}{//}{\,}}%
        #1%
    }

% Prime writing of partial derivate
    \newcommand\partialprime[2]{
% The following command works because xstring traits {...} like a single character.
        \StrSubstitute{#2}{^}{\@bracketIt}[\@index]%
        #1%
        ^{\prime}%
        _{\expandafter\StrSubstitute\expandafter{\@index}{//}{\,}}
    }

% Fractional writing of partial derivate
	\newcommand\partialfrac{\@ifstar{\@partialfracStar}{\@partialfracNoStar}}

    \newcommand\@partialfracNoStar[3][0]{%
        \frac{%
            \pp[#1]{#2}%
        }{%
            \StrSubstitute{\partial#3}{//}{\, \partial}[\@deno]%
            \expandafter\StrSubstitute\expandafter{\@deno}{\partial\cdots}{\,\cdots{}\,\partial}
        }
    }

    \newcommand\@partialfracStar[3][\relax]{
    	\@partialfracNoStar[#1]{}{#3} #2
    }
\makeatother


\begin{document}

\section{Differential calculus}

    \subsection{\texorpdfstring{The $\pp{}$ and $\dd{}$ operators}%
                               {The "rounded d" and "straight d" operators}}

        \subsubsection{Example of use}

\begin{tcblisting}{}
You can write $\dd{x}$ and $\pp{t}$ and also $ \dd[5]{x}$ or $\pp[n]{x}$.
\end{tcblisting}


        \subsubsection{Technical IDs}

\IDmacro{dd}{1}{1}

\IDmacro{pp}{1}{1}

\IDoption{} if used, this argument will be the exponent of the symbol $\pp{}$ or $\dd{}$.

\IDarg{} the variable of differentiation at the right of the symbol $\pp{}$ or $\dd{}$.



    \subsection{Total derivation}

        \subsubsection{Example of use \#1}

\begin{tcblisting}{}
$\displaystyle f'(a)
             = \derpow{f} (a) 
             = \derfrac{f}{x} (a)
             = \dersub{f}{x} (a)$
\end{tcblisting}


        \subsubsection{Example of use \#2}

\begin{tcblisting}{}
$\displaystyle f'''(a)
             = \derpow[3]{f} (a)
             = \derfrac[3]{f}{x} (a)
             = \dersub[3]{f}{x} (a)$
and
$\displaystyle \cos''' a = \derfrac[3]{\cos}{x} (a)$.
\end{tcblisting}


        \subsubsection{Example of use \#3}

\begin{tcblisting}{}
If $\displaystyle f(x) = \frac{1}{x^2+3}$, then we can write :
$\displaystyle \derpow[3]{f} (a)
             = \derfrac*[3]{\left( \frac{1}{x^2+3} \right)}{x} (a)$.
\end{tcblisting}


        \subsubsection{Technical IDs}

\IDmacro{derpow}{1}{1}

\IDoption{} if used, this argument will be the exponent of derivation put inside braces.

\IDarg{} the function to be differenciated.


\bigskip


\IDmacro{derfrac}{1}{2}

\IDmacro{derfrac*}{1}{2}

\IDmacro{dersub}{1}{2}

\IDoption{} if used, the exponent of derivation.

\IDarg{1} the function to be differenciated.

\IDarg{2} the variable used for the derivation.



    \subsection{Partial derivation}

        \subsubsection{Example of use \#1}

\begin{tcblisting}{}
$\displaystyle \partialfrac{f}{x} (a;b)
             = \partialsub{f}{x} (a;b)
             = \partialprime{f}{x} (a;b)$
\end{tcblisting}


        \subsubsection{Example of use \#2}

\begin{tcblisting}{}
$\displaystyle \partialfrac[3]{G}{f^2 // v} (a;b)
             = \partialfrac{G}{f^2 // v} (a;b)
             = \partialsub{G}{f^2 // v} (a;b)
             = \partialprime{G}{f^2 // v} (a;b)$
\end{tcblisting}


        \subsubsection{Example of use \#3}

\begin{tcblisting}{}
If $\displaystyle f(x;y) = \frac{cos(x y)}{x^2+y^2}$, then we can study
$\displaystyle \partialfrac[2]{f}{x // y}
= \partialfrac*[2]{\left( \frac{cos(x y)}{x^2 + y^2} \right)}{x // y}$.
\end{tcblisting}


        \subsubsection{Technical IDs}

\IDmacro{partialfrac}{1}{2}

\IDmacro{partialfrac*}{1}{2}

\IDoption{} if used, the exponent of $\pp$ associated to the function differenciated.

\IDarg{1} the function to be partially differenciated.

\IDarg{2} the variables used for the partial derivation. The syntax is particular : for example, \verb+x // y^3 // ...+ indicates regarding to the variables $x$ one time, $y$ three times... and so on.


\bigskip


\IDmacro*{partialsub}{2}

\IDmacro*{partialprime}{2}

\IDarg{1} the function to be partially differenciated.

\IDarg{2} the variables used for the partial derivation. The syntax is particular : for example, \verb+x // y^3 // ...+ indicates regarding to the variables $x$ one time, $y$ three times... and so on.

\end{document}
