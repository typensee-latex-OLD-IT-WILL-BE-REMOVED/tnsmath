\documentclass[12pt,a4paper]{article}

\makeatletter
	\usepackage[utf8]{inputenc}
\usepackage[T1]{fontenc}
\usepackage{ucs}

\usepackage[french]{babel,varioref}

\usepackage[top=2cm, bottom=2cm, left=1.5cm, right=1.5cm]{geometry}
\usepackage{enumitem}

\usepackage{multicol}

\usepackage{makecell}

\usepackage{color}
\usepackage{hyperref}
\hypersetup{
    colorlinks,
    citecolor=black,
    filecolor=black,
    linkcolor=black,
    urlcolor=black
}

\usepackage{amsthm}

\usepackage{tcolorbox}
\tcbuselibrary{listingsutf8}

\usepackage{ifplatform}

\usepackage{ifthen}

\usepackage{cbdevtool}


% MISC

\newtcblisting{latexex}{%
	sharp corners,%
	left=1mm, right=1mm,%
	bottom=1mm, top=1mm,%
	colupper=red!75!blue,%
	listing side text
}

\newtcblisting{latexex-flat}{%
	sharp corners,%
	left=1mm, right=1mm,%
	bottom=1mm, top=1mm,%
	colupper=red!75!blue,%
}

\newtcblisting{latexex-alone}{%
	sharp corners,%
	left=1mm, right=1mm,%
	bottom=1mm, top=1mm,%
	colupper=red!75!blue,%
	listing only
}


\newcommand\env[1]{\texttt{#1}}
\newcommand\macro[1]{\env{\textbackslash{}#1}}



\setlength{\parindent}{0cm}
\setlist{noitemsep}

\theoremstyle{definition}
\newtheorem*{remark}{Remarque}

\usepackage[raggedright]{titlesec}

\titleformat{\paragraph}[hang]{\normalfont\normalsize\bfseries}{\theparagraph}{1em}{}
\titlespacing*{\paragraph}{0pt}{3.25ex plus 1ex minus .2ex}{0.5em}


\newcommand\separation{
	\medskip
	\hfill\rule{0.5\textwidth}{0.75pt}\hfill
	\medskip
}


\newcommand\extraspace{
	\vspace{0.25em}
}


\newcommand\whyprefix[2]{%
	\textbf{\prefix{#1}}-#2%
}

\newcommand\mwhyprefix[2]{%
	\texttt{#1 = #1-#2}%
}

\newcommand\prefix[1]{%
	\texttt{#1}%
}


\newcommand\inenglish{\@ifstar{\@inenglish@star}{\@inenglish@no@star}}

\newcommand\@inenglish@star[1]{%
	\emph{\og #1 \fg}%
}

\newcommand\@inenglish@no@star[1]{%
	\@inenglish@star{#1} en anglais%
}


\newcommand\ascii{\texttt{ASCII}}


% Example
\newcounter{paraexample}[subsubsection]

\newcommand\@newexample@abstract[2]{%
	\paragraph{%
		#1%
		\if\relax\detokenize{#2}\relax\else {} -- #2\fi%
	}%
}



\newcommand\newparaexample{\@ifstar{\@newparaexample@star}{\@newparaexample@no@star}}

\newcommand\@newparaexample@no@star[1]{%
	\refstepcounter{paraexample}%
	\@newexample@abstract{Exemple \theparaexample}{#1}%
}

\newcommand\@newparaexample@star[1]{%
	\@newexample@abstract{Exemple}{#1}%
}


% Change log
\newcommand\topic{\@ifstar{\@topic@star}{\@topic@no@star}}

\newcommand\@topic@no@star[1]{%
	\textbf{\textsc{#1}.}%
}

\newcommand\@topic@star[1]{%
	\textbf{\textsc{#1} :}%
}







	\usepackage{07-big-small-O-and-co}
\makeatother


% == EXTRAS == %

\usepackage{amssymb}
\usepackage{textgreek}



\begin{document}

% \section{Analysis}

    \subsection{Comparaison asymptotique de suites et de fonctions}

        \subsubsection{\texorpdfstring{Les notations $\bigO{}$ et $\smallO{}$}%
                               {Les notations "grand O" et "petit O"}}

            \paragraph{Exemple d'utilisation 1}

\begin{tcblisting}{}
Vous pouvez utiliser les symboles $\bigO{}$ et $\smallO{}$ créés par Landau.
\end{tcblisting}


            \paragraph{Exemple d'utilisation 2}

\begin{tcblisting}{}
Vous pouvez écrire $\bigO{x} \neq \smallO{x}$ et $e^{t + \smallO{t}} = e^{\bigO{t}}$.
\end{tcblisting}


            \paragraph{Fiches techniques}

\IDmacro*{bigO}{1}

\IDmacro*{smallO}{1}

\IDarg{} non vide, cet argument sera mis entre des parenthèses après $\bigO{}$ ou $\smallO{}$.



        \subsubsection{\texorpdfstring{La notation $\bigomega{}$}%
                               {La notation "grand Omega"}}

            \paragraph{Exemple d'utilisation 1}

\begin{tcblisting}{}
Vous pouvez utiliser le symbole $\bigomega{}$ créé par Hardy et Littlewood.
\end{tcblisting}


            \paragraph{Exemple d'utilisation 2}

\begin{tcblisting}{}
$f(n) = \bigomega{g(n)}$ signifie :
$\exists (m, n_0)$ tel que $n \geqslant n_0$ implique $f(n) \geqslant m g(n)$.
\end{tcblisting}


            \paragraph{Fiche technique}

\IDmacro*{bigomega}{1}

\IDarg{} non vide, cet argument sera mis entre des parenthèses après $\bigomega{}$.



        \subsubsection{\texorpdfstring{La notation $\bigtheta{}$}%
                               {La notation "grand Theta"}}

            \paragraph{Exemple d'utilisation 1}

\begin{tcblisting}{}
Voici le dernier symbole $\bigtheta{}$ qui peut rendre service.
\end{tcblisting}


            \paragraph{Exemple d'utilisation 2}

\begin{tcblisting}{}
$f(n) = \bigtheta{g(n)}$ signifie : $\exists (m, M, n_0)$ tel que $n \geqslant n_0$
implique $m g(n) \leqslant f(n) \leqslant M g(n)$.
\end{tcblisting}


            \paragraph{Fiche technique}

\IDmacro*{bigtheta}{1}

\IDarg{} non vide, cet argument sera mis entre des parenthèses après $\bigtheta{}$.

\end{document}
