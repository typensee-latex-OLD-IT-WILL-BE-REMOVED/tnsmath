\documentclass[12pt,a4paper]{article}

% == FOR DOC AND TESTS - START == %

\usepackage[utf8]{inputenc}
\usepackage{ucs}
\usepackage[top=2cm, bottom=2cm, left=1.5cm, right=1.5cm]{geometry}

\usepackage{color}
\usepackage{hyperref}
\hypersetup{
    colorlinks,
    citecolor=black,
    filecolor=black,
    linkcolor=black,
    urlcolor=black
}

\usepackage{enumitem}

\usepackage{amsthm}

\usepackage{tcolorbox}
\tcbuselibrary{listings}

\usepackage{pgffor}
\usepackage{xstring}


% MISC

\setlength{\parindent}{0cm}
\setlist{noitemsep}

\theoremstyle{definition}
\newtheorem*{remark}{Remark}

\usepackage[raggedright]{titlesec}

\titleformat{\paragraph}[hang]{\normalfont\normalsize\bfseries}{\theparagraph}{1em}{}
\titlespacing*{\paragraph}{0pt}{3.25ex plus 1ex minus .2ex}{0.5em}


% Technical IDs

\newwrite\tempfile

\immediate\openout\tempfile=x-\jobname.macros-x.txt

\AtEndDocument{\immediate\closeout\tempfile}

\newcommand\IDconstant[1]{%
    \immediate\write\tempfile{constant@#1}%
}

\makeatletter
	\newcommand\IDmacro{\@ifstar{\@IDmacro@star}{\@IDmacro@no@star}}

    \newcommand\@IDmacro@no@star[3]{%
        \texttt{%
        	\textbackslash#1%
        	\IfStrEq{#2}{0}{}{%
        		\,\,[#2 Option%
				\IfStrEq{#2}{1}{}{s}]%
			}%
    	    \IfStrEq{#3}{}{}{%
	    		\,\,(#3 Argument%
				\IfStrEq{#3}{1}{}{s})%
			}
	   	}
        \immediate\write\tempfile{macro,#1,#2,#3}%
    }

    \newcommand\@IDmacro@star[2]{%
        \@IDmacro@no@star{#1}{0}{#2}%
    }

	\newcommand\@IDoptarg{\@ifstar{\@IDoptarg@star}{\@IDoptarg@no@star}}

	\newcommand\@IDoptarg@star[2]{%
    	\vspace{0.5em}
		--- \texttt{#1%
			\IfStrEq{#2}{}{:}{\,#2:}%
		}%
	}

	\newcommand\@IDoptarg@no@star[2]{%
    	\IfStrEq{#2}{}{%
			\@IDoptarg@star{#1}{}%
		}{%
			\@IDoptarg@star{#1}{\##2}%
		}%
	}

	\newcommand\IDkey[1]{%
    	\@IDoptarg*{Option}{{\itshape "#1"}}%
	}

	\newcommand\IDoption[1]{%
    	\@IDoptarg{Option}{#1}%
	}

	\newcommand\IDarg[1]{%
    	\@IDoptarg{Argument}{#1}%
	}
\makeatother

% == FOR DOC AND TESTS - END == %


% == EXTRAS == %

\usepackage{amssymb}
\usepackage{textgreek}

% == PACKAGES USED == %

\usepackage{bm}
\usepackage{graphicx}
\usepackage{yhmath}


% == DEFINITIONS == %

% Sources :
%     1) http://forum.mathematex.net/latex-f6/bonnes-commandes-de-base-t12278.html
%     2) http://tex.stackexchange.com/questions/30944/mathcalo-and-font-size
%     3) https://tex.stackexchange.com/a/53091/6880

\makeatletter
    \newcommand\@bigtAsymptoricOpe[2]{%
        \ensuremath{%
            \if\relax\detokenize{#2}\relax
                #1%
            \else
                \mathop{}\mathopen{}#1\mathopen{}\left( #2 \right)%
            \fi
        }%
    }

    \newcommand\bigomega[1]{%
        \@bigtAsymptoricOpe{\bm{\Omega}}{#1}%
    }

    \newcommand\bigtheta[1]{%
        \@bigtAsymptoricOpe{\bm{\Theta}}{#1}%
    }

    \newcommand\bigO[1]{%
        \@bigtAsymptoricOpe{\mathcal{O}}{#1}%
    }

    \newcommand\smallO[1]{%
        \if\relax\detokenize{#1}\relax
            \mathchoice{% * Display style
                {\scriptstyle\mathcal{O}}%
            }{%           * Text style
                {\scriptstyle\mathcal{O}}%
            }{%           * Script style
                {\scriptscriptstyle\mathcal{O}}%
            }{%           * Script script style
                %\scalebox{0.8}{$\scriptscriptstyle\mathcal{O}$}%
            }
        \else
            \mathchoice{% * Display style
                \operatorname{\scriptstyle\mathcal{O}}\!\left(#1\right)%
            }{%           * Text style
                \operatorname{\scriptstyle\mathcal{O}}\!\left(#1\right)%
            }{%           * Script style
                \operatorname{\scriptscriptstyle\mathcal{O}}\left(#1\right)%
            }{%           * Script script style
                \operatorname{\scalebox{0.8}{$\scriptscriptstyle\mathcal{O}$}}\left(#1\right)%
            }
        \fi
    }
\makeatother



\begin{document}

% \section{Analysis}

    \subsection{Asymptotic comparisons of sequences and functions}

        \subsubsection{\texorpdfstring{The $\bigO{}$ and $\smallO{}$ notations}%
                               {The "big O" and "small O" notations}}

            \paragraph{Example of use \#1}

\begin{tcblisting}{}
You can use the symbols $\bigO{}$ and $\smallO{}$ created by Landau.
\end{tcblisting}


            \paragraph{Example of use \#2}

\begin{tcblisting}{}
You can write $\bigO{x} \neq \smallO{x}$ and $e^{t + \smallO{t}} = e^{\bigO{t}}$.
\end{tcblisting}


            \paragraph{Technical IDs}

\IDmacro*{bigO}{1}

\IDmacro*{smallO}{1}

\IDarg{} the content inside the braces after the symbol $\bigO{}$ or $\smallO{}$.



        \subsubsection{\texorpdfstring{The $\bigomega{}$ notation}%
                               {The "big Omega" notation}}

            \paragraph{Example of use \#1}

\begin{tcblisting}{}
You can use the symbol $\bigomega{}$ created by Hardy and Littlewood.
\end{tcblisting}


            \paragraph{Example of use \#2}

\begin{tcblisting}{}
$f(n) = \bigomega{g(n)}$ means: $\exists (m, n_0)$ such that
$n \geqslant n_0$ implies $f(n) \geqslant m g(n)$.
\end{tcblisting}


            \paragraph{Technical ID}

\IDmacro*{bigomega}{1}

\IDarg{} the content inside the braces after the symbol $\bigomega{}$.



        \subsubsection{\texorpdfstring{The $\bigtheta{}$ notation}%
                               {The "big Theta" notation}}

            \paragraph{Example of use \#1}

\begin{tcblisting}{}
Here is the last symbol $\bigtheta{}$ that can be helpful.
\end{tcblisting}


            \paragraph{Example of use \#2}

\begin{tcblisting}{}
$f(n) = \bigtheta{g(n)}$ means: $\exists (m, M, n_0)$ such that $n \geqslant n_0$
implies $m g(n) \leqslant f(n) \leqslant M g(n)$.
\end{tcblisting}


            \paragraph{Technical ID}

\IDmacro*{bigtheta}{1}

\IDarg{} the content inside the braces after the symbol $\bigtheta{}$.

\end{document}
