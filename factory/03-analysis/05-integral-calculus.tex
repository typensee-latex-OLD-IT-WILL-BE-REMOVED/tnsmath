\documentclass[12pt,a4paper]{article}

% == FOR DOC AND TESTS - START == %

\usepackage[utf8]{inputenc}
\usepackage{ucs}
\usepackage[top=2cm, bottom=2cm, left=1.5cm, right=1.5cm]{geometry}

\usepackage{color}
\usepackage{hyperref}
\hypersetup{
    colorlinks,
    citecolor=black,
    filecolor=black,
    linkcolor=black,
    urlcolor=black
}

\usepackage{enumitem}

\usepackage{amsthm}

\usepackage{tcolorbox}
\tcbuselibrary{listings}

\usepackage{pgffor}
\usepackage{xstring}


% MISC

\setlength{\parindent}{0cm}
\setlist{noitemsep}

\theoremstyle{definition}
\newtheorem*{remark}{Remark}


% Technical IDs

\newwrite\tempfile

\immediate\openout\tempfile=x-\jobname.macros-x.txt

\AtEndDocument{\immediate\closeout\tempfile}

\newcommand\IDconstant[1]{%
    \immediate\write\tempfile{constant@#1}%
}

\makeatletter
	\newcommand\IDmacro{\@ifstar{\@IDmacroStar}{\@IDmacroNoStar}}
	
    \newcommand\@IDmacroNoStar[3]{%
        \texttt{%
        	\textbackslash#1%
        	\IfStrEq{#2}{0}{}{%
        		\,\,[#2 Option%
				\IfStrEq{#2}{1}{}{s}]%
			}%
    	    \IfStrEq{#3}{}{}{%
	    		\,\,(#3 Argument%
				\IfStrEq{#3}{1}{}{s})%
			}
	   	}
        \immediate\write\tempfile{macro@#1@#2@#3}%
    }

    \newcommand\@IDmacroStar[2]{%
        \@IDmacroNoStar{#1}{0}{#2}%
    }

	\newcommand\@IDoptarg{\@ifstar{\@IDoptargStar}{\@IDoptargNoStar}}
	
	\newcommand\@IDoptargStar[2]{%
    	\vspace{0.5em}
		--- \texttt{#1%
			\IfStrEq{#2}{}{:}{\,#2:}%
		}%
	}

	\newcommand\@IDoptargNoStar[2]{%
    	\IfStrEq{#2}{}{%
			\@IDoptargStar{#1}{}%
		}{%
			\@IDoptargStar{#1}{\##2}%
		}%
	}

	\newcommand\IDkey[1]{%
    	\@IDoptarg*{Option}{{\itshape "#1"}}%
	}

	\newcommand\IDoption[1]{%
    	\@IDoptarg{Option}{#1}%
	}

	\newcommand\IDarg[1]{%
    	\@IDoptarg{Argument}{#1}%
	}
\makeatother

% == FOR DOC AND TESTS - END == %


% == EXTRAS == %

\newcommand\dd[1]{d#1} % This operator is defined properly in "differential.tex"


% == PACKAGES USED == %

\usepackage{amsmath}
\usepackage{relsize}

% == DEFINITIONS == %

% Source for minimizing spaces between consecutive integrals :
%    * http://forum.mathematex.net/latex-f6/integrale-triple-t13312.html#p128016

% Consecutive integrals

\makeatletter
    \let\original@int\int
    \DeclareRobustCommand{\int}{%
        \original@int\operator@followup{\@ifnextchar{\int}{\!\!}{}}%
    }

    \def\operator@followup#1{%
        \@ifnextchar{^}{\operator@followup@sup{#1}}%
            {\operator@followup@nosup{#1}}%
    }

    \def\operator@followup@sup#1^#2{%
        \@ifnextchar{_}{\operator@followup@sup@sub{#1}^{#2}}%
            {\operator@followup@sup@nosub{#1}^{#2}}%
    }

    \def\operator@followup@sup@sub#1^#2_#3{%
        \operator@followup@{#1}{#2}{#3}%
    }

    \def\operator@followup@sup@nosub#1^#2{%
        \operator@followup@{#1}{#2}{}%
    }

    \def\operator@followup@nosup#1{%
        \@ifnextchar{_}{\operator@followup@nosup@sub{#1}}
            {\operator@followup@nosup@nosub{#1}}%
    }

    \def\operator@followup@nosup@sub#1_#2{%
        \@ifnextchar{^}{\operator@followup@nosup@sub@sup{#1}_{#2}}
            {\operator@followup@nosup@sub@nosup{#1}_{#2}}%
    }

    \def\operator@followup@nosup@sub@sup#1_#2^#3{%
        \operator@followup@{#1}{#3}{#2}%
    }

    \def\operator@followup@nosup@sub@nosup#1_#2{%
        \operator@followup@{#1}{}{#2}%
    }

    \def\operator@followup@nosup@nosub#1{%
        \operator@followup@{#1}{}{}%
    }

    \def\operator@followup@#1#2#3{%
        ^{#2}_{#3}#1%
    }

% The hook calculator
    \newcommand\hook{\@ifstar{\@hookStar}{\@hookNoStar}}
    \newcommand\@hookNoStar[3]{\ensuremath{ \left[ \vphantom{\relsize{1.25}{\text{$\displaystyle F_1^2$}}} \right. \!\! #1 \left. \vphantom{\relsize{1.25}{\text{$\displaystyle F_1^2$}}} \!\! \right]_{#2}^{#3} }}
    \newcommand\@hookStar[3]{\left[ \vphantom{\text{\relsize{1.1}$#1$}} #1 \right]_{#2}^{#3}}

    \newcommand\vhook{\@ifstar{\@vhookStar}{\@vhookNoStar}}
    \newcommand\@vhookNoStar[3]{#1 {\text{\relsize{1.5}$\rvert$}}_{#2}^{#3}}
    \newcommand\@vhookStar[3]{\left. #1 \right\rvert_{#2}^{#3}}

\makeatother



\begin{document}

\section{Integral calculus}

    \subsection{The hook operator - 1st version}

        \subsubsection{Example of use \#1}

\begin{tcblisting}{}
By definition, $\displaystyle \int_{a}^{b} f(x) \dd{x} = \hook{F(x)}{a}{b}$ where
$\hook{F(x)}{a}{b} = F(b) - F(a)$.
\end{tcblisting}


        \subsubsection{Example of use \#2}

\begin{tcblisting}{}
With the star version, the hooks stretch vertically like in
$\displaystyle \hook*{\frac{x - 1}{5 + x^2}}{a}{b} 
             = \hook{\frac{x - 1}{5 + x^2}}{a}{b}$.
\end{tcblisting}


        \subsubsection{Technical IDs}

\IDmacro*{hook}{3}

\IDmacro*{hook*}{3}

\IDarg{1} the content inside the hooks.

\IDarg{2} the lower bound displayed as an index.

\IDarg{3} the upper bound displayed as an exponent.



    \subsection{The hook operator - 2nd version}

        \subsubsection{Example of use \#1}

\begin{tcblisting}{}
You can use $\vhook{F(x)}{a}{b}$ instead of $\hook{F(x)}{a}{b}$.
\end{tcblisting}


        \subsubsection{Example of use \#2}

\begin{tcblisting}{}
With the star version, the left rule stretches vertically like in
$\displaystyle \vhook*{\frac{x - 1}{5 + x^2}}{a}{b} 
             = \vhook{\frac{x - 1}{5 + x^2}}{a}{b}$.
\end{tcblisting}


        \subsubsection{Technical IDs}

\IDmacro*{vhook}{3}

\IDmacro*{vhook*}{3}

\IDarg{1} the content before the vertical line $\vert$ .

\IDarg{2} the lower bound displayed as an index.

\IDarg{3} the upper bound displayed as an exponent.



    \subsection{Several integrals}

The package minimizes spacings between consecutive symbols of integration. Here is an example.

\begin{tcblisting}{}
$\displaystyle \int \int \int F(x;y;z) \dd{x} \dd{y} \dd{z}
= \int_{a}^{b} \int_{c}^{d} \int_{e}^{f} F(x;y;z) \dd{x} \dd{y} \dd{z}$
\end{tcblisting}


\begin{remark}
	By default, \LaTeX{} prints
	\makeatletter
    	$\displaystyle \original@int \original@int \original@int F(x;y;z) \dd{x} \dd{y} \dd{z}
    	= \original@int_{a}^{b} \original@int_{c}^{d} \original@int_{e}^{f} F(x;y;z) \dd{x} \dd{y} \dd{z}$.
	\makeatother
\end{remark}

\end{document}
