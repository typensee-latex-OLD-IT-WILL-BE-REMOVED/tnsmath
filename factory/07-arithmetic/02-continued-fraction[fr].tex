\documentclass[12pt,a4paper]{article}

\makeatletter
    \usepackage[utf8]{inputenc}
\usepackage[T1]{fontenc}
\usepackage{ucs}

\usepackage[french]{babel,varioref}

\usepackage[top=2cm, bottom=2cm, left=1.5cm, right=1.5cm]{geometry}
\usepackage{enumitem}

\usepackage{multicol}

\usepackage{makecell}

\usepackage{color}
\usepackage{hyperref}
\hypersetup{
    colorlinks,
    citecolor=black,
    filecolor=black,
    linkcolor=black,
    urlcolor=black
}

\usepackage{amsthm}

\usepackage{tcolorbox}
\tcbuselibrary{listingsutf8}

\usepackage{ifplatform}

\usepackage{ifthen}

\usepackage{cbdevtool}


% MISC

\newtcblisting{latexex}{%
	sharp corners,%
	left=1mm, right=1mm,%
	bottom=1mm, top=1mm,%
	colupper=red!75!blue,%
	listing side text
}

\newtcblisting{latexex-flat}{%
	sharp corners,%
	left=1mm, right=1mm,%
	bottom=1mm, top=1mm,%
	colupper=red!75!blue,%
}

\newtcblisting{latexex-alone}{%
	sharp corners,%
	left=1mm, right=1mm,%
	bottom=1mm, top=1mm,%
	colupper=red!75!blue,%
	listing only
}


\newcommand\env[1]{\texttt{#1}}
\newcommand\macro[1]{\env{\textbackslash{}#1}}



\setlength{\parindent}{0cm}
\setlist{noitemsep}

\theoremstyle{definition}
\newtheorem*{remark}{Remarque}

\usepackage[raggedright]{titlesec}

\titleformat{\paragraph}[hang]{\normalfont\normalsize\bfseries}{\theparagraph}{1em}{}
\titlespacing*{\paragraph}{0pt}{3.25ex plus 1ex minus .2ex}{0.5em}


\newcommand\separation{
	\medskip
	\hfill\rule{0.5\textwidth}{0.75pt}\hfill
	\medskip
}


\newcommand\extraspace{
	\vspace{0.25em}
}


\newcommand\whyprefix[2]{%
	\textbf{\prefix{#1}}-#2%
}

\newcommand\mwhyprefix[2]{%
	\texttt{#1 = #1-#2}%
}

\newcommand\prefix[1]{%
	\texttt{#1}%
}


\newcommand\inenglish{\@ifstar{\@inenglish@star}{\@inenglish@no@star}}

\newcommand\@inenglish@star[1]{%
	\emph{\og #1 \fg}%
}

\newcommand\@inenglish@no@star[1]{%
	\@inenglish@star{#1} en anglais%
}


\newcommand\ascii{\texttt{ASCII}}


% Example
\newcounter{paraexample}[subsubsection]

\newcommand\@newexample@abstract[2]{%
	\paragraph{%
		#1%
		\if\relax\detokenize{#2}\relax\else {} -- #2\fi%
	}%
}



\newcommand\newparaexample{\@ifstar{\@newparaexample@star}{\@newparaexample@no@star}}

\newcommand\@newparaexample@no@star[1]{%
	\refstepcounter{paraexample}%
	\@newexample@abstract{Exemple \theparaexample}{#1}%
}

\newcommand\@newparaexample@star[1]{%
	\@newexample@abstract{Exemple}{#1}%
}


% Change log
\newcommand\topic{\@ifstar{\@topic@star}{\@topic@no@star}}

\newcommand\@topic@no@star[1]{%
	\textbf{\textsc{#1}.}%
}

\newcommand\@topic@star[1]{%
	\textbf{\textsc{#1} :}%
}







    \usepackage{02-continued-fraction}
\makeatother


\begin{document}

%\subsection{Arithmétiques}

\subsection{Fractions continuées}

\subsubsection{Fractions continuées standard}

\paragraph{Exemple d'utilisation}

Dans l'exemple suivant, la notation en ligne semble être due à Alfred Pringsheim. La notation à gauche utilise toujours le maximum d'espace pour améliorer la lisibilité.

\begin{latexex-flat}
 $\contfrac {u_0 | u_1 | u_2 | \dots | u_n}
= \contfrac*{u_0 | u_1 | u_2 | \dots | u_n}$
\end{latexex-flat}


% ---------------------- %


\subsubsection{Fiches techniques}

\paragraph{Fractions continuées standard}

\IDmacro*{contfrac}{1}

\IDmacro*{contfrac*}{1}

\IDarg{} tous les éléments de la fraction continuée séparés par des \verb+|+.


% ---------------------- %


\subsubsection{Fractions continuées généralisées}

\paragraph{Exemple d'utilisation}

Voici comment écrire une fraction continuée généralisée.

\begin{latexex-flat}
 $\displaystyle
  \contfracgene {a | b | c | d | e | f | \dots | y | z}
= \contfracgene*{a | b | c | d | e | f | \dots | y | z}$
\end{latexex-flat}


% ---------------------- %


\subsubsection{Fiches techniques}

\paragraph{Fractions continuées généralisées}

\IDmacro*{contfracgene}{1}

\IDmacro*{contfracgene*}{1}

\IDarg{} tous les éléments de la fraction continuée généralisée séparés par des \verb+|+.


% ---------------------- %


\subsubsection{Comme une fraction continuée isolée}

\paragraph{Exemple d'utilisation}

La raison d'être de la macro ci-dessous vient juste de son usage en interne.

\begin{latexex}
$\singlecontfrac{a}{b}$
pour les fous\dots :-)
\end{latexex}


% ---------------------- %


\subsubsection{Fiches techniques}

\paragraph{Comme une fraction continuée isolée}

\IDmacro*{singlecontfrac}{2}

\IDarg{1} le pseudo numérateur.

\IDarg{2} le pseudo dénominateur.


% ---------------------- %


\subsubsection{\texorpdfstring{L'opérateur $\contfracope$}%
                               {L'opérateur K}}

\paragraph{Exemple d'utilisation 1}

La notation suivante est proche de celle qu'utilisait Carl Friedrich Gauss.

\begin{latexex-flat}
 $\displaystyle
  \contfracope_{k=1}^{n} (b_k:c_k)
= \cfrac{b_1}%
        {\contfracgene{c_1 | b_2 | c_2 | b_3 | \dots | b_n | c_n}}$
\end{latexex-flat}


\begin{remark}
    La lettre $\contfracope $vient de "kettenbruch" qui signifie "fraction continuée" en allemand.
\end{remark}


% ---------------------- %


\paragraph{Exemple d'utilisation 2}

\begin{latexex-flat}
 $\displaystyle
  u_0 + \contfracope_{k=1}^{n} (1:u_k)
= \contfrac{u_0 | u_1 | u_2 | \dots | u_n}$
\end{latexex-flat}


% ---------------------- %


\subsubsection{Fiches techniques}

\paragraph{\texorpdfstring{L'opérateur $\contfracope$}%
                          {L'opérateur K}}

La macro suivante sans argument a un comportement spécifique vis à vis des mises en index et en exposant. 


\separation


\IDmacro*{contfracope}{0}

\end{document}
