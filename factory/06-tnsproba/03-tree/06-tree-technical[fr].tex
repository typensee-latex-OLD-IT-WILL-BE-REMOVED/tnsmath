\documentclass[12pt,a4paper]{article}

\makeatletter
    \usepackage[utf8]{inputenc}
\usepackage[T1]{fontenc}
\usepackage{ucs}

\usepackage[french]{babel,varioref}

\usepackage[top=2cm, bottom=2cm, left=1.5cm, right=1.5cm]{geometry}
\usepackage{enumitem}

\usepackage{multicol}

\usepackage{makecell}

\usepackage{color}
\usepackage{hyperref}
\hypersetup{
    colorlinks,
    citecolor=black,
    filecolor=black,
    linkcolor=black,
    urlcolor=black
}

\usepackage{amsthm}

\usepackage{tcolorbox}
\tcbuselibrary{listingsutf8}

\usepackage{ifplatform}

\usepackage{ifthen}

\usepackage{cbdevtool}


% MISC

\newtcblisting{latexex}{%
	sharp corners,%
	left=1mm, right=1mm,%
	bottom=1mm, top=1mm,%
	colupper=red!75!blue,%
	listing side text
}

\newtcblisting{latexex-flat}{%
	sharp corners,%
	left=1mm, right=1mm,%
	bottom=1mm, top=1mm,%
	colupper=red!75!blue,%
}

\newtcblisting{latexex-alone}{%
	sharp corners,%
	left=1mm, right=1mm,%
	bottom=1mm, top=1mm,%
	colupper=red!75!blue,%
	listing only
}


\newcommand\env[1]{\texttt{#1}}
\newcommand\macro[1]{\env{\textbackslash{}#1}}



\setlength{\parindent}{0cm}
\setlist{noitemsep}

\theoremstyle{definition}
\newtheorem*{remark}{Remarque}

\usepackage[raggedright]{titlesec}

\titleformat{\paragraph}[hang]{\normalfont\normalsize\bfseries}{\theparagraph}{1em}{}
\titlespacing*{\paragraph}{0pt}{3.25ex plus 1ex minus .2ex}{0.5em}


\newcommand\separation{
	\medskip
	\hfill\rule{0.5\textwidth}{0.75pt}\hfill
	\medskip
}


\newcommand\extraspace{
	\vspace{0.25em}
}


\newcommand\whyprefix[2]{%
	\textbf{\prefix{#1}}-#2%
}

\newcommand\mwhyprefix[2]{%
	\texttt{#1 = #1-#2}%
}

\newcommand\prefix[1]{%
	\texttt{#1}%
}


\newcommand\inenglish{\@ifstar{\@inenglish@star}{\@inenglish@no@star}}

\newcommand\@inenglish@star[1]{%
	\emph{\og #1 \fg}%
}

\newcommand\@inenglish@no@star[1]{%
	\@inenglish@star{#1} en anglais%
}


\newcommand\ascii{\texttt{ASCII}}


% Example
\newcounter{paraexample}[subsubsection]

\newcommand\@newexample@abstract[2]{%
	\paragraph{%
		#1%
		\if\relax\detokenize{#2}\relax\else {} -- #2\fi%
	}%
}



\newcommand\newparaexample{\@ifstar{\@newparaexample@star}{\@newparaexample@no@star}}

\newcommand\@newparaexample@no@star[1]{%
	\refstepcounter{paraexample}%
	\@newexample@abstract{Exemple \theparaexample}{#1}%
}

\newcommand\@newparaexample@star[1]{%
	\@newexample@abstract{Exemple}{#1}%
}


% Change log
\newcommand\topic{\@ifstar{\@topic@star}{\@topic@no@star}}

\newcommand\@topic@no@star[1]{%
	\textbf{\textsc{#1}.}%
}

\newcommand\@topic@star[1]{%
	\textbf{\textsc{#1} :}%
}






\makeatother


% == EXTRAS == %


\begin{document}

%\section{Arbres pondérés}

\section{Fiches techniques}

\IDenv[n]{probatree}

\IDenv[n]{probatree*}

\IDenv[n]{probatree**}

\Content{} un arbre codé en utilisant la syntaxe supportée par le package \verb#forest#.

\extraspace

\IDkey{pweight}  pour écrire un poids sur le milieu d'une branche.

\IDkey{apweight} pour écrire un poids au-dessus le milieu d'une branche.

\IDkey{bpweight} pour écrire un poids en-dessous du milieu d'une branche.

\extraspace

\IDkey{pweight*} pour indiquer un poids sans l'imprimer.
Dans l'environnement \env{probatree**}, le poids sera affiché comme si on avait utilisé \verb#pweight#.

\IDkey{apweight*} pour indiquer un poids sans l'imprimer.
Dans l'environnement \env{probatree**}, le poids sera affiché comme si on avait utilisé \verb#apweight#.

\IDkey{bpweight*} pour indiquer un poids sans l'imprimer.
Dans l'environnement \env{probatree**}, le poids sera affiché comme si on avait utilisé \verb#bpweight#.

\extraspace

\IDkey{pframe} pour encadrer un sous-arbre depuis un noeud vers toutes les feuilles de celui-ci.


\separation


\IDmacro{ptreeFrame}{1}{3} \hfill \mwhyprefix{p}{robabilty}

\IDoption{} un système de type \texttt{clé=valeur}.

\begin{enumerate}
	\item \verb#col# : une couleur au format TikZ. La valeur par défaut est \verb#blue#.
\end{enumerate}


\IDargs{1..3} les noms de la sous-racine (à gauche), du noeud final en haut (à droite) et du noeud final en bas (à droite) tous indiqués via \verb#name = ...# \emph{(en fait l'ordre n'est pas important ici)}.


\separation


\IDmacro{aptreeFrame}{1}{3} \hfill \mwhyprefix{a}{auto}

\extraspace
\extraspace

Voir les indications précédentes excepté qu'ici on utilise le système de nommage automatisé dérivé de celui de \verb#forest#.


\separation


\IDmacro{ptreeComment}{1}{2}

\IDoption{} un système de type \texttt{clé=valeur}.

\begin{enumerate}
	\item \verb#col# : une couleur au format TikZ. La valeur par défaut est \verb#black#.

	\item \verb#dx# : une distance horizontale relative de décalage. La valeur par défaut est \verb#0cm#.

	\item \verb#dy# : une distance verticale relative de décalage. La valeur par défaut est \verb#0cm#.
\end{enumerate}

\IDarg{1} le nom de la feuille.

\IDarg{2} le texte du commentaire.


\separation


\IDmacro{aptreeComment}{1}{2}

\extraspace
\extraspace

Voir les indications précédentes excepté qu'ici on utilise le système de nommage automatisé dérivé de celui de \verb#forest#.


\separation


\IDmacro{ptreeFocus  }{1}{1}

\IDmacro{ptreeFocus* }{1}{1}

\IDmacro{ptreeFocus**}{1}{1}

\IDoption{} un système de type \texttt{clé=valeur}.

\begin{enumerate}
	\item \verb#col# : une couleur au format TikZ. La valeur par défaut est \verb#blue#.
\end{enumerate}

\IDarg{} les noms des noeuds indiqués via \verb#name = ...# à fournir dans le bon ordre et à séparer par des barres verticales \verb#|#.


\separation


\IDmacro{aptreeFocus  }{1}{1}

\IDmacro{aptreeFocus* }{1}{1}

\IDmacro{aptreeFocus**}{1}{1}

\extraspace
\extraspace

Voir les indications précédentes excepté qu'ici on utilise le système de nommage automatisé dérivé de celui de \verb#forest#.

\end{document}
