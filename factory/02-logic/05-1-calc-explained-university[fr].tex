\documentclass[12pt,a4paper]{article}

\makeatletter
    \usepackage[utf8]{inputenc}
\usepackage[T1]{fontenc}
\usepackage{ucs}

\usepackage[french]{babel,varioref}

\usepackage[top=2cm, bottom=2cm, left=1.5cm, right=1.5cm]{geometry}
\usepackage{enumitem}

\usepackage{multicol}

\usepackage{makecell}

\usepackage{color}
\usepackage{hyperref}
\hypersetup{
    colorlinks,
    citecolor=black,
    filecolor=black,
    linkcolor=black,
    urlcolor=black
}

\usepackage{amsthm}

\usepackage{tcolorbox}
\tcbuselibrary{listingsutf8}

\usepackage{ifplatform}

\usepackage{ifthen}

\usepackage{cbdevtool}


% MISC

\newtcblisting{latexex}{%
	sharp corners,%
	left=1mm, right=1mm,%
	bottom=1mm, top=1mm,%
	colupper=red!75!blue,%
	listing side text
}

\newtcblisting{latexex-flat}{%
	sharp corners,%
	left=1mm, right=1mm,%
	bottom=1mm, top=1mm,%
	colupper=red!75!blue,%
}

\newtcblisting{latexex-alone}{%
	sharp corners,%
	left=1mm, right=1mm,%
	bottom=1mm, top=1mm,%
	colupper=red!75!blue,%
	listing only
}


\newcommand\env[1]{\texttt{#1}}
\newcommand\macro[1]{\env{\textbackslash{}#1}}



\setlength{\parindent}{0cm}
\setlist{noitemsep}

\theoremstyle{definition}
\newtheorem*{remark}{Remarque}

\usepackage[raggedright]{titlesec}

\titleformat{\paragraph}[hang]{\normalfont\normalsize\bfseries}{\theparagraph}{1em}{}
\titlespacing*{\paragraph}{0pt}{3.25ex plus 1ex minus .2ex}{0.5em}


\newcommand\separation{
	\medskip
	\hfill\rule{0.5\textwidth}{0.75pt}\hfill
	\medskip
}


\newcommand\extraspace{
	\vspace{0.25em}
}


\newcommand\whyprefix[2]{%
	\textbf{\prefix{#1}}-#2%
}

\newcommand\mwhyprefix[2]{%
	\texttt{#1 = #1-#2}%
}

\newcommand\prefix[1]{%
	\texttt{#1}%
}


\newcommand\inenglish{\@ifstar{\@inenglish@star}{\@inenglish@no@star}}

\newcommand\@inenglish@star[1]{%
	\emph{\og #1 \fg}%
}

\newcommand\@inenglish@no@star[1]{%
	\@inenglish@star{#1} en anglais%
}


\newcommand\ascii{\texttt{ASCII}}


% Example
\newcounter{paraexample}[subsubsection]

\newcommand\@newexample@abstract[2]{%
	\paragraph{%
		#1%
		\if\relax\detokenize{#2}\relax\else {} -- #2\fi%
	}%
}



\newcommand\newparaexample{\@ifstar{\@newparaexample@star}{\@newparaexample@no@star}}

\newcommand\@newparaexample@no@star[1]{%
	\refstepcounter{paraexample}%
	\@newexample@abstract{Exemple \theparaexample}{#1}%
}

\newcommand\@newparaexample@star[1]{%
	\@newexample@abstract{Exemple}{#1}%
}


% Change log
\newcommand\topic{\@ifstar{\@topic@star}{\@topic@no@star}}

\newcommand\@topic@no@star[1]{%
	\textbf{\textsc{#1}.}%
}

\newcommand\@topic@star[1]{%
	\textbf{\textsc{#1} :}%
}






    % == PACKAGES USED == %

\RequirePackage{amsmath}
\RequirePackage{relsize}
\RequirePackage{xparse}


% == DEFINITIONS == %

% Settable texts
\@ifpackagewith{babel}{french}{
    \newcommand\lymathsep{;}
    \newcommand\lymathsubsep{,}

    \newcommand\textopchoice{choix}
    \newcommand\textopcond{cond}
    \newcommand\textopdef{déf}
    \newcommand\textophyp{hyp}
    \newcommand\textopid{id}
    \newcommand\textoptest{?}
}{
    \newcommand\lymathsep{,}
    \newcommand\lymathsubsep{;}

    \newcommand\textopchoice{choice}
    \newcommand\textopcond{cond}
    \newcommand\textopdef{def}
    \newcommand\textophyp{hyp}
    \newcommand\textopid{id}
    \newcommand\textoptest{?}
}


\newcommand\textexplainleft{\{}
\newcommand\textexplainright{\}}
\newcommand\textexplainspacein{2em}


% Tools - Apply same macro to all arguments

% #1        : main macro
% #2        : macro to apply to arguments
% #3 and #4 : the two arguments
\newcommand\@apply@macro@two@args[4]{%
    #1{#2{#3}}{#2{#4}}%
}


% Tools - Deco over a math symbol

\newcommand\@over@math@symbol[2]{%
	\mathrel{\overset{\mathrm{\text{\raisebox{.5ex}{#1}}}}{#2}}%
}


% Tools - Intervals

\newcommand\@extra@phantom{%
    \vphantom{\relsize{1.25}{\text{$\displaystyle F_1^2$}}}%
}

\newcommand\@interval@tool@star[5]{%
    \ensuremath{ \left#1 \@extra@phantom \right. \!\! #2 #3 #4 \left. \@extra@phantom \!\! \right#5}%
}

\newcommand\@interval@tool@no@star[5]{\ensuremath{ \left#1 #2 #3 #4 \right#5}}


% Tools - Multi-arguments
%
% Source : the following lines come directly for the following post
%
%    * https://tex.stackexchange.com/a/475291/6880

\ExplSyntaxOn
% General purpose macro for defining other macros
    \NewDocumentCommand{\makemultiargument}{mmmmmo}{
        \lymath_multiarg:nnnnnn{#1}{#2}{#3}{#4}{#5}{#6}
    }
 
% Allocate a private variable
    \seq_new:N \l__lymath_generic_seq

% The internal version of the general purpose macro
    \cs_new_protected:Nn \lymath_multiarg:nnnnnn{
        % #1 = separator
        % #2 = multiargument
        % #3 = code before
          % #4 = code between
          % #5 = code after
          % #6 = ornament to items

        % A group allows nesting
        \group_begin:
         % Split the multiargument into parts
        \seq_set_split:Nnn \l__lymath_generic_seq { #1 } { #2 }
        % Apply the ornament to the items
          \tl_if_novalue:nF { #6 }{
            \seq_set_eq:NN \l__lymath_temp_seq \l__lymath_generic_seq
            \seq_set_map:NNn \l__lymath_generic_seq \l__lymath_generic_seq { #6 }
           }
        % Execute the <code before>
          #3
        % Deliver the items, with the chosen material between them
          \seq_use:Nn \l__lymath_generic_seq { #4 }
          % Execute the <code after>
         #5
          % End the group started at the beginning
          \group_end:
    }    
\ExplSyntaxOff


    \usepackage{05-calc-explained}
\makeatother


\newcommand\RRp{RR+}
\newcommand\vimplies{\Downarrow}
\newcommand\viff{\Updownarrow}


\begin{document}

%\section{Logique et fondements}

\subsection{Détailler un raisonnement simple}

\subsubsection{Version pour le lycée et après} \label{explain-proof}

\newparaexample{Avec les réglages par défaut}

L'environnement \env{explain} permet de détailler les étapes principales d'un calcul ou d'un raisonnement simple en s'appuyant sur la macro \macro{explnext} dont le nom vient de \emph{\og \whyprefix{expl}{ain} \prefix{next} step \fg} soit \inenglish{expliquer la prochaine étape}
\footnote{
    Cet environnement utilise aussi le package \texttt{witharrows} qui est très sympathique pour expliquer des étapes de calcul.
}.
On dispose aussi de \macro{explnext*} pour des explications descendantes et/ou montantes 
\footnote{
    Les explications données ne doivent pas être trop longues car ce serait contre-productif.
}.

\medskip


Ci-dessous se trouve un exemple, très farfelu vers la fin, où l'on utilise les réglages par défaut.
Notons au passage que ce type de présentation n'est sûrement pas bien adaptée à un jeune public pour lequel une 2\ieme{} façon de détailler des calculs et/ou un raisonnement simple est proposée plus bas dans la section \ref{explain-proof-for-youngs}.

\begin{latexex-flat}
\begin{explain}
    (a + b)^2
        \explnext{On utilise $x^2 = x \cdot x$.}
    (a + b) (a + b)
        \explnext*{Double développement depuis la parenthèse gauche.}%
                  {Double factorisation pas facile.}
    a^2 + a b + b a + b^2
        \explnext*{}%
                  {Commutativité du produit.}
    a^2 + 2 a b + b^2
        \explnext*{Commutativité de l'addition.}%
                  {}
    a^2 + b^2 + 2 a b
\end{explain}
\end{latexex-flat}


\begin{remark}
    Il faut savoir que la mise en forme est celle d'une formule ce qui peut rendre service comme dans l'exemple suivant.

\begin{latexex}
Un calcul avec un placement pouvant être 
utile :
\begin{explain}
    (a + b)^2
        \explnext{Identité remarquable.}
    a^2 + b^2 + 2 a b
\end{explain}
\end{latexex}

Avec un retour à la ligne, il faudra donc si besoin gérer l'espacement vertical.

\begin{latexex}
Mon calcul pas trop proche.

\medskip
\begin{explain}
    (a + b)^2
        \explnext{Identité remarquable.}
    a^2 + b^2 + 2 a b
\end{explain}
\end{latexex}
\end{remark}


\begin{remark}
    Voici des petites choses à connaître sur les macros \macro{explnext} et \macro{explnext*}.
    \begin{enumerate}
        \item \macro{expltxt} est utilisée par \macro{explnext} pour mettre en forme le texte d'explication.

        \item \macro{expltxtup} et \macro{expltxtdown} sont utilisées par \macro{explnext*} décorer les textes d'explication juste avant leur mise en forme finale via \macro{expltxtupdown}.

        \item \macro{explnext} et \macro{explnext*} utilisent la macro constante \macro{expltxtspacein} pour l'espacement entre le symbole et la courte explication. Par défaut, cette macro vaut \verb+2em+.
    \end{enumerate}
 
\end{remark}


% ---------------------- %


\newparaexample{Utiliser un autre symbole globalement}

L'environnement \env{explain} possède plusieurs options dont l'une est \verb+ope+ qui vaut \verb+{=}+ par défaut. Ceci permet de faire ce qui suit sans effort.

\begin{latexex}
\begin{explain}[ope = \viff]
    x^2 + 10 x + 25 = 0
        \explnext{Identité remarquable.}
    (x + 5)^2 = 0
        \explnext{$P^2 = 0$ si et 
                  seulement si $P = 0$.}
    x = -5
\end{explain}
\end{latexex}


% ---------------------- %


\newparaexample{Juste utiliser des symboles}

Si l'argument obligatoire de la macro \macro{explnext} est vide alors seul le symbole est affiché \emph{(ne pas oublier les accolades vides)}. Voici un court exemple de ceci.

\begin{latexex}
\begin{explain}[ope = \viff]
    a^2 = b^2
        \explnext{}
    a = \pm b
\end{explain}
\end{latexex}


% ---------------------- %


\newparaexample{Utiliser un autre symbole localement}

La macro \macro{explnext} possède un argument optionnel qui utilise par défaut celui de l'environment. En utilisant cette option, on choisit alors localement le symbole à employer. Voici un exemple d'utilisation complètement farfelu bien que correct.

\begin{latexex}
\begin{explain}[ope = \viff]
    0 \leq a < b
        \explnext[\vimplies]%
                 {Croissance de $x^2$ 
                  sur $\RRp$.}
    a^2 < b^2
        \explnext{}
    a^2 - b^2 < 0
        \explnext{Identité remarquable.}
    (a - b)(a + b) < 0
        \explnext[\vimplies]{}
    a \neq b
\end{explain}
\end{latexex}


% ---------------------- %


\newparaexample{Choisir la mise en forme des explications}

Pour la mise en forme des explications à double sens, la macro \macro{explnext} fait appel à la macro \macro{expltxt}.
Par défaut, le package utilise la définition suivante.

\begin{latexex-alone}
\newcommand\expltxt[1]{%
    \text{\color{blue}\footnotesize \{\,{\itshape #1}\,\} }%
}
\end{latexex-alone}


Pour la mise en forme des explications à sens unique, la macro \macro{explnext*} fait appel aux macros \macro{expltxtup}, \macro{expltxtdown} et \macro{expltxtupdown}.
Par défaut le package utilise les définitions suivantes.

\begin{latexex-alone}
\newcommand\expltxtup[1]{%
    $\uparrow$ #1 $\uparrow$%
}

\newcommand\expltxtdown[1]{%
    $\downarrow$ #1 $\downarrow$%
}

\newcommand\expltxtupdown[2]{{%
    \displaystyle\footnotesize\color{blue}%
    \left\{\,%
        \genfrac{}{}{0pt}{}{%
            \text{\itshape\expltxtdown{\samesizeas{#1}{#2}}}%
        }{%
            \text{\itshape\expltxtup{\samesizeas{#2}{#1}}}%
        }%
    \,\right\}%
}}
\end{latexex-alone}


Nous allons expliquer comment obtenir l'affreux exemple ci-dessous montrant que l'on peut adapter si besoin la mise en forme \emph{(ne pas oublier de passer en mode texte via \macro{text})}.

% ==================== %

\bgroup

\newcommand\myexpltxt[2]{%
    \text{\color{#1} \footnotesize \itshape \bfseries #2}%
}

\renewcommand\expltxt[1]{%
    \myexpltxt{gray}{$\Downarrow$ #1 $\Uparrow$}%
}

\renewcommand\expltxtup[1]{%
    \myexpltxt{orange}{$\Uparrow$ #1 $\Uparrow$}%
}

\renewcommand\expltxtdown[1]{%
    \myexpltxt{red}{$\Downarrow$ #1 $\Downarrow$}%
}

\renewcommand\expltxtupdown[2]{%
    \displaystyle\color{blue!20!black!30!green}\genfrac{\langle}{\rangle}{1pt}{}{%
        \expltxtdown{\samesizeas{#1}{#2}}%
    }{%
        \expltxtup{\samesizeas{#2}{#1}}%
    }%
}


\begin{latexex}
\begin{explain}
    (a + b) (a + b)
        \explnext{Se souvenir de $P\cdot P = P^2$.}%
    (a + b)^2
        \explnext*{Id. Rm. - Dév.}%
                  {Id. Rm. - Facto.}
    a^2 + 2 a b + b^2
\end{explain}
\end{latexex}

\egroup


La mise en forme a été obtenue en utilisant le code \LaTeX{} suivant où la macro \macro{samesizeas\{\#1\}\{\#2\}} rend le texte \verb+#1+ aussi large que \verb+#2+ en ajoutant des espaces supplémentaires tout en centrant le résultat final si besoin.

\begin{latexex-alone}
\newcommand\myexpltxt[2]{%
    \text{\color{#1} \footnotesize \itshape \bfseries #2}%
}

\renewcommand\expltxt[1]{%
    \myexpltxt{gray}{$\Downarrow$ #1 $\Uparrow$}%
}

\renewcommand\expltxtup[1]{%
    \myexpltxt{orange}{$\Uparrow$ #1 $\Uparrow$}%
}

\renewcommand\expltxtdown[1]{%
    \myexpltxt{red}{$\Downarrow$ #1 $\Downarrow$}%
}

\renewcommand\expltxtupdown[2]{%
    \displaystyle\color{blue!20!black!30!green}%
    \genfrac{\langle}{\rangle}{1pt}{}{%
        \expltxtdown{\samesizeas{#1}{#2}}%
    }{%
        \expltxtup{\samesizeas{#2}{#1}}%
    }%
}
\end{latexex-alone}


% ==================== %


\end{document}
