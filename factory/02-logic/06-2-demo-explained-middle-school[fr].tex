\documentclass[12pt,a4paper]{article}

\makeatletter
    \usepackage[utf8]{inputenc}
\usepackage[T1]{fontenc}
\usepackage{ucs}

\usepackage[french]{babel,varioref}

\usepackage[top=2cm, bottom=2cm, left=1.5cm, right=1.5cm]{geometry}
\usepackage{enumitem}

\usepackage{multicol}

\usepackage{color}
\usepackage{hyperref}
\hypersetup{
    colorlinks,
    citecolor=black,
    filecolor=black,
    linkcolor=black,
    urlcolor=black
}

\usepackage{amsthm}

\usepackage{tcolorbox}
\tcbuselibrary{listingsutf8}

\usepackage{ifplatform}

\usepackage{ifthen}

\usepackage{cbdevtool}


% MISC

\newtcblisting{latexex}{%
	sharp corners,%
	left=1mm, right=1mm,%
	bottom=1mm, top=1mm,%
	colupper=red!75!blue,%
	listing side text
}

\newtcblisting{latexex-flat}{%
	sharp corners,%
	left=1mm, right=1mm,%
	bottom=1mm, top=1mm,%
	colupper=red!75!blue,%
}

\newtcblisting{latexex-alone}{%
	sharp corners,%
	left=1mm, right=1mm,%
	bottom=1mm, top=1mm,%
	colupper=red!75!blue,%
	listing only
}


\newcommand\env[1]{\texttt{#1}}
\newcommand\macro[1]{\env{\textbackslash{}#1}}



\setlength{\parindent}{0cm}
\setlist{noitemsep}

\theoremstyle{definition}
\newtheorem*{remark}{Remarque}

\usepackage[raggedright]{titlesec}

\titleformat{\paragraph}[hang]{\normalfont\normalsize\bfseries}{\theparagraph}{1em}{}
\titlespacing*{\paragraph}{0pt}{3.25ex plus 1ex minus .2ex}{0.5em}


\newcommand\separation{
	\medskip
	\hfill\rule{0.5\textwidth}{0.75pt}\hfill
	\medskip
}


\newcommand\extraspace{
	\vspace{0.25em}
}


\newcommand\ascii{\texttt{ASCII}}

    % == PACKAGES USED == %

\RequirePackage{amsmath}
\RequirePackage{relsize}
\RequirePackage{xparse}


% == DEFINITIONS == %

% Settable texts
\@ifpackagewith{babel}{french}{
    \newcommand\lymathsep{;}
    \newcommand\lymathsubsep{,}
    \newcommand\textopchoice{choix}
    \newcommand\textopcond{cond}
    \newcommand\textopdef{déf}
    \newcommand\textophyp{hyp}
    \newcommand\textopid{id}
}{
	\newcommand\lymathsep{,}
	\newcommand\lymathsubsep{;}
    \newcommand\textopchoice{choice}
    \newcommand\textopcond{cond}
    \newcommand\textopdef{def}
    \newcommand\textophyp{hyp}
    \newcommand\textopid{id}
}


\newcommand\textexplainleft{\{}
\newcommand\textexplainright{\}}
\newcommand\textexplainspacebefore{\qquad}
\newcommand\textexplainspacein{\qquad}


% Tools - Apply same macro to all arguments

% #1        : main macro
% #2        : macro to apply to arguments
% #3 and #4 : the two arguments
\newcommand\@apply@macro@two@args[4]{%
	#1{#2{#3}}{#2{#4}}%
}




% Tools - Intervals

\newcommand\@extra@phantom{%
	\vphantom{\relsize{1.25}{\text{$\displaystyle F_1^2$}}}%
}

\newcommand\@interval@tool@star[5]{%
	\ensuremath{ \left#1 \@extra@phantom \right. \!\! #2 #3 #4 \left. \@extra@phantom \!\! \right#5}%
}

\newcommand\@interval@tool@no@star[5]{\ensuremath{ \left#1 #2 #3 #4 \right#5}}


% Tools - Multi-arguments
%
% Source : the following lines come directly for the following post
%
%    * https://tex.stackexchange.com/a/475291/6880

\ExplSyntaxOn
% General purpose macro for defining other macros
	\NewDocumentCommand{\makemultiargument}{mmmmmo}{
		\lymath_multiarg:nnnnnn{#1}{#2}{#3}{#4}{#5}{#6}
	}
 
% Allocate a private variable
	\seq_new:N \l__lymath_generic_seq

% The internal version of the general purpose macro
	\cs_new_protected:Nn \lymath_multiarg:nnnnnn{
		% #1 = separator
		% #2 = multiargument
		% #3 = code before
	  	% #4 = code between
	  	% #5 = code after
	  	% #6 = ornament to items

		% A group allows nesting
		\group_begin:
	 	% Split the multiargument into parts
		\seq_set_split:Nnn \l__lymath_generic_seq { #1 } { #2 }
		% Apply the ornament to the items
	  	\tl_if_novalue:nF { #6 }{
	    	\seq_set_eq:NN \l__lymath_temp_seq \l__lymath_generic_seq
	    	\seq_set_map:NNn \l__lymath_generic_seq \l__lymath_generic_seq { #6 }
	   	}
		% Execute the <code before>
	  	#3
		% Deliver the items, with the chosen material between them
	  	\seq_use:Nn \l__lymath_generic_seq { #4 }
  		% Execute the <code after>
	 	#5
  		% End the group started at the beginning
	  	\group_end:
	}	
\ExplSyntaxOff


    \usepackage{06-demo-explained}
\makeatother


\newcommand\anglein[1]{#1}

\begin{document}

%\section{Logique et fondements}

%\subsection{Détailler un \og vrai \fg{} raisonnement}

\subsubsection{Un tableau pour le collège et le lycée} \label{explain-hard-proof-for-youngs}


\newparaexample{Avec les réglages par défaut}

L'environnement étoilé \env{demoexplain*} est différent de l'environnement \env{demoexplain} puisqu'il sert à indiquer trois choses et non juste deux comme le montre l'exemple suivant
\footnote{
	C'est pour cela qu'est proposé une version étoilée de l'environnement et non l'utilisation d'une option de l'environnement non étoilé. 
}.
Par contre, la syntaxe est très similaire.
Notez au passage la possibilité d'utiliser \macro{newline} pour forcer un retour à la ligne dans une cellule.

\begin{latexex-flat}
\begin{demoexplain*}
    \demostep
        $ABC$ est un triangle
        \newline équilatéral 
      & Dans un triangle équilatéral, les trois angles mesurent $60$\textdegree. 
      & $\anglein{ABC} = 60$\textdegree     
    \demostep
        Voir la conséquence \explref*{1} .
      & Simple calcul avec conversion en radians.
      & $\dfrac{1}{3} \anglein{ABC} = \dfrac{\pi}{9}$
\end{demoexplain*}
\end{latexex-flat}


% ---------------------- %


\newparaexample{Avec toutes les options}

Le système de référence marche ici aussi.
Par contre \env{demoexplain*} ne propose que \verb+start+ comme clé optionnelle avec le même fonctionnement que pour \env{demoexplain}.

\begin{latexex-flat}
\begin{demoexplain*}[start = last]
    \demostep[demo-first-geo-fact]
        $ABC$ est un triangle \newline équilatéral 
      & Dans un triangle équilatéral, les trois angles mesurent $60$\textdegree. 
      & $\anglein{ABC} = 60$\textdegree     
    \demostep
        Voir la conséquence \explref{demo-first-geo-fact} .
      & Simple calcul avec conversion en radians.
      & $\dfrac{1}{3} \anglein{ABC} = \dfrac{\pi}{9}$
\end{demoexplain*}
\end{latexex-flat}


% ---------------------- %


\subsubsection{Un tableau sur plusieurs pages}

Un tableau devant utiliser plusieurs pages sera scindé comme ci-dessous sans perte d'information
\footnote{
	Tout le travail est fait par l'environnement \env{longtable} du package éponyme.
}.

\begin{center}
	\frame{\includegraphics[scale = .5]{demo-explained-middleschool-broken[fr].png}}
\end{center}

\end{document}
