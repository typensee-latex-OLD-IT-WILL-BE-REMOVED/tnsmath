\documentclass[12pt,a4paper]{article}

\makeatletter
	\usepackage[utf8]{inputenc}
\usepackage[T1]{fontenc}
\usepackage{ucs}

\usepackage[french]{babel,varioref}

\usepackage[top=2cm, bottom=2cm, left=1.5cm, right=1.5cm]{geometry}
\usepackage{enumitem}

\usepackage{multicol}

\usepackage{makecell}

\usepackage{color}
\usepackage{hyperref}
\hypersetup{
    colorlinks,
    citecolor=black,
    filecolor=black,
    linkcolor=black,
    urlcolor=black
}

\usepackage{amsthm}

\usepackage{tcolorbox}
\tcbuselibrary{listingsutf8}

\usepackage{ifplatform}

\usepackage{ifthen}

\usepackage{cbdevtool}


% MISC

\newtcblisting{latexex}{%
	sharp corners,%
	left=1mm, right=1mm,%
	bottom=1mm, top=1mm,%
	colupper=red!75!blue,%
	listing side text
}

\newtcblisting{latexex-flat}{%
	sharp corners,%
	left=1mm, right=1mm,%
	bottom=1mm, top=1mm,%
	colupper=red!75!blue,%
}

\newtcblisting{latexex-alone}{%
	sharp corners,%
	left=1mm, right=1mm,%
	bottom=1mm, top=1mm,%
	colupper=red!75!blue,%
	listing only
}


\newcommand\env[1]{\texttt{#1}}
\newcommand\macro[1]{\env{\textbackslash{}#1}}



\setlength{\parindent}{0cm}
\setlist{noitemsep}

\theoremstyle{definition}
\newtheorem*{remark}{Remarque}

\usepackage[raggedright]{titlesec}

\titleformat{\paragraph}[hang]{\normalfont\normalsize\bfseries}{\theparagraph}{1em}{}
\titlespacing*{\paragraph}{0pt}{3.25ex plus 1ex minus .2ex}{0.5em}


\newcommand\separation{
	\medskip
	\hfill\rule{0.5\textwidth}{0.75pt}\hfill
	\medskip
}


\newcommand\extraspace{
	\vspace{0.25em}
}


\newcommand\whyprefix[2]{%
	\textbf{\prefix{#1}}-#2%
}

\newcommand\mwhyprefix[2]{%
	\texttt{#1 = #1-#2}%
}

\newcommand\prefix[1]{%
	\texttt{#1}%
}


\newcommand\inenglish{\@ifstar{\@inenglish@star}{\@inenglish@no@star}}

\newcommand\@inenglish@star[1]{%
	\emph{\og #1 \fg}%
}

\newcommand\@inenglish@no@star[1]{%
	\@inenglish@star{#1} en anglais%
}


\newcommand\ascii{\texttt{ASCII}}


% Example
\newcounter{paraexample}[subsubsection]

\newcommand\@newexample@abstract[2]{%
	\paragraph{%
		#1%
		\if\relax\detokenize{#2}\relax\else {} -- #2\fi%
	}%
}



\newcommand\newparaexample{\@ifstar{\@newparaexample@star}{\@newparaexample@no@star}}

\newcommand\@newparaexample@no@star[1]{%
	\refstepcounter{paraexample}%
	\@newexample@abstract{Exemple \theparaexample}{#1}%
}

\newcommand\@newparaexample@star[1]{%
	\@newexample@abstract{Exemple}{#1}%
}


% Change log
\newcommand\topic{\@ifstar{\@topic@star}{\@topic@no@star}}

\newcommand\@topic@no@star[1]{%
	\textbf{\textsc{#1}.}%
}

\newcommand\@topic@star[1]{%
	\textbf{\textsc{#1} :}%
}






	% == PACKAGES USED == %

\RequirePackage{amsmath}
\RequirePackage{relsize}
\RequirePackage{xparse}


% == DEFINITIONS == %

% Settable texts
\@ifpackagewith{babel}{french}{
    \newcommand\lymathsep{;}
    \newcommand\lymathsubsep{,}

    \newcommand\textopchoice{choix}
    \newcommand\textopcond{cond}
    \newcommand\textopdef{déf}
    \newcommand\textophyp{hyp}
    \newcommand\textopid{id}
    \newcommand\textoptest{?}
}{
    \newcommand\lymathsep{,}
    \newcommand\lymathsubsep{;}

    \newcommand\textopchoice{choice}
    \newcommand\textopcond{cond}
    \newcommand\textopdef{def}
    \newcommand\textophyp{hyp}
    \newcommand\textopid{id}
    \newcommand\textoptest{?}
}


\newcommand\textexplainleft{\{}
\newcommand\textexplainright{\}}
\newcommand\textexplainspacein{2em}


% Tools - Apply same macro to all arguments

% #1        : main macro
% #2        : macro to apply to arguments
% #3 and #4 : the two arguments
\newcommand\@apply@macro@two@args[4]{%
    #1{#2{#3}}{#2{#4}}%
}


% Tools - Deco over a math symbol

\newcommand\@over@math@symbol[2]{%
	\mathrel{\overset{\mathrm{\text{\raisebox{.5ex}{#1}}}}{#2}}%
}


% Tools - Intervals

\newcommand\@extra@phantom{%
    \vphantom{\relsize{1.25}{\text{$\displaystyle F_1^2$}}}%
}

\newcommand\@interval@tool@star[5]{%
    \ensuremath{ \left#1 \@extra@phantom \right. \!\! #2 #3 #4 \left. \@extra@phantom \!\! \right#5}%
}

\newcommand\@interval@tool@no@star[5]{\ensuremath{ \left#1 #2 #3 #4 \right#5}}


% Tools - Multi-arguments
%
% Source : the following lines come directly for the following post
%
%    * https://tex.stackexchange.com/a/475291/6880

\ExplSyntaxOn
% General purpose macro for defining other macros
    \NewDocumentCommand{\makemultiargument}{mmmmmo}{
        \lymath_multiarg:nnnnnn{#1}{#2}{#3}{#4}{#5}{#6}
    }
 
% Allocate a private variable
    \seq_new:N \l__lymath_generic_seq

% The internal version of the general purpose macro
    \cs_new_protected:Nn \lymath_multiarg:nnnnnn{
        % #1 = separator
        % #2 = multiargument
        % #3 = code before
          % #4 = code between
          % #5 = code after
          % #6 = ornament to items

        % A group allows nesting
        \group_begin:
         % Split the multiargument into parts
        \seq_set_split:Nnn \l__lymath_generic_seq { #1 } { #2 }
        % Apply the ornament to the items
          \tl_if_novalue:nF { #6 }{
            \seq_set_eq:NN \l__lymath_temp_seq \l__lymath_generic_seq
            \seq_set_map:NNn \l__lymath_generic_seq \l__lymath_generic_seq { #6 }
           }
        % Execute the <code before>
          #3
        % Deliver the items, with the chosen material between them
          \seq_use:Nn \l__lymath_generic_seq { #4 }
          % Execute the <code after>
         #5
          % End the group started at the beginning
          \group_end:
    }    
\ExplSyntaxOff


	\usepackage{01-equal-signs-n-co}
\makeatother
\usepackage{amssymb}
\newcommand\RR{RR}
\newcommand\setgeo[1]{#1}


\begin{document}


\section{Logique et fondements}

\subsection{Différents types d'égalités \og standard \fg}

D'un point de vue pédagogique, il peut être intéressant de disposer de différentes façon d'écrire une égalité, une non égalité ou une inégalité.
Bien entendu on tord les règles de typographie avec ce type de pratique mais c'est pour le bien de la communauté.


\subsubsection{Définir quelque chose}

L'exemple suivant montre trois façons de rédiger une égalité signifiant une définition
\footnote{
	Le symbole peu courant $\eqdef**$ est utilisé par le langage B qui permet de spécifier et prouver certains programmes.
}
\emph{(la section \ref{text-for-opes} explique comment est définit le texte \emph{\og \textopdef \fg})}.

\begin{latexex}
$f(x) \eqdef x^3 + 1$ ou
$f(x) \eqdef* x^3 + 1$
\end{latexex}


% ---------------------- %


\subsubsection{Indiquer une identité}

L'exemple suivant montre deux façons de rédiger des identités, la noptation symbolique n'étant pas standard \emph{(la section \ref{text-for-opes} explique comment est défini le texte \emph{\og \textopid \fg})}.

\begin{latexex}
$(a + b)^2 \eqid a^2 + b^2 + 2 a b$ ou
$(a + b)^2 \eqid* a^2 + b^2 + 2 a b$ .
\end{latexex}


% ---------------------- %


\subsubsection{Une égalité à vérifier ou non, une hypothèse, une condition}

Se reporter à la section \ref{text-for-opes} pour savoir comment sont définis les textes \emph{\og \textopcond \fg} et \emph{\og \textophyp \fg}.

\begin{latexex}
$(a + b)^3 \eqtest a^3 + b^3 + 3 a b$

$(a + b)^3 \neqid a^3 + b^3 + 3 a b$

$x \neqhyp 0$ ou $x \neqcond 0$

$x \eqcond 0$ ou $x \eqhyp 0$ ?
\end{latexex}


% ---------------------- %


\subsubsection{Une égalité indiquant le choix d'une valeur}

La section \ref{text-for-opes} permet de savoir comment le texte \emph{\og \textopchoice \fg} est défini.

\begin{latexex}
$x \geqcond 4$ implique
$x^2 \geqcons 16$.

Alors $x \eqchoice 123$ donne
$123^2 \geqappli 16$.
\end{latexex}


% ---------------------- %


\subsubsection{Une égalité indiquant l'équation d'une courbe que l'on utilise}

La section \ref{text-for-opes} permet de savoir comment les textes \emph{\og \textopplot \fg} et \emph{\og \textopappli \fg} sont définis \emph{(la macro \emph{\texttt{\textbackslash{}setgeo}} est définie dans la section \ref{set-geo})}.

\begin{latexex}
$M \in \setgeo{C}: y \eqplot x^2 + 3$
donne
$y_M \eqappli x_M^2 + 3$.
\end{latexex}


% ---------------------- %


\subsubsection{Différents types d'inéquations}

Le principe reste le même pour les symboles d'équations excepté qu'il n'y a ici aucune écriture purement symbolique. Voici un code \og fourre-tout \fg{} montrant quelques exemples.

\begin{latexex}
$x \leqtest x^2$ ou $x \ltest x^2$ ou
$x \geqhyp 1$ ou $x \gcond 2$.
\end{latexex}


% ---------------------- %


\subsubsection{Une table récapitulative}

La table \ref{table:decorations-operators} \vpageref{table:decorations-operators} fournit toutes les associations autorisées entre opérateurs de comparaison et décorations.


% ---------------------- %


\subsubsection{Textes utilisés} \label{text-for-opes}

Voici les macros définissant les textes utilisés qui tiennent compte de l'utilisation ou non de l'option \verb+french+ de \verb+babel+. Nous ne donnons que les versions françaises.

\vspace{-.5em}

% == All texts - START == %

\begin{multicols}{2}
    \macro{textopappli} donne \emph{\og \textopappli \fg}

    \macro{textopchoice} donne \emph{\og \textopchoice \fg}

    \macro{textopcond} donne \emph{\og \textopcond \fg}

    \macro{textopcons} donne \emph{\og \textopcons \fg}

    \macro{textopdef} donne \emph{\og \textopdef \fg}

    \macro{textophyp} donne \emph{\og \textophyp \fg}

    \macro{textopid} donne \emph{\og \textopid \fg}

    \macro{textopplot} donne \emph{\og \textopplot \fg}

    \macro{textoptest} donne \emph{\og \textoptest \fg}
\vfill\null\end{multicols}

% == All texts - END == %


% ---------------------- %


\subsubsection{Fiches techniques}

\paragraph{Les textes pour les opérateurs de \og comparaison algébrique \fg et de logique}

% == Technical infos - Texts - START == %

\foreach \k in {textopappli, textopchoice, textopcond, textopcons, textopdef, textophyp, textopid, textopplot, textoptest}{

	\IDmacro*{\k}{0}

}

% == Technical infos - Texts - END == %


% ---------------------- %


\paragraph{Les opérateurs de \og comparaison algébrique \fg}

% == Technical infos - Operators - START == %

\foreach \k in {eqappli, eqchoice, eqcond, eqcons, eqdef, eqdef*, eqhyp, eqid, eqid*, eqplot, eqtest}{

    \IDmacro*{\k}{0}
}
                
\separation

\foreach \k in {neqappli, neqchoice, neqcond, neqcons, neqhyp, neqid, neqtest}{

    \IDmacro*{\k}{0}
}
                
\separation

\foreach \k in {lappli, lchoice, lcond, lcons, lhyp, lplot, ltest}{

    \IDmacro*{\k}{0}
}
                
\separation

\foreach \k in {gappli, gchoice, gcond, gcons, ghyp, gplot, gtest}{

    \IDmacro*{\k}{0}
}
                
\separation

\foreach \k in {leqappli, leqchoice, leqcond, leqcons, leqhyp, leqplot, leqtest}{

    \IDmacro*{\k}{0}
}
                
% == Technical infos - Operators - END == %

\end{document}








