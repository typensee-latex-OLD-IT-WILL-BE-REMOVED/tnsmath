\documentclass[12pt,a4paper]{article}

\makeatletter
	\usepackage[utf8]{inputenc}
\usepackage[T1]{fontenc}
\usepackage{ucs}

\usepackage[french]{babel,varioref}

\usepackage[top=2cm, bottom=2cm, left=1.5cm, right=1.5cm]{geometry}
\usepackage{enumitem}

\usepackage{multicol}

\usepackage{color}
\usepackage{hyperref}
\hypersetup{
    colorlinks,
    citecolor=black,
    filecolor=black,
    linkcolor=black,
    urlcolor=black
}

\usepackage{amsthm}

\usepackage{tcolorbox}
\tcbuselibrary{listingsutf8}

\usepackage{ifplatform}

\usepackage{ifthen}

\usepackage{cbdevtool}


% MISC

\newtcblisting{latexex}{%
	sharp corners,%
	left=1mm, right=1mm,%
	bottom=1mm, top=1mm,%
	colupper=red!75!blue,%
	listing side text
}

\newtcblisting{latexex-flat}{%
	sharp corners,%
	left=1mm, right=1mm,%
	bottom=1mm, top=1mm,%
	colupper=red!75!blue,%
}

\newtcblisting{latexex-alone}{%
	sharp corners,%
	left=1mm, right=1mm,%
	bottom=1mm, top=1mm,%
	colupper=red!75!blue,%
	listing only
}


\newcommand\env[1]{\texttt{#1}}
\newcommand\macro[1]{\env{\textbackslash{}#1}}



\setlength{\parindent}{0cm}
\setlist{noitemsep}

\theoremstyle{definition}
\newtheorem*{remark}{Remarque}

\usepackage[raggedright]{titlesec}

\titleformat{\paragraph}[hang]{\normalfont\normalsize\bfseries}{\theparagraph}{1em}{}
\titlespacing*{\paragraph}{0pt}{3.25ex plus 1ex minus .2ex}{0.5em}


\newcommand\separation{
	\medskip
	\hfill\rule{0.5\textwidth}{0.75pt}\hfill
	\medskip
}


\newcommand\extraspace{
	\vspace{0.25em}
}


\newcommand\ascii{\texttt{ASCII}}

	% == PACKAGES USED == %

\RequirePackage{amsmath}
\RequirePackage{relsize}
\RequirePackage{xparse}


% == DEFINITIONS == %

% Settable texts
\@ifpackagewith{babel}{french}{
    \newcommand\lymathsep{;}
    \newcommand\lymathsubsep{,}
    \newcommand\textopchoice{choix}
    \newcommand\textopcond{cond}
    \newcommand\textopdef{déf}
    \newcommand\textophyp{hyp}
    \newcommand\textopid{id}
}{
	\newcommand\lymathsep{,}
	\newcommand\lymathsubsep{;}
    \newcommand\textopchoice{choice}
    \newcommand\textopcond{cond}
    \newcommand\textopdef{def}
    \newcommand\textophyp{hyp}
    \newcommand\textopid{id}
}


\newcommand\textexplainleft{\{}
\newcommand\textexplainright{\}}
\newcommand\textexplainspacebefore{\qquad}
\newcommand\textexplainspacein{\qquad}


% Tools - Apply same macro to all arguments

% #1        : main macro
% #2        : macro to apply to arguments
% #3 and #4 : the two arguments
\newcommand\@apply@macro@two@args[4]{%
	#1{#2{#3}}{#2{#4}}%
}




% Tools - Intervals

\newcommand\@extra@phantom{%
	\vphantom{\relsize{1.25}{\text{$\displaystyle F_1^2$}}}%
}

\newcommand\@interval@tool@star[5]{%
	\ensuremath{ \left#1 \@extra@phantom \right. \!\! #2 #3 #4 \left. \@extra@phantom \!\! \right#5}%
}

\newcommand\@interval@tool@no@star[5]{\ensuremath{ \left#1 #2 #3 #4 \right#5}}


% Tools - Multi-arguments
%
% Source : the following lines come directly for the following post
%
%    * https://tex.stackexchange.com/a/475291/6880

\ExplSyntaxOn
% General purpose macro for defining other macros
	\NewDocumentCommand{\makemultiargument}{mmmmmo}{
		\lymath_multiarg:nnnnnn{#1}{#2}{#3}{#4}{#5}{#6}
	}
 
% Allocate a private variable
	\seq_new:N \l__lymath_generic_seq

% The internal version of the general purpose macro
	\cs_new_protected:Nn \lymath_multiarg:nnnnnn{
		% #1 = separator
		% #2 = multiargument
		% #3 = code before
	  	% #4 = code between
	  	% #5 = code after
	  	% #6 = ornament to items

		% A group allows nesting
		\group_begin:
	 	% Split the multiargument into parts
		\seq_set_split:Nnn \l__lymath_generic_seq { #1 } { #2 }
		% Apply the ornament to the items
	  	\tl_if_novalue:nF { #6 }{
	    	\seq_set_eq:NN \l__lymath_temp_seq \l__lymath_generic_seq
	    	\seq_set_map:NNn \l__lymath_generic_seq \l__lymath_generic_seq { #6 }
	   	}
		% Execute the <code before>
	  	#3
		% Deliver the items, with the chosen material between them
	  	\seq_use:Nn \l__lymath_generic_seq { #4 }
  		% Execute the <code after>
	 	#5
  		% End the group started at the beginning
	  	\group_end:
	}	
\ExplSyntaxOff


	\usepackage{01-equal-signs-n-co}
\makeatother

\newcommand\RR{RR}
\newcommand\setgeo[1]{#1}


\begin{document}

\section{Logique et fondements}

\subsection{Différents types d'égalités \og standard \fg}

D'un point de vue pédagogique, il peut être intéressant de disposer de différentes façon d'écrire une égalité, une non égalité ou une inégalité.
Bien entendu on tord les règles de typographie avec ce type de pratique mais c'est pour le bien de la communauté.

\subsubsection{Définir quelque chose}

L'exemple suivant montre trois façons de rédiger une égalité signifiant une définition
\footnote{
	Le symbole peu courant $\eqdef**$ est utilisé par le langage B qui permet de spécifier et prouver certains programmes.
}
\emph{(la section \ref{text-for-opes} explique comment est définit le texte \emph{\og \textopdef \fg})}.

\begin{tcblisting}{}
La fonction $f$ est définie sur $\RR$ par $f(x) \eqdef x^3 + 1$ ou avec les écritures
symboliques $f(x) \eqdef* x^3 + 1$ et $f(x) \eqdef** x^3 + 1$.
\end{tcblisting}


\subsubsection{Indiquer une identité}

L'exemple suivant montre deux façons de rédiger des identités, la noptation symbolique n'étant pas standard \emph{(la section \ref{text-for-opes} explique comment est défini le texte \emph{\og \textopid \fg})}.

\begin{tcblisting}{}
$\forall (a ; b) \in \RR^2$, nous avons : $(a + b)^2 \eqid a^2 + b^2 + 2 a b$ .
On peut utiliser une écriture plus symbolique : $(a + b)^2 \eqid* a^2 + b^2 + 2 a b$ .
\end{tcblisting}


\subsubsection{Une égalité à vérifier ou non, une hypothèse, une condition}

Se reporter à la section \ref{text-for-opes} pour savoir comment sont définis les textes \emph{\og \textopcond \fg} et \emph{\og \textophyp \fg}.

\begin{tcblisting}{}
Est-il vrai que $(a + b)^3 \eqtest a^3 + b^3 + 3 a b$ ?
Non car $(a + b)^3 \neqid a^3 + b^3 + 3 a b$

A-t-on $x \neqhyp 0$ pour pouvoir écrire $\dfrac{1}{x}$ ?

Comme $x$ doit être non nul, je sait que $x \neqcond 0$.

Une autre condition $x \eqcond 0$ ou une autre hypothèse $x \eqhyp 0$ ?
\end{tcblisting}


\subsubsection{Une égalité indiquant le choix d'une valeur}

La section \ref{text-for-opes} permet de savoir comment le texte \emph{\og \textopchoice \fg} est défini.

\begin{tcblisting}{}
Sachant que si $x \geqcond 4$ alors $x^2 \geqcons 16$ alors $x \eqchoice 123$ nous
donne $123^2 \geqappli 16$.
\end{tcblisting}


\subsubsection{Une égalité indiquant l'équation d'une courbe que l'on utilise}

La section \ref{text-for-opes} permet de savoir comment les textes \emph{\og \textopplot \fg} et \emph{\og \textopappli \fg} sont définis \emph{(la macro \emph{\texttt{setgeo}} est définie dans la section \ref{set-geo})}.

\begin{tcblisting}{}
Si $M(4 ; y_M) \in \setgeo{C}: y \eqplot x^2 + 3$ alors $y_M \eqappli 4^2 + 3 = 19$.
\end{tcblisting}



\subsubsection{Différents types d'inéquations}

Le principe reste le même pour les symboles d'équations excepté qu'il n'y a ici aucune écriture purement symbolique et que l'on pas la version \emph{\og {choix} \fg}. Voici un code \og fourre-tout \fg{} pour voir ce que vous avez à votre disposition.

\begin{tcblisting}{}
A-t-on $x \leqtest x^2$ ou $x \ltest x^2$ ?

A moins que ce ne soit $x \geqtest x^2$ ou $x \gtest x^2$ qu'il faille vérifier.

On peut supposer $x \leqhyp 1$ ou avoir la condition $x \gcond 2$.

\end{tcblisting}



\subsubsection{Une table récapitulative}

La table \ref{table:decorations-operators} \vpageref{table:decorations-operators} fournit toutes les associations autorisées entre opérateurs de comparaison et décorations.



\subsection{Textes utilisés} \label{text-for-opes}

Voici les macros définissant les textes utilisés qui tiennent compte de l'utilisation ou non de l'option \verb+french+ de \verb+babel+. Nous ne donnons que les versions françaises.

% == All texts - START == %

\begin{multicols}{2}
    \verb+\textopappli+ donne \emph{\og \textopappli \fg}

    \verb+\textopchoice+ donne \emph{\og \textopchoice \fg}

    \verb+\textopcond+ donne \emph{\og \textopcond \fg}

    \verb+\textopcons+ donne \emph{\og \textopcons \fg}

    \verb+\textopdef+ donne \emph{\og \textopdef \fg}

    \verb+\textophyp+ donne \emph{\og \textophyp \fg}

    \verb+\textopid+ donne \emph{\og \textopid \fg}

    \verb+\textopplot+ donne \emph{\og \textopplot \fg}

    \verb+\textoptest+ donne \emph{\og \textoptest \fg}
\vfill\null\end{multicols}

% == All texts - END == %



\subsection{Fiches techniques}

\paragraph{Les opérateurs disponibles}

\begin{multicols}{2}
% == Technical infos - Operators - START == %

\foreach \k in {eqappli, eqchoice, eqcond, eqcons, eqdef, eqdef*, eqdef**, eqhyp, eqid, eqid*, eqplot, eqtest, neqappli, neqid, neqchoice, neqcond, neqcons, neqhyp, neqtest, lappli, lchoice, lcond, lcons, lhyp, ltest, gappli, gchoice, gcond, gcons, ghyp, gtest, leqappli, leqchoice, leqcond, leqcons, leqhyp, leqtest, geqappli, geqchoice, geqcond, geqcons, geqhyp, geqtest}{

	\IDmacro*{\k}{0}

}

% == Technical infos - Operators - END == %
\vfill\null\end{multicols}



\paragraph{Les textes disponibles}

\begin{multicols}{2}
% == Technical infos - Texts - START == %

\foreach \k in {textopappli, textopchoice, textopcond, textopcons, textopdef, textophyp, textopid, textopplot, textoptest}{

	\IDmacro*{\k}{0}

}

% == Technical infos - Texts - END == %
\vfill\null\end{multicols}

\end{document}
