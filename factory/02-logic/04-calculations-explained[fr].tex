\documentclass[12pt,a4paper]{article}

\makeatletter
    \usepackage[utf8]{inputenc}
\usepackage[T1]{fontenc}
\usepackage{ucs}

\usepackage[french]{babel,varioref}

\usepackage[top=2cm, bottom=2cm, left=1.5cm, right=1.5cm]{geometry}
\usepackage{enumitem}

\usepackage{multicol}

\usepackage{color}
\usepackage{hyperref}
\hypersetup{
    colorlinks,
    citecolor=black,
    filecolor=black,
    linkcolor=black,
    urlcolor=black
}

\usepackage{amsthm}

\usepackage{tcolorbox}
\tcbuselibrary{listingsutf8}

\usepackage{ifplatform}

\usepackage{ifthen}

\usepackage{cbdevtool}


% MISC

\newtcblisting{latexex}{%
	sharp corners,%
	left=1mm, right=1mm,%
	bottom=1mm, top=1mm,%
	colupper=red!75!blue,%
	listing side text
}

\newtcblisting{latexex-flat}{%
	sharp corners,%
	left=1mm, right=1mm,%
	bottom=1mm, top=1mm,%
	colupper=red!75!blue,%
}

\newtcblisting{latexex-alone}{%
	sharp corners,%
	left=1mm, right=1mm,%
	bottom=1mm, top=1mm,%
	colupper=red!75!blue,%
	listing only
}


\newcommand\env[1]{\texttt{#1}}
\newcommand\macro[1]{\env{\textbackslash{}#1}}



\setlength{\parindent}{0cm}
\setlist{noitemsep}

\theoremstyle{definition}
\newtheorem*{remark}{Remarque}

\usepackage[raggedright]{titlesec}

\titleformat{\paragraph}[hang]{\normalfont\normalsize\bfseries}{\theparagraph}{1em}{}
\titlespacing*{\paragraph}{0pt}{3.25ex plus 1ex minus .2ex}{0.5em}


\newcommand\separation{
	\medskip
	\hfill\rule{0.5\textwidth}{0.75pt}\hfill
	\medskip
}


\newcommand\extraspace{
	\vspace{0.25em}
}


\newcommand\ascii{\texttt{ASCII}}

    % == PACKAGES USED == %

\RequirePackage{amsmath}
\RequirePackage{relsize}
\RequirePackage{xparse}


% == DEFINITIONS == %

% Settable texts
\@ifpackagewith{babel}{french}{
    \newcommand\lymathsep{;}
    \newcommand\lymathsubsep{,}
    \newcommand\textopchoice{choix}
    \newcommand\textopcond{cond}
    \newcommand\textopdef{déf}
    \newcommand\textophyp{hyp}
    \newcommand\textopid{id}
}{
	\newcommand\lymathsep{,}
	\newcommand\lymathsubsep{;}
    \newcommand\textopchoice{choice}
    \newcommand\textopcond{cond}
    \newcommand\textopdef{def}
    \newcommand\textophyp{hyp}
    \newcommand\textopid{id}
}


\newcommand\textexplainleft{\{}
\newcommand\textexplainright{\}}
\newcommand\textexplainspacebefore{\qquad}
\newcommand\textexplainspacein{\qquad}


% Tools - Apply same macro to all arguments

% #1        : main macro
% #2        : macro to apply to arguments
% #3 and #4 : the two arguments
\newcommand\@apply@macro@two@args[4]{%
	#1{#2{#3}}{#2{#4}}%
}




% Tools - Intervals

\newcommand\@extra@phantom{%
	\vphantom{\relsize{1.25}{\text{$\displaystyle F_1^2$}}}%
}

\newcommand\@interval@tool@star[5]{%
	\ensuremath{ \left#1 \@extra@phantom \right. \!\! #2 #3 #4 \left. \@extra@phantom \!\! \right#5}%
}

\newcommand\@interval@tool@no@star[5]{\ensuremath{ \left#1 #2 #3 #4 \right#5}}


% Tools - Multi-arguments
%
% Source : the following lines come directly for the following post
%
%    * https://tex.stackexchange.com/a/475291/6880

\ExplSyntaxOn
% General purpose macro for defining other macros
	\NewDocumentCommand{\makemultiargument}{mmmmmo}{
		\lymath_multiarg:nnnnnn{#1}{#2}{#3}{#4}{#5}{#6}
	}
 
% Allocate a private variable
	\seq_new:N \l__lymath_generic_seq

% The internal version of the general purpose macro
	\cs_new_protected:Nn \lymath_multiarg:nnnnnn{
		% #1 = separator
		% #2 = multiargument
		% #3 = code before
	  	% #4 = code between
	  	% #5 = code after
	  	% #6 = ornament to items

		% A group allows nesting
		\group_begin:
	 	% Split the multiargument into parts
		\seq_set_split:Nnn \l__lymath_generic_seq { #1 } { #2 }
		% Apply the ornament to the items
	  	\tl_if_novalue:nF { #6 }{
	    	\seq_set_eq:NN \l__lymath_temp_seq \l__lymath_generic_seq
	    	\seq_set_map:NNn \l__lymath_generic_seq \l__lymath_generic_seq { #6 }
	   	}
		% Execute the <code before>
	  	#3
		% Deliver the items, with the chosen material between them
	  	\seq_use:Nn \l__lymath_generic_seq { #4 }
  		% Execute the <code after>
	 	#5
  		% End the group started at the beginning
	  	\group_end:
	}	
\ExplSyntaxOff


    \usepackage{04-calculations-explained}
\makeatother


\newcommand\RRp{RR+}
\newcommand\vimplies{\Downarrow}
\newcommand\viff{\Updownarrow}


\begin{document}

%\section{Logique et fondements}

\subsection{Détailler un raisonnement} \label{explain-proof}

\paragraph{Exemple 1 -- Avec les réglages par défaut}

L'environnement \env{explain} permet de détailler les étapes principales d'un calcul ou d'un raisonnement en s'appuyant sur la macro \macro{explnext} dont le nom vient de \emph{\og \textbf{expl}-ain \textbf{next} step \fg} soit \emph{\og expliquer la prochaine étape \fg}
\footnote{
    Le gros du travail est fait par l'environnement \env{flalign} du package \texttt{amsmath} qui est automatiquement chargé par \texttt{lymath}.
}.
Voici un exemple avec les réglages par défaut \emph{(notez les espaces verticaux ajoutés automatiquement avant et après)}.

\begin{latexex-flat}
\begin{explain}
    (a + b)^2
        \explnext{On utilise $x^2 = x \cdot x$.}
    (a + b) (a + b)
        \explnext{Double développement depuis la parenthèse gauche.}
    a^2 + a b + b a + b^2
        \explnext{Commutativité du produit.}
    a^2 + 2 a b + b^2
\end{explain}
\end{latexex-flat}


\begin{remark}
    La macro \macro{explnext} utilise les macros constantes suivantes.
    \begin{itemize}
        \item \macro{textexplainleft} et \macro{textexplainright} qui donnent $\textexplainleft$ et $\textexplainright$ respectivement par défaut.

        \item \macro{textexplainspacein} est l'espacement entre le symbole et la courte explication. Par défaut, cette macro vaut \verb+2em+.
    \end{itemize}
\end{remark}


% ---------------------- %


\paragraph{Exemple 2 -- Utiliser un autre symbole globalement}

L'environnement \env{explain} possède un argument optionnel qui est \verb+=+ par défaut. Ceci permet de faire ce qui suit sans effort.

\begin{latexex}
\begin{explain}[\viff]
    x^2 + 10 x + 25 = 0
        \explnext{Identité remarquable.}
    (x + 5)^2 = 0
        \explnext{$X^2 = 0$ si et 
                  seulement si $X = 0$.}
    x = -5
\end{explain}
\end{latexex}


% ---------------------- %


\paragraph{Exemple 3 -- Juste utiliser des symboles}

Si l'argument obligatoire de la macro \macro{explnext} est vide alors seul le symbole est affiché. Voici un court exemple de ceci.

\begin{latexex}
\begin{explain}[\viff]
    a^2 = b^2
        \explnext{}
    a = \pm b
\end{explain}
\end{latexex}


%% ---------------------- %


\paragraph{Exemple 4 -- Utiliser un autre symbole localement}

La macro \macro{explnext} possède un argument optionnel qui utilise par défaut celui de l'environment. En utilisant cette option, on choisit alors localement le symbole à employer. Voici un exemple farfelu d'utilisation.

\begin{latexex}
\begin{explain}[\viff]
    0 \leq a < b
        \explnext[\vimplies]%
                 {Croissance de $x^2$ 
                  sur $\RRp$.}
    a^2 < b^2
        \explnext{}
    a^2 - b^2 < 0
        \explnext{Identité remarquable.}
    (a - b)(a + b) < 0
        \explnext[\vimplies]{}
    a \neq b
\end{explain}
\end{latexex}


% ---------------------- %


\subsubsection{Fiches techniques}

\paragraph{Les textes pour détailler un raisonnement}

\IDmacro*{textexplainleft}{0}

\IDmacro*{textexplainright}{0}

\IDmacro*{textexplainspacein}{0}


% ---------------------- %


\paragraph{Détailler un raisonnement}

\IDenv{explain}{1}

\IDoption{} le symbole à utiliser dans l'environnement, la valeur par défaut étant \verb+=+ .


\separation


\IDmacro{explnext}{1}{1}

\IDoption{} le symbole à utiliser pour une explication, la valeur par défaut étant celle du symbole de l'environnement \env{explain} où \macro{explnext} est utilisé.

\IDarg{} le texte de l'explication qui peut être vide si aucune explication n'est à afficher.


\end{document}
