\documentclass[12pt,a4paper]{article}

\makeatletter
	\usepackage[utf8]{inputenc}
\usepackage[T1]{fontenc}
\usepackage{ucs}

\usepackage[french]{babel,varioref}

\usepackage[top=2cm, bottom=2cm, left=1.5cm, right=1.5cm]{geometry}
\usepackage{enumitem}

\usepackage{multicol}

\usepackage{color}
\usepackage{hyperref}
\hypersetup{
    colorlinks,
    citecolor=black,
    filecolor=black,
    linkcolor=black,
    urlcolor=black
}

\usepackage{amsthm}

\usepackage{tcolorbox}
\tcbuselibrary{listingsutf8}

\usepackage{ifplatform}

\usepackage{ifthen}

\usepackage{cbdevtool}


% MISC

\newtcblisting{latexex}{%
	sharp corners,%
	left=1mm, right=1mm,%
	bottom=1mm, top=1mm,%
	colupper=red!75!blue,%
	listing side text
}

\newtcblisting{latexex-flat}{%
	sharp corners,%
	left=1mm, right=1mm,%
	bottom=1mm, top=1mm,%
	colupper=red!75!blue,%
}

\newtcblisting{latexex-alone}{%
	sharp corners,%
	left=1mm, right=1mm,%
	bottom=1mm, top=1mm,%
	colupper=red!75!blue,%
	listing only
}


\newcommand\env[1]{\texttt{#1}}
\newcommand\macro[1]{\env{\textbackslash{}#1}}



\setlength{\parindent}{0cm}
\setlist{noitemsep}

\theoremstyle{definition}
\newtheorem*{remark}{Remarque}

\usepackage[raggedright]{titlesec}

\titleformat{\paragraph}[hang]{\normalfont\normalsize\bfseries}{\theparagraph}{1em}{}
\titlespacing*{\paragraph}{0pt}{3.25ex plus 1ex minus .2ex}{0.5em}


\newcommand\separation{
	\medskip
	\hfill\rule{0.5\textwidth}{0.75pt}\hfill
	\medskip
}


\newcommand\extraspace{
	\vspace{0.25em}
}


\newcommand\ascii{\texttt{ASCII}}

	% == PACKAGES USED == %

\RequirePackage{amsmath}
\RequirePackage{relsize}
\RequirePackage{xparse}


% == DEFINITIONS == %

% Settable texts
\@ifpackagewith{babel}{french}{
    \newcommand\lymathsep{;}
    \newcommand\lymathsubsep{,}
    \newcommand\textopchoice{choix}
    \newcommand\textopcond{cond}
    \newcommand\textopdef{déf}
    \newcommand\textophyp{hyp}
    \newcommand\textopid{id}
}{
	\newcommand\lymathsep{,}
	\newcommand\lymathsubsep{;}
    \newcommand\textopchoice{choice}
    \newcommand\textopcond{cond}
    \newcommand\textopdef{def}
    \newcommand\textophyp{hyp}
    \newcommand\textopid{id}
}


\newcommand\textexplainleft{\{}
\newcommand\textexplainright{\}}
\newcommand\textexplainspacebefore{\qquad}
\newcommand\textexplainspacein{\qquad}


% Tools - Apply same macro to all arguments

% #1        : main macro
% #2        : macro to apply to arguments
% #3 and #4 : the two arguments
\newcommand\@apply@macro@two@args[4]{%
	#1{#2{#3}}{#2{#4}}%
}




% Tools - Intervals

\newcommand\@extra@phantom{%
	\vphantom{\relsize{1.25}{\text{$\displaystyle F_1^2$}}}%
}

\newcommand\@interval@tool@star[5]{%
	\ensuremath{ \left#1 \@extra@phantom \right. \!\! #2 #3 #4 \left. \@extra@phantom \!\! \right#5}%
}

\newcommand\@interval@tool@no@star[5]{\ensuremath{ \left#1 #2 #3 #4 \right#5}}


% Tools - Multi-arguments
%
% Source : the following lines come directly for the following post
%
%    * https://tex.stackexchange.com/a/475291/6880

\ExplSyntaxOn
% General purpose macro for defining other macros
	\NewDocumentCommand{\makemultiargument}{mmmmmo}{
		\lymath_multiarg:nnnnnn{#1}{#2}{#3}{#4}{#5}{#6}
	}
 
% Allocate a private variable
	\seq_new:N \l__lymath_generic_seq

% The internal version of the general purpose macro
	\cs_new_protected:Nn \lymath_multiarg:nnnnnn{
		% #1 = separator
		% #2 = multiargument
		% #3 = code before
	  	% #4 = code between
	  	% #5 = code after
	  	% #6 = ornament to items

		% A group allows nesting
		\group_begin:
	 	% Split the multiargument into parts
		\seq_set_split:Nnn \l__lymath_generic_seq { #1 } { #2 }
		% Apply the ornament to the items
	  	\tl_if_novalue:nF { #6 }{
	    	\seq_set_eq:NN \l__lymath_temp_seq \l__lymath_generic_seq
	    	\seq_set_map:NNn \l__lymath_generic_seq \l__lymath_generic_seq { #6 }
	   	}
		% Execute the <code before>
	  	#3
		% Deliver the items, with the chosen material between them
	  	\seq_use:Nn \l__lymath_generic_seq { #4 }
  		% Execute the <code after>
	 	#5
  		% End the group started at the beginning
	  	\group_end:
	}	
\ExplSyntaxOff


	\usepackage{01-polynomial-and-co}
\makeatother


\newcommand\ZZ{ZZ}
\newcommand\QQ{QQ}
\newcommand\RR{RR}
\newcommand\CC{CC}



\begin{document}

\section{Algèbre}

\subsection{Polynômes, séries formelles et compagnie}

\subsubsection{Polynômes et fractions polynômiales}

\paragraph{Exemple d'utilisation 1 : Polynômes}

\begin{tcblisting}{}
$\polyset{\RR}{X}$ est l'ensemble des polynômes à coefficients réels en la variable
$X$, et $\polyset{\RR}{X | Y | Z}$ est l'ensemble des polynômes à coefficients réels
en les variables $X$ , $Y$ et $Z$.
\end{tcblisting}



\paragraph{Exemple d'utilisation 2 : Fractions polynômiales}

\begin{tcblisting}{}
$\polyfracset{\QQ}{T}$ et $\polyfracset{\QQ}{S_1 | S_2 | \dots | S_k}$ permettent
d'indiquer des ensemble de fractions polynomiales à coefficients rationnels.
\end{tcblisting}



\subsubsection{Séries formelles et leurs corps de fractions}

\paragraph{Exemple d'utilisation 1 : Séries formelles}

\begin{tcblisting}{}
$\serieset{\CC}{X}$ et $\serieset{\CC}{T | O | P}$ permettent de travailler avec des
séries formelles à coefficients complexes.
\end{tcblisting}



\paragraph{Exemple d'utilisation 2 : Corps des fractions de séries formelles}

\begin{tcblisting}{}
$\seriefracset{\ZZ}{X}$ et $\seriefracset{\ZZ}{Z | T | O | P}$ permettent de travailler
avec des fractions de séries formelles à coefficients entiers.
\end{tcblisting}



\subsubsection{Polynômes de Laurent et séries formelles de Laurent}

\paragraph{Exemple d'utilisation 1 : Polynômes de Laurent}

\begin{tcblisting}{}
$\polylaurentset{\RR}{X} = \polyset{\RR}{X | X^{-1}}$ est l'ensemble des polynômes
réels de Laurent en $X$. On propose de généraliser comme suit (notation non standard) :
$\polylaurentset{\RR}{X_1 | X_2} = \polyset{\RR}{X_1 | X_1^{-1} | X_2 | X_2^{-1}}$
\end{tcblisting}



\paragraph{Exemple d'utilisation 2 : Séries formelles de Laurent}

\begin{tcblisting}{}
$\serielaurentset{\QQ}{X} = \serieset{\QQ}{X | X^{-1}}$ est l'ensemble des séries
formelles rationnelles de Laurent en $X$. On généralise via, notation non standard,
$\serielaurentset{\QQ}{X_1 | X_2} = \serieset{\QQ}{X_1 | X_1^{-1} | X_2 | X_2^{-1}}$
\end{tcblisting}



\subsubsection{Toutes les fiches techniques}

\IDmacro*{polyset}{2}

\IDmacro*{polyfracset}{2}

\IDmacro*{serieset}{2}

\IDmacro*{seriefracset}{2}

\IDmacro*{polylaurentset}{2}

\IDmacro*{serielaurentset}{2}

\IDarg{1} l'ensemble auquel les coefficients appartiennent.

\IDarg{2} cet argument est une suite de "morceaux" séparés par des barres \verb+|+, chaque morceau étant une variable formelle.

\end{document}
