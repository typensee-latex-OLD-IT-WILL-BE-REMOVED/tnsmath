\documentclass[12pt,a4paper]{article}

\usepackage[french]{babel}

\makeatletter
	\usepackage[utf8]{inputenc}
\usepackage[T1]{fontenc}
\usepackage{ucs}

\usepackage[french]{babel,varioref}

\usepackage[top=2cm, bottom=2cm, left=1.5cm, right=1.5cm]{geometry}
\usepackage{enumitem}

\usepackage{multicol}

\usepackage{color}
\usepackage{hyperref}
\hypersetup{
    colorlinks,
    citecolor=black,
    filecolor=black,
    linkcolor=black,
    urlcolor=black
}

\usepackage{amsthm}

\usepackage{tcolorbox}
\tcbuselibrary{listingsutf8}

\usepackage{ifplatform}

\usepackage{ifthen}

\usepackage{cbdevtool}


% MISC

\newtcblisting{latexex}{%
	sharp corners,%
	left=1mm, right=1mm,%
	bottom=1mm, top=1mm,%
	colupper=red!75!blue,%
	listing side text
}

\newtcblisting{latexex-flat}{%
	sharp corners,%
	left=1mm, right=1mm,%
	bottom=1mm, top=1mm,%
	colupper=red!75!blue,%
}

\newtcblisting{latexex-alone}{%
	sharp corners,%
	left=1mm, right=1mm,%
	bottom=1mm, top=1mm,%
	colupper=red!75!blue,%
	listing only
}


\newcommand\env[1]{\texttt{#1}}
\newcommand\macro[1]{\env{\textbackslash{}#1}}



\setlength{\parindent}{0cm}
\setlist{noitemsep}

\theoremstyle{definition}
\newtheorem*{remark}{Remarque}

\usepackage[raggedright]{titlesec}

\titleformat{\paragraph}[hang]{\normalfont\normalsize\bfseries}{\theparagraph}{1em}{}
\titlespacing*{\paragraph}{0pt}{3.25ex plus 1ex minus .2ex}{0.5em}


\newcommand\separation{
	\medskip
	\hfill\rule{0.5\textwidth}{0.75pt}\hfill
	\medskip
}


\newcommand\extraspace{
	\vspace{0.25em}
}


\newcommand\ascii{\texttt{ASCII}}

	
	\usepackage{tnscom}
\makeatother


\begin{document}

\section{Utiliser des unités S.I.}

Comme l'excellent package \verb#siunitx# est chargé en coulisse il devient facile de travailler avec des unités de mesure tout en respectant
\textbf{%
les conventions d'écriture qui seront françaises
	\footnote{
		Notez dans l'exemple l'écriture de $\num{1230}$ n'utilise pas d'espace contrairement à celle de $\num{12300}$.
	}
dès lors que vous aurez chargé \texttt{babel} avec l'option \texttt{french}%
} comme c'est le cas pour cette documentation.
Notez que les espaces dans \verb#\num{123 000}# et \verb#\num{1230 * 100}# sont inutiles mais qu'ils facilitent la relecture du code.

\begin{latexex}
$\ang{180} = \SI{\pi}{\radian}$ ,
$\SI{1}{km} = \SI{1e3}{m}$ avec aussi
$\num{123 000} = \num{1230 * 100}$
\end{latexex}

\end{document}
