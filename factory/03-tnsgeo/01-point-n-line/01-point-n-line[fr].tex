\documentclass[12pt,a4paper]{article}

\makeatletter
	\usepackage[utf8]{inputenc}
\usepackage[T1]{fontenc}
\usepackage{ucs}

\usepackage[french]{babel,varioref}

\usepackage[top=2cm, bottom=2cm, left=1.5cm, right=1.5cm]{geometry}
\usepackage{enumitem}

\usepackage{multicol}

\usepackage{makecell}

\usepackage{color}
\usepackage{hyperref}
\hypersetup{
    colorlinks,
    citecolor=black,
    filecolor=black,
    linkcolor=black,
    urlcolor=black
}

\usepackage{amsthm}

\usepackage{tcolorbox}
\tcbuselibrary{listingsutf8}

\usepackage{ifplatform}

\usepackage{ifthen}

\usepackage{cbdevtool}


% MISC

\newtcblisting{latexex}{%
	sharp corners,%
	left=1mm, right=1mm,%
	bottom=1mm, top=1mm,%
	colupper=red!75!blue,%
	listing side text
}

\newtcblisting{latexex-flat}{%
	sharp corners,%
	left=1mm, right=1mm,%
	bottom=1mm, top=1mm,%
	colupper=red!75!blue,%
}

\newtcblisting{latexex-alone}{%
	sharp corners,%
	left=1mm, right=1mm,%
	bottom=1mm, top=1mm,%
	colupper=red!75!blue,%
	listing only
}


\newcommand\env[1]{\texttt{#1}}
\newcommand\macro[1]{\env{\textbackslash{}#1}}



\setlength{\parindent}{0cm}
\setlist{noitemsep}

\theoremstyle{definition}
\newtheorem*{remark}{Remarque}

\usepackage[raggedright]{titlesec}

\titleformat{\paragraph}[hang]{\normalfont\normalsize\bfseries}{\theparagraph}{1em}{}
\titlespacing*{\paragraph}{0pt}{3.25ex plus 1ex minus .2ex}{0.5em}


\newcommand\separation{
	\medskip
	\hfill\rule{0.5\textwidth}{0.75pt}\hfill
	\medskip
}


\newcommand\extraspace{
	\vspace{0.25em}
}


\newcommand\whyprefix[2]{%
	\textbf{\prefix{#1}}-#2%
}

\newcommand\mwhyprefix[2]{%
	\texttt{#1 = #1-#2}%
}

\newcommand\prefix[1]{%
	\texttt{#1}%
}


\newcommand\inenglish{\@ifstar{\@inenglish@star}{\@inenglish@no@star}}

\newcommand\@inenglish@star[1]{%
	\emph{\og #1 \fg}%
}

\newcommand\@inenglish@no@star[1]{%
	\@inenglish@star{#1} en anglais%
}


\newcommand\ascii{\texttt{ASCII}}


% Example
\newcounter{paraexample}[subsubsection]

\newcommand\@newexample@abstract[2]{%
	\paragraph{%
		#1%
		\if\relax\detokenize{#2}\relax\else {} -- #2\fi%
	}%
}



\newcommand\newparaexample{\@ifstar{\@newparaexample@star}{\@newparaexample@no@star}}

\newcommand\@newparaexample@no@star[1]{%
	\refstepcounter{paraexample}%
	\@newexample@abstract{Exemple \theparaexample}{#1}%
}

\newcommand\@newparaexample@star[1]{%
	\@newexample@abstract{Exemple}{#1}%
}


% Change log
\newcommand\topic{\@ifstar{\@topic@star}{\@topic@no@star}}

\newcommand\@topic@no@star[1]{%
	\textbf{\textsc{#1}.}%
}

\newcommand\@topic@star[1]{%
	\textbf{\textsc{#1} :}%
}







	\usepackage{01-point-n-line}
\makeatother



\begin{document}

\chapter{Géométrie}

\section{Points et lignes}

\subsection{Points}

\newparaexample{Sans indice}

\begin{latexex}
$\pt{I}$
\end{latexex}


% ---------------------- %


\newparaexample{Avec un indice}

\begin{latexex}
$\pt*{I}{1}$ ou
$\pt*{I}{2}$
\end{latexex}


% ---------------------- %


\subsection{Fiches techniques}

\IDmacro*{pt}{1}

\IDarg{} un texte donnant le nom d'un point.


\separation


\IDmacro*{pt*}{2}

\IDarg{1} un texte indiquant $\pt{UP}$ dans le nom $\pt*{UP}{down}$ d'un point.

\IDarg{2} un texte indiquant $down$ dans le nom $\pt*{UP}{down}$ d'un point.


% ---------------------- %


\subsection{Lignes}

\newparaexample{Les droites}

Dans l'exemple suivant, le préfixe \prefix{g} est pour \whyprefix{g}{éometrie} tandis que \prefix{p} est pour \whyprefix{p}{oint}.

\begin{latexex}
$\gline{A}{B}$ ,
$\gline{\pt{A}}{\pt{B}}$ ou
$\pgline{A}{B}$
\end{latexex}


% ---------------------- %


\newparaexample{Les segments}

Les macros \macro{segment} et \macro{psegment} ont un comportement similaire à \macro{gline} et \macro{pgline}.

\begin{latexex}
$\segment{A}{B}$ ,
$\segment{\pt{A}}{\pt{B}}$ ou
$\psegment{A}{B}$
\end{latexex}


% ---------------------- %


\newparaexample{Les demi-droites}

Dans l'exemple suivant, le préfixe \prefix{h} est pour \whyprefix{h}{alf}  soit \inenglish{moitié}.

\begin{latexex}
$\hgline{A}{B}$ ,
$\hgline{\pt{A}}{\pt{B}}$ ou
$\phgline{A}{B}$
\end{latexex}


% ---------------------- %


\newparaexample{D'autres demi-droites}

Ce qui suit nécessite d'utilise l'argument optionnel de \macro{gline} et \macro{pgline}. La valeur \prefix{OC} provient de \whyprefix{O}{pened} -- \whyprefix{C}{losed} soit \inenglish{ouvert -- fermé}.

\begin{latexex}
$\gline[OC]{A}{B}$ ,
$\gline[OC]{\pt{A}}{\pt{B}}$ ou
$\pgline[OC]{A}{B}$
\end{latexex}


\begin{remark}
	Les segments utilisent en fait l'option \prefix{C} et les demi-droites standard l'option \prefix{CO}.
	La valeur par défaut est \prefix{O}.
\end{remark}


% ---------------------- %


\subsection{Fiches techniques}

\IDmacro{gline }{1}{2}   où \quad \mwhyprefix{g}{eometry}

\IDmacro{pgline}{1}{2}  où \quad \mwhyprefix{p}{oint}
                              et \mwhyprefix{g}{eometry}

\IDoption{} pour indiquer les parenthèses ou crochets à utiliser, les valeurs possibles étant \prefix{O}, valeur par défaut, \prefix{C}, \prefix{CO} et \prefix{OC}.

\IDarg{1} le 1\ier{} point géométrique.

\IDarg{2} le 2\ieme{} point géométrique.


\separation


\IDmacro*{hgline  }{2}   où \quad \mwhyprefix{h}{alf}
                             et \mwhyprefix{g}{eometry}

\IDmacro*{phgline }{2}  où \quad \mwhyprefix{p}{oint},
                                \mwhyprefix{h}{alf}
                             et \mwhyprefix{g}{eometry}

\extraspace

\IDmacro*{segment }{2}

\IDmacro*{psegment}{2}  où \quad \mwhyprefix{p}{oint}

\IDarg{1} le 1\ier{} point géométrique.

\IDarg{2} le 2\ieme{} point géométrique.


% ---------------------- %


\subsection{Droites parallèles ou non}

Les opérateurs \macro{parallel} et \macro{nparallel} utilisent des obliques au lieu de barres verticales comme le montre l'exemple qui suit où \macro{stdnparallel} est un alias de \macro{nparallel} fourni par le package \verb+amssymb+, et \macro{stdparallel} est un alias de la version standard de \macro{parallel} proposée par \LaTeX{}.

\begin{latexex}
$\pgline{A}{B} \parallel \pgline{C}{D}$
au lieu de
$\pgline{A}{B}
 \stdparallel \pgline{C}{D}$

$\pgline{E}{F} \nparallel \pgline{G}{H}$
au lieu de
$\pgline{E}{F}
 \stdnparallel \pgline{G}{H}$
\end{latexex}


% ---------------------- %


\subsection{Fiches techniques}

\IDmacro*{parallel    }{0}

\IDmacro*{nparallel   }{0}

\extraspace

\IDmacro*{stdparallel }{0}

\IDmacro*{stdnparallel}{0}


\end{document}
