\documentclass[12pt,a4paper]{article}

\makeatletter
	\usepackage[utf8]{inputenc}
\usepackage[T1]{fontenc}
\usepackage{ucs}

\usepackage[french]{babel,varioref}

\usepackage[top=2cm, bottom=2cm, left=1.5cm, right=1.5cm]{geometry}
\usepackage{enumitem}

\usepackage{multicol}

\usepackage{color}
\usepackage{hyperref}
\hypersetup{
    colorlinks,
    citecolor=black,
    filecolor=black,
    linkcolor=black,
    urlcolor=black
}

\usepackage{amsthm}

\usepackage{tcolorbox}
\tcbuselibrary{listingsutf8}

\usepackage{ifplatform}

\usepackage{ifthen}

\usepackage{cbdevtool}


% MISC

\newtcblisting{latexex}{%
	sharp corners,%
	left=1mm, right=1mm,%
	bottom=1mm, top=1mm,%
	colupper=red!75!blue,%
	listing side text
}

\newtcblisting{latexex-flat}{%
	sharp corners,%
	left=1mm, right=1mm,%
	bottom=1mm, top=1mm,%
	colupper=red!75!blue,%
}

\newtcblisting{latexex-alone}{%
	sharp corners,%
	left=1mm, right=1mm,%
	bottom=1mm, top=1mm,%
	colupper=red!75!blue,%
	listing only
}


\newcommand\env[1]{\texttt{#1}}
\newcommand\macro[1]{\env{\textbackslash{}#1}}



\setlength{\parindent}{0cm}
\setlist{noitemsep}

\theoremstyle{definition}
\newtheorem*{remark}{Remarque}

\usepackage[raggedright]{titlesec}

\titleformat{\paragraph}[hang]{\normalfont\normalsize\bfseries}{\theparagraph}{1em}{}
\titlespacing*{\paragraph}{0pt}{3.25ex plus 1ex minus .2ex}{0.5em}


\newcommand\separation{
	\medskip
	\hfill\rule{0.5\textwidth}{0.75pt}\hfill
	\medskip
}


\newcommand\extraspace{
	\vspace{0.25em}
}


\newcommand\ascii{\texttt{ASCII}}

	% == PACKAGES USED == %

\RequirePackage{amsmath}
\RequirePackage{relsize}
\RequirePackage{xparse}


% == DEFINITIONS == %

% Settable texts
\@ifpackagewith{babel}{french}{
    \newcommand\lymathsep{;}
    \newcommand\lymathsubsep{,}
    \newcommand\textopchoice{choix}
    \newcommand\textopcond{cond}
    \newcommand\textopdef{déf}
    \newcommand\textophyp{hyp}
    \newcommand\textopid{id}
}{
	\newcommand\lymathsep{,}
	\newcommand\lymathsubsep{;}
    \newcommand\textopchoice{choice}
    \newcommand\textopcond{cond}
    \newcommand\textopdef{def}
    \newcommand\textophyp{hyp}
    \newcommand\textopid{id}
}


\newcommand\textexplainleft{\{}
\newcommand\textexplainright{\}}
\newcommand\textexplainspacebefore{\qquad}
\newcommand\textexplainspacein{\qquad}


% Tools - Apply same macro to all arguments

% #1        : main macro
% #2        : macro to apply to arguments
% #3 and #4 : the two arguments
\newcommand\@apply@macro@two@args[4]{%
	#1{#2{#3}}{#2{#4}}%
}




% Tools - Intervals

\newcommand\@extra@phantom{%
	\vphantom{\relsize{1.25}{\text{$\displaystyle F_1^2$}}}%
}

\newcommand\@interval@tool@star[5]{%
	\ensuremath{ \left#1 \@extra@phantom \right. \!\! #2 #3 #4 \left. \@extra@phantom \!\! \right#5}%
}

\newcommand\@interval@tool@no@star[5]{\ensuremath{ \left#1 #2 #3 #4 \right#5}}


% Tools - Multi-arguments
%
% Source : the following lines come directly for the following post
%
%    * https://tex.stackexchange.com/a/475291/6880

\ExplSyntaxOn
% General purpose macro for defining other macros
	\NewDocumentCommand{\makemultiargument}{mmmmmo}{
		\lymath_multiarg:nnnnnn{#1}{#2}{#3}{#4}{#5}{#6}
	}
 
% Allocate a private variable
	\seq_new:N \l__lymath_generic_seq

% The internal version of the general purpose macro
	\cs_new_protected:Nn \lymath_multiarg:nnnnnn{
		% #1 = separator
		% #2 = multiargument
		% #3 = code before
	  	% #4 = code between
	  	% #5 = code after
	  	% #6 = ornament to items

		% A group allows nesting
		\group_begin:
	 	% Split the multiargument into parts
		\seq_set_split:Nnn \l__lymath_generic_seq { #1 } { #2 }
		% Apply the ornament to the items
	  	\tl_if_novalue:nF { #6 }{
	    	\seq_set_eq:NN \l__lymath_temp_seq \l__lymath_generic_seq
	    	\seq_set_map:NNn \l__lymath_generic_seq \l__lymath_generic_seq { #6 }
	   	}
		% Execute the <code before>
	  	#3
		% Deliver the items, with the chosen material between them
	  	\seq_use:Nn \l__lymath_generic_seq { #4 }
  		% Execute the <code after>
	 	#5
  		% End the group started at the beginning
	  	\group_end:
	}	
\ExplSyntaxOff


	\usepackage{03-2-cartesian-axes}
\makeatother


% == EXTRA == %

\usepackage[f]{esvect}
\usepackage{relsize}
\usepackage{yhmath}
\usepackage{xstring}


\makeatletter
    \newcommand\pt[1]{\mathrm{#1}}


	\newcommand\@no@point[1]{%
		\IfStrEq{#1}{i}{%
			\imath%
		}{%
			\IfStrEq{#1}{j}{%
				\jmath%
			}{%
				#1
			}%
		}%
	}

	\newcommand\vect{\@ifstar{\@vect@star}{\@vect@no@star}}
	\newcommand*\@vect@star[1]{\vv*{\@no@point{#1}}}
	\newcommand*\@vect@no@star[1]{\vv{\@no@point{#1}}}
\makeatother



\begin{document}

% \section{Géométrie}

\subsection{Nommer un repère}

\paragraph{Exemple d'utilisation 1 -- La méthode basique}

Commençons par la manière la plus basique d'écrire un repère \textit{(nous verrons d'autres méthodes qui peuvent être plus efficaces)}.

\begin{latexex}
$\axes{\pt{O} %
     | \pt{I} | \pt{J}}$
\end{latexex}


% ---------------------- %


\paragraph{Exemple d'utilisation 2 -- La méthode basique en version étoilée}

Dans l'exemple ci-dessous, on voit que la version étoilée produit des petites parenthèses.
\begin{latexex}
$\displaystyle
 \axes{\pt{O} %
     | \frac{7}{3} \vect{i} %
     | \vect{j}}$
ou
$\displaystyle
 \axes*{\pt{O} %
     | \frac{7}{3} \vect{i} %
     | \vect{j}}$
\end{latexex}


% ---------------------- %


\paragraph{Exemple d'utilisation 3 -- La méthode basique en dimension quelconque}

Il faut au minimum deux "morceaux" séparés par des barres \verb+|+, cas de la dimension $1$, mais il n'y a pas de maximum, cas d'une dimension quelconque $n > 0$.

\begin{latexex}
$\axes{\pt{O} %
     | \vect*{i}{1} %
     | \vect*{i}{2} %
     | \vect*{i}{3} %
     | \dots %
     | \vect*{i}{9} %
     | \vect*{i}{10} %
     | \vect*{i}{11} %
     | \vect*{i}{12}}$
\end{latexex}


% ---------------------- %


\paragraph{Exemple d'utilisation 4 -- Repère affine}

Dans l'exemple suivant, le préfixe \verb+p+ est pour \textbf{p}-oint.

\begin{latexex}
$\paxes{O | I | J | K}$
au lieu de
$\axes{\pt{O} %
     | \pt{I} | \pt{J} | \pt{K}}$
\end{latexex}


% ---------------------- %


\paragraph{Exemple d'utilisation 5 -- Repère vectoriel (méthode 1)}

Dans l'exemple suivant, le préfixe \verb+v+ est pour \textbf{v}-ecteur.

\begin{latexex}
$\vaxes{\pt{O} | i | j}$
au lieu de
$\axes{\pt{O} | \vect{i} | \vect{j}}$
\end{latexex}


% ---------------------- %


\paragraph{Exemple d'utilisation 6 -- Repère vectoriel (méthode 2)}

Dans l'exemple suivant, le préfixe \verb+pv+ permet de combiner ensemble les fonctionnalités proposées par les préfixes \verb+p+ et \verb+v+.

\begin{latexex}
$\pvaxes{O | i | j}$
au lieu de
$\axes{\pt{O} | \vect{i} | \vect{j}}$
\end{latexex}


% ---------------------- %


\subsubsection{Fiches techniques}

\paragraph{Nommer un repère}

\IDmacro*{axes}{1}

\IDmacro*{axes*}{1}

\IDarg{} l'argument est une suite de "morceaux" séparés par des barres \verb+|+.

\begin{itemize}[topsep=0pt]
	\item Le premier morceau est l'origine du repère.

	\item Les morceaux suivants sont des points ou des vecteurs qui "définissent" chaque axe.
\end{itemize}


\separation

\IDmacro*{paxes}{1} où \quad \verb+p = p-oint+

\IDarg{} l'argument est une suite de "morceaux" séparés par des barres \verb+|+.

\begin{itemize}[topsep=0pt]
	\item Le premier morceau est le nom de l'origine du repère sur laquelle la macro-commande \macro{pt} sera automatiquement appliquée.

	\item Viennent ensuite les noms des points "définissant" chaque axe. Pour chacun de ces points la macro-commande \macro{pt} sera automatiquement appliquée.
\end{itemize}


\separation

\IDmacro*{vaxes}{1} où \quad \verb+v = v-ector+

\IDarg{} l'argument est une suite de "morceaux" séparés par des barres \verb+|+.

\begin{itemize}[topsep=0pt]
	\item Le premier morceau est l'origine du repère.

	\item Viennent ensuite les noms des vecteurs "définissant" chaque axe. Pour chacun de ces vecteurs la macro-commande \macro{vect} sera automatiquement appliquée.
\end{itemize}


\separation

\IDmacro*{pvaxes}{3} où \quad \verb&pv = p + v&

\IDarg{} l'argument est une suite de "morceaux" séparés par des barres \verb+|+.

\begin{itemize}[topsep=0pt]
	\item Le premier morceau est le nom de l'origine du repère sur laquelle la macro-commande \macro{pt} sera automatiquement appliquée.

	\item Viennent ensuite les noms des vecteurs "définissant" chaque axe. Pour chacun de ces vecteurs la macro-commande \macro{vect} sera automatiquement appliquée.
\end{itemize}

\end{document}
