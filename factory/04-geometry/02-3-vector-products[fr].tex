\documentclass[12pt,a4paper]{article}

\makeatletter
	\usepackage[utf8]{inputenc}
\usepackage[T1]{fontenc}
\usepackage{ucs}

\usepackage[french]{babel,varioref}

\usepackage[top=2cm, bottom=2cm, left=1.5cm, right=1.5cm]{geometry}
\usepackage{enumitem}

\usepackage{multicol}

\usepackage{makecell}

\usepackage{color}
\usepackage{hyperref}
\hypersetup{
    colorlinks,
    citecolor=black,
    filecolor=black,
    linkcolor=black,
    urlcolor=black
}

\usepackage{amsthm}

\usepackage{tcolorbox}
\tcbuselibrary{listingsutf8}

\usepackage{ifplatform}

\usepackage{ifthen}

\usepackage{cbdevtool}


% MISC

\newtcblisting{latexex}{%
	sharp corners,%
	left=1mm, right=1mm,%
	bottom=1mm, top=1mm,%
	colupper=red!75!blue,%
	listing side text
}

\newtcblisting{latexex-flat}{%
	sharp corners,%
	left=1mm, right=1mm,%
	bottom=1mm, top=1mm,%
	colupper=red!75!blue,%
}

\newtcblisting{latexex-alone}{%
	sharp corners,%
	left=1mm, right=1mm,%
	bottom=1mm, top=1mm,%
	colupper=red!75!blue,%
	listing only
}


\newcommand\env[1]{\texttt{#1}}
\newcommand\macro[1]{\env{\textbackslash{}#1}}



\setlength{\parindent}{0cm}
\setlist{noitemsep}

\theoremstyle{definition}
\newtheorem*{remark}{Remarque}

\usepackage[raggedright]{titlesec}

\titleformat{\paragraph}[hang]{\normalfont\normalsize\bfseries}{\theparagraph}{1em}{}
\titlespacing*{\paragraph}{0pt}{3.25ex plus 1ex minus .2ex}{0.5em}


\newcommand\separation{
	\medskip
	\hfill\rule{0.5\textwidth}{0.75pt}\hfill
	\medskip
}


\newcommand\extraspace{
	\vspace{0.25em}
}


\newcommand\whyprefix[2]{%
	\textbf{\prefix{#1}}-#2%
}

\newcommand\mwhyprefix[2]{%
	\texttt{#1 = #1-#2}%
}

\newcommand\prefix[1]{%
	\texttt{#1}%
}


\newcommand\inenglish{\@ifstar{\@inenglish@star}{\@inenglish@no@star}}

\newcommand\@inenglish@star[1]{%
	\emph{\og #1 \fg}%
}

\newcommand\@inenglish@no@star[1]{%
	\@inenglish@star{#1} en anglais%
}


\newcommand\ascii{\texttt{ASCII}}


% Example
\newcounter{paraexample}[subsubsection]

\newcommand\@newexample@abstract[2]{%
	\paragraph{%
		#1%
		\if\relax\detokenize{#2}\relax\else {} -- #2\fi%
	}%
}



\newcommand\newparaexample{\@ifstar{\@newparaexample@star}{\@newparaexample@no@star}}

\newcommand\@newparaexample@no@star[1]{%
	\refstepcounter{paraexample}%
	\@newexample@abstract{Exemple \theparaexample}{#1}%
}

\newcommand\@newparaexample@star[1]{%
	\@newexample@abstract{Exemple}{#1}%
}


% Change log
\newcommand\topic{\@ifstar{\@topic@star}{\@topic@no@star}}

\newcommand\@topic@no@star[1]{%
	\textbf{\textsc{#1}.}%
}

\newcommand\@topic@star[1]{%
	\textbf{\textsc{#1} :}%
}






	% == PACKAGES USED == %

\RequirePackage{amsmath}
\RequirePackage{relsize}
\RequirePackage{xparse}


% == DEFINITIONS == %

% Settable texts
\@ifpackagewith{babel}{french}{
    \newcommand\lymathsep{;}
    \newcommand\lymathsubsep{,}

    \newcommand\textopchoice{choix}
    \newcommand\textopcond{cond}
    \newcommand\textopdef{déf}
    \newcommand\textophyp{hyp}
    \newcommand\textopid{id}
    \newcommand\textoptest{?}
}{
    \newcommand\lymathsep{,}
    \newcommand\lymathsubsep{;}

    \newcommand\textopchoice{choice}
    \newcommand\textopcond{cond}
    \newcommand\textopdef{def}
    \newcommand\textophyp{hyp}
    \newcommand\textopid{id}
    \newcommand\textoptest{?}
}


\newcommand\textexplainleft{\{}
\newcommand\textexplainright{\}}
\newcommand\textexplainspacein{2em}


% Tools - Apply same macro to all arguments

% #1        : main macro
% #2        : macro to apply to arguments
% #3 and #4 : the two arguments
\newcommand\@apply@macro@two@args[4]{%
    #1{#2{#3}}{#2{#4}}%
}


% Tools - Deco over a math symbol

\newcommand\@over@math@symbol[2]{%
	\mathrel{\overset{\mathrm{\text{\raisebox{.5ex}{#1}}}}{#2}}%
}


% Tools - Intervals

\newcommand\@extra@phantom{%
    \vphantom{\relsize{1.25}{\text{$\displaystyle F_1^2$}}}%
}

\newcommand\@interval@tool@star[5]{%
    \ensuremath{ \left#1 \@extra@phantom \right. \!\! #2 #3 #4 \left. \@extra@phantom \!\! \right#5}%
}

\newcommand\@interval@tool@no@star[5]{\ensuremath{ \left#1 #2 #3 #4 \right#5}}


% Tools - Multi-arguments
%
% Source : the following lines come directly for the following post
%
%    * https://tex.stackexchange.com/a/475291/6880

\ExplSyntaxOn
% General purpose macro for defining other macros
    \NewDocumentCommand{\makemultiargument}{mmmmmo}{
        \lymath_multiarg:nnnnnn{#1}{#2}{#3}{#4}{#5}{#6}
    }
 
% Allocate a private variable
    \seq_new:N \l__lymath_generic_seq

% The internal version of the general purpose macro
    \cs_new_protected:Nn \lymath_multiarg:nnnnnn{
        % #1 = separator
        % #2 = multiargument
        % #3 = code before
          % #4 = code between
          % #5 = code after
          % #6 = ornament to items

        % A group allows nesting
        \group_begin:
         % Split the multiargument into parts
        \seq_set_split:Nnn \l__lymath_generic_seq { #1 } { #2 }
        % Apply the ornament to the items
          \tl_if_novalue:nF { #6 }{
            \seq_set_eq:NN \l__lymath_temp_seq \l__lymath_generic_seq
            \seq_set_map:NNn \l__lymath_generic_seq \l__lymath_generic_seq { #6 }
           }
        % Execute the <code before>
          #3
        % Deliver the items, with the chosen material between them
          \seq_use:Nn \l__lymath_generic_seq { #4 }
          % Execute the <code after>
         #5
          % End the group started at the beginning
          \group_end:
    }    
\ExplSyntaxOff


	\usepackage{02-3-vector-products}
\makeatother


% == EXTRA == %

\usepackage[f]{esvect}
\usepackage{relsize}
\usepackage{yhmath}
\usepackage{xstring}


\makeatletter
    \newcommand\pt[1]{\mathrm{#1}}

	\newcommand\@no@point[1]{%
		\IfStrEq{#1}{i}{%
			\imath%
		}{%
			\IfStrEq{#1}{j}{%
				\jmath%
			}{%
				#1
			}%
		}%
	}

	\newcommand\vect{\@ifstar{\@vect@star}{\@vect@no@star}}
	\newcommand*\@vect@star[1]{\vv*{\@no@point{#1}}}
	\newcommand*\@vect@no@star[1]{\vv{\@no@point{#1}}}
\makeatother



\begin{document}

% \section{Géométrie}

%\subsection{Vecteurs}

		\subsubsection{Produit scalaire -- Écriture minimaliste}

\paragraph{Exemple d'utilisation - Version longue}

\begin{tcblisting}{}
En mathématique, il est usage d'écrire un produit scalaire avec un point via
$\dotprod{\dfrac{1}{2} \vect{i}}{\vect{j}}$ .
\end{tcblisting}


\paragraph{Exemple d'utilisation - Version courte mais restrictive}

Dans l'exemple suivant, le préfixe \verb+v+ est pour \textbf{v}-ector.

\begin{tcblisting}{}
On peut aussi parfois juste taper $\vdotprod{i}{j}$ .
\end{tcblisting}


\paragraph{Fiches techniques}

\IDmacro*{dotprod}{2}

\IDarg{1} le premier vecteur qu'il faut taper via la macro \verb+\vect+.

\IDarg{2} le second vecteur qu'il faut taper via la macro \verb+\vect+.


\bigskip


\IDmacro*{vdotprod}{2} où \quad \verb+v = v-ector+

\IDarg{1} le nom du premier vecteur sans utiliser la macro \verb+\vect+.

\IDarg{2} le nom du second vecteur sans utiliser la macro \verb+\vect+.



		\subsubsection{Produit scalaire -- Écriture \og physicienne \fg}

Dans l'exemple suivant, le préfixe \verb+a+ est pour \textbf{a}-ngle, et  \verb+v+ pour \textbf{v}-ector.

\begin{tcblisting}{}
Les physiciens pourront utiliser
$\displaystyle \adotprod{\frac{1}{2} \vect{i}}{\vect{j}}$ ,
$\displaystyle \adotprod*{\frac{1}{2} \vect{i}}{\vect{j}}$ ,
$\displaystyle \vadotprod{i}{j}$
ou
$\displaystyle \vadotprod*{i}{j}$ .
\end{tcblisting}


\paragraph{Fiches techniques}

\IDmacro*{adotprod}{2} où \quad \verb&a = a-ngle&

\IDmacro*{adotprod*}{2}

\IDarg{1} le premier vecteur qu'il faut taper via la macro \verb+\vect+.

\IDarg{2} le second vecteur qu'il faut taper via la macro \verb+\vect+.


\bigskip


\IDmacro*{vadotprod}{2} où \quad \verb&a = a-ngle& et \verb+v = v-ector+

\IDarg{1} le nom du premier vecteur sans utiliser la macro \verb+\vect+.

\IDarg{2} le nom du second vecteur sans utiliser la macro \verb+\vect+.





		\subsubsection{Produit vectoriel}

\paragraph{Exemple d'utilisation - Version longue}

\begin{tcblisting}{}
Un produit vectoriel peut s'écrire via $\crossprod{\dfrac{1}{2} \vect{i}}{\vect{j}}$ .
\end{tcblisting}


\paragraph{Exemple d'utilisation - Version courte mais restrictive}

\begin{tcblisting}{}
Dans certain cas, un produit vectoriel s'écrit vite via $\vcrossprod{i}{j}$ .
\end{tcblisting}


\paragraph{Fiche technique}

\IDmacro*{crossprod}{2}

\IDarg{1} le premier vecteur qu'il faut taper via la macro \verb+\vect+.

\IDarg{2} le second vecteur qu'il faut taper via la macro \verb+\vect+.


\bigskip


\IDmacro*{vcrossprod}{2} où \quad \verb+v = v-ector+

\IDarg{1} le nom du premier vecteur sans utiliser la macro \verb+\vect+.

\IDarg{2} le nom du second vecteur sans utiliser la macro \verb+\vect+.

\end{document}
