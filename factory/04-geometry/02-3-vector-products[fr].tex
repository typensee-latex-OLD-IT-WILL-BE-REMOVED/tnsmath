\documentclass[12pt,a4paper]{article}

\makeatletter
	\usepackage[utf8]{inputenc}
\usepackage[T1]{fontenc}
\usepackage{ucs}

\usepackage[french]{babel,varioref}

\usepackage[top=2cm, bottom=2cm, left=1.5cm, right=1.5cm]{geometry}
\usepackage{enumitem}

\usepackage{multicol}

\usepackage{color}
\usepackage{hyperref}
\hypersetup{
    colorlinks,
    citecolor=black,
    filecolor=black,
    linkcolor=black,
    urlcolor=black
}

\usepackage{amsthm}

\usepackage{tcolorbox}
\tcbuselibrary{listingsutf8}

\usepackage{ifplatform}

\usepackage{ifthen}

\usepackage{cbdevtool}


% MISC

\newtcblisting{latexex}{%
	sharp corners,%
	left=1mm, right=1mm,%
	bottom=1mm, top=1mm,%
	colupper=red!75!blue,%
	listing side text
}

\newtcblisting{latexex-flat}{%
	sharp corners,%
	left=1mm, right=1mm,%
	bottom=1mm, top=1mm,%
	colupper=red!75!blue,%
}

\newtcblisting{latexex-alone}{%
	sharp corners,%
	left=1mm, right=1mm,%
	bottom=1mm, top=1mm,%
	colupper=red!75!blue,%
	listing only
}


\newcommand\env[1]{\texttt{#1}}
\newcommand\macro[1]{\env{\textbackslash{}#1}}



\setlength{\parindent}{0cm}
\setlist{noitemsep}

\theoremstyle{definition}
\newtheorem*{remark}{Remarque}

\usepackage[raggedright]{titlesec}

\titleformat{\paragraph}[hang]{\normalfont\normalsize\bfseries}{\theparagraph}{1em}{}
\titlespacing*{\paragraph}{0pt}{3.25ex plus 1ex minus .2ex}{0.5em}


\newcommand\separation{
	\medskip
	\hfill\rule{0.5\textwidth}{0.75pt}\hfill
	\medskip
}


\newcommand\extraspace{
	\vspace{0.25em}
}


\newcommand\ascii{\texttt{ASCII}}


	\usepackage{02-3-vector-products}
\makeatother


% == EXTRA == %

\usepackage[f]{esvect}

% == EXTRA == %

\usepackage[f]{esvect}
\usepackage{relsize}
\usepackage{yhmath}
\usepackage{xstring}


\makeatletter
    \newcommand\@interval@tool@star[5]{%
    	\ensuremath{ \left#1 \vphantom{\relsize{1.25}{\text{$\displaystyle F_1^2$}}} \right. \!\! #2 #3 #4 \left. \vphantom{\relsize{1.25}{\text{$\displaystyle F_1^2$}}} \!\! \right#5}%
	}

    \newcommand\@interval@tool@no@star[5]{\ensuremath{ \left#1 #2 #3 #4 \right#5}}
    
    \newcommand\pt[1]{\mathrm{#1}}

	\newcommand\@no@point[1]{%
		\IfStrEq{#1}{i}{%
			\imath%
		}{%
			\IfStrEq{#1}{j}{%
				\jmath%
			}{%
				#1
			}%
		}%
	}
	
	\newcommand\vect{\@ifstar{\@vect@star}{\@vect@no@star}}
	\newcommand*\@vect@star[1]{\vv*{\@no@point{#1}}}
	\newcommand*\@vect@no@star[1]{\vv{\@no@point{#1}}}
\makeatother



\begin{document}

% \section{Géométrie}

%    \subsection{Vecteurs}

		\subsubsection{Produit scalaire}

            \paragraph{Exemple d'utilisation 1}

\begin{tcblisting}{}
En mathématique, il est usage d'écrire un produit scalaire avec un point via
$\dotprod{\vect{i}}{\vect{j}}$ .
\end{tcblisting}


            \paragraph{Exemple d'utilisation 2}

Dans l'exemple suivant, le préfixe \verb+a+ est pour \textbf{a}-ngle.

\begin{tcblisting}{}
Les physiciens pourront utiliser
$\displaystyle \adotprod{\frac{1}{2} \vect{i}}{\vect{j}}$
ou
$\displaystyle \adotprod*{\frac{1}{2} \vect{i}}{\vect{j}}$ .
\end{tcblisting}


            \paragraph{Fiches techniques}

\IDmacro*{dotprod}{2}

\IDarg{1} le premier vecteur.

\IDarg{2} le second vecteur.


\bigskip

\IDmacro*{adotprod}{2} où \quad \verb&a = a-ngle&

\IDmacro*{adotprod*}{2}

\IDarg{1} le premier vecteur.

\IDarg{2} le second vecteur.





		\subsubsection{Produit vectoriel}

            \paragraph{Exemple d'utilisation}

\begin{tcblisting}{}
Un produit vectoriel peut s'écrire via $\crossprod{\vect{i}}{\vect{j}}$ .
\end{tcblisting}


            \paragraph{Fiche technique}

\IDmacro*{crossprod}{2}

\IDarg{1} le premier vecteur.

\IDarg{2} le second vecteur.

\end{document}
