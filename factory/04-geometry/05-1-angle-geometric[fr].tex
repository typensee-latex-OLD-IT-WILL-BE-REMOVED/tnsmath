\documentclass[12pt,a4paper]{article}

\makeatletter
	\usepackage[utf8]{inputenc}
\usepackage[T1]{fontenc}
\usepackage{ucs}

\usepackage[french]{babel,varioref}

\usepackage[top=2cm, bottom=2cm, left=1.5cm, right=1.5cm]{geometry}
\usepackage{enumitem}

\usepackage{multicol}

\usepackage{color}
\usepackage{hyperref}
\hypersetup{
    colorlinks,
    citecolor=black,
    filecolor=black,
    linkcolor=black,
    urlcolor=black
}

\usepackage{amsthm}

\usepackage{tcolorbox}
\tcbuselibrary{listingsutf8}

\usepackage{ifplatform}

\usepackage{ifthen}

\usepackage{cbdevtool}


% MISC

\newtcblisting{latexex}{%
	sharp corners,%
	left=1mm, right=1mm,%
	bottom=1mm, top=1mm,%
	colupper=red!75!blue,%
	listing side text
}

\newtcblisting{latexex-flat}{%
	sharp corners,%
	left=1mm, right=1mm,%
	bottom=1mm, top=1mm,%
	colupper=red!75!blue,%
}

\newtcblisting{latexex-alone}{%
	sharp corners,%
	left=1mm, right=1mm,%
	bottom=1mm, top=1mm,%
	colupper=red!75!blue,%
	listing only
}


\newcommand\env[1]{\texttt{#1}}
\newcommand\macro[1]{\env{\textbackslash{}#1}}



\setlength{\parindent}{0cm}
\setlist{noitemsep}

\theoremstyle{definition}
\newtheorem*{remark}{Remarque}

\usepackage[raggedright]{titlesec}

\titleformat{\paragraph}[hang]{\normalfont\normalsize\bfseries}{\theparagraph}{1em}{}
\titlespacing*{\paragraph}{0pt}{3.25ex plus 1ex minus .2ex}{0.5em}


\newcommand\separation{
	\medskip
	\hfill\rule{0.5\textwidth}{0.75pt}\hfill
	\medskip
}


\newcommand\extraspace{
	\vspace{0.25em}
}


\newcommand\ascii{\texttt{ASCII}}


	\usepackage{05-1-angle-geometric}
\makeatother


% == EXTRA == %

\makeatletter
	\newcommand\@no@point[1]{%
		\if#1i%
			\imath%
		\else%
			\if#1j%
				\jmath%
			\else%
				#1
			\fi%
		\fi%
	}
\makeatother



\begin{document}

% \section{Géométrie}

\subsection{Angles}

\subsubsection{Angles géométriques \og intérieurs \fg}

\newparaexample{}

\begin{latexex}
$\anglein {ABCDEF}$

$\anglein* {A}{rot}$

$\anglein {A_{rot}}$
\end{latexex}


% ---------------------- %


\newparaexample{Cacher les points du i et du j}

\begin{latexex}
$\anglein{i}$ et
$\anglein*{j}{2}$
\end{latexex}


% ---------------------- %


\subsubsection{Fiche technique}

\paragraph{Angles géométriques \og intérieurs \fg}

\IDmacro*{anglein}{1} où \quad \mwhyprefix{in}{terior}

\IDarg{} un texte donnant le nom d'un angle intérieur.


\separation


\IDmacro*{anglein*}{2} où \quad \mwhyprefix{in}{terior}

\IDarg{1} un texte indiquant $up$ dans le nom $\anglein*{up}{down}$ d'un angle intérieur.

\IDarg{2} un texte indiquant $down$ dans le nom $\anglein*{up}{down}$ d'un angle intérieur.

\end{document}
