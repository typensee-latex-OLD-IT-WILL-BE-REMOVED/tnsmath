\documentclass[12pt,a4paper]{article}

\makeatletter
	\usepackage[utf8]{inputenc}
\usepackage[T1]{fontenc}
\usepackage{ucs}

\usepackage[french]{babel,varioref}

\usepackage[top=2cm, bottom=2cm, left=1.5cm, right=1.5cm]{geometry}
\usepackage{enumitem}

\usepackage{multicol}

\usepackage{makecell}

\usepackage{color}
\usepackage{hyperref}
\hypersetup{
    colorlinks,
    citecolor=black,
    filecolor=black,
    linkcolor=black,
    urlcolor=black
}

\usepackage{amsthm}

\usepackage{tcolorbox}
\tcbuselibrary{listingsutf8}

\usepackage{ifplatform}

\usepackage{ifthen}

\usepackage{cbdevtool}


% MISC

\newtcblisting{latexex}{%
	sharp corners,%
	left=1mm, right=1mm,%
	bottom=1mm, top=1mm,%
	colupper=red!75!blue,%
	listing side text
}

\newtcblisting{latexex-flat}{%
	sharp corners,%
	left=1mm, right=1mm,%
	bottom=1mm, top=1mm,%
	colupper=red!75!blue,%
}

\newtcblisting{latexex-alone}{%
	sharp corners,%
	left=1mm, right=1mm,%
	bottom=1mm, top=1mm,%
	colupper=red!75!blue,%
	listing only
}


\newcommand\env[1]{\texttt{#1}}
\newcommand\macro[1]{\env{\textbackslash{}#1}}



\setlength{\parindent}{0cm}
\setlist{noitemsep}

\theoremstyle{definition}
\newtheorem*{remark}{Remarque}

\usepackage[raggedright]{titlesec}

\titleformat{\paragraph}[hang]{\normalfont\normalsize\bfseries}{\theparagraph}{1em}{}
\titlespacing*{\paragraph}{0pt}{3.25ex plus 1ex minus .2ex}{0.5em}


\newcommand\separation{
	\medskip
	\hfill\rule{0.5\textwidth}{0.75pt}\hfill
	\medskip
}


\newcommand\extraspace{
	\vspace{0.25em}
}


\newcommand\whyprefix[2]{%
	\textbf{\prefix{#1}}-#2%
}

\newcommand\mwhyprefix[2]{%
	\texttt{#1 = #1-#2}%
}

\newcommand\prefix[1]{%
	\texttt{#1}%
}


\newcommand\inenglish{\@ifstar{\@inenglish@star}{\@inenglish@no@star}}

\newcommand\@inenglish@star[1]{%
	\emph{\og #1 \fg}%
}

\newcommand\@inenglish@no@star[1]{%
	\@inenglish@star{#1} en anglais%
}


\newcommand\ascii{\texttt{ASCII}}


% Example
\newcounter{paraexample}[subsubsection]

\newcommand\@newexample@abstract[2]{%
	\paragraph{%
		#1%
		\if\relax\detokenize{#2}\relax\else {} -- #2\fi%
	}%
}



\newcommand\newparaexample{\@ifstar{\@newparaexample@star}{\@newparaexample@no@star}}

\newcommand\@newparaexample@no@star[1]{%
	\refstepcounter{paraexample}%
	\@newexample@abstract{Exemple \theparaexample}{#1}%
}

\newcommand\@newparaexample@star[1]{%
	\@newexample@abstract{Exemple}{#1}%
}


% Change log
\newcommand\topic{\@ifstar{\@topic@star}{\@topic@no@star}}

\newcommand\@topic@no@star[1]{%
	\textbf{\textsc{#1}.}%
}

\newcommand\@topic@star[1]{%
	\textbf{\textsc{#1} :}%
}






	\usepackage{01-general-sets}

	\usepackage{02-intervals}
\makeatother


\begin{document}

\section{Intervalles}

\subsection{Intervalles réels - Notation française (?\,)}

\newparaexample{}

Dans cet exemple, la syntaxe fait référence à 
\whyprefix{O}{pened} et \whyprefix{C}{losed}
pour
\inenglish{ouvert et fermé}.
Nous verrons que \prefix{CC} et \prefix{OO} sont contractés en \prefix{C} et \prefix{O}.
Notez au passage que la macro utilisée résout un problème d'espacement vis à vis du signe $=$ .

\begin{latexex}
$I = ]a ; b] = \intervalOC{a}{b}$
\end{latexex}


% ---------------------- %


\newparaexample{}

Les crochets s'étendent verticalement automatiquement. Pour empêcher cela, il suffit d'utiliser la version étoilée de la macro.
Dans ce cas, les crochets restent tout de même un peu plus grands que des crochets utilisés directement. Voici un exemple.

\begin{latexex}
$\displaystyle
 \intervalC{ \frac{1}{2} }{ 1^{2^{3}} }
 =
 [ \frac{1}{2} ; 1^{2^{3}} ]
 =
 \intervalC*{ \frac{1}{2} }{ 1^{2^{3}} }$
\end{latexex}


% ---------------------- %


\subsection{Fiches techniques}

Pour toutes les macros ci-dessous, la version non étoilée produit des délimiteurs qui s'étirent si besoin verticalement, tandis que la version étoilée ne le fait pas.


\separation


% Docs for french real intervals - START

\IDmacro*{intervalCO}{2}

\IDmacro*{intervalCO*}{2}

\IDarg{1} borne inférieure $a$ de l'intervalle $\intervalCO{a}{b}$.

\IDarg{2} borne supérieure $b$ de l'intervalle $\intervalCO{a}{b}$.


\separation


\IDmacro*{intervalC}{2}

\IDmacro*{intervalC*}{2}

\IDarg{1} borne inférieure $a$ de l'intervalle $\intervalC{a}{b}$.

\IDarg{2} borne supérieure $b$ de l'intervalle $\intervalC{a}{b}$.


\separation


\IDmacro*{intervalO}{2}

\IDmacro*{intervalO*}{2}

\IDarg{1} borne inférieure $a$ de l'intervalle $\intervalO{a}{b}$.

\IDarg{2} borne supérieure $b$ de l'intervalle $\intervalO{a}{b}$.


\separation


\IDmacro*{intervalOC}{2}

\IDmacro*{intervalOC*}{2}

\IDarg{1} borne inférieure $a$ de l'intervalle $\intervalOC{a}{b}$.

\IDarg{2} borne supérieure $b$ de l'intervalle $\intervalOC{a}{b}$.

% Docs for french real intervals - END


% ---------------------- %


\subsection{Intervalles réels -- Notation américaine}

Dans l'exemple suivant la syntaxe fait référence à \whyprefix{P}{arenthèse}. Cette notation est utilisée aux États Unis.

\begin{latexex}
$\intervalPC{a}{b} = \intervalOC{a}{b}$
et
$\intervalP{a}{b} = \intervalO{a}{b}$.
\end{latexex}


% ---------------------- %


\subsection{Fiches techniques}

Pour toutes les macros ci-dessous, la version non étoilée produit des délimiteurs qui s'étirent si besoin verticalement, tandis que la version étoilée ne le fait pas.


\separation


% Docs for american real intervals - START

\IDmacro*{intervalCP}{2}

\IDmacro*{intervalCP*}{2}

\IDarg{1} borne inférieure $a$ de l'intervalle $\intervalCP{a}{b}$.

\IDarg{2} borne supérieure $b$ de l'intervalle $\intervalCP{a}{b}$.


\separation


\IDmacro*{intervalP}{2}

\IDmacro*{intervalP*}{2}

\IDarg{1} borne inférieure $a$ de l'intervalle $\intervalP{a}{b}$.

\IDarg{2} borne supérieure $b$ de l'intervalle $\intervalP{a}{b}$.


\separation


\IDmacro*{intervalPC}{2}

\IDmacro*{intervalPC*}{2}

\IDarg{1} borne inférieure $a$ de l'intervalle $\intervalPC{a}{b}$.

\IDarg{2} borne supérieure $b$ de l'intervalle $\intervalPC{a}{b}$.

% Docs for american real intervals - END


% ---------------------- %


\subsection{Intervalles discrets d'entiers}


Dans l'exemple suivant la syntaxe fait référence à $\ZZ$ l'ensemble des entiers relatifs.

\begin{latexex}
 $\ZintervalC{-1}{4} =
  \{ -1 ; 0 ; 1 ; 2 ; 3 ; 4 \}$
 
 $\ZintervalC{-1}{4} =
  \ZintervalO{-2}{5}$.
\end{latexex}


% ---------------------- %


\subsection{Fiches techniques}

Pour toutes les macros ci-dessous, la version non étoilée produit des délimiteurs qui s'étirent si besoin verticalement, tandis que la version étoilée ne le fait pas.


\separation


% Docs for discrete intervals - START

\IDmacro*{ZintervalCO}{2}

\IDmacro*{ZintervalCO*}{2}

\IDarg{1} borne inférieure $a$ de l'intervalle $\ZintervalCO{a}{b}$.

\IDarg{2} borne supérieure $b$ de l'intervalle $\ZintervalCO{a}{b}$.


\separation


\IDmacro*{ZintervalC}{2}

\IDmacro*{ZintervalC*}{2}

\IDarg{1} borne inférieure $a$ de l'intervalle $\ZintervalC{a}{b}$.

\IDarg{2} borne supérieure $b$ de l'intervalle $\ZintervalC{a}{b}$.


\separation


\IDmacro*{ZintervalO}{2}

\IDmacro*{ZintervalO*}{2}

\IDarg{1} borne inférieure $a$ de l'intervalle $\ZintervalO{a}{b}$.

\IDarg{2} borne supérieure $b$ de l'intervalle $\ZintervalO{a}{b}$.


\separation


\IDmacro*{ZintervalOC}{2}

\IDmacro*{ZintervalOC*}{2}

\IDarg{1} borne inférieure $a$ de l'intervalle $\ZintervalOC{a}{b}$.

\IDarg{2} borne supérieure $b$ de l'intervalle $\ZintervalOC{a}{b}$.

% Docs for discrete intervals - END

\end{document}
