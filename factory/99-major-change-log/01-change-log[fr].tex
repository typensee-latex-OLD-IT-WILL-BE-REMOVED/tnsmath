\documentclass[12pt,a4paper]{article}

\makeatletter
	\usepackage[utf8]{inputenc}
\usepackage[T1]{fontenc}
\usepackage{ucs}

\usepackage[french]{babel,varioref}

\usepackage[top=2cm, bottom=2cm, left=1.5cm, right=1.5cm]{geometry}
\usepackage{enumitem}

\usepackage{multicol}

\usepackage{makecell}

\usepackage{color}
\usepackage{hyperref}
\hypersetup{
    colorlinks,
    citecolor=black,
    filecolor=black,
    linkcolor=black,
    urlcolor=black
}

\usepackage{amsthm}

\usepackage{tcolorbox}
\tcbuselibrary{listingsutf8}

\usepackage{ifplatform}

\usepackage{ifthen}

\usepackage{cbdevtool}


% MISC

\newtcblisting{latexex}{%
	sharp corners,%
	left=1mm, right=1mm,%
	bottom=1mm, top=1mm,%
	colupper=red!75!blue,%
	listing side text
}

\newtcblisting{latexex-flat}{%
	sharp corners,%
	left=1mm, right=1mm,%
	bottom=1mm, top=1mm,%
	colupper=red!75!blue,%
}

\newtcblisting{latexex-alone}{%
	sharp corners,%
	left=1mm, right=1mm,%
	bottom=1mm, top=1mm,%
	colupper=red!75!blue,%
	listing only
}


\newcommand\env[1]{\texttt{#1}}
\newcommand\macro[1]{\env{\textbackslash{}#1}}



\setlength{\parindent}{0cm}
\setlist{noitemsep}

\theoremstyle{definition}
\newtheorem*{remark}{Remarque}

\usepackage[raggedright]{titlesec}

\titleformat{\paragraph}[hang]{\normalfont\normalsize\bfseries}{\theparagraph}{1em}{}
\titlespacing*{\paragraph}{0pt}{3.25ex plus 1ex minus .2ex}{0.5em}


\newcommand\separation{
	\medskip
	\hfill\rule{0.5\textwidth}{0.75pt}\hfill
	\medskip
}


\newcommand\extraspace{
	\vspace{0.25em}
}


\newcommand\whyprefix[2]{%
	\textbf{\prefix{#1}}-#2%
}

\newcommand\mwhyprefix[2]{%
	\texttt{#1 = #1-#2}%
}

\newcommand\prefix[1]{%
	\texttt{#1}%
}


\newcommand\inenglish{\@ifstar{\@inenglish@star}{\@inenglish@no@star}}

\newcommand\@inenglish@star[1]{%
	\emph{\og #1 \fg}%
}

\newcommand\@inenglish@no@star[1]{%
	\@inenglish@star{#1} en anglais%
}


\newcommand\ascii{\texttt{ASCII}}


% Example
\newcounter{paraexample}[subsubsection]

\newcommand\@newexample@abstract[2]{%
	\paragraph{%
		#1%
		\if\relax\detokenize{#2}\relax\else {} -- #2\fi%
	}%
}



\newcommand\newparaexample{\@ifstar{\@newparaexample@star}{\@newparaexample@no@star}}

\newcommand\@newparaexample@no@star[1]{%
	\refstepcounter{paraexample}%
	\@newexample@abstract{Exemple \theparaexample}{#1}%
}

\newcommand\@newparaexample@star[1]{%
	\@newexample@abstract{Exemple}{#1}%
}


% Change log
\newcommand\topic{\@ifstar{\@topic@star}{\@topic@no@star}}

\newcommand\@topic@no@star[1]{%
	\textbf{\textsc{#1}.}%
}

\newcommand\@topic@star[1]{%
	\textbf{\textsc{#1} :}%
}






\makeatother



\begin{document}

\newpage

\section{Historique}

Tous les changements sont décrits en anglais uniquement dans le dossier \verb+change-log+ : voir le code source de \verb+lymath+ sur \verb+github+. Nous ne donnons ici qu'un très bref historique de \verb+lymath+ côté utilisateur principalement.

\begin{description}[leftmargin=1em]
	\setlength\itemsep{1em}


 
% --------------- %

%	\item[2019-????]  Nouvelle version mineure \verb+0.5.0-beta+ dont voici les principaux changements.
%	\begin{itemize}
%		\item Redéfinition de \verb+\sqrt+ pour un meilleur espacement.

%		\item new math operator \verb+\divides+, \verb+\notdivides+ et \verb+\modulo+
%	\end{itemize}

% --------------- %

	\item[2019-09-06]  Nouvelle version mineure \verb+0.4.0-beta+ dont voici les principaux changements.
	\begin{itemize}
		\item Dans \emph{\og Logique et fondements \fg}, différents types de signes d'inéquation et de non égalité pour des cas de test, d'hypothèse faite et de condition à vérifier.

		\item Intégration du package \verb+tkz-tab+ pour rédiger des tableaux de variations et de signes.

		\item Intégration du package \verb+nicematrix+ pour écrire des matrices.
	\end{itemize}

% --------------- %

	\item[2019-07-23] Nouvelle version mineure \verb+0.3.0-beta+ dont voici les principaux changements.
	\begin{itemize}
		\item Une nouvelle section \emph{\og Logique et fondements \fg} a été ajoutée.
		\begin{itemize}
			\item Trois types de signes $=$ décorés sont proposés : voir les macros \verb+\eqdef+ , \verb+\eqid+ et \verb+\eqtest+.

			\item Via la macro \verb+\explain+, il devient facile d'expliquer des étapes de raisonnement ou des calculs.
		\end{itemize}

		\item Pour les ensembles, la macro \verb+\fieldset+ a été renommé \verb+\algeset+ et la macro \verb+\PP+ permet d'indiquer l'ensemble des nombres premiers.

		\item En géométrie, il y a quelques nouveautés.
		\begin{itemize}
			\item La macro \verb+\hangleorient+ permet l'écriture d'angles orientés avec un chapeau en plus.

			\item Les macros \verb+\vangleorient+ et \verb+\vhangleorient+ évite d'avoir à utiliser \verb+\vect+ lorsque l'on a juste des vecturs simples nommés et non coefficientés.

			\item De même pour les macros \verb+\vdotprod+, \verb+\vadotprod+ et \verb+\vcroosprod+.
		\end{itemize}

		\item Ajout de \verb+\lymathsubsep+ qui définit le séparateur des arguments de second niveau.
	\end{itemize}

% --------------- %

	\item[2019-02-21] Nouvelle version mineure \verb+0.2.0-beta+ dont voici les principaux changements.
	\begin{itemize}
		\item L'usage de \verb+//+ pour les macros-commandes avec un nombre quelconque d'arguments a été remplacé par celui de \verb+|+.

		\item En géométrie, il y a diverses nouveautés.
		\begin{itemize}
			\item Ajout de l'écriture de coordonnées, de produits scalaires et de produits vectoriels.

			\item \verb+\axis+ a été correctement traduit en \verb+\axes+.

			\item Les macros \verb+\gpaxis+ et \verb+\gpvaxis+ deviennent \verb+\paxes+ et \verb+\pvaxes+ pour être cohérent avec \verb+\pt+ qui a remplacé l'ancien \verb+\gpt+.
		\end{itemize}

		\item En analyse, ajout de la macro commande étoilée \verb+\derpow*+ pour la gestion automatique des primes d'une dérivée.

		\item Une nouvelle section "algèbre" propose des macros pour écrire des ensembles de polynômes, de fractions polynomiales, de séries formelles, de fractions de séries formelles, et aussi de polynômes et de séries formelles de Laurent.

		\item Redéfinition de \verb+\frac+ et \verb+\dfrac+ pour obtenir des traits de fraction un peu plus longs.

		\item Ajout de \verb+\lymathsep+ qui définit le séparateur d'arguments.
	\end{itemize}

% --------------- %

	\item[2017-11-01] Nouvelle version mineure \verb+0.1.0-beta+ : pour les ensembles, les fonctions et la géométrie, il y a eu des changements et l'ajout de nouveaux outils.

% --------------- %

	\item[2017-10-21] Historique court de \verb+lymath+ ajouté au présent document.

% --------------- %

	\item[2017-10-18] Nouvelle version "patchée" \verb+0.0.2-beta+ : de nouveaux outils pour le calcul différentiel.

% --------------- %

	\item[2017-10-06] Nouvelle version "patchée" \verb+0.0.1-beta+ : de nouveaux outils pour l'arithmétique, la géométrie, le calcul intégral et le calcul différentiel.

% --------------- %

	\item[2017-10-02] Première version \verb+0.0.0-beta+ du package.
\end{description}

\end{document}
