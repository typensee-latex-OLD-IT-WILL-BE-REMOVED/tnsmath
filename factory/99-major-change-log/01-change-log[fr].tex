\documentclass[12pt,a4paper]{article}

\makeatletter
    \usepackage[utf8]{inputenc}
\usepackage[T1]{fontenc}
\usepackage{ucs}

\usepackage[french]{babel,varioref}

\usepackage[top=2cm, bottom=2cm, left=1.5cm, right=1.5cm]{geometry}
\usepackage{enumitem}

\usepackage{multicol}

\usepackage{makecell}

\usepackage{color}
\usepackage{hyperref}
\hypersetup{
    colorlinks,
    citecolor=black,
    filecolor=black,
    linkcolor=black,
    urlcolor=black
}

\usepackage{amsthm}

\usepackage{tcolorbox}
\tcbuselibrary{listingsutf8}

\usepackage{ifplatform}

\usepackage{ifthen}

\usepackage{cbdevtool}


% MISC

\newtcblisting{latexex}{%
	sharp corners,%
	left=1mm, right=1mm,%
	bottom=1mm, top=1mm,%
	colupper=red!75!blue,%
	listing side text
}

\newtcblisting{latexex-flat}{%
	sharp corners,%
	left=1mm, right=1mm,%
	bottom=1mm, top=1mm,%
	colupper=red!75!blue,%
}

\newtcblisting{latexex-alone}{%
	sharp corners,%
	left=1mm, right=1mm,%
	bottom=1mm, top=1mm,%
	colupper=red!75!blue,%
	listing only
}


\newcommand\env[1]{\texttt{#1}}
\newcommand\macro[1]{\env{\textbackslash{}#1}}



\setlength{\parindent}{0cm}
\setlist{noitemsep}

\theoremstyle{definition}
\newtheorem*{remark}{Remarque}

\usepackage[raggedright]{titlesec}

\titleformat{\paragraph}[hang]{\normalfont\normalsize\bfseries}{\theparagraph}{1em}{}
\titlespacing*{\paragraph}{0pt}{3.25ex plus 1ex minus .2ex}{0.5em}


\newcommand\separation{
	\medskip
	\hfill\rule{0.5\textwidth}{0.75pt}\hfill
	\medskip
}


\newcommand\extraspace{
	\vspace{0.25em}
}


\newcommand\whyprefix[2]{%
	\textbf{\prefix{#1}}-#2%
}

\newcommand\mwhyprefix[2]{%
	\texttt{#1 = #1-#2}%
}

\newcommand\prefix[1]{%
	\texttt{#1}%
}


\newcommand\inenglish{\@ifstar{\@inenglish@star}{\@inenglish@no@star}}

\newcommand\@inenglish@star[1]{%
	\emph{\og #1 \fg}%
}

\newcommand\@inenglish@no@star[1]{%
	\@inenglish@star{#1} en anglais%
}


\newcommand\ascii{\texttt{ASCII}}


% Example
\newcounter{paraexample}[subsubsection]

\newcommand\@newexample@abstract[2]{%
	\paragraph{%
		#1%
		\if\relax\detokenize{#2}\relax\else {} -- #2\fi%
	}%
}



\newcommand\newparaexample{\@ifstar{\@newparaexample@star}{\@newparaexample@no@star}}

\newcommand\@newparaexample@no@star[1]{%
	\refstepcounter{paraexample}%
	\@newexample@abstract{Exemple \theparaexample}{#1}%
}

\newcommand\@newparaexample@star[1]{%
	\@newexample@abstract{Exemple}{#1}%
}


% Change log
\newcommand\topic{\@ifstar{\@topic@star}{\@topic@no@star}}

\newcommand\@topic@no@star[1]{%
	\textbf{\textsc{#1}.}%
}

\newcommand\@topic@star[1]{%
	\textbf{\textsc{#1} :}%
}






\makeatother


\begin{document}

\newpage

\section{Historique}

Nous ne donnons ici qu'un très bref historique récent de \verb+lymath+ à destination de l'utilisateur principalement.
Tous les changements sont disponibles uniquement en anglais dans le dossier \verb+change-log+ : voir le code source de \verb+lymath+ sur \verb+github+.

\begin{description}
% Changes shown - START

    \medskip
    \item[2020-06-08] Nouvelle version mineure \verb+0.7.0-beta+.
    
    \begin{itemize}[itemsep=.5em]
        \item En analyse, il y a eu les changements suivants.
        \begin{itemize}[itemsep=.5em]
            \item Pour éviter des conflits avec d'autres packages les renommages suivants ont dû être faits.
            \begin{itemize}[itemsep=.5em]
                \item \macro{ch} et \macro{ach} sont devenus \macro{fch} et \macro{afch} où \prefix{f} est pour \whyprefix{f}{rench}.
    
                \item \macro{sh} et \macro{ash} sont devenus \macro{fsh} et \macro{afsh}.
    
                \item \macro{th} et \macro{ath} sont devenus \macro{fth} et \macro{afth}.
            \end{itemize}
    
    		\item Les macros \macro{acosh}, \macro{asinh} et \macro{atanh} été ajoutées.
    
     		\item \macro{derpar} et \macro{derpar*} servent à rédiger des dérivées avec des parenthèses extensibles. En coulisse, \macro{derpow} et \macro{derpow*} sont appelées.
    
            Pour utiliser des parenthèses non extensibles, on passera par \macro{sderpar} et \macro{sderpar*} où \prefix{s} est pour \whyprefix{s}{mall}.
    
    		\item Ajout de \macro{stdint} pour rendre public l'opérateur intégral proposé par défaut par \LaTeX.
        \end{itemize}
    
        % ------------ %
    
        \item En géométrie, il y a eu une suppression et trois ajouts.
        \begin{itemize}[itemsep=.5em]
            \item La macro \macro{pts} a été supprimée car sans signification sémantique puisqu'un point peut être nommé avec deux lettres.
    
            \item Les macros \macro{gline} et \macro{pgline} servent à indiquer des droites.
    
            \item La macro \macro{hgline}, avec \prefix{h} pour \whyprefix{h}{alf}, est pour les demi-droites.
    
            \item La macro \macro{segment} est utile pour les segments.
        \end{itemize}
    
        % ------------ %
    
        \item En logique, voici les améliorations apportées.
        \begin{itemize}[itemsep=.5em]
            \item Pour les inégalités, on peut maintenant utiliser le décorateur \verb+plot+.
    
    		\item Ajout des versions négatives des opérateurs logiques verticaux.
        \end{itemize}
    
        % ------------ %
    
        \item Pour les probabilités, il y a eu les modifications ci-après.
        \begin{itemize}[itemsep=.5em]
            \item Ajout de \macro{proba} pour écrire des probabilités.
    
    		\item Le comportement de \macro{probacond} a été modifié pour le rendre plus logique.
        \end{itemize}
    \end{itemize}

% ------------------------ %

    \medskip
    \item[2019-10-21] Nouvelle version sous-mineure \verb+0.6.3-beta+.
    
    \begin{itemize}[itemsep=.5em]
        \item Pour les intervalles, \macro{CSinterval} a été déplacée dans le package \verb+lyalgo+ disponible à l'adresse \url{https://github.com/bc-latex/ly-algo}.
    
        % ------------ %
    
        \item En logique, il y a eu les modifications suivantes.
        \begin{itemize}[itemsep=.5em]
            \item \macro{eqdef**} a été supprimé. Voir la macro \macro{Store*} du package \verb+lyalgo+.
    
            \item Différentes versions de l'opérateur $\exists$ via \macro{existsone} et \macro{existmulti} avec leurs versions négatives \macro{nexistsone} et \macro{nexistmulti}.
    
            \item Deux nouvelles macros \macro{eqplot} et \macro{eqappli} pour indiquer une équation de courbe et l'application d'une identité à des variables. Ceci s'accompagne de l'ajout des macros \macro{textopplot} et \macro{textopappli}.
    
            \item Ajout des formes négatives \macro{niff}, \macro{nimplies} et \macro{nliesimp}.
    
            \item Les décorations \verb+cons+, \verb+appli+ et \verb+choice+ sont utilisables avec les opérateurs \macro{iff}, \macro{implies} et \macro{liesimp} et leurs formes négatives.
    
            \item Une macro \macro{textoptest} a été ajoutée afin de rendre personnalisable tous les textes décorant les symboles.
        \end{itemize}
    
        % ------------ %
    
        \item En analyse, il y a eu les renommages suivants.
        \begin{itemize}[itemsep=.5em]
            \item \macro{hypergeo} est devenu \macro{seqhypergeo}.
    
            \item \macro{suprageo} est devenu \macro{seqsuprageo}.
        \end{itemize}
    
        % ------------ %
    
        \item En géométrie, \macro{notparallel} est devenu \macro{nparallel}.
    \end{itemize}

% ------------------------ %

    \medskip
    \item[2019-10-14] Nouvelle version sous-mineure \verb+0.6.2-beta+.
    
    \begin{itemize}[itemsep=.5em]
        \item En algèbre, il y a eu les renommages ci-dessous qui avaient été oubliés.
        \begin{itemize}[itemsep=.5em]
            \item \macro{polyset} est devenu \macro{setpoly}.
    
            \item \macro{polyfracset} est devenu \macro{setpolyfrac}.
    
            \item \macro{serieset} est devenu \macro{setserie}.
    
            \item \macro{seriefracset} est devenu \macro{setseriefrac}.
    
            \item \macro{polylaurentset} est devenu \macro{setpolylaurent}.
    
            \item \macro{serielaurentset} est devenu \macro{setserielaurent}.
        \end{itemize}
    \end{itemize}

% ------------------------ %

    \medskip
    \item[2019-10-13] Nouvelle version sous-mineure \verb+0.6.1-beta+.
    
    \begin{itemize}[itemsep=.5em]
        \item En logique, la macro \macro{explain} possède maintenant un argument optionnel pour indiquer l'espacement avant le symbole.
              Ceci s'accompagne de la suppression des macros obsolètes \macro{explain*} et \macro{textexplainspacebefore}.
    
        % ------------ %
    
        \item En probabilité, voici ce qui a évolué.
        \begin{itemize}[itemsep=.5em]
            \item Les macros \macro{probacond} et \macro{probacond*} n'ont plus d'argument optionnel. Pour obtenir l'écriture fractionnaire, il faut utiliser \macro{probacond**} ou \macro{dprobacond**}.
    
            \item Les environnements \verb+probatree+ et \verb+probatree*+ ont trois nouvelles clés.
                  La clé \verb+frame+ permet d'encadrer un sous-arbre, et les clés \verb+apweight+ et \verb+bpweight+ permettent d'écrire des poids dessus/dessous une branche.
        \end{itemize}
    
        % ------------ %
    
        \item Pour les ensembles, il y a eu les renommages suivants par souci de cohérence.
        \begin{itemize}[itemsep=.5em]
            \item \macro{algeset} est devenu \macro{setalge}.
    
            \item \macro{geoset} est devenu \macro{setgeo}.
    
            \item \macro{geneset} est devenu \macro{setgene}.
    
            \item \macro{probaset} est devenu \macro{setproba}.
    
            \item \macro{specialset} est devenu \macro{setspecial}.
        \end{itemize}
    \end{itemize}

% ------------------------ %

    \medskip
    \item[2019-10-10] Nouvelle version mineure \verb+0.6.0-beta+.
    
    \begin{itemize}[itemsep=.5em]
        \item Des nouveaux outils spécifiques aux probabilités.
        \begin{itemize}[itemsep=.5em]
            \item Les macros \macro{probacond} et \macro{probacond*} servent à écrire des probabilités conditionnelles.
    
            \item Les environnements \verb+probatree+ et \verb+probatree*+ simplifient la production d'arbres probabilistes pondérés ou non.
        \end{itemize}
    
        % ------------ %
    
        \item En géométrie, la macro \macro{notparallel} a été rajoutée.
    
        % ------------ %
    
        \item Un nouveau type d'intervalle pour l'informatique théorique via la macro \macro{CSinterval} afin d'obtenir quelque chose comme \verb+a..b+.
    
        % ------------ %
    
        \item En logique, il y a deux nouvelles macros sémantiques \macro{neqid} et \macro{eqchoice}.
    \end{itemize}

% ------------------------ %

    \medskip
    \item[2019-09-27] Nouvelle version mineure \verb+0.5.0-beta+.
    
    \begin{itemize}[itemsep=.5em]
        \item Ajout des macros \macro{dsum} et \macro{dprod} qui sont vis à vis de \macro{sum} et \macro{prod} des équivalents de \macro{dfrac} pour \macro{frac}.
    
        % ------------ %
    
        \item En arithmétique, ajout des opérateurs \macro{divides}, \macro{notdivides} et \macro{modulo}.
    
        % ------------ %
    
        \item En géométrie, une nouvelle macro et un opérateur modifié.
        \begin{itemize}[itemsep=.5em]
            \item \macro{pts} permet d'indiquer plusieurs points.
    
            \item \macro{parallel} utilise des obliques pour symboliser le parallélisme au lieu de barres verticales.
        \end{itemize}
    
        % ------------ %
    
        \item En logique, il y a les nouveautés suivantes.
        \begin{itemize}[itemsep=.5em]
            \item La version doublement étoilée \macro{eqdef**} donne une deuxième écriture symbolique d'un symbole égal de type définition \emph{(cette notation vient du langage B)}.
    
            \item Ajout de \macro{liesimp} comme alias de \macro{Longleftarrow}.
    
            \item Les macros \macro{vimplies}, \macro{viff} et \macro{vliesimp} sont des versions verticales de \macro{implies}, \macro{iff} et \macro{liesimp}.
    
            \item Comme pour les égalités, il existe les macros \macro{impliestest}, \macro{iffhyp} ... etc.
        \end{itemize}
    \end{itemize}

% ------------------------ %

    \medskip
    \item[2019-09-06] Nouvelle version mineure \verb+0.4.0-beta+.
    
    \begin{itemize}[itemsep=.5em]
        \item Dans \emph{\og Logique et fondements \fg}, différents types de signes d'inéquation et de non égalité pour des cas de test, d'hypothèse faite et de condition à vérifier.
    
        % ------------ %
    
        \item Intégration du package \verb+tkz-tab+ pour rédiger des tableaux de variations et de signes.
    
        % ------------ %
    
        \item Intégration du package \verb+nicematrix+ pour écrire des matrices.
    \end{itemize}

% ------------------------ %

% Changes shown - END 
\end{description}

\end{document}
