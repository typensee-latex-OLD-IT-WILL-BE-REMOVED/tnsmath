Nouvelle version mineure \verb+1.9.0-beta+.


% -------------- %
% -- tnsana -- %
% -------------- %


\begin{center}
    \textbf{\textsc{Analyse [0.8.0-beta]}}
\end{center}

\begin{itemize}[itemsep=.5em]
    \item \topic*{Fonctions avec un paramètre optionnel}
          les macros \macro{lg} et \macro{ln} peuvent s'utiliser avec une base.
    
\end{itemize}


\separation


% -------------- %
% -- tnsarith -- %
% -------------- %


\begin{center}
    \textbf{\textsc{Arithmétique [0.2.0-beta]}}
\end{center}

\begin{itemize}[itemsep=.5em]
    \item \topic*{Fractions continuées}
    	  ajout de \macro{scontfrac}, \macro{scontfrac*}, \macro{scontfracgene} et \macro{scontfracgene*} qui donnent juste la partie fractionnaire.
\end{itemize}


\separation


% -------------- %
% -- tnsgeo -- %
% -------------- %


\begin{center}
    \textbf{\textsc{Géométrie [0.3.0-beta]}}
\end{center}

\begin{itemize}[itemsep=.5em]
    \item \topic*{Vecteur}
          ajout de la macro \macro{pvect} pour ne pas avoir à taper \macro{pt}.

	% -------------- %

    \item \topic*{Produit vectoriel et déterminant de deux vecteurs} nouveau changement de l'API.
    \begin{itemize}[itemsep=.5em]
        \item \macro{calccrossprod}, \macro{vcalccrossprod}, \macro{calcdetplane} et \macro{vcalcdetplane} permettent de tracer une croix fléchée ou non à la place de la boucle fléchée.

        \item \macro{coordcrossprod} a été supprimé. Il faut à la place utiliser l'une des options \verb#exp#, \verb#texp# et \verb#cexp# de \macro{vcalccrossprod} et \macro{calccrossprod}.

        \item Plus aucune version étoilée simple ou double pour \macro{calccrossprod}, \macro{vcalccrossprod}, \macro{calcdetplane} et \macro{vcalcdetplane}.
    \end{itemize}

	% -------------- %

\end{itemize}


\separation


% -------------- %
% -- tnsproba -- %
% -------------- %


\begin{center}
    \textbf{\textsc{Probabilité [0.8.0-beta]}}
\end{center}

\begin{itemize}[itemsep=.5em]
    \item \topic*{Arbre de probabilités}
          le mode mathématique est activé par défaut pour les noms des noeuds et les poids \emph{(plus besoin de taper plein de \texttt{\$})}.
\end{itemize}


\separation
