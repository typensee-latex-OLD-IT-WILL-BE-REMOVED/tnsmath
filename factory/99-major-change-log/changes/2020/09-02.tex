Nouvelle version mineure \verb+1.10.0-beta+.


% -------------- %
% -- tnsgeo -- %
% -------------- %


\begin{center}
    \textbf{\textsc{Géométrie [0.4.0-beta]}}
\end{center}

\begin{itemize}[itemsep=.5em]
    \item \topic*{Vecteur}
          retour de la macro \macro{norm*}.

	% -------------- %

\end{itemize}


\separation


% -------------- %
% -- tnslinalg -- %
% -------------- %


\begin{center}
    \textbf{\textsc{Algèbre linéaire [0.1.0-beta]}}
\end{center}

\begin{itemize}[itemsep=.5em]
    \item \topic*{Déterminant $2 \times 2$} changement de l'API.
    \begin{itemize}[itemsep=.5em]
        \item \macro{calcdettwo} sert à obtenir au choix les versions développée ou bien celles matricielles avec pour décorations supplémentaires une croix fléchée ou non.

        \item Suppression de \macro{calcdettwo*}.
    \end{itemize}

	% -------------- %

\end{itemize}


\separation


% -------------- %
% -- tnsproba -- %
% -------------- %


\begin{center}
    \textbf{\textsc{Probabilité [0.9.0-beta]}}
\end{center}

\begin{itemize}[itemsep=.5em]
    \item \topic*{Calculs de l'espérance dans le cas fini}
    	  \macro{calcexpval} rend facile la définition d'une variable aléatoire finie avec la possibilité de détailler le calcul de son espérance.


% ---------------------- %


    \item \topic*{Arbre de probabilités}
    	  pas mal de changements dans l'API et quelques nouveautés.

    \begin{itemize}[itemsep=.5em]
        \item \verb#\begin{probatree}<hideall># \verb#...# \verb#\end{probatree}#  remplace l'usage de l'environnement \env{probatree*}.


        \item \verb#\begin{probatree}<showall># \verb#...# \verb#\end{probatree}# remplace l'usage de l'environnement \env{probatree**}.


        \item De nouvelles macros permettent d'agir sur le texte des noeuds.

        \begin{enumerate}
        	\item \macro{ptreeTextOf} renvoie le texte d'un noeud.

        	\item \macro{ptreeNodeColor} change les couleurs du texte et/ou du fond d'un noeud.

        	\item \macro{ptreeNodeNewText} change le texte en plus éventuellement  les couleurs du texte et/ou du fond d'un noeud.
        \end{enumerate}


        \item \macro{ptreeFocus} a évolué
              \footnote{
              		Malheureusement ces changements n'ont pas pu être faits pour \macro{aptreeFocus} principalement car l'équivalent automatique de \macro{ptreeTextOf} n'a pu être implémenté.
			  }.

        \begin{enumerate}
        	\item \macro{ptreeFocus[frame = start]} remplace l'ancien \macro{ptreeFocus}.

        	\item \macro{ptreeFocus[frame = none]} remplace \macro{ptreeFocus**} qui n'existe plus.

        	\item \macro{ptreeFocus[frame = nostart]} est utilisé par défaut et remplace \macro{ptreeFocus*} qui n'existe plus. On obtient dans ce cas une mise en valeur encadrant tous les noeuds sauf le tout 1\ier{}.

        	\item Les clés optionnelles \verb#lcol#, \verb#tcol# et \verb#bcol# permettent de choisir la couleur des arrêtes et des cadres, celle du texte et enfin celle du fond.
        \end{enumerate}


        \item \verb#tcol# remplace l'ancien \verb#col# de \macro{ptreeComment} et \macro{aptreeComment}.

		\item \verb#lcol# remplace l'ancien \verb#col# de \macro{ptreeFrame}, \macro{aptreeFrame} et \macro{aptreeFocus}.



    \end{itemize}
\end{itemize}


\begin{center}
    \textbf{\textsc{Probabilité [0.10.0-beta]}}
\end{center}

\begin{itemize}[itemsep=.5em]
    \item \topic{Généralités}

    \begin{itemize}[itemsep=.5em]
        \item $\mathrm{P}$ est le nom par défaut d'une probabilité.

		\item Les noms des probabilités, des espérances, des variances et des écarts-types utilisent tous une police droite via \macro{mathrm} en coulisse.
    \end{itemize}


% ---------------------- %


    \item \topic{Espérance d'une variable aléatoire finie}

    \begin{itemize}[itemsep=.5em]
        \item \macro{expval} est utilisée en coulisse pour rédiger les espérances dans les détails des calculs.
        
		\item L'ordre des produits dans le calcul numérique est le même que celui dans la somme formelle.
    \end{itemize}
\end{itemize}


\separation
