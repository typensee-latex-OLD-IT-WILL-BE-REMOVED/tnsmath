Nouvelle version majeure \verb+1.0.0-beta+.

\begin{itemize}[itemsep=.5em]
    \item Le changement vers une nouvelle version majeure se justifie par de nouvelles règles strictes pour définir les signatures des macros. À l'avenir ces règles seront appliquées tout le temps sauf dans de très rares cas.
    \begin{center}
		\bfseries\itshape
		Ceci a créé beaucoup de nouvelles façons de rédiger.
    \end{center}



% ------------------------ %


    \separation
    \item \topic*{Algèbre linéaire}
          \macro{calcdettwo} est un outil pédagogique pour expliquer le calcul d'un déterminant $2\times2$.


% ------------------------ %


    \separation
    \item \topic{Analyse}
    \begin{itemize}[itemsep=.5em]
        \item Calcul intégral.
        
        \begin{itemize}[itemsep=.5em, label=$\rightarrow$]
            \item \macro{integrate} et \macro{dintegrate} servent à rédiger des intégrales simples.

            \item \macro{hook} s'utilise différemment : on doit taper \macro{hook\{a\}\{b\}\{F(x)\}\{x\}} avec l'obligation de donner la variable.
            
            \item \macro{hook} marche avec des options.
                  Du coup \macro{vhook} and \macro{vhook*} ont été supprimées mais les mises en forme correspondantes existent toujours via \macro{hook[r]} et \macro{hook[sr]}.
	    \end{itemize}
	    
	    % :::::::::::::::::::::::: %
        
        \item Dérivées totales.
        
        \begin{itemize}[itemsep=.5em, label=$\rightarrow$]
            \item Il ne reste plus que trois macros : \macro{sder}, \macro{der} et \macro{derope}.

            \item \macro{sder} et \macro{sder[e]} remplacent \macro{derpow*} et \macro{derpow} avec en plus la possibilité d'ajout automatique de parenthèses pour faire comme avec \macro{derpar*}, \macro{derpar}, \macro{sderpar*} et \macro{sderpar} avant.


            \item \macro{der} s'utilise en indiquant la variable de dérivation. Cette macro propose différentes options pour différentes mises en forme 
                  \emph{(on peut toujours obtenir la même chose que ce que proposaient \macro{derfrac}, \macro{derfrac*} et \macro{dersub})}.

            \item \macro{derope} sert à écrire un opérateur fonctionnel.
	    \end{itemize}

	    % :::::::::::::::::::::::: %

        \item Dérivées partielles.
        
        \begin{itemize}[itemsep=.5em, label=$\rightarrow$]
            \item Il ne reste plus que deux macros : \macro{pder} et \macro{pderope}.

            \item \macro{pder} possède des options permettant d'obtenir le même résultat qu'avec les anciennes macros \macro{partialfrac} et \macro{partialsub}.

            \item La mise en forme proposée par \macro{partialprime} n'a pas été gardée.

            \item \macro{pderope} sert à écrire un opérateur fonctionnel.
	    \end{itemize}
    \end{itemize}


% ------------------------ %


    \separation
    \item \topic{Arithmétique}
    \begin{itemize}[itemsep=.5em]
        \item \macro{notdivides} a été renommée \macro{ndivides}.

        \item Ajout de \macro{nequiv}.
    \end{itemize}


% ------------------------ %


    \separation
    \item \topic{Géométrie}
    \begin{itemize}[itemsep=.5em]
        \item \macro{calcdetplane}, \macro{calcdetplane*}, \macro{vcalcdetplane} et \macro{vcalcdetplane*} permettent de détailler le calcul du déterminant de deux vecteurs en dimension $2$ \emph{(utile pour le critère de colinéarité)}.

	    % :::::::::::::::::::::::: %

        \item \macro{calccrossprod}, \macro{calccrossprod*}, \macro{vcalccrossprod} et \macro{vcalccrossprod*} permettent de détailler le calcul du produit vectoriel de deux vecteurs en dimension $3$.

        \item \macro{coordcrossprod} permet d'obtenir formellement les coordonnées d'un produit vectoriel avec des formats du type $(y z' - z y' , z x' - x z ' , x y' - y x')$.

		% :::::::::::::::::::::::: %

        \item Angles orientés.
        
        \begin{itemize}[itemsep=.5em, label=$\rightarrow$]
            \item \macro{angleorient} devient l'unique macro pour rédiger des angles orientés de différentes façons via des options.
            
            \item \macro{angleorient*}  a été remplacée par \macro{angleorient[sp]}.
            
            \item \macro{hangleorient}  a été remplacée par \macro{angleorient[h]}.
                  
            \item \macro{hangleorient*} a été remplacée par \macro{angleorient[sh]}.
	    \end{itemize}

	    % :::::::::::::::::::::::: %

        \item Produit scalaire.
        
        \begin{itemize}[itemsep=.5em, label=$\rightarrow$]
            \item \macro{dotprod} devient l'unique macro pour rédiger des produits scalaires de différentes façons grâce à des options.
            
            \item \macro{adotprod}  a été remplacée par \macro{dotprod[a]}.

            \item \macro{adotprod*} a été remplacée par \macro{dotprod[sa]}.
	    \end{itemize}

	    % :::::::::::::::::::::::: %

        \item Coordonnées.
        
        \begin{itemize}[itemsep=.5em, label=$\rightarrow$]
            \item Il faut passer via l'une des macros : \macro{coord}, \macro{pcoord}, \macro{pcoord*}, \macro{vcoord} et \macro{vcoord*}.
                  Toutes ces macros proposent des options pour choisir la mise en forme : des parenthèses en mode horizontal, des crochets en mode vertical...

            \item \macro{coord} est pour des coordonnées seules.

            \item \macro{pcoord} et \macro{pcoord*} sont pour un point avec ses coordonnées.

            \item \macro{vcoord} et \macro{vcoord*} sont pour un vecteur avec ses coordonnées.

            \item La version étoilée \macro{coord*} a été supprimée. 
	    \end{itemize}



        \item Norme.
        
        \begin{itemize}[itemsep=.5em, label=$\rightarrow$]
            \item \macro{norm} fonctionne maintenant avec des options.
                  Du coup \macro{norm*} a été supprimée mais la mise en forme correspondante existe toujours via \macro{norm[s]}.

			\item \macro{vnorm} évite d'avoir à utiliser \macro{vect} pour des vecteurs juste nommés.
	    \end{itemize}
    \end{itemize}


% ------------------------ %


    \separation
    \item \topic{Logique}
    \begin{itemize}[itemsep=.5em]
        \item La macro \macro{explain} a été supprimée pour être remplacée par l'environnement \env{explain} qui est redoutable d'efficacité pour détailler un calcul ou un raisonnement simple.
        
        \item \env{explain} est complété par les deux environnements \env{aexplain} et \env{aexplain*} qui utilisent des flèches pour les indications.

        \item Les environnements \env{demoexplain} et \env{demoexplain*} permettent de rédiger de \emph{\og vraies \fg} démonstrations via des tableaux efficaces.

        \item Toutes les macros négatives avec pour préfixe \prefix{not} auparavant utilisent maintenant juste le préfixe \prefix{n}.

        \item Tous les opérateurs de comparaison ont une version négative.
    \end{itemize}


% ------------------------ %


    \separation
    \item \topic{Probabilités}
    \begin{itemize}[itemsep=.5em]
        \item \macro{probacond**} et \macro{dprobacond**} sont devenues \macro{probacondexp*} et \macro{probacondexp} respectivement.
        
        \item \macro{expval} est une nouvelle macro pour l'écriture symbolique de l'espérance d'une  variable aléatoire.
    \end{itemize}
\end{itemize}

