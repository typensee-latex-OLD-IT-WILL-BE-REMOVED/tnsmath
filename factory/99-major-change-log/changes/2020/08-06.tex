Nouvelle version mineure \verb+1.7.0-beta+.


% -------------- %
% -- tnsana -- %
% -------------- %


\begin{center}
    \textbf{\textsc{Analyse [0.6.0-beta]}}
\end{center}

\begin{itemize}[itemsep=.5em]
    \item \topic{Définition explicite d'une fonction}
    \begin{itemize}[itemsep=.5em]
        \item Une nouvelle macro \macro{txfuncdef} produit une version textuelle courte.

        \item Omission possible des ensembles, via des arguments vides, quand on utilise \macro{funcdef[h]} ou \macro{txfuncdef}.
    \end{itemize}

% ------------------------ %

    \item \topic*{Fonctions avec un paramètre}
          les macros \macro{expb} et \macro{logb} ont été remplacées par \macro{exp} et \macro{log} qui ont un argument optionnel pour indiquer éventuellement une base.

% ------------------------ %

    \item \topic*{Dérivation partielle}
          \macro{pder[ei]} fonctionne maintenant aussi avec des variables indexées.
          
\end{itemize}


\begin{center}
    \textbf{\textsc{Analyse [0.7.0-beta]}}
\end{center}

\begin{itemize}[itemsep=.5em]
    \item \topic*{Définition explicite d'une fonction}
          les macros \macro{funcdef} et \macro{txtfuncdef} ont été déplacées dans \verb#tnssets# qui est disponible sur \url{https://github.com/typensee-latex/tnssets.git}.
    
\end{itemize}


\separation


% -------------- %
% -- tnsgeo -- %
% -------------- %


\begin{center}
    \textbf{\textsc{Géométrie [0.2.0-beta]}}
\end{center}

\begin{itemize}[itemsep=.5em]
    \item \topic*{Critère de colinéarité}
          ajout de la macro \macro{colicriteria}.

	% -------------- %

    \item \topic*{Produit vectoriel} changement de l'API.
    \begin{itemize}[itemsep=.5em]
        \item \macro{vcalccrossprod*} devient \macro{vcalccrossprod**}.

        \item \macro{vcalccrossprod*} dessine des produits en croix à la place des boucles.
    \end{itemize}

	% -------------- %

\end{itemize}


\separation


% -------------- %
% -- tnsproba -- %
% -------------- %


\begin{center}
    \textbf{\textsc{Probabilité [0.4.0-beta]}}
\end{center}

\begin{itemize}[itemsep=.5em]
    \item \topic*{Arbre}
    	  possibilité de mettre en valeur un chemin via \macro{ptreeFocus},  \macro{ptreeFocus*} ou \macro{ptreeFocus**}.
\end{itemize}


\begin{center}
    \textbf{\textsc{Probabilité [0.5.0-beta]}}
\end{center}

\begin{itemize}[itemsep=.5em]
    \item \topic{Arbre de probabilités}
    \begin{itemize}[itemsep=.5em]
        \item \macro{ptreeFocus}, \macro{ptreeFocus*} et \macro{ptreeFocus**} fonctionnent avec un multi-argument pour pourvoir indiquer un chemin sur plusieurs noeuds.
        
        \item Suppression de la clé \macro{pcomment}.
        
        \item Ajout des macros \macro{ptreeComment} et \macro{ptreeComment*} qui simplifient la saisie.
    \end{itemize}
\end{itemize}


\separation


% ------------- %
% -- tnssets -- %
% ------------- %


\begin{center}
    \textbf{\textsc{Théorie générale des ensembles [0.2.0-beta]}}
\end{center}

\begin{itemize}[itemsep=.5em]
    \item \topic{Composition d'applications}
    \begin{itemize}[itemsep=.5em]
        \item \macro{compo} est un opérateur de composition de deux applications.

        \item \macro{multicomp} permet d'indiquer des compositions successives d'une application par elle-même.
    \end{itemize}
\end{itemize}


\begin{center}
    \textbf{\textsc{Théorie générale des ensembles [0.3.0-beta]}}
\end{center}

\begin{itemize}[itemsep=.5em]
    \item \topic*{Définition explicite d'une fonction}
          intégration de deux macros proposées avant par \verb#tnsana# disponible sur \url{https://github.com/typensee-latex/tnsana.git}.

    \begin{itemize}[itemsep=.5em]
        \item \macro{funcdef} peut produire trois versions symboliques.

        \item \macro{txtfuncdef} produit une version textuelle courte.
    \end{itemize}


    \item \topic{Fonctions spéciales}

    \begin{itemize}[itemsep=.5em]
        \item Ajout de \macro{id} pour la fonction identité.

        \item Ajout de \macro{caract} et \macro{caractone} pour deux versions de la fonction caratéristque d'un ensemble.
    \end{itemize}
\end{itemize}


\separation
