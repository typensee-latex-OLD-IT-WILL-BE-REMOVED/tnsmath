Nouvelle version mineure \verb+0.7.0-beta+.

\begin{itemize}[itemsep=.5em]
    \item En analyse, il y a eu les changements suivants.
    \begin{itemize}[itemsep=.5em]
        \item Pour éviter des conflits avec d'autres packages les renommages suivants ont dû être faits.
        \begin{itemize}[itemsep=.5em]
            \item \macro{ch} et \macro{ach} sont devenus \macro{fch} et \macro{afch} où \prefix{f} est pour \whyprefix{f}{rench}.

            \item \macro{sh} et \macro{ash} sont devenus \macro{fsh} et \macro{afsh}.

            \item \macro{th} et \macro{ath} sont devenus \macro{fth} et \macro{afth}.
        \end{itemize}

		\item Les macros \macro{acosh}, \macro{asinh} et \macro{atanh} ont été ajoutées.

 		\item \macro{derpar} et \macro{derpar*} servent à rédiger des dérivées avec des parenthèses extensibles. En coulisse, \macro{derpow} et \macro{derpow*} sont appelées.

        Pour utiliser des parenthèses non extensibles, on passera par \macro{sderpar} et \macro{sderpar*} où \prefix{s} est pour \whyprefix{s}{mall}.

		\item Ajout de \macro{stdint} pour rendre public l'opérateur intégral proposé par défaut par \LaTeX.
    \end{itemize}

    % ------------ %

    \item En géométrie, il y a eu une suppression et trois ajouts.
    \begin{itemize}[itemsep=.5em]
        \item La macro \macro{pts} a été supprimée car sans signification sémantique puisqu'un point peut être nommé avec deux lettres.

        \item Les macros \macro{gline} et \macro{pgline} servent à indiquer des droites définies par deux points.

        \item La macro \macro{hgline}, avec \prefix{h} pour \whyprefix{h}{alf}, est pour les demi-droites définies par deux points.

        \item La macro \macro{segment} est utile pour les segments définis par deux points.
    \end{itemize}

    % ------------ %

    \item En logique, voici les améliorations apportées.
    \begin{itemize}[itemsep=.5em]
        \item Pour les inégalités, on peut maintenant utiliser le décorateur \verb+plot+.

		\item Ajout des versions négatives des opérateurs logiques verticaux.
    \end{itemize}

    % ------------ %

    \item Pour les probabilités, il y a eu les modifications ci-après.
    \begin{itemize}[itemsep=.5em]
        \item Ajout de \macro{proba} pour écrire des probabilités.

		\item Le comportement de \macro{probacond} a été modifié pour le rendre plus logique.
    \end{itemize}
\end{itemize}
