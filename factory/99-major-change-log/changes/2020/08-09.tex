Nouvelle version mineure \verb+1.8.0-beta+.


% -------------- %
% -- tnslog -- %
% -------------- %


\begin{center}
    \textbf{\textsc{Méthodes formelles en logique [0.1.0-beta]}}
\end{center}

\begin{itemize}[itemsep=.5em]
    \item \topic*{Macros \og textuelles \fg}
          le préfixe \verb#txt# remplace l'ancien \verb#text#.
\end{itemize}


\separation


% -------------- %
% -- tnsproba -- %
% -------------- %


\begin{center}
    \textbf{\textsc{Probabilité [0.6.0-beta]}}
\end{center}

\begin{itemize}[itemsep=.5em]
    \item \topic{Arbre de probabilités}
    \begin{itemize}[itemsep=.5em]
        \item \macro{aptreeFocus}, \macro{aptreeFocus*} et \macro{aptreeFocus**} permettent d'utiliser le système de nommage automatique des noeuds proposé par \verb#forest#.
        
        \item Il en va de même pour \macro{aptreeComment} et \macro{aptreeFrame}.
    \end{itemize}
\end{itemize}


\begin{center}
    \textbf{\textsc{Probabilité [0.7.0-beta]}}
\end{center}

\begin{itemize}[itemsep=.5em]
    \item \topic{Arbre de probabilités}
    \begin{itemize}[itemsep=.5em]
        \item Utilisation obligatoire de \verb#col=...# pour indiquer une couleur à toutes les macros de décoration.
        
        \item Les macros \macro{ptreeComment}, \macro{ptreeComment*}, 
              \macro{aptreeComment} et \macro{aptreeComment*}
              ont deux clés \verb#dx# et \verb#dy# pour indiquer un décalage relatif.
        
        \item Ajout de l'environnement \env{probatree**} qui force l'affichage de tous les poids !
    \end{itemize}
\end{itemize}


\separation


% -------------- %
% -- tnsseq -- %
% -------------- %


\begin{center}
    \textbf{\textsc{Suites [0.1.0-beta]}}
\end{center}

\begin{itemize}[itemsep=.5em]
    \item \topic*{Comparaison asymptotique}
          ce sont de vrais opérateurs mathématiques qui sont définis en coulisse \emph{(du coup les macros \macro{bigO}, \macro{smallO}, \macro{bigOmega} et \macro{bigTheta} n'ont plus d'argument)}.   
\end{itemize}


\separation
