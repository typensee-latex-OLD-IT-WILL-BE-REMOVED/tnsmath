Nouvelle version mineure \verb+0.3.0-beta+.

\begin{itemize}[itemsep=.5em]
    \item Une nouvelle section \emph{\og Logique et fondements \fg} a été ajoutée.
    \begin{itemize}[itemsep=.5em]
        \item Trois types de signes $=$ décorés sont proposés : voir les macros \macro{eqdef} , \macro{eqid} et \macro{eqtest}.

        \item Via la macro \macro{explain}, il devient facile d'expliquer des étapes de raisonnement ou des calculs.
    \end{itemize}

    % ------------ %

    \item Pour les ensembles, la macro \macro{fieldset} a été renommé \macro{algeset} et la macro \macro{PP} permet d'indiquer l'ensemble des nombres premiers.

    % ------------ %

    \item En géométrie, il y a quelques nouveautés.
    \begin{itemize}[itemsep=.5em]
        \item La macro \macro{hangleorient} permet l'écriture d'angles orientés avec un chapeau en plus.

        \item Les macros \macro{vangleorient} et \macro{vhangleorient} évite d'avoir à utiliser \macro{vect} lorsque l'on a juste des vecturs simples nommés et non coefficientés.

        \item De même pour les macros \macro{vdotprod}, \macro{vadotprod} et \macro{vcroosprod}.
    \end{itemize}

    % ------------ %

    \item Ajout de \macro{lymathsubsep} qui définit le séparateur des arguments de second niveau.
\end{itemize}
