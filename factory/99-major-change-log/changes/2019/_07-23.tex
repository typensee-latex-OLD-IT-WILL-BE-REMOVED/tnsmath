Nouvelle version mineure \verb+0.3.0-beta+.

\begin{itemize}[itemsep=.5em]
    \item Une nouvelle section \emph{\og Logique et fondements \fg} a été ajoutée.
    \begin{itemize}[itemsep=.5em]
        \item Trois types de signes $=$ décorés sont proposés : voir les macros \verb+\eqdef+ , \verb+\eqid+ et \verb+\eqtest+.

        \item Via la macro \verb+\explain+, il devient facile d'expliquer des étapes de raisonnement ou des calculs.
    \end{itemize}

    % ------------ %

    \item Pour les ensembles, la macro \verb+\fieldset+ a été renommé \verb+\algeset+ et la macro \verb+\PP+ permet d'indiquer l'ensemble des nombres premiers.

    % ------------ %

    \item En géométrie, il y a quelques nouveautés.
    \begin{itemize}[itemsep=.5em]
        \item La macro \verb+\hangleorient+ permet l'écriture d'angles orientés avec un chapeau en plus.

        \item Les macros \verb+\vangleorient+ et \verb+\vhangleorient+ évite d'avoir à utiliser \verb+\vect+ lorsque l'on a juste des vecturs simples nommés et non coefficientés.

        \item De même pour les macros \verb+\vdotprod+, \verb+\vadotprod+ et \verb+\vcroosprod+.
    \end{itemize}

    % ------------ %

    \item Ajout de \verb+\lymathsubsep+ qui définit le séparateur des arguments de second niveau.
\end{itemize}
