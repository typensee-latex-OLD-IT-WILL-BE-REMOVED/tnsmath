\documentclass[12pt,a4paper]{article}

\usepackage{color}
\usepackage{hyperref}
\hypersetup{
    colorlinks,
    citecolor=black,
    filecolor=black,
    linkcolor=black,
    urlcolor=black
}

\usepackage{lymath}

\usepackage{amssymb}
\usepackage{textgreek}
\usepackage[raggedright]{titlesec}

\titleformat{\paragraph}[hang]{\normalfont\normalsize\bfseries}{\theparagraph}{1em}{}
\titlespacing*{\paragraph}{0pt}{3.25ex plus 1ex minus .2ex}{0.5em}


\newcommand\ascii{\texttt{ASCII}}

\usepackage[utf8]{inputenc}
\usepackage{ucs}
\usepackage[top=2cm, bottom=2cm, left=1.5cm, right=1.5cm]{geometry}

\usepackage{color}
\usepackage{hyperref}
\hypersetup{
    colorlinks,
    citecolor=black,
    filecolor=black,
    linkcolor=black,
    urlcolor=black
}

\usepackage{enumitem}

\usepackage{tcolorbox}
\tcbuselibrary{listings}

\usepackage{pgffor}
\usepackage{xstring}

\setlength{\parindent}{0cm}
\setlist{noitemsep}

% Technical IDs
\newwrite\tempfile

\immediate\openout\tempfile=x-\jobname.macros-x.txt

\AtEndDocument{\immediate\closeout\tempfile}

\newcommand\IDconstant[1]{%
    \immediate\write\tempfile{constant@#1}%
}

\makeatletter
	\newcommand\IDmacro{\@ifstar{\@IDmacroStar}{\@IDmacroNoStar}}
	
    \newcommand\@IDmacroNoStar[3]{%
        \texttt{%
        	\textbackslash#1%
        	\IfStrEq{#2}{0}{}{%
        		\,\,[#2 Option%
				\IfStrEq{#2}{1}{}{s}]%
			}%
    	    \,\,(#3 Argument%
				\IfStrEq{#3}{1}{}{s})%
	   	}
        \immediate\write\tempfile{macro@#1@#2@#3}%
    }

    \newcommand\@IDmacroStar[2]{%
        \@IDmacroNoStar{#1}{0}{#2}%
    }

	\newcommand\@IDoptarg[2]{%
    	\vspace{0.5em}
		--- \texttt{#1 \##2:}%
	}

	\newcommand\IDoption[1]{%
    	\@IDoptarg{Option}{#1}%
	}

	\newcommand\IDarg[1]{%
    	\@IDoptarg{Argument}{#1}%
	}
\makeatother

\begin{document}

\section{Miscellaneous}



\subsection{Semi-colon and spacing}

\begin{tcblisting}{}
Same spacing around the semi-colon in $A(x;y)$. So cute !
\end{tcblisting}


\section{Sets}



\subsection{Different kind of sets}

    \subsubsection{Sets for geometry}

		\paragraph{Example of use}

\begin{tcblisting}{}
You can semantically write $\geoset{C}$, $\geoset{D}$ and $\geoset{d}$
but you can't write things like \verb+$\geoset{ABC}$+.
\end{tcblisting}


		\paragraph{Technical IDs}

\IDmacro*{geoset}{1}

\IDarg{1} one single \ascii{} letter indicating a geometrical set.



    \subsubsection{Sets for probability}

		\paragraph{Example of use}

\begin{tcblisting}{}
You can semantically write $\probaset{E}$ and $\probaset{G}$
but you can't write things like \verb+$\probaset{ABC}$+.
\end{tcblisting}


		\paragraph{Technical IDs}

\IDmacro*{probaset}{1}

\IDarg{1} one single \ascii{} upper letter indicating a probabilistic set.



    \subsubsection{Sets for rings and fields theory}

		\paragraph{Example of use}

\begin{tcblisting}{}
You can semantically write $\fieldset{A}$, $\fieldset{K}$, $\fieldset{h}$ 
and $\fieldset{k}$ but you can't write things like \verb+$\fieldset{ABC}$+.
\end{tcblisting}


		\paragraph{Technical IDs}

\IDmacro*{fieldset}{1}

\IDarg{1} either one of the letters \texttt{h} and \texttt{k}, or one single \ascii{} upper letter indicating a field or ring like set.



    \subsubsection{Classical sets}

\begin{tcblisting}{}
You can directly use $\nullset$, $\NN$, $\ZZ$, $\DD$, $\QQ$, $\RR$, $\CC$,
$\HH$ and $\OO$.
\end{tcblisting}



    \subsubsection{Classical sets with suffixes}

\begin{tcblisting}{}
It is easy to type $\RRn$, $\RRp$, $\RRs$, $\RRsn$ and $\RRsp$.
\end{tcblisting}


We have used suffixes \verb+n+ for \verb+Negative+, \verb+p+ for \verb+Positive+, and \verb+s+ for \verb+Star+ with the additional composite suffixes \verb+sn+ et \verb+sp+.

\medskip

Note that you can't use \verb+$\CCn$+ for $\specialset{\CC}{n}$ because the set $\CC$ doesn't have any standard powerful ordered structure. Take a look at the next section to see how to write $\specialset{\CC}{n}$ if you need it.

\medskip

The following table shows when you can add one of the suffixes \verb+n+, \verb+p+, \verb+s+, \verb+sn+ and \verb+sp+.

% == Table of suffixes - START == %
\newcommand\xx{\phantom{$\times$}}
\begin{table}[h]
    \caption{Suffixes}
    \begin{center}
        \begin{tabular}{c|c|c|c|c|c}
  & \verb+n+ & \verb+p+ & \verb+s+ & \verb+sn+ & \verb+sp+ \\
\hline \verb+N+ & \xx & \xx & $\times$ & \xx & \xx \\
\hline \verb+Z+ & $\times$ & $\times$ & $\times$ & $\times$ & $\times$ \\
\hline \verb+D+ & $\times$ & $\times$ & $\times$ & $\times$ & $\times$ \\
\hline \verb+Q+ & $\times$ & $\times$ & $\times$ & $\times$ & $\times$ \\
\hline \verb+R+ & $\times$ & $\times$ & $\times$ & $\times$ & $\times$ \\
\hline \verb+C+ & \xx & \xx & $\times$ & \xx & \xx \\
\hline \verb+H+ & \xx & \xx & $\times$ & \xx & \xx \\
\hline \verb+O+ & \xx & \xx & $\times$ & \xx & \xx \\
        \end{tabular}
    \end{center}
    \label{default}
\end{table}
% == Table of suffixes - END == %



    \subsubsection{Suffixes on demand}

		\paragraph{Example of use}

\begin{tcblisting}{}
You can indeed write things like $\specialset{\CC}{n}$ or $\specialset{\HH}{sp}$. 
There is also $\specialset*{\probaset{P}}{n}$ with another formatting.
\end{tcblisting}


		\paragraph{Technical IDs}

\IDmacro*{specialset}{2}

\IDmacro*{specialset*}{2}

\IDarg{1} the set to be "suffixed".

\IDarg{2} one of the suffixes \verb+n+, \verb+p+, \verb+s+, \verb+sn+ and \verb+sp+.




\subsection{Intervals}

    \subsubsection{Real intervals - French (?) notation}

		\paragraph{Example of use}

\begin{tcblisting}{}
In $\intervalOC{a}{b} = ]a ; b] = \intervalOC{a}{b}$, you can see that the macro
used solves a spacing problem, and that the delimiters are a little bigger.

% The syntax refers to O-pened and C-losed but CC and OO are reduced to C and O.

\medskip

You can use the star version of a macro if you want the delimiters to stretch 
vertically.

$\displaystyle \intervalC{ \frac{1}{2} }{ 1^{2^{3}} }
             = [ \frac{1}{2} ; 1^{2^{3}} ] 
             = \intervalC*{ \frac{1}{2} }{ 1^{2^{3}} }$
\end{tcblisting}


		\paragraph{Technical IDs}

For all the macros above, the star version produces intervals with delimiters that fit vertically with the bounds of the interval.

\medskip

% Docs for french real intervals - START

\medskip

\IDmacro*{intervalCO}{2}

\IDmacro*{intervalCO*}{2}

\IDarg{1} lower bound $a$ of the interval $\intervalCO{a}{b}$.

\IDarg{2} upper bound $b$ of the interval $\intervalCO{a}{b}$.

\medskip

\IDmacro*{intervalC}{2}

\IDmacro*{intervalC*}{2}

\IDarg{1} lower bound $a$ of the interval $\intervalC{a}{b}$.

\IDarg{2} upper bound $b$ of the interval $\intervalC{a}{b}$.

\medskip

\IDmacro*{intervalO}{2}

\IDmacro*{intervalO*}{2}

\IDarg{1} lower bound $a$ of the interval $\intervalO{a}{b}$.

\IDarg{2} upper bound $b$ of the interval $\intervalO{a}{b}$.

\medskip

\IDmacro*{intervalOC}{2}

\IDmacro*{intervalOC*}{2}

\IDarg{1} lower bound $a$ of the interval $\intervalOC{a}{b}$.

\IDarg{2} upper bound $b$ of the interval $\intervalOC{a}{b}$.

% Docs for french real intervals - END



    \subsubsection{Real intervals - American notation}

		\paragraph{Example of use}

\begin{tcblisting}{}
A semi-closed interval $\intervalPC{a}{b} = (a ; b]$ and an opened one
$\intervalP{a}{b} = (a ; b)$.

% The syntax refers to P-arenthesis.
\end{tcblisting}


		\paragraph{Technical IDs}

For all the macros above, the star version produces intervals with delimiters that fit vertically with the bounds of the interval.

\medskip

% Docs for american real intervals - START

\medskip

\IDmacro*{intervalCP}{2}

\IDmacro*{intervalCP*}{2}

\IDarg{1} lower bound $a$ of the interval $\intervalCP{a}{b}$.

\IDarg{2} upper bound $b$ of the interval $\intervalCP{a}{b}$.

\medskip

\IDmacro*{intervalP}{2}

\IDmacro*{intervalP*}{2}

\IDarg{1} lower bound $a$ of the interval $\intervalP{a}{b}$.

\IDarg{2} upper bound $b$ of the interval $\intervalP{a}{b}$.

\medskip

\IDmacro*{intervalPC}{2}

\IDmacro*{intervalPC*}{2}

\IDarg{1} lower bound $a$ of the interval $\intervalPC{a}{b}$.

\IDarg{2} upper bound $b$ of the interval $\intervalPC{a}{b}$.

% Docs for american real intervals - END



    \subsubsection{Discrete intervals of integers}

		\paragraph{Example of use}

\begin{tcblisting}{}
By definition, $\ZintervalC{-1}{4} = \{ -1 ; 0 ; 1 ; 2 ; 3 ; 4 \}$. So we also have
$\ZintervalC{-1}{4} = \ZintervalO{-2}{5}$.

% The syntax refers to Z the set of integers.
\end{tcblisting}


		\paragraph{Technical IDs}

% Docs for discrete intervals - START

\medskip

\IDmacro*{ZintervalCO}{2}

\IDmacro*{ZintervalCO*}{2}

\IDarg{1} lower bound $a$ of the interval $\ZintervalCO{a}{b}$.

\IDarg{2} upper bound $b$ of the interval $\ZintervalCO{a}{b}$.

\medskip

\IDmacro*{ZintervalC}{2}

\IDmacro*{ZintervalC*}{2}

\IDarg{1} lower bound $a$ of the interval $\ZintervalC{a}{b}$.

\IDarg{2} upper bound $b$ of the interval $\ZintervalC{a}{b}$.

\medskip

\IDmacro*{ZintervalO}{2}

\IDmacro*{ZintervalO*}{2}

\IDarg{1} lower bound $a$ of the interval $\ZintervalO{a}{b}$.

\IDarg{2} upper bound $b$ of the interval $\ZintervalO{a}{b}$.

\medskip

\IDmacro*{ZintervalOC}{2}

\IDmacro*{ZintervalOC*}{2}

\IDarg{1} lower bound $a$ of the interval $\ZintervalOC{a}{b}$.

\IDarg{2} upper bound $b$ of the interval $\ZintervalOC{a}{b}$.

% Docs for discrete intervals - END


\section{Analysis}



\subsection{Constants}

    \subsubsection{Classical constants}

		\paragraph{Complete list}
		
% List of classical constants

\foreach \k in {ggamma, ppi, ttau, ee, ii, jj, kk}{\IDconstant{\k}}

\begin{tcblisting}{}
List of all classical constants where $\ttau = \frac{\ppi}{2}$ is the youngest one:
$\ggamma$, $\ppi$, $\ttau$, $\ee$, $\ii$, $\jj$ and $\kk$.
\end{tcblisting}


		\paragraph{Remark}

Take care that \verb+{\Large $\ppi \neq \pi$}+ produces {\Large $\ppi \neq \pi$}. As you can see, the symbols are not the same. Indeed, this is true for all the greek constants.



    \subsubsection{User's latine constants}

		\paragraph{Example of use}

\begin{tcblisting}{}
It is easy to write $\ct{a} x^2 + \ct{b} x + \ct{c}$ instead of $a x^2 + b x + c$
such as to stress the fact that $\ct{a}$, $\ct{b}$ and $\ct{c}$ are constants.
\end{tcblisting}


		\paragraph{Technical IDs}

\IDmacro*{ct}{1}

\IDarg{1} a latine text, and not a formula, indicated one constant.




\subsection{Special functions}

    \subsubsection{Some examples of use}

\begin{tcblisting}{}
Some additional special functions :
$\ch x \neq ch x$, $\ppcm(x;y)$, $\lg x =\logb{2} x$ and $\expb{6} y$.
\end{tcblisting}


    \subsubsection{Functions without parameter}

All the following macros don't have any parameter.

% List of functions without parameter

\foreach \k in {pgcd, ppcm, ch, sh, th, ach, ash, ath, arccosh, arcsinh, arctanh, acos, asin, atan}{\IDconstant{\k}}

\medskip

\begin{tabular*}{\textwidth}%
                {@{\extracolsep{\fill}}*{4}{l}}
    \verb+\pgcd+ & \verb+\ppcm+ & \verb+\ch+ & \verb+\sh+\\
    \verb+\th+ & \verb+\ach+ & \verb+\ash+ & \verb+\ath+\\
    \verb+\arccosh+ & \verb+\arcsinh+ & \verb+\arctanh+ & \verb+\acos+\\
    \verb+\asin+ & \verb+\atan+ &  & \\
\end{tabular*}

















    \subsubsection{Functions with parameters}

		\paragraph{The complete list}

All the following macros have at least one parameter.

% List of functions with parameters

\medskip

\begin{tabular*}{\textwidth}%
                {@{\extracolsep{\fill}}*{4}{l}}
    \verb+\expb+ \, (1 parameter) & \verb+\logb+ \, (1 parameter) &  & \\
\end{tabular*}


		\paragraph{Technical IDs}

\IDmacro*{expb}{1}

\IDmacro*{logb}{1}

\IDarg{1} the base of the exponential or the logarithm




\subsection{Extended notations for special sequences}

		\paragraph{Example of use}

\begin{tcblisting}{}
Sometimes we need to write $\seqplus{F}{1}{2}$ or $\hypergeo{F}{1}{2}$
and crazy men really (?) love $\suprageo{F}{1}{2}{3}{4}$.
\end{tcblisting}


		\paragraph{Technical IDs}

\IDmacro*{seqplus}{2}

\IDarg{1} the right exponent like expression.

\IDarg{2} the right indice.


\bigskip


\IDmacro*{hypergeo}{2}

\IDarg{1} the left indice.

\IDarg{2} the right indice.


\bigskip


\IDmacro*{suprageo}{4}

\IDarg{1} the left indice.

\IDarg{2} the right indice.

\IDarg{3} the right exponent.

\IDarg{4} the left exponent.




\subsection{Differential calculus}

    \subsubsection{\texorpdfstring{The $\pp{}$ and $\dd{}$ operators}%
                               {The "rounded d" and "straight d" operators}}

        \paragraph{Example of use}

\begin{tcblisting}{}
You can write $\dd{x}$ and $\pp{t}$, but also $ \dd[5]{x}$ et $\pp[n]{x}$.
\end{tcblisting}


        \paragraph{Technical IDs}

\IDmacro{dd}{1}{1}

\IDmacro{pp}{1}{1}

\IDoption{1} if used, this argument will be the exponent of the symbol $\pp{}$ or $\dd{}$.

\IDarg{1} the variable of differentiation at the right of the symbol $\pp{}$ or $\dd{}$.



    \subsubsection{Total derivation}

        \paragraph{Example of use}

\begin{tcblisting}{}
$\displaystyle cos' a = \derpow{\cos} (a) = \derfrac{\cos}{x} (a)$
and
$\displaystyle cos''' a = \derpow[3]{\cos} (a) = \derfrac[3]{\cos}{x} (a)$
\end{tcblisting}


        \paragraph{Technical IDs}

\IDmacro{derpow}{1}{1}

\IDoption{1} if used, this argument will be the exponent of derivation put inside braces.

\IDarg{1} the function to be differenciated.


\bigskip


\IDmacro{derfrac}{1}{2}

\IDoption{1} if used, the exponent of derivation.

\IDarg{1} the function to be differenciated.

\IDarg{2} the variable used for the derivation.



    \subsubsection{Partial derivation}

        \paragraph{Example of use}

\begin{tcblisting}{}
$\displaystyle \partialfrac{f}{x} (a;b)
             = \partialsub{f}{x} (a;b)
             = \partialprime{f}{x} (a;b)$

\medskip

$\displaystyle \partialfrac[3]{G}{f^{2} // v} (a;b)
             = \partialfrac{G}{f^{2} // v} (a;b)
             = \partialsub{G}{f^{2} // v} (a;b)
             = \partialprime{G}{f^{2} // v} (a;b)$
\end{tcblisting}


        \paragraph{Technical IDs}

\IDmacro{partialfrac}{1}{2}

\IDoption{1} if used, the exponent of $\pp$ associated to the function differenciated.

\IDarg{1} the function to be partially differenciated.

\IDarg{2} the variables used for the partial derivation. The syntax is particular : for example, \verb+x // y^3 // ...+ indicates regarding to the variables $x$ one time, $y$ three times... and so on.


\bigskip


\IDmacro*{partialsub}{2}

\IDmacro*{partialprime}{2}

\IDarg{1} the function to be partially differenciated.

\IDarg{2} the variables used for the partial derivation. The syntax is particular : for example, \verb+x // y^3 // ...+ indicates regarding to the variables $x$ one time, $y$ three times... and so on.




\subsection{Integral calculus}

    \subsubsection{The hook operator}

		\paragraph{Example of use}

\begin{tcblisting}{}
By definition, $\displaystyle \int_{a}^{b} f(x) \dd{x} = \hook{F(x)}{a}{b}$
where $\hook{F(x)}{a}{b} = \hook*{F(x)}{a}{b} = F(b) - F(a)$.
\end{tcblisting}


		\paragraph{Technical IDs}

\IDmacro*{hook}{3}

\IDarg{1} the content inside the hooks.

\IDarg{2} the lower bound displayed as an index.

\IDarg{3} the upper bound displayed as an exponent.


\bigskip


\IDmacro*{hook*}{3}

\IDarg{1} the content before the vertical line $\vert$ .

\IDarg{2} the lower bound displayed as an index.

\IDarg{3} the upper bound displayed as an exponent.



    \subsubsection{Several integrals}

The package minimizes spacings between consecutive symbols of integration. Here is an example.

\begin{tcblisting}{}
$\displaystyle \int \int \int F(x;y;z) \dd{x} \dd{y} \dd{z}
= \int_{a}^{b} \int_{c}^{d} \int_{e}^{f} F(x;y;z) \dd{x} \dd{y} \dd{z}$
\end{tcblisting}

The default behavior is the following one :
\makeatletter
    $\displaystyle \original@int \original@int \original@int F(x;y;z) \dd{x} \dd{y} \dd{z}
    = \original@int_{a}^{b} \original@int_{c}^{d} \original@int_{e}^{f} F(x;y;z) \dd{x} \dd{y} \dd{z}$.
\makeatother




\subsection{Asymptotic comparisons of sequences and functions}

    \subsubsection{\texorpdfstring{The $\bigO{}$ and $\smallO{}$ notations}%
                               {The "big O" and "small O" notations}}

		\paragraph{Example of use}

\begin{tcblisting}{}
Let's see how to use the symbols $\bigO{}$ and $\smallO{}$ created by Landau.

You can write $\bigO{x} \neq \smallO{x}$ and $e^{t + \smallO{t}} = e^{\bigO{t}}$.
\end{tcblisting}


		\paragraph{Technical IDs}

\IDmacro*{bigO}{1}

\IDmacro*{smallO}{1}

\IDarg{1} the content inside the braces after the symbol $\bigO{}$ or $\smallO{}$.



    \subsubsection{\texorpdfstring{The \textOmega{} notation}%
                               {The "big Omega" notation}}

		\paragraph{Example of use}

\begin{tcblisting}{}
Let's see how to use the symbol $\bigomega{}$ created by Hardy and Littlewood.

$f(n) = \bigomega{g(n)}$ means: $\exists (m, n_0)$ such that
$n \geqslant n_0$ implies $f(n) \geqslant m g(n)$.
\end{tcblisting}


		\paragraph{Technical IDs}

\IDmacro*{bigomega}{1}

\IDarg{1} the content inside the braces after the symbol $\bigomega{}$.



    \subsubsection{\texorpdfstring{The \textTheta{} notation}%
                               {The "big Theta" notation}}

		\paragraph{Example of use}

\begin{tcblisting}{}
Let's see how to use the symbol $\bigtheta{}$.

$f(n) = \bigtheta{g(n)}$ means: $\exists (m, M, n_0)$ such that
$n \geqslant n_0$ implies $m g(n) \leqslant f(n) \leqslant M g(n)$.
\end{tcblisting}


		\paragraph{Technical IDs}

\IDmacro*{bigtheta}{1}

\IDarg{1} the content inside the braces after the symbol $\bigtheta{}$.



\end{document}
