\documentclass[12pt,a4paper]{article}

\usepackage{color}
\usepackage{hyperref}
\hypersetup{
    colorlinks,
    citecolor=black,
    filecolor=black,
    linkcolor=black,
    urlcolor=black
}

\usepackage{ifplatform}

\usepackage{lymath}

\usepackage{amssymb}
\usepackage{textgreek}
\usepackage[raggedright]{titlesec}

\titleformat{\paragraph}[hang]{\normalfont\normalsize\bfseries}{\theparagraph}{1em}{}
\titlespacing*{\paragraph}{0pt}{3.25ex plus 1ex minus .2ex}{0.5em}


\newcommand\ascii{\texttt{ASCII}}

\usepackage[utf8]{inputenc}
\usepackage{ucs}
\usepackage[top=2cm, bottom=2cm, left=1.5cm, right=1.5cm]{geometry}

\usepackage{color}
\usepackage{hyperref}
\hypersetup{
    colorlinks,
    citecolor=black,
    filecolor=black,
    linkcolor=black,
    urlcolor=black
}

\usepackage{enumitem}

\usepackage{amsthm}

\usepackage{tcolorbox}
\tcbuselibrary{listings}

\usepackage{pgffor}
\usepackage{xstring}


% MISC

\setlength{\parindent}{0cm}
\setlist{noitemsep}

\theoremstyle{definition}
\newtheorem*{remark}{Remark}


% Technical IDs

\newwrite\tempfile

\immediate\openout\tempfile=x-\jobname.macros-x.txt

\AtEndDocument{\immediate\closeout\tempfile}

\newcommand\IDconstant[1]{%
    \immediate\write\tempfile{constant@#1}%
}

\makeatletter
	\newcommand\IDmacro{\@ifstar{\@IDmacroStar}{\@IDmacroNoStar}}
	
    \newcommand\@IDmacroNoStar[3]{%
        \texttt{%
        	\textbackslash#1%
        	\IfStrEq{#2}{0}{}{%
        		\,\,[#2 Option%
				\IfStrEq{#2}{1}{}{s}]%
			}%
    	    \IfStrEq{#3}{}{}{%
	    		\,\,(#3 Argument%
				\IfStrEq{#3}{1}{}{s})%
			}
	   	}
        \immediate\write\tempfile{macro@#1@#2@#3}%
    }

    \newcommand\@IDmacroStar[2]{%
        \@IDmacroNoStar{#1}{0}{#2}%
    }

	\newcommand\@IDoptarg{\@ifstar{\@IDoptargStar}{\@IDoptargNoStar}}
	
	\newcommand\@IDoptargStar[2]{%
    	\vspace{0.5em}
		--- \texttt{#1%
			\IfStrEq{#2}{}{:}{\,#2:}%
		}%
	}

	\newcommand\@IDoptargNoStar[2]{%
    	\IfStrEq{#2}{}{%
			\@IDoptargStar{#1}{}%
		}{%
			\@IDoptargStar{#1}{\##2}%
		}%
	}

	\newcommand\IDkey[1]{%
    	\@IDoptarg*{Option}{{\itshape "#1"}}%
	}

	\newcommand\IDoption[1]{%
    	\@IDoptarg{Option}{#1}%
	}

	\newcommand\IDarg[1]{%
    	\@IDoptarg{Argument}{#1}%
	}
\makeatother

\begin{document}

\title{\texttt{lymath} package:\\semantic low level math formulas\\{\footnotesize (developped and tested on \macosxname{})}}
\author{Christophe BAL}
\date{2017-10-21}

\maketitle


\section{Introduction}



\LaTeX{} is a great tool for writing mathematics, maybe it is the best, but the power of \LaTeX{} allows very poor semantic writings.
The modest purpose of the package \verb+lymath+ is to give some semantic macros to write elementary mathematic formulas.
Here is an example of \LaTeX{} code that you can use with \verb+lymath+.

\begin{tcblisting}{listing only}
% Without lyxam

Knowing that $\frac{df}{dx}(x) = 4 cos(x^2)$ on $[a ; b]$, we have: 
$\int_a^b cos(x^2) dx = \left[ \frac{1){4} f(x) \right]_a^b$.

% With lyxam

Knowing that $\derfrac{f}{x}(x) = 4 cos(x^2)$ on $\intervalC{a}{b}$, we have: 
$\int_a^b cos(x^2) \dd{x} = \hook{\frac{1){4} f(x)}{a}{b}$.
\end{tcblisting}

Even if some commands are longest to write than direct \LaTeX{} commands, there are two benefits.
\begin{enumerate}
	\item The formatting inside your document is consistent.

	\item \verb+lymath+ resolves some "complex" problems automatically for you.
\end{enumerate}


\section{Miscellaneous}



\subsection{Semi-colon and spacing with the french option of \texttt{babel}}

\begin{tcblisting}{}
Only \textbf{if you use \texttt{babel} with the option \texttt{francais}}, then
you will see the same spacing around the semi-colon in $A(x;y)$. Que c'est beau !
\end{tcblisting}


\section{Sets}



\subsection{Different kind of sets}

    \subsubsection{Sets for geometry}

        \paragraph{Example of use}

\begin{tcblisting}{}
You can semantically write $\geoset{C}$, $\geoset{D}$ and $\geoset{d}$ but you can
not write things like \verb+$\geoset{ABC}$+.
\end{tcblisting}


        \paragraph{Technical ID}

\IDmacro*{geoset}{1}

\IDarg{} one single \ascii{} letter indicating a geometrical set.



    \subsubsection{Sets for probability}

        \paragraph{Example of use}

\begin{tcblisting}{}
You can semantically write $\probaset{E}$ and $\probaset{G}$ but you can not write
things like \verb+$\probaset{ABC}$+.
\end{tcblisting}


        \paragraph{Technical ID}

\IDmacro*{probaset}{1}

\IDarg{} one single upper \ascii{} upper letter indicating a probabilistic set.



    \subsubsection{Sets for rings and fields theory}

        \paragraph{Example of use}

\begin{tcblisting}{}
You can semantically write $\fieldset{A}$, $\fieldset{K}$, $\fieldset{h}$ and
$\fieldset{k}$, but you can't write things like \verb+$\fieldset{ABC}$+.
\end{tcblisting}


        \paragraph{Technical ID}

\IDmacro*{fieldset}{1}

\IDarg{} either one of the letters \texttt{h} and \texttt{k}, or one single upper \ascii{} letter indicating a field or ring like set.



    \subsubsection{Classical sets}

\begin{tcblisting}{}
You can directly use $\nullset$, $\NN$, $\ZZ$, $\DD$, $\QQ$, $\RR$, $\CC$, $\HH$
and $\OO$.
\end{tcblisting}



    \subsubsection{Classical sets with suffixes}

\begin{tcblisting}{}
It is easy to type $\RRn$, $\RRp$, $\RRs$, $\RRsn$ and $\RRsp$.
\end{tcblisting}


We have used suffixes \verb+n+ for \verb+Negative+, \verb+p+ for \verb+Positive+, and \verb+s+ for \verb+Star+ with the additional composite suffixes \verb+sn+ et \verb+sp+.

\medskip

Note that you can't use \verb+$\CCn$+ for $\specialset{\CC}{n}$ because the set $\CC$ doesn't have any standard powerful ordered structure. Take a look at the next section to see how to write $\specialset{\CC}{n}$ if you need it.

\medskip

The following table shows when you can add one of the suffixes \verb+n+, \verb+p+, \verb+s+, \verb+sn+ and \verb+sp+.

% == Table of suffixes - START == %
\newcommand\xx{\phantom{$\times$}}
\begin{table}[h]
    \caption{Suffixes}
    \begin{center}
        \begin{tabular}{c|c|c|c|c|c}
  & \verb+n+ & \verb+p+ & \verb+s+ & \verb+sn+ & \verb+sp+ \\
\hline \verb+N+ & \xx & \xx & $\times$ & \xx & \xx \\
\hline \verb+Z+ & $\times$ & $\times$ & $\times$ & $\times$ & $\times$ \\
\hline \verb+D+ & $\times$ & $\times$ & $\times$ & $\times$ & $\times$ \\
\hline \verb+Q+ & $\times$ & $\times$ & $\times$ & $\times$ & $\times$ \\
\hline \verb+R+ & $\times$ & $\times$ & $\times$ & $\times$ & $\times$ \\
\hline \verb+C+ & \xx & \xx & $\times$ & \xx & \xx \\
\hline \verb+H+ & \xx & \xx & $\times$ & \xx & \xx \\
\hline \verb+O+ & \xx & \xx & $\times$ & \xx & \xx \\
        \end{tabular}
    \end{center}
    \label{default}
\end{table}
% == Table of suffixes - END == %



    \subsubsection{Suffixes on demand}

        \paragraph{Example of use}

\begin{tcblisting}{}
You can indeed write things like $\specialset{\CC}{n}$ or $\specialset{\HH}{sp}$. 
There is also $\specialset*{\probaset{P}}{n}$ with another formatting.
\end{tcblisting}


        \paragraph{Technical IDs}

\IDmacro*{specialset}{2}

\IDmacro*{specialset*}{2}

\IDarg{1} the set to be "suffixed".

\IDarg{2} one of the suffixes \verb+n+, \verb+p+, \verb+s+, \verb+sn+ and \verb+sp+.




\subsection{Intervals}

    \subsubsection{Real intervals - French (?) notation}

        \paragraph{Example of use \#1}

\begin{tcblisting}{}
In $\intervalOC{a}{b} = ]a ; b] = \intervalOC{a}{b}$, you can see that the macro
used solves a spacing problem, and that the delimiters are a little bigger.

% The syntax refers to O-pened and C-losed but CC and OO are reduced to C and O.
\end{tcblisting}


        \paragraph{Example of use \#2}

\begin{tcblisting}{}
You can use the star version of a macro if you want the delimiters to stretch 
vertically.

$\displaystyle \intervalC{ \frac{1}{2} }{ 1^{2^{3}} }
             = [ \frac{1}{2} ; 1^{2^{3}} ] 
             = \intervalC*{ \frac{1}{2} }{ 1^{2^{3}} }$
\end{tcblisting}


        \paragraph{Technical IDs}

For all the macros above, the star version produces intervals with delimiters that fit vertically with the bounds of the interval.


\bigskip


% Docs for french real intervals - START

\IDmacro*{intervalCO}{2}

\IDmacro*{intervalCO*}{2}

\IDarg{1} lower bound $a$ of the interval $\intervalCO{a}{b}$.

\IDarg{2} upper bound $b$ of the interval $\intervalCO{a}{b}$.


\bigskip


\IDmacro*{intervalC}{2}

\IDmacro*{intervalC*}{2}

\IDarg{1} lower bound $a$ of the interval $\intervalC{a}{b}$.

\IDarg{2} upper bound $b$ of the interval $\intervalC{a}{b}$.


\bigskip


\IDmacro*{intervalO}{2}

\IDmacro*{intervalO*}{2}

\IDarg{1} lower bound $a$ of the interval $\intervalO{a}{b}$.

\IDarg{2} upper bound $b$ of the interval $\intervalO{a}{b}$.


\bigskip


\IDmacro*{intervalOC}{2}

\IDmacro*{intervalOC*}{2}

\IDarg{1} lower bound $a$ of the interval $\intervalOC{a}{b}$.

\IDarg{2} upper bound $b$ of the interval $\intervalOC{a}{b}$.

% Docs for french real intervals - END



    \subsubsection{Real intervals - American notation}

        \paragraph{Example of use}

\begin{tcblisting}{}
A semi-closed interval $\intervalPC{a}{b} = (a ; b]$ and an opened one
$\intervalP{a}{b} = (a ; b)$.

% The syntax refers to P-arenthesis.
\end{tcblisting}


        \paragraph{Technical IDs}

For all the macros above, the star version produces intervals with delimiters that fit vertically with the bounds of the interval.


\bigskip


% Docs for american real intervals - START

\IDmacro*{intervalCP}{2}

\IDmacro*{intervalCP*}{2}

\IDarg{1} lower bound $a$ of the interval $\intervalCP{a}{b}$.

\IDarg{2} upper bound $b$ of the interval $\intervalCP{a}{b}$.


\bigskip


\IDmacro*{intervalP}{2}

\IDmacro*{intervalP*}{2}

\IDarg{1} lower bound $a$ of the interval $\intervalP{a}{b}$.

\IDarg{2} upper bound $b$ of the interval $\intervalP{a}{b}$.


\bigskip


\IDmacro*{intervalPC}{2}

\IDmacro*{intervalPC*}{2}

\IDarg{1} lower bound $a$ of the interval $\intervalPC{a}{b}$.

\IDarg{2} upper bound $b$ of the interval $\intervalPC{a}{b}$.

% Docs for american real intervals - END



    \subsubsection{Discrete intervals of integers}

        \paragraph{Example of use}

\begin{tcblisting}{}
By definition, $\ZintervalC{-1}{4} = \{ -1 ; 0 ; 1 ; 2 ; 3 ; 4 \}$. So we also have
$\ZintervalC{-1}{4} = \ZintervalO{-2}{5}$.

% The syntax refers to Z the set of integers.
\end{tcblisting}


        \paragraph{Technical IDs}

For all the macros above, the star version produces intervals with delimiters that fit vertically with the bounds of the interval.


\bigskip


% Docs for discrete intervals - START

\IDmacro*{ZintervalCO}{2}

\IDmacro*{ZintervalCO*}{2}

\IDarg{1} lower bound $a$ of the interval $\ZintervalCO{a}{b}$.

\IDarg{2} upper bound $b$ of the interval $\ZintervalCO{a}{b}$.


\bigskip


\IDmacro*{ZintervalC}{2}

\IDmacro*{ZintervalC*}{2}

\IDarg{1} lower bound $a$ of the interval $\ZintervalC{a}{b}$.

\IDarg{2} upper bound $b$ of the interval $\ZintervalC{a}{b}$.


\bigskip


\IDmacro*{ZintervalO}{2}

\IDmacro*{ZintervalO*}{2}

\IDarg{1} lower bound $a$ of the interval $\ZintervalO{a}{b}$.

\IDarg{2} upper bound $b$ of the interval $\ZintervalO{a}{b}$.


\bigskip


\IDmacro*{ZintervalOC}{2}

\IDmacro*{ZintervalOC*}{2}

\IDarg{1} lower bound $a$ of the interval $\ZintervalOC{a}{b}$.

\IDarg{2} upper bound $b$ of the interval $\ZintervalOC{a}{b}$.

% Docs for discrete intervals - END


\section{Analysis}



\subsection{Constants}

    \subsubsection{Classical constants}

        \paragraph{Complete list}
       
% List of classical constants

\foreach \k in {ggamma, ppi, ttau, ee, ii, jj, kk}{\IDconstant{\k}}

\begin{tcblisting}{}
List of all classical constants where $\ttau = \frac{\ppi}{2}$ is the youngest one:
$\ggamma$, $\ppi$, $\ttau$, $\ee$, $\ii$, $\jj$ and $\kk$.
\end{tcblisting}


\begin{remark}
	Take care that \verb+{\Large $\ppi \neq \pi$}+ produces {\Large $\ppi \neq \pi$}. As you can see, the symbols are not the same. Indeed, this is true for all the greek constants.
\end{remark}



    \subsubsection{User's latine constants}

        \paragraph{Example of use}

\begin{tcblisting}{}
It is easy to write $\ct{a} x^2 + \ct{b} x + \ct{c}$ instead of $a x^2 + b x + c$
such as to stress the fact that $\ct{a}$, $\ct{b}$ and $\ct{c}$ are constants.
\end{tcblisting}


        \paragraph{Technical ID}

\IDmacro*{ct}{1}

\IDarg{} a latine text, and not a formula, indicating one constant.




\subsection{Special functions}

    \subsubsection{Some examples of use}

\begin{tcblisting}{}
Some additional special functions :
$\ch x \neq ch x$, $\ppcm(x;y)$, $\lg x =\logb{2} x$ and $\expb{6} y$.
\end{tcblisting}


    \subsubsection{Functions without parameter}

All the following macros don't have any parameter.

% List of functions without parameter

\foreach \k in {pgcd, ppcm, ch, sh, th, ach, ash, ath, arccosh, arcsinh, arctanh, acos, asin, atan}{\IDconstant{\k}}

\medskip

\begin{tabular*}{\textwidth}%
                {@{\extracolsep{\fill}}*{4}{l}}
    \verb+\pgcd+ & \verb+\ppcm+ & \verb+\ch+ & \verb+\sh+\\
    \verb+\th+ & \verb+\ach+ & \verb+\ash+ & \verb+\ath+\\
    \verb+\arccosh+ & \verb+\arcsinh+ & \verb+\arctanh+ & \verb+\acos+\\
    \verb+\asin+ & \verb+\atan+ &  & \\
\end{tabular*}



    \subsubsection{Functions with parameters}

        \paragraph{The complete list}

All the following macros have at least one parameter.

% List of functions with parameters

\medskip

\begin{tabular*}{\textwidth}%
                {@{\extracolsep{\fill}}*{4}{l}}
    \verb+\expb+ \, (1 parameter) & \verb+\logb+ \, (1 parameter) &  & \\
\end{tabular*}


        \paragraph{Technical IDs}

\IDmacro*{expb}{1}

\IDmacro*{logb}{1}

\IDarg{} the base of the exponential or the logarithm




\subsection{Extended notations for special sequences}

        \paragraph{Example of use}

\begin{tcblisting}{}
Sometimes we need to write $\seqplus{F}{1}{2}$ or $\hypergeo{F}{1}{2}$ and crazy
-wo-men really (?) love $\suprageo{F}{1}{2}{3}{4}$.
\end{tcblisting}


        \paragraph{Technical IDs}

\IDmacro*{seqplus}{2}

\IDarg{1} the right exponent like expression.

\IDarg{2} the right indice.


\bigskip


\IDmacro*{hypergeo}{2}

\IDarg{1} the left indice.

\IDarg{2} the right indice.


\bigskip


\IDmacro*{suprageo}{4}

\IDarg{1} the left indice.

\IDarg{2} the right indice.

\IDarg{3} the right exponent.

\IDarg{4} the left exponent.




\subsection{Differential calculus}

    \subsubsection{\texorpdfstring{The $\pp{}$ and $\dd{}$ operators}%
                               {The "rounded d" and "straight d" operators}}

        \paragraph{Example of use}

\begin{tcblisting}{}
You can write $\dd{x}$ and $\pp{t}$ and also $ \dd[5]{x}$ or $\pp[n]{x}$.
\end{tcblisting}


        \paragraph{Technical IDs}

\IDmacro{dd}{1}{1}

\IDmacro{pp}{1}{1}

\IDoption{} if used, this argument will be the exponent of the symbol $\pp{}$ or $\dd{}$.

\IDarg{} the variable of differentiation at the right of the symbol $\pp{}$ or $\dd{}$.



    \subsubsection{Total derivation}

        \paragraph{Example of use \#1}

\begin{tcblisting}{}
$\displaystyle f'(a)
             = \derpow{f} (a) 
             = \derfrac{f}{x} (a)
             = \dersub{f}{x} (a)$
\end{tcblisting}


        \paragraph{Example of use \#2}

\begin{tcblisting}{}
$\displaystyle f'''(a)
             = \derpow[3]{f} (a)
             = \derfrac[3]{f}{x} (a)
             = \dersub[3]{f}{x} (a)$
and
$\displaystyle \cos''' a = \derfrac[3]{\cos}{x} (a)$.
\end{tcblisting}


        \paragraph{Example of use \#3}

\begin{tcblisting}{}
If $\displaystyle f(x) = \frac{1}{x^2+3}$, then we can write :
$\displaystyle \derpow[3]{f} (a)
             = \derfrac*[3]{\left( \frac{1}{x^2+3} \right)}{x} (a)$.
\end{tcblisting}


        \paragraph{Technical IDs}

\IDmacro{derpow}{1}{1}

\IDoption{} if used, this argument will be the exponent of derivation put inside braces.

\IDarg{} the function to be differenciated.


\bigskip


\IDmacro{derfrac}{1}{2}

\IDmacro{derfrac*}{1}{2}

\IDmacro{dersub}{1}{2}

\IDoption{} if used, the exponent of derivation.

\IDarg{1} the function to be differenciated.

\IDarg{2} the variable used for the derivation.



    \subsubsection{Partial derivation}

        \paragraph{Example of use \#1}

\begin{tcblisting}{}
$\displaystyle \partialfrac{f}{x} (a;b)
             = \partialsub{f}{x} (a;b)
             = \partialprime{f}{x} (a;b)$
\end{tcblisting}


        \paragraph{Example of use \#2}

\begin{tcblisting}{}
$\displaystyle \partialfrac[3]{G}{f^2 // v} (a;b)
             = \partialfrac{G}{f^2 // v} (a;b)
             = \partialsub{G}{f^2 // v} (a;b)
             = \partialprime{G}{f^2 // v} (a;b)$
\end{tcblisting}


        \paragraph{Example of use \#3}

\begin{tcblisting}{}
If $\displaystyle f(x;y) = \frac{cos(x y)}{x^2+y^2}$, then we can study
$\displaystyle \partialfrac[2]{f}{x // y}
= \partialfrac*[2]{\left( \frac{cos(x y)}{x^2 + y^2} \right)}{x // y}$.
\end{tcblisting}


        \paragraph{Technical IDs}

\IDmacro{partialfrac}{1}{2}

\IDmacro{partialfrac*}{1}{2}

\IDoption{} if used, the exponent of $\pp$ associated to the function differenciated.

\IDarg{1} the function to be partially differenciated.

\IDarg{2} the variables used for the partial derivation. The syntax is particular : for example, \verb+x // y^3 // ...+ indicates regarding to the variables $x$ one time, $y$ three times... and so on.


\bigskip


\IDmacro*{partialsub}{2}

\IDmacro*{partialprime}{2}

\IDarg{1} the function to be partially differenciated.

\IDarg{2} the variables used for the partial derivation. The syntax is particular : for example, \verb+x // y^3 // ...+ indicates regarding to the variables $x$ one time, $y$ three times... and so on.




\subsection{Integral calculus}

    \subsubsection{The hook operator - 1st version}

        \paragraph{Example of use \#1}

\begin{tcblisting}{}
By definition, $\displaystyle \int_{a}^{b} f(x) \dd{x} = \hook{F(x)}{a}{b}$ where
$\hook{F(x)}{a}{b} = F(b) - F(a)$.
\end{tcblisting}


        \paragraph{Example of use \#2}

\begin{tcblisting}{}
With the star version, the hooks stretch vertically like in
$\displaystyle \hook*{\frac{x - 1}{5 + x^2}}{a}{b}
             = \hook{\frac{x - 1}{5 + x^2}}{a}{b}$.
\end{tcblisting}


        \paragraph{Technical IDs}

\IDmacro*{hook}{3}

\IDmacro*{hook*}{3}

\IDarg{1} the content inside the hooks.

\IDarg{2} the lower bound displayed as an index.

\IDarg{3} the upper bound displayed as an exponent.



    \subsubsection{The hook operator - 2nd version}

        \paragraph{Example of use \#1}

\begin{tcblisting}{}
You can use $\vhook{F(x)}{a}{b}$ instead of $\hook{F(x)}{a}{b}$.
\end{tcblisting}


        \paragraph{Example of use \#2}

\begin{tcblisting}{}
With the star version, the rule at the right stretches vertically like in
$\displaystyle \vhook*{\frac{x - 1}{5 + x^2}}{a}{b}
             = \vhook{\frac{x - 1}{5 + x^2}}{a}{b}$.
\end{tcblisting}


        \paragraph{Technical IDs}

\IDmacro*{vhook}{3}

\IDmacro*{vhook*}{3}

\IDarg{1} the content before the vertical line $\vert$ .

\IDarg{2} the lower bound displayed as an index.

\IDarg{3} the upper bound displayed as an exponent.



    \subsubsection{Several integrals}

The package minimizes spacings between consecutive symbols of integration. Here is an example.

\begin{tcblisting}{}
$\displaystyle \int \int \int F(x;y;z) \dd{x} \dd{y} \dd{z}
= \int_{a}^{b} \int_{c}^{d} \int_{e}^{f} F(x;y;z) \dd{x} \dd{y} \dd{z}$
\end{tcblisting}


\begin{remark}
	By default, \LaTeX{} prints
	\makeatletter
    	$\displaystyle \original@int \original@int \original@int F(x;y;z) \dd{x} \dd{y} \dd{z}
    	= \original@int_{a}^{b} \original@int_{c}^{d} \original@int_{e}^{f} F(x;y;z) \dd{x} \dd{y} \dd{z}$.
	\makeatother
\end{remark}




\subsection{Asymptotic comparisons of sequences and functions}

    \subsubsection{\texorpdfstring{The $\bigO{}$ and $\smallO{}$ notations}%
                               {The "big O" and "small O" notations}}

        \paragraph{Example of use}

\begin{tcblisting}{}
Let's see how to use the symbols $\bigO{}$ and $\smallO{}$ created by Landau.

\medskip

You can write $\bigO{x} \neq \smallO{x}$ and $e^{t + \smallO{t}} = e^{\bigO{t}}$.
\end{tcblisting}


        \paragraph{Technical IDs}

\IDmacro*{bigO}{1}

\IDmacro*{smallO}{1}

\IDarg{} the content inside the braces after the symbol $\bigO{}$ or $\smallO{}$.



    \subsubsection{\texorpdfstring{The $\bigomega{}$ notation}%
                               {The "big Omega" notation}}

        \paragraph{Example of use}

\begin{tcblisting}{}
Let's see how to use the symbol $\bigomega{}$ created by Hardy and Littlewood.

\medskip

$f(n) = \bigomega{g(n)}$ means: $\exists (m, n_0)$ such that
$n \geqslant n_0$ implies $f(n) \geqslant m g(n)$.
\end{tcblisting}


        \paragraph{Technical ID}

\IDmacro*{bigomega}{1}

\IDarg{} the content inside the braces after the symbol $\bigomega{}$.



    \subsubsection{\texorpdfstring{The $\bigtheta{}$ notation}%
                               {The "big Theta" notation}}

        \paragraph{Example of use}

\begin{tcblisting}{}
Let's see how to use the symbol $\bigtheta{}$.

\medskip

$f(n) = \bigtheta{g(n)}$ means: $\exists (m, M, n_0)$ such that $n \geqslant n_0$ 
implies $m g(n) \leqslant f(n) \leqslant M g(n)$.
\end{tcblisting}


        \paragraph{Technical ID}

\IDmacro*{bigtheta}{1}

\IDarg{} the content inside the braces after the symbol $\bigtheta{}$.


\section{Geometry}



\subsection{Points}

	\subsubsection{Example of use}

\begin{tcblisting}{}
We can call $\pt{I}$ the middle of $[\pt{A}\pt{B}]$.
\end{tcblisting}



	\subsubsection{Technical ID}

\IDmacro*{pt}{1}

\IDarg{} one text indicating the name of a point.




\subsection{Vectors}

	\subsubsection{Example of use}

\begin{tcblisting}{}
Here is one vector $\vect{ABCDEF}$ using a lot of letters and you can write
$\vect*{e}{rot}$ instead of $\vect{e_{rot}}$.
\end{tcblisting}



	\subsubsection{Technical IDs}

\IDmacro*{vect}{1}

\IDarg{} one text indicating the name of a vector.


\bigskip


\IDmacro*{vect*}{2}

\IDarg{1} one text indicating $up$ in the name $\vect*{up}{down}$ of a vector.

\IDarg{2} one text indicating $down$ in the name $\vect*{up}{down}$ of a vector.




\subsection{Inner angles}

	\subsubsection{Example of use}

\begin{tcblisting}{}
Here is one inner angle $\anglein{ABCDEF}$ using a lot of letters, and you can 
write $\anglein*{A}{rot}$ instead of $\anglein{A_{rot}}$.
\end{tcblisting}



	\subsubsection{Technical IDs}

\IDmacro*{anglein}{1}

\IDarg{} one text indicating the name of an inner angle.


\bigskip


\IDmacro*{anglein*}{2}

\IDarg{1} one text indicating $up$ in the name $\anglein*{up}{down}$ of an inner angle.

\IDarg{2} one text indicating $down$ in the name $\anglein*{up}{down}$ of an inner angle.




\subsection{Circular arcs}

	\subsubsection{Example of use}

\begin{tcblisting}{}
Here is one arc $\arc{ABCDEF}$ using a lot of letters, and you can write 
$\arc*{A}{rot}$ instead of $\arc{A_{rot}}$.
\end{tcblisting}



	\subsubsection{Technical IDs}

\IDmacro*{arc}{1}

\IDarg{} one text indicating the name of a circular arc.


\bigskip


\IDmacro*{arc*}{2}

\IDarg{1} one text indicating $up$ in the name $\arc*{up}{down}$ of a circular arc.

\IDarg{2} one text indicating $down$ in the name $\arc*{up}{down}$ of a circular arc.


\section{Arithmetic}



\subsection{Continued fractions}

	\subsubsection{Standard continued fractions}

        \paragraph{Example of use}

\begin{tcblisting}{}
It is easy to write what is just after where the inline notation seems to have been 
introduced by Alfred Pringsheim (the left notation is always space consuming for a 
better readability).

$ \contfrac{u_0 // u_1 // u_2 // \dots // u_n}
= \contfrac*{u_0 // u_1 // u_2 // \dots // u_n}$
\end{tcblisting}


        \paragraph{Technical IDs}

\IDmacro*{contfrac}{1}

\IDmacro*{contfrac*}{1}

\IDarg{} all the elements of the continued fraction separated by \verb+//+.




	\subsubsection{Generalized continued fractions}

        \paragraph{Example of use}

\begin{tcblisting}{}
You can use similar notations for generalized continued fractions :

$\displaystyle \contfracgene{a // b // c // d // e // f // \dots // y // z}
             = \contfracgene*{a // b // c // d // e // f // \dots // y // z}$
\end{tcblisting}


        \paragraph{Technical IDs}

\IDmacro*{contfracgene}{1}

\IDmacro*{contfracgene*}{1}

\IDarg{} all the elements of the generalized continued fraction separated by \verb+//+.



	\subsubsection{Single like continued fraction}

        \paragraph{Example of use}

\begin{tcblisting}{}
Crazy men really (?) need to write things like $\singlecontfrac{a}{b}$.

% The existence of this macro just comes from its use internally.
\end{tcblisting}


        \paragraph{Technical ID}

\IDmacro*{singlecontfrac}{2}

\IDarg{1} the pseudo numerator

\IDarg{2} the pseudo denominator



    \subsubsection{\texorpdfstring{The $\contfracope$ operator}%
                               {The K operator}}

        \paragraph{Example of use \#1}

\IDconstant{contfracope}

\begin{tcblisting}{}
The following notation is very closed to the one used by Carl Friedrich Gauss:

$\displaystyle
  \contfracope_{k=1}^{n} (b_k:c_k)
= \cfrac{b_1}{\contfracgene{c_1 // b_2 // c_2 // b_3 // \dots // b_n // c_n}}$
\end{tcblisting}


\begin{remark}
	The letter $\contfracope$ comes from "kettenbruch" which means "continued
fraction" in german.
\end{remark}


        \paragraph{Example of use \#2}

\begin{tcblisting}{}
$\displaystyle
  u_0 + \contfracope_{k=1}^{n} (1:u_k)
= \contfrac{u_0 // u_1 // u_2 // \dots // u_n}$
\end{tcblisting}


\section{Change log}



\begin{description}
	\item[2017-10-21] Major changes are now added in this PDF documentation.

	\item[2017-10-18] New minor version \verb+0.0.2-beta+ : for differential calculus, two new star versions of \verb+\derfrac+ and \verb+\partialfrac+ which use an operator like notation.

	\item[2017-10-06] New minor version \verb+0.0.1-beta+ : additional tools for arithmetic, geometry, integral calculus and differential calculus.

	\item[2017-10-02] First version of the package.
\end{description}



\end{document}
