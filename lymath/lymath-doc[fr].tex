%\documentclass[12pt,a4paper]{scrartcl}
\documentclass[12pt,a4paper]{article}

\makeatletter % Technical doc - START

\usepackage[utf8]{inputenc}
\usepackage[T1]{fontenc}
\usepackage{ucs}

\usepackage[french]{babel,varioref}

\usepackage[top=2cm, bottom=2cm, left=1.5cm, right=1.5cm]{geometry}
\usepackage{enumitem}

\usepackage{color}
\usepackage{hyperref}
\hypersetup{
    colorlinks,
    citecolor=black,
    filecolor=black,
    linkcolor=black,
    urlcolor=black
}

\usepackage{amsthm}

\usepackage{tcolorbox}
\tcbuselibrary{listingsutf8}

\usepackage{ifplatform}

\usepackage{pgffor}
\usepackage{xstring}


% MISC

\tcbset{%
	sharp corners,%
	left=1mm, right=1mm,%
	bottom=1mm, top=1mm,%
	colupper=red!75!blue% 
}

\setlength{\parindent}{0cm}
\setlist{noitemsep}

\theoremstyle{definition}
\newtheorem*{remark}{Remarque}

\usepackage[raggedright]{titlesec}

\titleformat{\paragraph}[hang]{\normalfont\normalsize\bfseries}{\theparagraph}{1em}{}
\titlespacing*{\paragraph}{0pt}{3.25ex plus 1ex minus .2ex}{0.5em}

\newcommand\ascii{\texttt{ASCII}}


% Technical IDs

\newwrite\tempfile

\immediate\openout\tempfile=x-\jobname.macros-x.txt

\AtEndDocument{\immediate\closeout\tempfile}

\newcommand\IDconstant[1]{%
    \immediate\write\tempfile{constant@#1}%
}


\newcommand\IDmacro{\@ifstar{\@IDmacro@star}{\@IDmacro@no@star}}

\newcommand\@IDmacro@no@star[3]{%
    \texttt{%
    	\textbackslash#1 <macro>%
    	\IfStrEq{#2}{0}{}{%
    		\,\,[#2 Option%
				\IfStrEq{#2}{1}{}{s}]%
			}%
	    \IfStrEq{#3}{}{}{%
	    		\,\,(#3 Argument%
				\IfStrEq{#3}{1}{}{s})%
			}
	   	}
    \immediate\write\tempfile{macro,#1,#2,#3}%
}

\newcommand\@IDmacro@star[2]{%
    \@IDmacro@no@star{#1}{0}{#2}%
}



\newcommand\IDenv{\@ifstar{\@IDenv@star}{\@IDenv@no@star}}

\newcommand\@IDenv@no@star[2]{%
    \texttt{%
    	#1 <env>%
    	\IfStrEq{#2}{0}{}{%
    		\,\,[#2 Option%
				\IfStrEq{#2}{1}{}{s}]%
			}%
	   	}
    \immediate\write\tempfile{env,#1,#2}%
}

\newcommand\@IDenv@star[1]{%
    \@IDenv@no@star{#1}{0}%
}


\newcommand\@IDoptarg{\@ifstar{\@IDoptarg@star}{\@IDoptarg@no@star}}

\newcommand\@IDoptarg@star[2]{%
	\vspace{0.5em}
	\textbf{---} \texttt{#1%
		\IfStrEq{#2}{}{:}{\,#2:}%
	}%
}

\newcommand\@IDoptarg@no@star[2]{%
	\IfStrEq{#2}{}{%
		\@IDoptarg@star{#1}{}%
	}{%
		\@IDoptarg@star{#1}{#2}%
	}%
}


\newcommand\IDkey[1]{%
	\@IDoptarg*{Option}{{\itshape "#1"}}%
}


\newcommand\IDoption[1]{%
	\@IDoptarg{Option}{#1}%
}


\newcommand\IDarg[1]{%
	\@IDoptarg{Argument}{#1}%
}


\newcommand\Content[1]{%
	\@IDoptarg{Contenu}{#1}%
}

\makeatother % Technical doc - END


\usepackage{lymath}


\begin{document}

\renewcommand\labelitemi{\raisebox{0.125em}{\tiny\textbullet}}
\renewcommand{\labelitemii}{---}

\title{%
	Le package \texttt{lymath}:\\%
	des formules plus sémantiques\\%
	{\footnotesize Code source disponible sur \url{https://github.com/bc-latex/ly-math}.}\\%
{\footnotesize Version \texttt{0.6.2-beta} développée et testée sur \macosxname{}.}%
}
\author{Christophe BAL}
\date{2019-10-14}

\maketitle


\vspace{2em}

\hrule

\tableofcontents

\vspace{1.5em}

\hrule

\newpage



\section{Introduction}

\LaTeX{} est un excellent langage, pour ne pas dire le meilleur, pour rédiger des documents contenant des formules mathématiques.
Malheureusement toute la puissance de \LaTeX{} permet d'écrire des codes très peu sémantiques.
Le modeste but du package \verb+lymath+ est de fournir quelques macros sémantiques pour la rédaction de formules mathématiques élémentaires. Considérons le code \LaTeX{} suivant.

\begin{tcblisting}{listing only}
Sachant que $\frac{df}{dx}(x) = 4 cos(x^2)$ sur $[a ; b]$ , nous avons :

$\int_a^b cos(x^2) dx = \left[ \frac{1}{4} f(x) \right]_a^b$.
\end{tcblisting}


Avec \verb+lymath+, vous pouvez écrire le code suivant.

\begin{tcblisting}{listing only}
Sachant que $\derfrac{f}{x}(x) = 4 cos(x^2)$ sur $\intervalC{a}{b}$, nous avons :

$\int_a^b cos(x^2) \dd{x} = \hook{\frac{1}{4} f(x)}{a}{b}$.
\end{tcblisting}


Même si certaines commandes sont plus longues à écrire que ce que permet \LaTeX{}, il y a trois avantages à utiliser des commandes sémantiques.
\begin{enumerate}
	\item La mise en forme dans votre document sera consistante.

	\item Il est facile de changer une mise en forme sur l'ensemble d'un document.

	\item \verb+lymath+ résout certains problèmes "complexes" pour vous.
\end{enumerate}



\section{Comment lire cette documentation ?}

Le choix a été fait de fournir des exemples comme documentation du package suivis de fiches techniques des macros-commandes. Les exemples se présentent comme ci-dessous \textit{(un code \LaTeX{} suivi de sa mise en forme)}.

\begin{tcblisting}{}
Sachant que $\displaystyle \frac{df}{dx}(x) = 4 cos(x^2)$ sur $[a ; b]$ , nous avons :
$\displaystyle \int_a^b cos(x^2) dx = \left[ \frac{1}{4} f(x) \right]_a^b$.
\end{tcblisting}



\section{A propos des macros}

\subsection{Règles de nommage}

\subsubsection{Les macros de même \og type \fg}

Les macros partageant une même fonctionnalité mathématique auront toute le même préfixe comme par exemple pour \verb+\derpow+ , \verb+\derfrac+ \dots{} utilisables pour rédiger des dérivées de fonctions.
Ce choix est assumé même si pour les macros du type \verb+\set...+ on obtient un nom pouvant faire penser à \emph{\og régler ... \fg} au lieu de \emph{\og ensemble du type ... \fg}.



\subsubsection{Les macros en mode \texttt{displaystyle}}

Les macros évitant d'avoir à taper \verb+displaystyle+ auront un nom commençant par la lettre \verb+d+.



\subsection{Versions étoilées}

Les versions étoilées proposent des mises en forme correspondant aux cas les moins usuels : par exemple une macro utilisant des parenthèses rendra ces dernières extensibles sauf dans sa version étoilée.



\subsection{Les arguments, deux conventions à connaître}

\subsubsection{Nombre fixé d'arguments}

Dans ce cas, c'est la syntaxe \LaTeX{} usuelle qui sera à utiliser comme dans \verb+\derfrac{f}{x}+.



\subsubsection{Nombre variable d'arguments}

Certaines macros offrent la possibilité de fournir un nombre variable d'arguments comme dans \verb+\coord{x | y | z | t}+ et \verb+\coord{x | y}+.
Ceci se fait en utilisant un seul argument, au sens de \LaTeX{}, dont le contenu est formé de morceaux séparés par des traits verticaux \verb+|+.
Ainsi dans \verb+\coord{x | y | z | t}+, l'unique argument \verb+x | y | z | t+, au sens de \LaTeX{}, sera analysé par \verb+lyxam+ comme étant formé des quatre arguments \verb+x+ , \verb+y+ , \verb+z+ et \verb+t+.




\section{Deux séparateurs d'arguments par défaut}

La macro \verb+\lymathsep+ définit le séparateur d'arguments de premier niveau, et \verb+\lymathsubsep+ celui des arguments de deuxième niveau.
Cette documentation utilisant l'option \verb+french+ de \verb+babel+, la valeur de 
\verb+\lymathsep+ est \fbox{\,\lymathsep$\vphantom{F}$\,} 
et celle de
\verb+\lymathsubsep+ est \fbox{\,\lymathsubsep$\vphantom{F}$\,} .
Sans ce choix, les valeurs de \verb+\lymathsep+ et \verb+\lymathsubsep+ seront \fbox{\,\lymathsubsep$\vphantom{F}$\,} et \fbox{\,\lymathsep$\vphantom{F}$\,} respectivement.




\section{Quelques gestions d'espaces}

\subsection{Espace et fraction}

Quand on utilise \verb+\frac+ ou \verb+\dfrac+, de petits espaces sont automatiquement ajoutés pour éviter d'avoir des traits de fraction trop petits. Le comportement par   défaut se retrouve en utilisant les macros \verb+\stdfrac+ et \verb+\stddfrac+ . Voici un exemple.

\begin{tcblisting}{}
Vous avez $\frac{2}{3} = \stdfrac{2}{3}$ et $\dfrac{2}{3} = \stddfrac{2}{3}$ .
\end{tcblisting}




%\section{Quelques gestions d'espaces}

\subsection{Espace et racines n-ièmes d'un réel}

\verb+\sqrt+ a été redéfini pour ajouter un peu d'espaces. Le comportement par défaut se retrouve en utilisant la macro \verb+\stdsqrt+ . Voici un exemple.


\begin{tcblisting}{}
Vous avez $\sqrt{2} = \stdsqrt{2}$ et $\sqrt[n]{45} = \stdsqrt[n]{45}$ .
\end{tcblisting}




%\section{Quelques gestions d'espaces}

\subsection{Sommes et produits en mode ligne}

Pour limiter l'espace, \LaTeX{} affiche $\sum_{k=0}^{n}$ et non $\dsum_{k=0}^{n}$ sauf si l'on utilise la commande \verb+\displaystyle+.
Les macros \verb+\dsum+ et \verb+\dprod+ permettent de se passer de \verb+\displaystyle+.
Voici un exemple.


\begin{tcblisting}{}
$\dsum_{k=0}^{n} 2^k = \sum_{k=0}^{n} 2^k = 2^{n+1} - 1$

$\dprod_{k=1}^{n} k = \prod_{k=1}^{n} k = k !$
\end{tcblisting}




%\section{Quelques gestions d'espaces}

\subsection{Espace et point-virgule avec l'option \texttt{french} de \texttt{babel}}

\textbf{Seulement si vous utilisez \texttt{babel} avec l'option \texttt{french}}, comme c'est le cas dans cette documentation, alors vous verrez le même espacement autour du point-virgule dans $A(x;y)$. Que c'est beau !




\section{Logique et fondements}

\subsection{Différents types d'égalités \og standard \fg}

D'un point de vue pédagogique, il peut être intéressant de disposer de différentes façon d'écrire une égalité, une non égalité ou une inégalité. Voici ce qui est proposé.

\subsubsection{Définir quelque chose}

L'exemple suivant montre trois façons de rédiger une égalité signifiant une définition
\footnote{
	Le symbole peu courant $\eqdef**$ est utilisé par le langage B qui permet de spécifier et prouver certains programmes.
}
\emph{(la section \ref{text-for-opes} explique comment est définit le texte \emph{\og \textopdef \fg})}.

\begin{tcblisting}{}
La fonction $f$ est définie sur $\RR$ par $f(x) \eqdef x^3 + 1$ ou avec les écritures
symboliques $f(x) \eqdef* x^3 + 1$ et $f(x) \eqdef** x^3 + 1$.
\end{tcblisting}


\subsubsection{Indiquer une identité}

L'exemple suivant montre deux façons de rédiger des identités, la noptation symbolique n'étant pas standard \emph{(la section \ref{text-for-opes} explique comment est défini le texte \emph{\og \textopid \fg})}.

\begin{tcblisting}{}
$\forall (a ; b) \in \RR^2$, nous avons : $(a + b)^2 \eqid a^2 + b^2 + 2 a b$ .
On peut utiliser une écriture plus symbolique : $(a + b)^2 \eqid* a^2 + b^2 + 2 a b$ .
\end{tcblisting}


\subsubsection{Une égalité à vérifier ou non, une hypothèse, une condition}

Se reporter à la section \ref{text-for-opes} pour savoir comment sont définis les textes \emph{\og \textopcond \fg} et \emph{\og \textophyp \fg}.

\begin{tcblisting}{}
Est-il vrai que $(a + b)^3 \eqtest a^3 + b^3 + 3 a b$ ?
Non car $(a + b)^3 \neqid a^3 + b^3 + 3 a b$

A-t-on $x \neqhyp 0$ pour pouvoir écrire $\dfrac{1}{x}$ ?

Comme $x$ doit être non nul, je sait que $x \neqcond 0$.

Une autre condition $x \eqcond 0$ ou une autre hypothèse $x \eqhyp 0$ ?
\end{tcblisting}


\subsubsection{Une égalité indiquant le choix d'une valeur}

La section \ref{text-for-opes} permet de savoir comment le texte \emph{\og \textopchoice \fg} est défini.

\begin{tcblisting}{}
Sachant que si $x \geq 4$ alors $x^2 \geq 16$, on peut choisir $x \eqchoice 123$
et affirmer que $123^2 \geq 16$.
\end{tcblisting}



\subsubsection{Différents types d'inéquations}

Le principe reste le même pour les symboles d'équations excepté qu'il n'y a ici aucune écriture purement symbolique et que l'on pas la version \emph{\og {choix} \fg}. Voici un code \og fourre-tout \fg{} pour voir ce que vous avez à votre disposition.

\begin{tcblisting}{}
A-t-on $x \leqtest x^2$ ou $x \ltest x^2$ ?

A moins que ce ne soit $x \geqtest x^2$ ou $x \gtest x^2$ qu'il faille vérifier.

On peut supposer $x \leqhyp 1$ ou avoir la condition $x \gcond 2$.

\end{tcblisting}


	\subsection{Textes utilisés} \label{text-for-opes}

Voici les macros définissant les textes utilisés qui tiennet compte de l'utilisation ou non de l'option \verb+french+ de \verb+babel+.

\begin{enumerate}
	\item \verb+\textopchoice+ donne \emph{\og choix \fg} en français et \emph{\og choice \fg} sinon.

	\item \verb+\textopcond+ donne \emph{\og cond \fg} dans tous les cas.

	\item \verb+\textopdef+ donne \emph{\og déf \fg} en français et \emph{\og def \fg} sinon.

	\item \verb+\textophyp+ donne \emph{\og hyp \fg} dans tous les cas.

	\item \verb+\textopid+ donne \emph{\og id \fg} dans tous les cas.
\end{enumerate}




%\section{Logique et fondements}

\subsection{Équivalences et implications}

\subsubsection{Des symboles supplémentaires}

En plus des opérateurs \verb+\iff+ et \verb+\implies+ proposés par \LaTeX{}, il a été ajouté l'opérateur \verb+\liesimp+, où l'on a inversé les groupes syllabiques de \verb+\implies+, un opérateur pour pour obtenir $\liesimp$ . Voici un exemple d'utilisation \emph{(penser aussi aux preuves d'équivalence par double implication)}.

\begin{tcblisting}{}
Un théorème de la logique : $(A \implies B) \iff (B \liesimp A)$
\end{tcblisting}


Tout comme pour les égalités, il existe des versions de type test, hypothèse et condition.

\begin{tcblisting}{}
Équivalences : $A \ifftest B \iffhyp C \iffcond D$

Implications directes : $A \impliestest B \implieshyp C \impliescond D$

Implications réciproques : $A \liesimptest B \liesimphyp C \liesimpcond D$
\end{tcblisting}


\subsubsection{Équivalences et implications verticales}

Dans la section \ref{explain-proof} est expliqué comment détailler les étapes d'un raisonnement. Avec cet outil, il devient utile d'avoir des versions verticales non décorées des symboles d'équivalence et d'implication. Voici comment les obtenir.

\begin{tcblisting}{}
\begin{tabular}{ccc}
    $A$       &   $B$           &   $C$         \\
    $\viff$   &   $\vimplies$   &   $\vliesimp$ \\
    $D$       &   $E$           &   $F$
\end{tabular}
\end{tcblisting}




%\section{Logique et fondements}

\subsection{Détailler un raisonnement} \label{explain-proof}

	\paragraph{Un exemple sur une petite étape}

La macro \verb+\explain+ prend deux arguments obligatoires : le premier est un symbole et le second une courte explication. Commençons par une simple étape de raisonnement détaillée comme suit.

\begin{tcblisting}{}
$0 \leq a < b$

$\explain{\vimplies}{Par croissance de la fonction carrée sur $\RRp$.}$

$a^2 \leq b^2$
\end{tcblisting}

Si vous souhaitez un espace devant le symbole, il suffit d'indiquer l'espace via une distance grâce à l'argument optionnel comme ci-après.

\begin{tcblisting}{}
$0 \leq a < b$

$\explain[1.5em]{\vimplies}{Par croissance de la fonction carrée sur $\RRp$.}$

$a^2 \leq b^2$
\end{tcblisting}


\verb+\explain+ utilise les macros constantes suivantes.
\begin{itemize}
	\item \verb+\textexplainleft+ et \verb+\textexplainright+ qui donnent $\textexplainleft$ et $\textexplainright$ respectivement par défaut.

	\item \verb+\textexplainspacein+ est l'espacement entre le symbole et la courte explication. Par défaut, cette macro vaut \verb+2em+.
\end{itemize}



	\paragraph{Détailler des calculs}

Pour finir, voici un exemple avec l'environnement \verb+flalign+ du package \verb+amsmath+ \emph{(qui est automatiquement chargé par \emph{\texttt{lymath}})}. Il est alors relativement rapide de détailler un calcul.

\begin{tcblisting}{}
\begin{flalign*}
	& (a + b)^2
	&&\\
	& \explain{=}{On utilise $x^2 = x \cdot x$.}
	&&\\
	& (a + b) (a + b)
	&&\\
	& \explain{=}{Double développement depuis la parenthèse gauche.}
	&&\\
	& a^2 + a b + b a + b^2
	&&\\
	& \explain{=}{Commutativité du produit.}
	&&\\
	& a^2 + 2 a b + b^2
	&&\\
\end{flalign*}
\end{tcblisting}



\paragraph{Fiches techniques}

\IDmacro{explain}{1}{2}

\IDoption{} espacement avant le symbole. Valeur par défaut : \verb+0em+.

\IDarg{1} un symbole.

\IDarg{2} une courte explication.




\section{Ensembles}

\subsection{Différents types d'ensembles}

\subsubsection{Ensembles versus accolades}

\paragraph{Exemple d'utilisation 1}

\begin{tcblisting}{}
Un ensemble de beaux nombres : $\setgene{1 ; 3 ; 5}$ .
\end{tcblisting}


\paragraph{Exemple d'utilisation 2}

\begin{tcblisting}{}
Choisissez votre camp :
$\displaystyle \setgene{\frac{1}{3} ; \frac{5}{7} ; \frac{9}{11}}$
ou
$\displaystyle \setgene*{\frac{1}{3} ; \frac{5}{7} ; \frac{9}{11}}$ .
\end{tcblisting}


\paragraph{Fiches techniques}

\IDmacro*{setgene}{1}

\IDmacro*{setgene*}{1}

\IDarg{} la définition de l'ensemble.



\subsubsection{Ensembles pour la géométrie}

\paragraph{Exemple d'utilisation 1}

\begin{tcblisting}{}
Vous pouvez écrire sémantiquement $\setgeo{C}$, $\setgeo{D}$ et $\setgeo{d}$ mais pas
taper \verb+$\setgeo{ABC}$+.
\end{tcblisting}


\paragraph{Exemple d'utilisation 2}

\begin{tcblisting}{}
Pour les indices, utilisez $\setgeo*{C}{1}$, $\setgeo*{C}{2}$ \dots
\end{tcblisting}


\paragraph{Fiches techniques}

\IDmacro*{setgeo}{1}

\IDarg{} un seul caractère \ascii{} indiquant un ensemble géométrique.


\bigskip


\IDmacro*{setgeo*}{2}

\IDarg{1} un seul caractère \ascii{} indiquant $\setgeo{U}$ dans le nom $\setgeo*{U}{d}$ d'un ensemble géométrique.

\IDarg{2} un texte donnant $d$ dans le nom $\setgeo*{U}{d}$ d'un ensemble géométrique.



\subsubsection{Ensembles probabilistes}

\paragraph{Exemple d'utilisation 1}

\begin{tcblisting}{}
Vous pouvez écrire sémantiquement $\setproba{E}$ et $\setproba{G}$ mais pas taper
\verb+$\setproba{ABC}$+.
\end{tcblisting}


\paragraph{Exemple d'utilisation 2}

\begin{tcblisting}{}
Pour les indices, utilisez $\setproba*{E}{1}$, $\setproba*{E}{2}$ \dots
\end{tcblisting}


\paragraph{Fiches techniques}

\IDmacro*{setproba}{1}

\IDarg{} un seul caractère \ascii{} majuscule indiquant un ensemble probabiliste.


\bigskip


\IDmacro*{setproba*}{2}

\IDarg{1} un seul caractère \ascii{} majuscule indiquant $\setproba{U}$ dans le nom $\setproba*{U}{d}$ d'un ensemble probabiliste.

\IDarg{2} un texte donnant $d$ dans le nom $\setproba*{U}{d}$ d'un ensemble probabiliste.



\subsubsection{Ensembles pour l'algèbre générale}

\paragraph{Exemple d'utilisation 1}

\begin{tcblisting}{}
Vous pouvez écrire sémantiquement $\setalge{A}$, $\setalge{K}$, $\setalge{h}$ et
$\setalge{k}$ mais pas taper \verb+$\setalge{ABC}$+.
\end{tcblisting}


\paragraph{Exemple d'utilisation 2}

\begin{tcblisting}{}
Pour les indices, utilisez $\setalge*{k}{1}$, $\setalge*{k}{2}$ \dots
\end{tcblisting}


\paragraph{Fiches techniques}

\IDmacro*{setalge}{1}

\IDarg{} soit l'une des lettres  \texttt{h} et \texttt{k}, soit un seul caractère \ascii{} majuscule indiquant un ensemble de type anneau ou corps.


\bigskip


\IDmacro*{setalge*}{2}

\IDarg{1} un seul caractère \ascii{} indiquant $\setalge{U}$ dans le nom $\setalge*{U}{d}$ d'un ensemble de type anneau ou corps.

\IDarg{2} un texte donnant $d$ dans le nom $\setalge*{U}{d}$ d'un ensemble de type anneau ou corps.



\subsection{Ensembles classiques}

\subsubsection{La liste complète}

\begin{tcblisting}{}
Vous pouvez utiliser directement $\nullset$, $\NN$, $\ZZ$, $\DD$, $\QQ$, $\RR$, $\CC$,
mais aussi $\PP$ pour l'ensemble des nombres premiers, $\HH$ pour les quaternions et
enfin $\OO$ pour les octonions.
\end{tcblisting}



\subsubsection{Ensembles classiques suffixés}

\begin{tcblisting}{}
Il est facile de taper $\RRn$, $\RRp$, $\RRs$, $\RRsn$ et $\RRsp$.
\end{tcblisting}


Nous avons utilisé les suffixes \verb+n+ pour \verb+Negatif+, \verb+p+ pour \verb+Positif+, et \verb+s+ pour \verb+star+, soit "étoile" en anglais. Il y a aussi les suffixes composites \verb+sn+ et \verb+sp+.

\medskip

Notez qu'il est interdit d'utiliser \verb+$\CCn$+ pour $\setspecial{\CC}{n}$ car l'ensemble $\CC$ ne possède pas de structure ordonnée standard. Jetez un oeil à la section suivante pour apprendre à taper $\setspecial{\CC}{n}$ si vous en avez besoin. L'interdiction est ici purement sémantique !

\medskip

\begin{remark}
	La table \ref{table:suffixes-sets} \vpageref{table:suffixes-sets} montre les associations autorisées entre ensembles classiques et suffixes.
\end{remark}

% == Table of suffixes - START == %

\newcommand\xx{\phantom{$\times$}}
\begin{table}[h]
    \caption{Suffixes}
    \begin{center}
        \begin{tabular}{c|c|c|c|c|c}
              & \verb+n+ & \verb+p+ & \verb+s+ & \verb+sn+ & \verb+sp+ \\
            \hline \verb+N+ & \xx & \xx & $\times$ & \xx & \xx \\
            \hline \verb+P+ & \xx & \xx & \xx & \xx & \xx \\
            \hline \verb+Z+ & $\times$ & $\times$ & $\times$ & $\times$ & $\times$ \\
            \hline \verb+D+ & $\times$ & $\times$ & $\times$ & $\times$ & $\times$ \\
            \hline \verb+Q+ & $\times$ & $\times$ & $\times$ & $\times$ & $\times$ \\
            \hline \verb+R+ & $\times$ & $\times$ & $\times$ & $\times$ & $\times$ \\
            \hline \verb+C+ & \xx & \xx & $\times$ & \xx & \xx \\
            \hline \verb+H+ & \xx & \xx & $\times$ & \xx & \xx \\
            \hline \verb+O+ & \xx & \xx & $\times$ & \xx & \xx \\
        \end{tabular}
    \end{center}
    \label{table:suffixes-sets}
\end{table}

% == Table of suffixes - END == %



\subsection{Des suffixes à la carte}

\paragraph{Exemple d'utilisation}

\begin{tcblisting}{}
Il est tout de même possible d'écrire $\setspecial{\CC}{n}$ ou $\setspecial{\HH}{sp}$.
Il y a aussi $\setspecial*{\setproba{P}}{n}$ avec une autre mise en forme.
\end{tcblisting}


\paragraph{Fiches techniques}

\IDmacro*{setspecial}{2}

\IDmacro*{setspecial*}{2}

\IDarg{1} l'ensemble à "suffixer".

\IDarg{2} l'un des suffixes \verb+n+, \verb+p+, \verb+s+, \verb+sn+ ou \verb+sp+.




% \section{Sets}

\subsection{Intervalles}

\subsubsection{Intervalles réels - Notation française (?\,)}

\paragraph{Exemple d'utilisation 1}

Dans cet exemple, la syntaxe fait référence à \textbf{O}-pened et \textbf{C}-losed pour "ouvert" et "fermé" en anglais.
Nous verrons que \textbf{CC} et \textbf{OO} sont contractés en \textbf{C} et \textbf{O}.

\begin{tcblisting}{}
Dans $I = ]a ; b] = \intervalOC{a}{b}$, vous constatez que la macro utilisée résout
un problème d'espacement vis à vis du signe $=$ .
\end{tcblisting}


\paragraph{Exemple d'utilisation 2}

Les crochets s'étendent verticalement automatiquement. Pour empêcher cela, il suffit d'utiliser la version étoilée de la macro.
Dans ce cas, les crochets restent tout de même un peu plus grands que des crochets utilisés directement. Voici un exemple.

\begin{tcblisting}{}
$\displaystyle \intervalC{ \frac{1}{2} }{ 1^{2^{3}} }
             = [ \frac{1}{2} ; 1^{2^{3}} ]
             = \intervalC*{ \frac{1}{2} }{ 1^{2^{3}} }$
\end{tcblisting}


\paragraph{Fiches techniques}

Pour toutes les macros ci-dessous, la version non étoilée produit des délimiteurs qui s'étirent si besoin verticalement, tandis que la version étoilée ne le fait pas.


\bigskip


% Docs for french real intervals - START

\IDmacro*{intervalCO}{2}

\IDmacro*{intervalCO*}{2}

\IDarg{1} borne inférieure $a$ de l'intervalle $\intervalCO{a}{b}$.

\IDarg{2} borne supérieure $b$ de l'intervalle $\intervalCO{a}{b}$.


\bigskip


\IDmacro*{intervalC}{2}

\IDmacro*{intervalC*}{2}

\IDarg{1} borne inférieure $a$ de l'intervalle $\intervalC{a}{b}$.

\IDarg{2} borne supérieure $b$ de l'intervalle $\intervalC{a}{b}$.


\bigskip


\IDmacro*{intervalO}{2}

\IDmacro*{intervalO*}{2}

\IDarg{1} borne inférieure $a$ de l'intervalle $\intervalO{a}{b}$.

\IDarg{2} borne supérieure $b$ de l'intervalle $\intervalO{a}{b}$.


\bigskip


\IDmacro*{intervalOC}{2}

\IDmacro*{intervalOC*}{2}

\IDarg{1} borne inférieure $a$ de l'intervalle $\intervalOC{a}{b}$.

\IDarg{2} borne supérieure $b$ de l'intervalle $\intervalOC{a}{b}$.

% Docs for french real intervals - END



\subsubsection{Intervalles réels - Notation américaine}

\paragraph{Exemple d'utilisation}

Dans cet exemple, la syntaxe fait référence à \textbf{P}-arenthèse.

\begin{tcblisting}{}
Aux États Unis, un intervalle semi-fermé s'écrit $\intervalPC{a}{b} = (a ; b]$ et
un intervalle ouvert se tape $\intervalP{a}{b} = (a ; b)$.
\end{tcblisting}


\paragraph{Fiches techniques}

Pour toutes les macros ci-dessous, la version non étoilée produit des délimiteurs qui s'étirent si besoin verticalement, tandis que la version étoilée ne le fait pas.


\bigskip


% Docs for american real intervals - START

\IDmacro*{intervalCP}{2}

\IDmacro*{intervalCP*}{2}

\IDarg{1} borne inférieure $a$ de l'intervalle $\intervalCP{a}{b}$.

\IDarg{2} borne supérieure $b$ de l'intervalle $\intervalCP{a}{b}$.


\bigskip


\IDmacro*{intervalP}{2}

\IDmacro*{intervalP*}{2}

\IDarg{1} borne inférieure $a$ de l'intervalle $\intervalP{a}{b}$.

\IDarg{2} borne supérieure $b$ de l'intervalle $\intervalP{a}{b}$.


\bigskip


\IDmacro*{intervalPC}{2}

\IDmacro*{intervalPC*}{2}

\IDarg{1} borne inférieure $a$ de l'intervalle $\intervalPC{a}{b}$.

\IDarg{2} borne supérieure $b$ de l'intervalle $\intervalPC{a}{b}$.

% Docs for american real intervals - END



\subsubsection{Intervalles discrets d'entiers}

\paragraph{Exemple d'utilisation}

Dans l'exemple, la syntaxe fait référence à $\ZZ$ l'ensemble des entiers relatifs.

\begin{tcblisting}{}
Par définition, $\ZintervalC{-1}{4} = \{ -1 ; 0 ; 1 ; 2 ; 3 ; 4 \}$. Donc nous avons
$\ZintervalC{-1}{4} = \ZintervalO{-2}{5}$.
\end{tcblisting}


\paragraph{Fiches techniques}

Pour toutes les macros ci-dessous, la version non étoilée produit des délimiteurs qui s'étirent si besoin verticalement, tandis que la version étoilée ne le fait pas.


\bigskip


% Docs for discrete intervals - START

\IDmacro*{ZintervalCO}{2}

\IDmacro*{ZintervalCO*}{2}

\IDarg{1} borne inférieure $a$ de l'intervalle $\ZintervalCO{a}{b}$.

\IDarg{2} borne supérieure $b$ de l'intervalle $\ZintervalCO{a}{b}$.


\bigskip


\IDmacro*{ZintervalC}{2}

\IDmacro*{ZintervalC*}{2}

\IDarg{1} borne inférieure $a$ de l'intervalle $\ZintervalC{a}{b}$.

\IDarg{2} borne supérieure $b$ de l'intervalle $\ZintervalC{a}{b}$.


\bigskip


\IDmacro*{ZintervalO}{2}

\IDmacro*{ZintervalO*}{2}

\IDarg{1} borne inférieure $a$ de l'intervalle $\ZintervalO{a}{b}$.

\IDarg{2} borne supérieure $b$ de l'intervalle $\ZintervalO{a}{b}$.


\bigskip


\IDmacro*{ZintervalOC}{2}

\IDmacro*{ZintervalOC*}{2}

\IDarg{1} borne inférieure $a$ de l'intervalle $\ZintervalOC{a}{b}$.

\IDarg{2} borne supérieure $b$ de l'intervalle $\ZintervalOC{a}{b}$.

% Docs for discrete intervals - END


\subsubsection{Intervalles discrets d'entiers à la sauce \og informatique \fg}

\paragraph{Exemple d'utilisation}

Dans l'exemple, la syntaxe fait référence à \og Computer Science \fg{} soit \og Informatique Théorique \fg{}.

\begin{tcblisting}{}
Par définition, $\CSinterval{4}{7} = \{ 4 ; 5 ; 6 ; 7 \}$.
\end{tcblisting}


\paragraph{Fiches techniques}

\IDmacro*{CSinterval}{2}

\IDarg{1} borne inférieure $a$ de l'intervalle $\CSinterval{a}{b}$.

\IDarg{2} borne supérieure $b$ de l'intervalle $\CSinterval{a}{b}$.




$\smallsetminus$
\section{Géométrie}

\subsection{Points et lignes}

\subsubsection{Points}

\paragraph{Exemple d'utilisation 1}

\begin{tcblisting}{}
$\pt{I}$ indique un point nommé "I".
\end{tcblisting}


\paragraph{Exemple d'utilisation 2}

\begin{tcblisting}{}
Une liste de points : $\pt*{I}{1}$, $\pt*{I}{2}$ \dots
\end{tcblisting}


\paragraph{Exemple d'utilisation 3}

\begin{tcblisting}{}
Une droite $(\pts{AB})$ au lieu de $(\pt{A}\pt{B})$.
\end{tcblisting}


\paragraph{Fiches techniques}

\IDmacro*{pt}{1}

\IDarg{} un texte donnant le nom d'un point.


\bigskip


\IDmacro*{pt*}{2}

\IDarg{1} un texte indiquant $\pt{UP}$ dans le nom $\pt*{UP}{down}$ d'un point.

\IDarg{2} un texte indiquant $down$ dans le nom $\pt*{UP}{down}$ d'un point.


\bigskip


\IDmacro*{pts}{1}

\IDarg{} un texte indiquant des noms de points non indicés.



\subsubsection{Droites parallèles ou non}

Les opérateurs \verb+\parallel+ et \verb+\notparallel+ utilisent des obliques au lieu de barres verticales comme le montre l'exemple qui suit où \verb+\nparallel+ est fourni par le package \verb+amssymb+, et \verb+\stdparallel+ est un alias de la version standard de \verb+\parallel+ proposée par \LaTeX{}.

\begin{tcblisting}{}
$(\pts{AB}) \parallel (\pts{CD})$ et $(\pts{EF}) \notparallel (\pts{GH})$ au lieu de
$(\pts{AB}) \stdparallel (\pts{CD})$ et de $(\pts{EF}) \nparallel (\pts{GH})$.
\end{tcblisting}




% \section{Géométrie}

\subsection{Vecteurs}

		\subsubsection{Les écrire}

\paragraph{Exemple d'utilisation 1}

\begin{tcblisting}{}
Voici un vecteur $\vect{ABCDEF}$ avec beaucoup de lettres et vous pouvez écrire
$\vect*{e}{rot}$ au lieu de $\vect{e_{rot}}$.
\end{tcblisting}


\paragraph{Exemple d'utilisation 2}

\begin{tcblisting}{}
Vous pouvez écrire $\vect{i}$ and $\vect*{j}{2}$ sans point.
\end{tcblisting}



\paragraph{Fiches techniques}

\IDmacro*{vect}{1}

\IDarg{} un texte donnant le nom d'un vecteur.


\bigskip


\IDmacro*{vect*}{2}

\IDarg{1} un texte indiquant $up$ dans le nom $\vect*{up}{down}$ d'un vecteur.

\IDarg{2} un texte indiquant $down$ dans le nom $\vect*{up}{down}$ d'un vecteur.




% \section{Géométrie}

%\subsection{Vecteurs}

		\subsubsection{Norme}

\paragraph{Exemple d'utilisation}

\begin{tcblisting}{}
Nous pouvons écrire $\norm{\vect{i}}$, $\displaystyle \norm{\frac{2}{7} \vect*{e}{k}}$,
ou $\displaystyle \norm*{\frac{2}{7} \vect*{e}{k}}$ avec de petites barres verticales.
\end{tcblisting}


\begin{remark}
	Le code \LaTeX{} vient directement de ce message : \url{https://tex.stackexchange.com/a/43009/6880}.
\end{remark}


\paragraph{Fiches techniques}

\IDmacro*{norm}{1}

\IDmacro*{norm*}{1}

\IDarg{} le vecteur sur lequel appliquer la norme.




% \section{Géométrie}

%\subsection{Vecteurs}

		\subsubsection{Produit scalaire -- Écriture minimaliste}

\paragraph{Exemple d'utilisation - Version longue}

\begin{tcblisting}{}
En mathématique, il est usage d'écrire un produit scalaire avec un point via
$\dotprod{\dfrac{1}{2} \vect{i}}{\vect{j}}$ .
\end{tcblisting}


\paragraph{Exemple d'utilisation - Version courte mais restrictive}

Dans l'exemple suivant, le préfixe \verb+v+ est pour \textbf{v}-ector.

\begin{tcblisting}{}
On peut aussi parfois juste taper $\vdotprod{i}{j}$ .
\end{tcblisting}


\paragraph{Fiches techniques}

\IDmacro*{dotprod}{2}

\IDarg{1} le premier vecteur qu'il faut taper via la macro \verb+\vect+.

\IDarg{2} le second vecteur qu'il faut taper via la macro \verb+\vect+.


\bigskip


\IDmacro*{vdotprod}{2} où \quad \verb+v = v-ector+

\IDarg{1} le nom du premier vecteur sans utiliser la macro \verb+\vect+.

\IDarg{2} le nom du second vecteur sans utiliser la macro \verb+\vect+.



		\subsubsection{Produit scalaire -- Écriture \og physicienne \fg}

Dans l'exemple suivant, le préfixe \verb+a+ est pour \textbf{a}-ngle, et  \verb+v+ pour \textbf{v}-ector.

\begin{tcblisting}{}
Les physiciens pourront utiliser
$\displaystyle \adotprod{\frac{1}{2} \vect{i}}{\vect{j}}$ ,
$\displaystyle \adotprod*{\frac{1}{2} \vect{i}}{\vect{j}}$ ,
$\displaystyle \vadotprod{i}{j}$
ou
$\displaystyle \vadotprod*{i}{j}$ .
\end{tcblisting}


\paragraph{Fiches techniques}

\IDmacro*{adotprod}{2} où \quad \verb&a = a-ngle&

\IDmacro*{adotprod*}{2}

\IDarg{1} le premier vecteur qu'il faut taper via la macro \verb+\vect+.

\IDarg{2} le second vecteur qu'il faut taper via la macro \verb+\vect+.


\bigskip


\IDmacro*{vadotprod}{2} où \quad \verb&a = a-ngle& et \verb+v = v-ector+

\IDarg{1} le nom du premier vecteur sans utiliser la macro \verb+\vect+.

\IDarg{2} le nom du second vecteur sans utiliser la macro \verb+\vect+.





		\subsubsection{Produit vectoriel}

\paragraph{Exemple d'utilisation - Version longue}

\begin{tcblisting}{}
Un produit vectoriel peut s'écrire via $\crossprod{\dfrac{1}{2} \vect{i}}{\vect{j}}$ .
\end{tcblisting}


\paragraph{Exemple d'utilisation - Version courte mais restrictive}

\begin{tcblisting}{}
Dans certain cas, un produit vectoriel s'écrit vite via $\vcrossprod{i}{j}$ .
\end{tcblisting}


\paragraph{Fiche technique}

\IDmacro*{crossprod}{2}

\IDarg{1} le premier vecteur qu'il faut taper via la macro \verb+\vect+.

\IDarg{2} le second vecteur qu'il faut taper via la macro \verb+\vect+.


\bigskip


\IDmacro*{vcrossprod}{2} où \quad \verb+v = v-ector+

\IDarg{1} le nom du premier vecteur sans utiliser la macro \verb+\vect+.

\IDarg{2} le nom du second vecteur sans utiliser la macro \verb+\vect+.




%\section{Géométrie}

\subsection{Coordonnées}

\paragraph{Exemple d'utilisation 1}

\begin{tcblisting}{}
On peut choisir d'écrire
$\displaystyle \pt{I} \coord{\frac{1}{3} | -4 | 0}$
ou bien
$\displaystyle \pt{I} \coord*{\frac{1}{3} | -4 | 0}$ .
\end{tcblisting}


\paragraph{Exemple d'utilisation 2}

Dans l'exemple suivant, le préfixe \verb+v+ est pour \textbf{v}-ertical.

\begin{tcblisting}{}
Pour les vecteurs, on peut préférer $\vect{i} \vcoord{3 | -4 | 0}$, voire
$\vect{i} \vcoord*{3 | -4 | 0}$ , à la place de $\vect{i} \coord{3 | -4 | 0}$
afin de bien différencier les coordonnées de points de celles de vecteurs.
\end{tcblisting}


\paragraph{Fiches techniques}

\IDmacro*{coord}{1}

\IDmacro*{coord*}{1}

\IDarg{} l'argument est une suite de "morceaux" séparés par des barres \verb+|+ , chaque morceau étant une coordonnée. Il peut n'y avoir qu'un seul morceau.



\bigskip

\IDmacro*{vcoord}{1} où \quad \verb&v = v-ertical&

\IDmacro*{vcoord*}{1} pour des crochets à la place de parenthèses

\IDarg{} l'argument est une suite de "morceaux" séparés par des barres \verb+|+ , chaque morceau étant une coordonnée. Il peut n'y avoir qu'un seul morceau.




% \section{Géométrie}

\subsection{Nommer un repère}

\paragraph{Exemple d'utilisation 1 -- La méthode basique}

Commençons par la manière la plus basique d'écrire un repère \textit{(nous verrons d'autres méthodes qui peuvent être plus efficaces)}.

\begin{tcblisting}{}
Dans le plan, trois points $\pt{O}$, $\pt{I}$ and $\pt{J}$ non alignés définissent
un repère cartésien $\axes{\pt{O} | \pt{I} | \pt{J}}$.
\end{tcblisting}


\paragraph{Exemple d'utilisation 2 -- La méthode basique en version étoilée}

Dans l'exemple ci-dessous, on voit que la version étoilée produit des petites parenthèses.
\begin{tcblisting}{}
$\displaystyle \axes{\pt{O} | \frac{7}{3} \vect{i} | \vect{j}}$
ou
$\displaystyle \axes*{\pt{O} | \frac{7}{3} \vect{i} | \vect{j}}$
\end{tcblisting}


\paragraph{Exemple d'utilisation 3 -- La méthode basique en dimension quelconque}

Il faut au minimum deux "morceaux" séparés par des barres \verb+|+, cas de la dimension $1$, mais il n'y a pas de maximum, cas d'une dimension quelconque $n > 0$.

\begin{tcblisting}{}
$\axes{\pt{O} | \vect*{i}{1} | \vect*{i}{2} | \vect*{i}{3} | \dots |
 \vect*{i}{9} | \vect*{i}{10} | \vect*{i}{11} | \vect*{i}{12}}$
\end{tcblisting}


\paragraph{Exemple d'utilisation 4 -- Repère affine}

Dans l'exemple suivant, le préfixe \verb+p+ est pour \textbf{p}-oint.

\begin{tcblisting}{}
$\paxes{O | I | J | K}$ évite de taper $\axes{\pt{O} | \pt{I} | \pt{J} | \pt{K}}$.
\end{tcblisting}


\paragraph{Exemple d'utilisation 5 -- Repère vectoriel (méthode 1)}

Dans l'exemple suivant, le préfixe \verb+v+ est pour \textbf{v}-ecteur.

\begin{tcblisting}{}
$\vaxes{\pt{O} | i | j}$ est un raccourci de $\axes{\pt{O} | \vect{i} | \vect{j}}$.
\end{tcblisting}


\paragraph{Exemple d'utilisation 6 -- Repère vectoriel (méthode 2)}

Dans l'exemple suivant, le préfixe \verb+pv+ permet de combiner à la fois les fonctionnalités proposés par les préfixes \verb+p+ et \verb+v+.

\begin{tcblisting}{}
$\pvaxes{O | i | j}$ donne rapidement $\axes{\pt{O} | \vect{i} | \vect{j}}$.
\end{tcblisting}



\paragraph{Fiches techniques}

\IDmacro*{axes}{1}

\IDmacro*{axes*}{1}

\IDarg{} l'argument est une suite de "morceaux" séparés par des barres \verb+|+.

\begin{itemize}[topsep=0pt]
	\item Le premier morceau est l'origine du repère.

	\item Les morceaux suivants sont des points ou des vecteurs qui "définissent" chaque axe.
\end{itemize}


\bigskip

\IDmacro*{paxes}{1} où \quad \verb+p = p-oint+

\IDarg{} l'argument est une suite de "morceaux" séparés par des barres \verb+|+.

\begin{itemize}[topsep=0pt]
	\item Le premier morceau est le nom de l'origine du repère sur laquelle la macro-commande \verb+\pt+ sera automatiquement appliquée.

	\item Viennent ensuite les noms des points "définissant" chaque axe. Pour chacun de ces points la macro-commande \verb+\pt+ sera automatiquement appliquée.
\end{itemize}


\bigskip

\IDmacro*{vaxes}{1} où \quad \verb+v = v-ector+

\IDarg{} l'argument est une suite de "morceaux" séparés par des barres \verb+|+.

\begin{itemize}[topsep=0pt]
	\item Le premier morceau est l'origine du repère.

	\item Viennent ensuite les noms des vecteurs "définissant" chaque axe. Pour chacun de ces vecteurs la macro-commande \verb+\vect+ sera automatiquement appliquée.
\end{itemize}


\bigskip

\IDmacro*{pvaxes}{3} où \quad \verb&pv = p + v&

\IDarg{} l'argument est une suite de "morceaux" séparés par des barres \verb+|+.

\begin{itemize}[topsep=0pt]
	\item Le premier morceau est le nom de l'origine du repère sur laquelle la macro-commande \verb+\pt+ sera automatiquement appliquée.

	\item Viennent ensuite les noms des vecteurs "définissant" chaque axe. Pour chacun de ces vecteurs la macro-commande \verb+\vect+ sera automatiquement appliquée.
\end{itemize}




% \section{Géométrie}

\subsection{Arcs circulaires}

\paragraph{Exemple d'utilisation 1}

\begin{tcblisting}{}
Voici un arc $\arc{ABCDEF}$ utilisant beaucoup de lettres, et vous pouvez écrire
$\arc*{A}{rot}$ à la place de $\arc{A_{rot}}$.
\end{tcblisting}


\paragraph{Exemple d'utilisation 2}

\begin{tcblisting}{}
$\arc{i}$ et $\arc*{j}{2}$ n'affiche pas de point sous l'arc.
\end{tcblisting}



\paragraph{Fiches techniques}

\IDmacro*{arc}{1}

\IDarg{} un texte donnant le nom d'un arc circulaire.


\bigskip


\IDmacro*{arc*}{2}

\IDarg{1} un texte indiquant $up$ dans le nom $\arc*{up}{down}$ d'un arc circulaire.

\IDarg{2} un texte indiquant $down$ dans le nom $\arc*{up}{down}$ d'un arc circulaire.




% \section{Géométrie}

\subsection{Angles}

		\subsubsection{Angles géométriques intérieurs}

\paragraph{Exemple d'utilisation 1}

\begin{tcblisting}{}
Voici un angle géométrique intérieur $\anglein{ABCDEF}$ avec un long nom, et vous
pouvez écrire $\anglein*{A}{rot}$ au lieu de $\anglein{A_{rot}}$.
\end{tcblisting}


\paragraph{Exemple d'utilisation 2}

\begin{tcblisting}{}
Vous pouvez aussi écrire $\anglein{i}$ and $\anglein*{j}{2}$ sans point.
\end{tcblisting}



\paragraph{Fiches techniques}

\IDmacro*{anglein}{1}

\IDarg{} un texte donnant le nom d'un angle intérieur.


\bigskip


\IDmacro*{anglein*}{2}

\IDarg{1} un texte indiquant $up$ dans le nom $\anglein*{up}{down}$ d'un angle intérieur.

\IDarg{2} un texte indiquant $down$ dans le nom $\anglein*{up}{down}$ d'un angle intérieur.




% \section{Géométrie}

%\subsection{Angles}

		\subsubsection{Angles orientés de vecteurs}

\paragraph{Sans chapeau - Version longue}

\begin{tcblisting}{}
En mathématique, il est d'usage de noter les angles orientés via
$\displaystyle \angleorient{\frac{1}{2} \vect{i}}{\vect{j}}$
ou
$\displaystyle \angleorient*{\frac{1}{2} \vect{i}}{\vect{j}}$ .
\end{tcblisting}



\paragraph{Sans chapeau - Version courte mais restrictive}

Dans l'exemple suivant, le préfixe \verb+v+ est pour \textbf{v}-ector.
\begin{tcblisting}{}
Avec des noms de vecteurs utilisant juste des lettres, on peut juste taper
$\displaystyle \vangleorient{i}{j}$
ou
$\displaystyle \vangleorient*{i}{j}$ (la seconde écriture n'apporte rien de nouveau).
\end{tcblisting}



\paragraph{Avec un chapeau}

Dans l'exemple suivant, le préfixe \verb+h+ est pour \textbf{h}-at, et \verb+v+ pour \textbf{v}-ector.

\begin{tcblisting}{}
Si vous préférez les angles orientés avec un chapeau, tapez les alors via
$\displaystyle \hangleorient{\frac{1}{2} \vect{i}}{\vect{j}}$ ,
$\displaystyle \hangleorient*{\frac{1}{2} \vect{i}}{\vect{j}}$ ,
$\displaystyle \hvangleorient{i}{j}$
ou
$\displaystyle \hvangleorient*{i}{j}$ (la dernière écriture n'apportant rien de neuf).
\end{tcblisting}



\paragraph{Fiche technique}

\IDmacro*{angleorient}{2}

\IDmacro*{hangleorient}{2}  où \quad \verb+h = h-at+

\IDarg{1} le premier vecteur qu'il faut taper via la macro \verb+\vect+.

\IDarg{2} le second vecteur qu'il faut taper via la macro \verb+\vect+.


\bigskip


\IDmacro*{vangleorient}{2} où \quad \verb+v = v-ector+

\IDmacro*{hvangleorient}{2} où \quad \verb+h = h-at+ et \verb+v = v-ector+

\IDarg{1} le nom du premier vecteur sans utiliser la macro \verb+\vect+.

\IDarg{2} le nom du second vecteur sans utiliser la macro \verb+\vect+.




\section{Analyse}

\subsection{Constantes}

\subsubsection{Constantes classiques}

\paragraph{La liste complète}

% List of classical constants - START

\foreach \k in {ggamma, ppi, ttau, ee, ii, jj, kk}{\IDconstant{\k}}

\begin{tcblisting}{}
Voici la liste des constantes classiques où $\ttau = 2 \ppi$ est la benjamine :
$\ggamma$, $\ppi$, $\ttau$, $\ee$, $\ii$, $\jj$ and $\kk$.
\end{tcblisting}

% List of classical constants - END


\begin{remark}
	Faites attention car \verb+{\Large $\ppi \neq \pi$}+ produit {\Large $\ppi \neq \pi$}. Comme vous le constatez, les symboles ne sont pas identiques. Ceci est vraie pour toutes les constantes grecques.
\end{remark}



\subsubsection{Constantes latines personnelles}

\paragraph{Exemple d'utilisation}

\begin{tcblisting}{}
Il est aisé d'écrire $\ct{a} x^2 + \ct{b} x + \ct{c}$ au lieu de $a x^2 + b x + c$
afin de souligner que $\ct{a}$, $\ct{b}$ et $\ct{c}$ sont des constantes.
\end{tcblisting}


\paragraph{Fiche technique}

\IDmacro*{ct}{1}

\IDarg{} un texte utilisant l'alphabet latin.




% \section{Analysis}

\subsection{La fonction valeur absolue}

\paragraph{Un exemple d'utilisation}

\begin{tcblisting}{}
Il est facile d'écrire $\abs{2}$ ou $\displaystyle \abs{\frac{3}{5}}$ voire aussi
$\displaystyle \abs*{\frac{3}{5}}$ si vous préférez des petits traits verticaux.
\end{tcblisting}


\begin{remark}
	Le code \LaTeX{} vient directement de ce poste : \url{https://tex.stackexchange.com/a/43009/6880}.
\end{remark}


\paragraph{Fiches techniques}

\IDmacro*{abs}{1}

\IDmacro*{abs*}{1}

\IDarg{} l'expression à laquelle on applique la fonction valeur absolue.




% \section{Analysis}

\subsection{Fonctions nommées spéciales}

\subsubsection{Un exemple d'utilisation}

\begin{tcblisting}{}
Quelques fonctions nommées supplémentaires (voir la liste ecomplète ci-dessous) :
$\ch x \neq ch x$ , $\ppcm(x;y)$ , $\lg x =\logb{2} x$ and $\expb{6} y$.
\end{tcblisting}


\subsubsection{Fonctions nommées sans paramètre}

Toutes les macros suivantes n'ont aucun paramètre.

\medskip

% List of functions without parameter - START

\foreach \k in {ch, sh, th, ach, ash, ath, arccosh, arcsinh, arctanh, acos, asin, atan, pgcd, ppcm}{\IDconstant{\k}}

\begin{tabular*}{\textwidth}{@{\extracolsep{\fill}}*{4}{l}}
    \verb+\ch+ & \verb+\sh+ & \verb+\th+\\
    \verb+\ach+ & \verb+\ash+ & \verb+\ath+\\
    \verb+\arccosh+ & \verb+\arcsinh+ & \verb+\arctanh+\\
    \verb+\acos+ & \verb+\asin+ & \verb+\atan+\\
    \verb+\pgcd+ & \verb+\ppcm+ & \\
\end{tabular*}

% List of functions without parameter - END



\subsubsection{Fonctions nommées avec un paramètre}

\paragraph{La liste complète}

Toutes les macros suivantes ont un seul paramètre.

\medskip

% List of functions with parameters - START

\begin{tabular*}{\textwidth}{@{\extracolsep{\fill}}*{4}{l}}
    \verb+\expb+ \, (1 paramètre) & \verb+\logb+ \, (1 paramètre) &  & \\
\end{tabular*}

% List of functions with parameters - END



\paragraph{Fiches techniques}

\IDmacro*{expb}{1}

\IDmacro*{logb}{1}

\IDarg{} la base de l'exponentielle ou du logarithme.




% \section{Analysis}

\subsection{Des notations complémentaires pour des suites spéciales}

\paragraph{Exemple d'utilisation}

\begin{tcblisting}{}
Parfois nous avons besoin d'écrire $\seqplus{F}{1}{2}$ ou $\hypergeo{F}{1}{2}$ et
le fou (?\,) aime vraiment $\suprageo{F}{1}{2}{3}{4}$.
\end{tcblisting}


\paragraph{Fiches techniques}

\IDmacro*{seqplus}{2}

\IDarg{1} l'exposant à droite.

\IDarg{2} l'indice à droite.


\bigskip


\IDmacro*{hypergeo}{2}

\IDarg{1} l'indice à gauche.

\IDarg{2} l'indice à droite.


\bigskip


\IDmacro*{suprageo}{4}

\IDarg{1} l'indice à gauche.

\IDarg{2} l'indice à droite.

\IDarg{3} l'exposant à droite.

\IDarg{4} l'exposant à gauche.




% \section{Analysis}

\subsection{Calcul différentiel}

\subsubsection{\texorpdfstring{Les opérateurs $\pp{}$ et $\dd{}$}%
                               {Les opérateurs "d rond" et "d droit"}}

\paragraph{Exemple d'utilisation}

\begin{tcblisting}{}
Vous pouvez écrire $\dd{f}$ et $\pp{t}$ et aussi $ \dd[5]{x}$ ou $\pp[n]{x}$.
\end{tcblisting}


\paragraph{Fiches techniques}

\IDmacro{dd}{1}{1}

\IDmacro{pp}{1}{1}

\IDoption{} utilisée, cette option sera mise en exposant du symbole $\pp{}$ ou $\dd{}$.

\IDarg{} la variable de différentiation à droite du symbole $\pp{}$ ou $\dd{}$.



\subsubsection{Dérivation totale}

\paragraph{Exemple d'utilisation 1}

\begin{tcblisting}{}
$\displaystyle \derpow*{f} (a) = \derpow{f} (a)
                               = \derfrac{f}{x} (a)
                               = \dersub{f}{x} (a)$
\end{tcblisting}


\paragraph{Exemple d'utilisation 2}

\begin{tcblisting}{}
$\displaystyle \derpow*[3]{f}(a) = \derpow[3]{f} (a)
                                 = \derfrac[3]{f}{x} (a)
                                 = \dersub[3]{f}{x} (a)$
et $\displaystyle \derpow*[10]{\cos} a = \derfrac[10]{\cos}{x} (a)$ avec beaucoup trop
de primes à gauche (mais le compte y est).
\end{tcblisting}


\paragraph{Exemple d'utilisation 3}

\begin{tcblisting}{}
Si $\displaystyle f(x) = \frac{1}{x^2+3}$ alors nous avons :
   $\displaystyle \derpow[3]{f} (a)
                = \derfrac*[3]{\left( \frac{1}{x^2+3} \right)}{x} (a)$.
\end{tcblisting}


\paragraph{Fiches techniques}

\IDmacro{derpow}{1}{1}

\IDmacro{derpow*}{1}{1}

\IDoption{} utilisée, cette option sera l'exposant de dérivation mis entre des parenthèses pour la version non étoilée, et le nombre de primes pour la version étoilée.

\IDarg{} la fonction à différencier.


\bigskip


\IDmacro{derfrac}{1}{2}

\IDmacro{derfrac*}{1}{2}

\IDmacro{dersub}{1}{2}

\IDoption{} utilisée, cette option sera l'exposant de dérivation.

\IDarg{1} la fonction à dériver.

\IDarg{2} la variable.



\subsubsection{Dérivation partielle}

\paragraph{Exemple d'utilisation 1}

\begin{tcblisting}{}
$\displaystyle \partialfrac{f}{x} (a;b)
             = \partialsub{f}{x} (a;b)
             = \partialprime{f}{x} (a;b)$
\end{tcblisting}


\paragraph{Exemple d'utilisation 2}

\begin{tcblisting}{}
$\displaystyle \partialfrac[3]{G}{f^2 | v} (a;b)
             = \partialfrac{G}{f^2 | v} (a;b)
             = \partialsub{G}{f^2 | v} (a;b)
             = \partialprime{G}{f^2 | v} (a;b)$
\end{tcblisting}


\paragraph{Exemple d'utilisation 3}

\begin{tcblisting}{}
Si $\displaystyle f(x;y) = \frac{cos(x y)}{x^2+y^2}$ alors nous avons
   $\displaystyle \partialfrac[2]{f}{x | y}
                = \partialfrac*[2]{\left( \frac{cos(x y)}{x^2 + y^2} \right)}{x | y}$.
\end{tcblisting}


\paragraph{Fiches techniques}

\IDmacro{partialfrac}{1}{2}

\IDmacro{partialfrac*}{1}{2}

\IDoption{} utilisée, cette option sera l'exposant total de dérivation mis en exposant de $\pp$.

\IDarg{1} la fonction à dériver partiellement.

\IDarg{2} les variables utilisées pour la dérivation partielle en utilisant la syntaxe suivante : par exemple, \verb+x | y^3 | ...+ indique de dériver suivant $x$ une fois, puis suivant $y$ trois fois... etc.


\bigskip


\IDmacro*{partialsub}{2}

\IDmacro*{partialprime}{2}

\IDarg{1} la fonction à dériver partiellement.

\IDarg{2} les variables utilisées pour la dérivation partielle en utilisant la syntaxe suivante : par exemple, \verb+x | y^3 | ...+ indique de dériver suivant $x$ une fois, puis suivant $y$ trois fois... etc.




% \section{Analysis}

\subsection{Calcul intégral}

\subsubsection{L'opérateur crochet -- 1\textsuperscript{ère} version}

\paragraph{Exemple d'utilisation 1}

\begin{tcblisting}{}
Par définition, $\displaystyle \int_{a}^{b} f(x) \dd{x} = \hook{F(x)}{a}{b}$ où
$\hook{F(x)}{a}{b} = F(b) - F(a)$.
\end{tcblisting}


\paragraph{Exemple d'utilisation 2}

Par défaut, les crochets s'étirent verticalement si besoin, mais si cela vous dérange, vous pouvez faire appel à la version étoilée de la macro comme dans l'exemple suivant

\begin{tcblisting}{}
$\displaystyle \hook{\frac{x - 1}{5 + x^2}}{a}{b}
             = \hook*{\frac{x - 1}{5 + x^2}}{a}{b}$.
\end{tcblisting}


\paragraph{Fiches techniques}

\IDmacro*{hook}{3}

\IDmacro*{hook*}{3}

\IDarg{1} le contenu entre les crochets.

\IDarg{2} la borne inférieure affichée en indice.

\IDarg{3} la borne supérieure affichée en exposant.



\subsubsection{L'opérateur crochet -- 2\textsuperscript{nde} version}

\paragraph{Exemple d'utilisation 1}

\begin{tcblisting}{}
Vous pouvez utiliser $\vhook{F(x)}{a}{b}$ au lieu de $\hook{F(x)}{a}{b}$.
\end{tcblisting}


\paragraph{Exemple d'utilisation 2}

Tout comme avec la première version de l'opérateur crochet, vous pouvez utiliser une version étoilée pour empêcher l'étirement verticalement du trait vertical. Voici un exemple.

\begin{tcblisting}{}
$\displaystyle \vhook{\frac{x - 1}{5 + x^2}}{a}{b}
             = \vhook*{\frac{x - 1}{5 + x^2}}{a}{b}$.
\end{tcblisting}


\paragraph{Fiches techniques}

\IDmacro*{vhook}{3}

\IDmacro*{vhook*}{3}

\IDarg{1} le contenu avant le trait vertical.

\IDarg{2} la borne inférieure affichée en indice.

\IDarg{3} la borne supérieure affichée en exposant.



\subsubsection{Intégrales multiples}

Le package réduit les espacements entres des symboles $\int$ successifs. Voici un exemple.

\begin{tcblisting}{}
$\displaystyle
  \int \int \int F(x;y;z) \dd{x} \dd{y} \dd{z}
= \int_{a}^{b} \int_{c}^{d} \int_{e}^{f} F(x;y;z) \dd{x} \dd{y} \dd{z}$
\end{tcblisting}


\begin{remark}
	Par défaut, \LaTeX{} affiche
	\makeatletter
    	$\displaystyle \original@int \original@int \original@int F(x;y;z) \dd{x} \dd{y} \dd{z}
    	= \original@int_{a}^{b} \original@int_{c}^{d} \original@int_{e}^{f} F(x;y;z) \dd{x} \dd{y} \dd{z}$.
	\makeatother
\end{remark}




% \section{Analysis}

\subsection{Tableaux de variation et de signe}

\paragraph{Comment ça marche ?}

Tout le boulot est fait par le package \verb+tkz-tab+ auquel on impose le choix d'une pointe de flèche plus visible. Nous vous demandons donc de vous reporter à la documentation de \verb+tkz-tab+ pour savoir comment s'y prendre.


\paragraph{Un exemple de tableaux de signes}

\begin{tcblisting}{}
\begin{tikzpicture}
\tkzTabInit{$x$ / 1 , $\cos(x)$ / 1}{$0$, $\frac{\pi}{2}$, $\pi$}
\tkzTabLine{, +, z, -, }
\end{tikzpicture}
\end{tcblisting}


\paragraph{Un exemple de tableaux de variation}

\begin{tcblisting}{}
\begin{tikzpicture}
    \tkzTabInit{$x$ / 1 , $f(x)$ / 1.5}{$-\infty$, $p$, $+\infty$}
    \tkzTabVar{+/ , -/ $f(p)$, +/ }
\end{tikzpicture}
\end{tcblisting}


\paragraph{Un exemple de tableaux de variation avec une dérivée}

\begin{tcblisting}{}
\begin{tikzpicture}
    \tkzTabInit{$x$ / 1 , $\cos(x)$ / 1, $\sin(x)$ / 1.5}{$0$, $\frac{\pi}{2}$, $\pi$}
    \tkzTabLine{, +, z, -, }
    \tkzTabVar{-/ 0, +/ 1, -/ 0}
\end{tikzpicture}
\end{tcblisting}




% \section{Analysis}

\subsection{Comparaison asymptotique de suites et de fonctions}

\subsubsection{\texorpdfstring{Les notations $\bigO{}$ et $\smallO{}$}%
                               {Les notations "grand O" et "petit O"}}

\paragraph{Exemple d'utilisation 1}

\begin{tcblisting}{}
Vous pouvez utiliser les symboles $\bigO{}$ et $\smallO{}$ créés par Landau.
\end{tcblisting}


\paragraph{Exemple d'utilisation 2}

\begin{tcblisting}{}
Vous pouvez écrire $\bigO{x} \neq \smallO{x}$ et $e^{t + \smallO{t}} = e^{\bigO{t}}$.
\end{tcblisting}


\paragraph{Fiches techniques}

\IDmacro*{bigO}{1}

\IDmacro*{smallO}{1}

\IDarg{} non vide, cet argument sera mis entre des parenthèses après $\bigO{}$ ou $\smallO{}$.



\subsubsection{\texorpdfstring{La notation $\bigomega{}$}%
                               {La notation "grand Omega"}}

\paragraph{Exemple d'utilisation 1}

\begin{tcblisting}{}
Vous pouvez utiliser le symbole $\bigomega{}$ créé par Hardy et Littlewood.
\end{tcblisting}


\paragraph{Exemple d'utilisation 2}

\begin{tcblisting}{}
$f(n) = \bigomega{g(n)}$ signifie :
$\exists (m, n_0)$ tel que $n \geqslant n_0$ implique $f(n) \geqslant m g(n)$.
\end{tcblisting}


\paragraph{Fiche technique}

\IDmacro*{bigomega}{1}

\IDarg{} non vide, cet argument sera mis entre des parenthèses après $\bigomega{}$.



\subsubsection{\texorpdfstring{La notation $\bigtheta{}$}%
                               {La notation "grand Theta"}}

\paragraph{Exemple d'utilisation 1}

\begin{tcblisting}{}
Voici le dernier symbole $\bigtheta{}$ qui peut rendre service.
\end{tcblisting}


\paragraph{Exemple d'utilisation 2}

\begin{tcblisting}{}
$f(n) = \bigtheta{g(n)}$ signifie : $\exists (m, M, n_0)$ tel que $n \geqslant n_0$
implique $m g(n) \leqslant f(n) \leqslant M g(n)$.
\end{tcblisting}


\paragraph{Fiche technique}

\IDmacro*{bigtheta}{1}

\IDarg{} non vide, cet argument sera mis entre des parenthèses après $\bigtheta{}$.




\section{Probabilité}

\subsection{Probabilité conditionnelle}

\paragraph{Un exemple type}

\begin{tcblisting}{}
Écrire des probabilités conditionnelles :
$\probacond{p}{A}{B} = \probacond*{p}{A}{B}
                     = \probacond**{p}{A}{B}
                     = \dprobacond**{p}{A}{B}$.
\end{tcblisting}


\paragraph{Fiche technique}

\IDmacro*{probacond}{3}

\IDmacro*{probacond*}{3}

\IDmacro*{probacond**}{3}

\IDmacro*{dprobacond**}{3}

\IDarg{1} le nom de la probabilité.

\IDarg{2} l'ensemble dont on veut calculer la probabilité.

\IDarg{3} l'ensemble qui donne la condition.




%\section{Probabilité}

\subsection{Arbres pondérés}

\paragraph{Que se passe-t-il en coulisse ?}

Le gros du travail est fait par le package \verb+forest+ qui utilise \verb+TiKz+. Ceci permet de faire des choses sympathiques comme dans le 2\ieme{} exemple ci-dessous.


\paragraph{Un exemple type}

Dans le code suivant l'environnement \verb+probatree+ utilise en coulisse celui nommé \verb+forest+ du package \verb+forest+. Des réglages spécifiques sont faits pour obtenir le résultat ci-après. A cela s'ajoute les styles spéciaux \verb+pweight+, \verb+apweight+ et \verb+bpweight+ qui facilitent l'écriture des pondérations sur les branches
\footnote{
	\texttt{pweight} vient de \emph{\og probability \fg} et \emph{\og weight\fg} soit \emph{\og probabilité \fg} et \emph{\og poids\fg} en anglais.
	Quant à \texttt{a} et \texttt{b} au début de \texttt{apweight} et \texttt{bpweight} respectivement, ils viennent de \emph{\og above \fg} et \emph{\og below\fg} soit \emph{\og dessus \fg} et \emph{\og dessous\fg} en anglais.
}.

\begin{tcblisting}{}
\begin{probatree}
[
    [$A$, pweight = $a$
        [$B$, pweight = $b$]
        [$C$, pweight = $c$]
    ]
    [$D$, pweight = $d$
        [$E$, apweight = $e$]
        [$F$, bpweight = $f$]
    ]
]
\end{probatree}
\end{tcblisting}


\paragraph{Un exemple décoré facilement}

Via la clé \verb+frame+, il est très aisé d'encadrer un sous-arbre comme le montre l'exemple suivant. Dans l'exemple ci-après nous utilisons la bidouille \verb+{},s sep  = 1.3cm+ qui évite que les cadres se superposent.

\begin{tcblisting}{}
\begin{probatree}
[{},s sep  = 1.3cm
    [$A$, pweight=$a$, frame=red
        [$B$, pweight=$b$]
        [$C$, pweight=$c$]
    ]
    [$D$, pweight=$d$, frame=blue
        [$E$, pweight=$e$
        	[$F$, pweight=$f$]
        	[$G$, pweight=$g$]
		]
        [$H$, pweight=$h$
        	[$I$, pweight=$i$]
        	[$J$, pweight=$j$]
        ]
    ]
]
\end{probatree}
\end{tcblisting}


\paragraph{Un exemple décoré à la main}

En utilisant la machinerie de \verb+TiKz+ il est facile de décorer un arbre de probabilité comme ci-dessous où le cadre s'appuie sur trois noeuds nommés. Notons que cet exemple n'est pas faisable avec la clé \verb+frame+.

\begin{tcblisting}{}
\begin{probatree}
[
    [$A$, pweight = $a$, name = left
        [$B$, pweight = $b$, name = topright
	        [$C$, pweight = $c$]
	        [$D$, pweight = $d$]
       	]
	    [$F$, pweight = $f$, name = bottomright]
    ]
    [$G$, pweight = $g$]
]
\node[draw = blue, thick, rounded corners, fit = (left)(topright)(bottomright)] {};
\end{probatree}
\end{tcblisting}



\paragraph{Un exemple de poids cachés partout}

On peut cacher tous les poids via l'environnement étoilé \verb+probatree*+ sans avoir à retaper un arbre où les pondérations ont déjà été indiquées.

\begin{tcblisting}{}
\begin{probatree*}
[$A$, pweight = $a$
    [$B$, pweight = $b$]
    [$C$, pweight = $c$]
]
\end{probatree*}
\end{tcblisting}


\paragraph{Un exemple de poids cachés localement}

Pour ne cacher que certains poids, il faudra utiliser, à la main, le style \verb+pweight*+ comme dans l'exemple ci-dessous.

\begin{tcblisting}{}
\begin{probatree}
[
    [$A$, pweight = $a$
        [$B$, pweight* = $b$]
        [$C$, pweight = $c$]
    ]
    [$D$, pweight* = $d$]
]
\end{probatree}
\end{tcblisting}


\paragraph{Fiches techniques}

\IDenv*{probatree}

\IDenv*{probatree*}

\Content{} un arbre codé en utilisant la syntaxe supportée par le package \verb+forest+.

\IDkey{pweight} pour écrire un poids sur le milieu d'une branche.

\IDkey{apweight} pour écrire un poids au-dessus le milieu d'une branche.

\IDkey{bpweight} pour écrire un poids en-dessous du milieu d'une branche.

\IDkey{frame} pour encadre un sous-arbre.




\section{Arithmétique}

\subsection{Opérateurs de base}

Pour des raisons d'expressivité des codes \LaTeX{}, les opérateurs binaires \verb+\divides+, \verb+\notdivides+ et \verb+\modulo+ ont été ajoutés comme alias de \verb+\mid+, \verb+\nmid+ et \verb+\bmod+ respectivement qui sont proposés par le package \verb+amssymb+.

\begin{tcblisting}{}
$10 \divides 150$ et $10 \notdivides 154$ c'est mieux que $10 | 150$ et $10 \not| 154$.

De plus, il est facile d'écrire que $a \equiv b \modulo p \iff p \divides (a - b)$.
\end{tcblisting}




%\subsection{Arithmétiques}

\subsection{Fractions continuées}

\subsubsection{Fractions continuées standard}

\paragraph{Exemple d'utilisation}

Dans l'exemple suivant, la notation en ligne semble être due à Alfred Pringsheim. La notation à gauche utilise toujours le maximum d'espace pour améliorer la lisibilité.

\begin{tcblisting}{}
$ \contfrac{u_0 | u_1 | u_2 | \dots | u_n}
= \contfrac*{u_0 | u_1 | u_2 | \dots | u_n}$
\end{tcblisting}


\paragraph{Fiches techniques}

\IDmacro*{contfrac}{1}

\IDmacro*{contfrac*}{1}

\IDarg{} tous les éléments de la fraction continuée séparés par des \verb+|+.




\subsubsection{Fractions continuées généralisées}

\paragraph{Exemple d'utilisation}

Voici comment écrire une fraction continuée généralisée.

\begin{tcblisting}{}
$\displaystyle \contfracgene{a | b | c | d | e | f | \dots | y | z}
             = \contfracgene*{a | b | c | d | e | f | \dots | y | z}$
\end{tcblisting}


\paragraph{Fiches techniques}

\IDmacro*{contfracgene}{1}

\IDmacro*{contfracgene*}{1}

\IDarg{} tous les éléments de la fraction continuée généralisée séparés par des \verb+|+.



\subsubsection{Comme une fraction continuée isolée}

\paragraph{Exemple d'utilisation}

La raison d'être de la macro ci-dessous vient juste de son usage en interne.

\begin{tcblisting}{}
Les fous (?\,) adorent vraiment écrire des choses comme $\singlecontfrac{a}{b}$.
\end{tcblisting}


\paragraph{Fiche technique}

\IDmacro*{singlecontfrac}{2}

\IDarg{1} le pseudo numérateur.

\IDarg{2} le pseudo dénominateur.



\subsubsection{\texorpdfstring{L'opérateur $\contfracope$}%
                               {L'opérateur K}}

\paragraph{Exemple d'utilisation 1}

\IDconstant{contfracope}

La notation suivante est proche de celle qu'utilisait Carl Friedrich Gauss.

\begin{tcblisting}{}
$\displaystyle
  \contfracope_{k=1}^{n} (b_k:c_k)
= \cfrac{b_1}{\contfracgene{c_1 | b_2 | c_2 | b_3 | \dots | b_n | c_n}}$
\end{tcblisting}


\begin{remark}
    La lettre $\contfracope$ vient de "kettenbruch" qui signifie "fraction continuée" en allemand.
\end{remark}


\paragraph{Exemple d'utilisation 2}

\begin{tcblisting}{}
$\displaystyle
  u_0 + \contfracope_{k=1}^{n} (1:u_k)
= \contfrac{u_0 | u_1 | u_2 | \dots | u_n}$
\end{tcblisting}




\section{Algèbre}

\subsection{Polynômes, séries formelles et compagnie}

\subsubsection{Polynômes et fractions polynômiales}

\paragraph{Exemple d'utilisation 1 : Polynômes}

\begin{tcblisting}{}
$\setpoly{\RR}{X}$ est l'ensemble des polynômes à coefficients réels en la variable
$X$, et $\setpoly{\RR}{X | Y | Z}$ est l'ensemble des polynômes à coefficients réels
en les variables $X$ , $Y$ et $Z$.
\end{tcblisting}



\paragraph{Exemple d'utilisation 2 : Fractions polynômiales}

\begin{tcblisting}{}
$\setpolyfrac{\QQ}{T}$ et $\setpolyfrac{\QQ}{S_1 | S_2 | \dots | S_k}$ permettent
d'indiquer des ensemble de fractions polynomiales à coefficients rationnels.
\end{tcblisting}



\subsubsection{Séries formelles et leurs corps de fractions}

\paragraph{Exemple d'utilisation 1 : Séries formelles}

\begin{tcblisting}{}
$\setserie{\CC}{X}$ et $\setserie{\CC}{T | O | P}$ permettent de travailler avec des
séries formelles à coefficients complexes.
\end{tcblisting}



\paragraph{Exemple d'utilisation 2 : Corps des fractions de séries formelles}

\begin{tcblisting}{}
$\setseriefrac{\ZZ}{X}$ et $\setseriefrac{\ZZ}{Z | T | O | P}$ permettent de travailler
avec des fractions de séries formelles à coefficients entiers.
\end{tcblisting}



\subsubsection{Polynômes de Laurent et séries formelles de Laurent}

\paragraph{Exemple d'utilisation 1 : Polynômes de Laurent}

\begin{tcblisting}{}
$\setpolylaurent{\RR}{X} = \setpoly{\RR}{X | X^{-1}}$ est l'ensemble des polynômes
réels de Laurent en $X$. On propose de généraliser comme suit (notation non standard) :
$\setpolylaurent{\RR}{X_1 | X_2} = \setpoly{\RR}{X_1 | X_1^{-1} | X_2 | X_2^{-1}}$
\end{tcblisting}



\paragraph{Exemple d'utilisation 2 : Séries formelles de Laurent}

\begin{tcblisting}{}
$\setserielaurent{\QQ}{X} = \setserie{\QQ}{X | X^{-1}}$ est l'ensemble des séries
formelles rationnelles de Laurent en $X$. On généralise via, notation non standard,
$\setserielaurent{\QQ}{X_1 | X_2} = \setserie{\QQ}{X_1 | X_1^{-1} | X_2 | X_2^{-1}}$
\end{tcblisting}



\subsubsection{Toutes les fiches techniques}

\IDmacro*{setpoly}{2}

\IDmacro*{setpolyfrac}{2}

\IDmacro*{setserie}{2}

\IDmacro*{setseriefrac}{2}

\IDmacro*{setpolylaurent}{2}

\IDmacro*{setserielaurent}{2}

\IDarg{1} l'ensemble auquel les coefficients appartiennent.

\IDarg{2} cet argument est une suite de "morceaux" séparés par des barres \verb+|+, chaque morceau étant une variable formelle.




%\section{Algèbre}

\subsection{Matrices}

\paragraph{Comment ça marche ?}

Tout le boulot est fait par le package \verb+nicematrix+ auquel on impose l'option \verb+transparent+. Veuillez vous reporter à la documentation de \verb+nicematrix+ pour savoir comment s'y prendre.


\paragraph{Exemple 1 tiré de la documentation de \texttt{nicematrix}}

\begin{tcblisting}{}
$\begin{pmatrix}
    1      & \cdots & \cdots & 1      \\
    0      & \ddots &        & \vdots \\
    \vdots & \ddots & \ddots & \vdots \\
    0      & \cdots & 0      & 1
\end{pmatrix}$
\end{tcblisting}


\paragraph{Exemple 2 tiré de la documentation de \texttt{nicematrix}}

\begin{tcblisting}{}
$\begin{pNiceMatrix}[name=mymatrix]
    1 & 2 & 3 \\
    4 & 5 & 6 \\
    7 & 8 & 9
\end{pNiceMatrix}$

\tikz[remember picture,overlay]
\draw (mymatrix-2-2) circle (2mm) ;
\end{tcblisting}


\paragraph{Exemple 3 tiré de la documentation de \texttt{nicematrix}}

\begin{tcblisting}{}
$\left(
    \begin{NiceArray}{CCCC:C}
        1  & 2  & 3  & 4  & 5  \\
        6  & 7  & 8  & 9  & 10 \\
        11 & 12 & 13 & 14 &15
    \end{NiceArray}
\right)$
\end{tcblisting}$




\newpage

\section{Historique}

Nous ne donnons ici qu'un très bref historique de \verb+lymath+ côté utilisateur principalement.
Tous les changements sont disponibles uniquement en anglais dans le dossier \verb+change-log+ : voir le code source de \verb+lymath+ sur \verb+github+.

\begin{description}[leftmargin=1em]
    \setlength\itemsep{1em}


% --------------- %

    \item[2019-10-13] Nouvelle version sous-mineure \verb+0.6.1-beta+.
    \begin{itemize}
        \item En logique, la macro \verb+\explain+ possède maintenant un argument optionnel pour indiquer l'espacement avant le symbole. Les macros obsolètes \verb+\explain*+ et \verb+\textexplainspacebefore+ ont été supprimées.

        \item En probabilité, voici ce qui a évolué.
        \begin{itemize}
        	\item Les macros \verb+\probacond+ et \verb+\probacond*+ n'ont plus d'argument optionnel. Pour obtenir l'écriture fractionnaire, il faut utiliser \verb+\probacond**+ ou \verb+\dprobacond**+.

        	\item Les environnements \verb+probatree+ et \verb+probatree*+ ont trois nouvelles clés.
			      La clé \verb+frame+ permet d'encadrer un sous-arbre, et les clés \verb+apweight+ et \verb+bpweight+ permettent d'écrire des poids dessus/dessous une branche.
        \end{itemize}
    
        \item Pour les ensembles, il y a eu les renommages suivants par souci de cohérence.
        \begin{itemize}
        	\item \verb+\algeset+ est devenu \verb+\setalge+.
        	\item \verb+\geoset+ est devenu \verb+\setgeo+.
        	\item \verb+\geneset+ est devenu \verb+\setgene+.
        	\item \verb+\probaset+ est devenu \verb+\setproba+.
        	\item \verb+\specialset+ est devenu \verb+\setspecial+.
        \end{itemize}
    \end{itemize}  
    

% --------------- %

    \item[2019-10-10] Nouvelle version mineure \verb+0.6.0-beta+.
    \begin{itemize}
        \item Des nouveaux outils spécifiques aux probabilités.
        \begin{itemize}
            \item Les macros \verb+\probacond+ et \verb+\probacond*+ servent à écrire des probabilités conditionnelles.

            \item Les environnements \verb+probatree+ et \verb+probatree*+ simplifient la production d'arbres probabilistes pondérés ou non.
        \end{itemize}

        \item En géométrie, la macro \verb+\notparallel+ a été rajoutée.

        \item Un nouveau type d'intervalle pour l'informatique théorique via la macro \verb+\CSinterval+ afin d'obtenir quelque chose comme \verb+a..b+.

        \item En logique, il y a deux nouvelles macros sémantiques \verb+\neqid+ et \verb+\eqchoice+.
    \end{itemize}


% --------------- %

    \item[2019-09-27] Nouvelle version mineure \verb+0.5.0-beta+.
    \begin{itemize}
        \item Ajout des macros \verb+\dsum+ et \verb+\dprod+ qui sont vis à vis de \verb+\sum+ et \verb+\prod+ des équivalents de \verb+\dfrac+ pour \verb+\frac+.

        \item En arithmétique, ajout des opérateurs \verb+\divides+, \verb+\notdivides+ et \verb+\modulo+.

        \item En géométrie, une nouvelle macro et un opérateur modifié.
        \begin{itemize}
            \item \verb+\pts+ permet d'indiquer plusieurs points.

            \item \verb+\parallel+ utilise des obliques pour symboliser le parallélisme au lieu de barres verticales.
        \end{itemize}

        \item En logique, il y a les nouveautés suivantes.
        \begin{itemize}
            \item La version doublement étoilée \verb+\eqdef**+ donne une deuxième écriture symbolique d'un symbole égal de type définition \emph{(cette notation vient du langage B)}.

            \item Ajout de \verb+\liesimp+ comme alias de \verb+\Longleftarrow+.

            \item Les macros \verb+\vimplies+, \verb+\viff+ et \verb+\vliesimp+ sont des versions verticales de \verb+\implies+, \verb+\iff+ et \verb+\liesimp+.

            \item Comme pour les égalités, il existe les macros \verb+\impliestest+, \verb+\iffhyp+ ... etc.
        \end{itemize}
    \end{itemize}

% --------------- %

    \item[2019-09-06]  Nouvelle version mineure \verb+0.4.0-beta+.
    \begin{itemize}
        \item Dans \emph{\og Logique et fondements \fg}, différents types de signes d'inéquation et de non égalité pour des cas de test, d'hypothèse faite et de condition à vérifier.

        \item Intégration du package \verb+tkz-tab+ pour rédiger des tableaux de variations et de signes.

        \item Intégration du package \verb+nicematrix+ pour écrire des matrices.
    \end{itemize}

% --------------- %

    \item[2019-07-23] Nouvelle version mineure \verb+0.3.0-beta+.
    \begin{itemize}
        \item Une nouvelle section \emph{\og Logique et fondements \fg} a été ajoutée.
        \begin{itemize}
            \item Trois types de signes $=$ décorés sont proposés : voir les macros \verb+\eqdef+ , \verb+\eqid+ et \verb+\eqtest+.

            \item Via la macro \verb+\explain+, il devient facile d'expliquer des étapes de raisonnement ou des calculs.
        \end{itemize}

        \item Pour les ensembles, la macro \verb+\fieldset+ a été renommé \verb+\algeset+ et la macro \verb+\PP+ permet d'indiquer l'ensemble des nombres premiers.

        \item En géométrie, il y a quelques nouveautés.
        \begin{itemize}
            \item La macro \verb+\hangleorient+ permet l'écriture d'angles orientés avec un chapeau en plus.

            \item Les macros \verb+\vangleorient+ et \verb+\vhangleorient+ évite d'avoir à utiliser \verb+\vect+ lorsque l'on a juste des vecturs simples nommés et non coefficientés.

            \item De même pour les macros \verb+\vdotprod+, \verb+\vadotprod+ et \verb+\vcroosprod+.
        \end{itemize}

        \item Ajout de \verb+\lymathsubsep+ qui définit le séparateur des arguments de second niveau.
    \end{itemize}

% --------------- %

    \item[2019-02-21] Nouvelle version mineure \verb+0.2.0-beta+.
    \begin{itemize}
        \item L'usage de \verb+//+ pour les macros-commandes avec un nombre quelconque d'arguments a été remplacé par celui de \verb+|+.

        \item En géométrie, il y a diverses nouveautés.
        \begin{itemize}
            \item Ajout de l'écriture de coordonnées, de produits scalaires et de produits vectoriels.

            \item \verb+\axis+ a été correctement traduit en \verb+\axes+.

            \item Les macros \verb+\gpaxis+ et \verb+\gpvaxis+ deviennent \verb+\paxes+ et \verb+\pvaxes+ pour être cohérent avec \verb+\pt+ qui a remplacé l'ancien \verb+\gpt+.
        \end{itemize}

        \item En analyse, ajout de la macro commande étoilée \verb+\derpow*+ pour la gestion automatique des primes d'une dérivée.

        \item Une nouvelle section "algèbre" propose des macros pour écrire des ensembles de polynômes, de fractions polynomiales, de séries formelles, de fractions de séries formelles, et aussi de polynômes et de séries formelles de Laurent.

        \item Redéfinition de \verb+\frac+ et \verb+\dfrac+ pour obtenir des traits de fraction un peu plus longs.

        \item Ajout de \verb+\lymathsep+ qui définit le séparateur d'arguments.
    \end{itemize}

% --------------- %

    \item[2017-11-01] Nouvelle version mineure \verb+0.1.0-beta+ : pour les ensembles, les fonctions et la géométrie, il y a eu des changements et l'ajout de nouveaux outils.

% --------------- %

    \item[2017-10-21] Historique court de \verb+lymath+ ajouté au présent document.

% --------------- %

    \item[2017-10-18] Nouvelle version "patchée" \verb+0.0.2-beta+ : de nouveaux outils pour le calcul différentiel.

% --------------- %

    \item[2017-10-06] Nouvelle version "patchée" \verb+0.0.1-beta+ : de nouveaux outils pour l'arithmétique, la géométrie, le calcul intégral et le calcul différentiel.

% --------------- %

    \item[2017-10-02] Première version \verb+0.0.0-beta+ du package.
\end{description}



\end{document}
