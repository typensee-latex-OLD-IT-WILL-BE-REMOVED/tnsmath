\documentclass[12pt,a4paper]{scrartcl}

\makeatletter % Technical doc - START

\usepackage[utf8]{inputenc}
\usepackage[T1]{fontenc}
\usepackage{ucs}

\usepackage[french]{babel,varioref}

\usepackage[top=2cm, bottom=2cm, left=1.5cm, right=1.5cm]{geometry}
\usepackage{enumitem}

\usepackage{color}
\usepackage{hyperref}
\hypersetup{
    colorlinks,
    citecolor=black,
    filecolor=black,
    linkcolor=black,
    urlcolor=black
}

\usepackage{amsthm}

\usepackage{tcolorbox}
\tcbuselibrary{listingsutf8}

\usepackage{ifplatform}

\usepackage{pgffor}
\usepackage{xstring}


% MISC

\tcbset{%
	sharp corners,%
	left=1mm, right=1mm,%
	bottom=1mm, top=1mm,%
	colupper=red!75!blue% 
}

\setlength{\parindent}{0cm}
\setlist{noitemsep}

\theoremstyle{definition}
\newtheorem*{remark}{Remarque}

\usepackage[raggedright]{titlesec}

\titleformat{\paragraph}[hang]{\normalfont\normalsize\bfseries}{\theparagraph}{1em}{}
\titlespacing*{\paragraph}{0pt}{3.25ex plus 1ex minus .2ex}{0.5em}

\newcommand\ascii{\texttt{ASCII}}


% Technical IDs

\newwrite\tempfile

\immediate\openout\tempfile=x-\jobname.macros-x.txt

\AtEndDocument{\immediate\closeout\tempfile}

\newcommand\IDconstant[1]{%
    \immediate\write\tempfile{constant@#1}%
}

\newcommand\IDmacro{\@ifstar{\@IDmacro@star}{\@IDmacro@no@star}}

\newcommand\@IDmacro@no@star[3]{%
    \texttt{%
    	\textbackslash#1%
    	\IfStrEq{#2}{0}{}{%
    		\,\,[#2 Option%
				\IfStrEq{#2}{1}{}{s}]%
			}%
	    \IfStrEq{#3}{}{}{%
	    		\,\,(#3 Argument%
				\IfStrEq{#3}{1}{}{s})%
			}
	   	}
    \immediate\write\tempfile{macro,#1,#2,#3}%
}

\newcommand\@IDmacro@star[2]{%
    \@IDmacro@no@star{#1}{0}{#2}%
}

\newcommand\@IDoptarg{\@ifstar{\@IDoptarg@star}{\@IDoptarg@no@star}}

\newcommand\@IDoptarg@star[2]{%
	\vspace{0.5em}
	\textbf{---} \texttt{#1%
		\IfStrEq{#2}{}{:}{\,#2:}%
	}%
}

\newcommand\@IDoptarg@no@star[2]{%
	\IfStrEq{#2}{}{%
		\@IDoptarg@star{#1}{}%
	}{%
		\@IDoptarg@star{#1}{#2}%
	}%
}

\newcommand\IDkey[1]{%
	\@IDoptarg*{Option}{{\itshape "#1"}}%
}

\newcommand\IDoption[1]{%
	\@IDoptarg{Option}{#1}%
}

\newcommand\IDarg[1]{%
	\@IDoptarg{Argument}{#1}%
}

\makeatother % Technical doc - END


\usepackage{lymath}


\begin{document}

\renewcommand\labelitemi{\raisebox{0.125em}{\tiny\textbullet}}
\renewcommand{\labelitemii}{---}

\title{%	
	Le package \texttt{lymath}:\\%
	des formules plus sémantiques\\%
	{\footnotesize Code source disponible sur \url{https://github.com/bc-latex/ly-math}.}\\%
{\footnotesize Version \texttt{0.2.0-beta} développée et testée sur \macosxname{}.}%
}
\author{Christophe BAL}
\date{2019-02-21}

\maketitle


\vspace{2em}

\hrule

\tableofcontents

\vspace{1.5em}

\hrule

\newpage



\section{Introduction}

\LaTeX{} est un excellent langage, pour ne pas dire le meilleur, pour rédiger des documents contenant des formules mathématiques.
Malheureusement toute la puissance de \LaTeX{} permet d'écrire des codes très peu sémantiques.
Le modeste but du package \verb+lymath+ est de fournir quelques macros sémantiques pour la rédaction de formules mathématiques élémentaires. Considérons le code \LaTeX{} suivant.

\begin{tcblisting}{listing only}
Sachant que $\frac{df}{dx}(x) = 4 cos(x^2)$ sur $[a ; b]$ , nous avons :

$\int_a^b cos(x^2) dx = \left[ \frac{1}{4} f(x) \right]_a^b$.
\end{tcblisting}


Avec \verb+lymath+, vous pouvez écrire le code suivant.

\begin{tcblisting}{listing only}
Sachant que $\derfrac{f}{x}(x) = 4 cos(x^2)$ sur $\intervalC{a}{b}$, nous avons :

$\int_a^b cos(x^2) \dd{x} = \hook{\frac{1}{4} f(x)}{a}{b}$.
\end{tcblisting}


Même si certaines commandes sont plus longues à écrire que ce que permet \LaTeX{}, il y a trois avantages à utiliser des commandes sémantiques.
\begin{enumerate}
	\item La mise en forme dans votre document sera consistante.

	\item Il est facile de changer une mise en forme sur l'ensemble d'un document.

	\item \verb+lymath+ résout certains problèmes "complexes" pour vous.
\end{enumerate}



\section{Comment lire cette documentation ?}

Le choix a été fait de fournir des exemples comme documentation du package suivis de fiches techniques des macros-commandes. Les exemples se présentent comme ci-dessous \textit{(un code \LaTeX{} suivi de sa mise en forme)}.

\begin{tcblisting}{}
Sachant que $\displaystyle \frac{df}{dx}(x) = 4 cos(x^2)$ sur $[a ; b]$ , nous avons :
$\displaystyle \int_a^b cos(x^2) dx = \left[ \frac{1}{4} f(x) \right]_a^b$.
\end{tcblisting}



\section{Versions étoilées}

Généralement les versions étoilés proposent des mises en forme faisant un peu moins de travail qua la macro non étoilé \textit{(il y a quelques exceptions)}. Par exemple une macro utilisant des parenthèses extensibles aura une version étoilé qui n'utilisera que des parenthèses non extensibles.



\section{A propos des arguments \LaTeX, une convention à connaître}

\verb+lymath+ propose des macros avec un \textbf{nombre fixé d'arguments}. Dans ce cas, la syntaxe \LaTeX{} est gardée comme dans \verb+\derfrac{f}{x}+ .

\medskip

Par contre, pour les macros avec un \textbf{nombre variable d'arguments}, la convention sera toujours d'utiliser un seul argument, au sens \LaTeX{}, dont le contenu sera formé de morceaux séparés par des traits verticaux \verb+|+ comme dans \verb+\coord{3 | -4 | 0}+ où l'unique argument \verb+3 | -4 | 0+ contient les trois morceaux \verb+3+ , \verb+-4+ et \verb+0+ .




\section{Un séparateur d'arguments par défaut}

La macro \verb+\lymathsep+ définit le séparateur d'arguments dans les formules écrites. Cette documentation utilisant l'option \verb+french+ de \verb+babel+, la valeur de \verb+\lymathsep+ est \fbox{\,\lymathsep$\vphantom{F}$\,} . Sans ce choix, la valeur de \verb+\lymathsep+ sera \fbox{\,,$\vphantom{F}$\,} .




\section{Quelques gestions d'espaces}

	\subsection{Espace et fraction}

Si vous utilisez \verb+\frac+ ou \verb+\dfrac+ alors de petits espaces sont automatiquement ajoutés pour éviter d'avoir des traits de fraction trop petits. Le comportement par   défaut se retrouve en utilisant les macros \verb+\stdfrac+ et \verb+\stddfrac+ . Voici un exemple.

\begin{tcblisting}{}
Vous avez $\frac{2}{3} = \stdfrac{2}{3}$ et $\dfrac{2}{3} = \stddfrac{2}{3}$ .
\end{tcblisting}




%\section{Quelques gestions d'espaces}

	\subsection{Espace et point-virgule avec l'option \texttt{french} de \texttt{babel}}

Seulement \textbf{si vous utilisez \texttt{babel} avec l'option \texttt{french}}, comme c'est le cas dans cette documentation, alors vous verrez le même espacement autour du point-virgule dans $A(x;y)$. Que c'est beau !




\section{Ensembles}

    \subsection{Différents types d'ensembles}

        \subsubsection{Ensembles versus accolades}

            \paragraph{Exemple d'utilisation 1}

\begin{tcblisting}{}
Un ensemble de beaux nombres : $\geneset{1 ; 3 ; 5}$ .
\end{tcblisting}


            \paragraph{Exemple d'utilisation 2}

\begin{tcblisting}{}
Choisissez votre camp :
$\displaystyle \geneset{\frac{1}{3} ; \frac{5}{7} ; \frac{9}{11}}$
ou 
$\displaystyle \geneset*{\frac{1}{3} ; \frac{5}{7} ; \frac{9}{11}}$ .
\end{tcblisting}


            \paragraph{Fiches techniques}

\IDmacro*{geneset}{1}

\IDmacro*{geneset*}{1}

\IDarg{} la définition de l'ensemble.


        \subsubsection{Ensembles pour la géométrie}

            \paragraph{Exemple d'utilisation 1}

\begin{tcblisting}{}
Vous pouvez écrire sémantiquement $\geoset{C}$, $\geoset{D}$ et $\geoset{d}$ mais pas
taper \verb+$\geoset{ABC}$+.
\end{tcblisting}


            \paragraph{Exemple d'utilisation 2}

\begin{tcblisting}{}
Pour les indices, utilisez $\geoset*{C}{1}$, $\geoset*{C}{2}$ \dots
\end{tcblisting}


            \paragraph{Fiches techniques}

\IDmacro*{geoset}{1}

\IDarg{} un seul caractère \ascii{} indiquant un ensemble géométrique.


\bigskip


\IDmacro*{geoset*}{2}

\IDarg{1} un seul caractère \ascii{} indiquant $\geoset{U}$ dans le nom $\geoset*{U}{d}$ d'un ensemble géométrique.

\IDarg{2} un texte donnant $d$ dans le nom $\geoset*{U}{d}$ d'un ensemble géométrique.


        \subsubsection{Ensembles probabilistes}

            \paragraph{Exemple d'utilisation 1}

\begin{tcblisting}{}
Vous pouvez écrire sémantiquement $\probaset{E}$ et $\probaset{G}$ mais pas taper
\verb+$\probaset{ABC}$+.
\end{tcblisting}


            \paragraph{Exemple d'utilisation 2}

\begin{tcblisting}{}
Pour les indices, utilisez $\probaset*{E}{1}$, $\probaset*{E}{2}$ \dots
\end{tcblisting}


            \paragraph{Fiches techniques}

\IDmacro*{probaset}{1}

\IDarg{} un seul caractère \ascii{} majuscule indiquant un ensemble probabiliste.


\bigskip


\IDmacro*{probaset*}{2}

\IDarg{1} un seul caractère \ascii{} majuscule indiquant $\probaset{U}$ dans le nom $\probaset*{U}{d}$ d'un ensemble probabiliste.

\IDarg{2} un texte donnant $d$ dans le nom $\probaset*{U}{d}$ d'un ensemble probabiliste.



        \subsubsection{Ensembles de type anneau ou corps}

            \paragraph{Exemple d'utilisation 1}

\begin{tcblisting}{}
Vous pouvez écrire sémantiquement $\fieldset{A}$, $\fieldset{K}$, $\fieldset{h}$ et
$\fieldset{k}$ mais pas taper \verb+$\fieldset{ABC}$+.
\end{tcblisting}


            \paragraph{Exemple d'utilisation 2}

\begin{tcblisting}{}
Pour les indices, utilisez $\fieldset*{k}{1}$, $\fieldset*{k}{2}$ \dots
\end{tcblisting}


            \paragraph{Fiches techniques}

\IDmacro*{fieldset}{1}

\IDarg{} soit l'une des lettres  \texttt{h} et \texttt{k}, soit un seul caractère \ascii{} majuscule indiquant un ensemble de type anneau ou corps.


\bigskip


\IDmacro*{fieldset*}{2}

\IDarg{1} un seul caractère \ascii{} indiquant $\fieldset{U}$ dans le nom $\fieldset*{U}{d}$ d'un ensemble de type anneau ou corps.

\IDarg{2} un texte donnant $d$ dans le nom $\fieldset*{U}{d}$ d'un ensemble de type anneau ou corps.



        \subsubsection{Ensembles classiques}

\begin{tcblisting}{}
Vous pouvez utiliser directement $\nullset$, $\NN$, $\ZZ$, $\DD$, $\QQ$, $\RR$, $\CC$,
$\HH$ et $\OO$.
\end{tcblisting}



        \subsubsection{Ensembles classiques suffixés}

\begin{tcblisting}{}
Il est facile de taper $\RRn$, $\RRp$, $\RRs$, $\RRsn$ et $\RRsp$.
\end{tcblisting}


Nous avons utilisé les suffixes \verb+n+ pour \verb+Negatif+, \verb+p+ pour \verb+Positif+, et \verb+s+ pour \verb+star+, soit "étoile" en anglais. Il y a aussi les suffixes composites \verb+sn+ et \verb+sp+.

\medskip

Notez qu'il est interdit d'utiliser \verb+$\CCn$+ pour $\specialset{\CC}{n}$ car l'ensemble $\CC$ ne possède pas de structure ordonnée standard. Jetez un oeil à la section suivante pour apprendre à taper $\specialset{\CC}{n}$ si vous en avez besoin. L'interdiction est ici purement sémantique !

\medskip

\begin{remark}
	La table \ref{table:suffixes-sets} \vpageref{table:suffixes-sets} montre les associations autorisées entre ensembles classiques et suffixes.
\end{remark}

% == Table of suffixes - START == %

\newcommand\xx{\phantom{$\times$}}
\begin{table}[h]
    \caption{Suffixes}
    \begin{center}
        \begin{tabular}{c|c|c|c|c|c}
              & \verb+n+ & \verb+p+ & \verb+s+ & \verb+sn+ & \verb+sp+ \\
            \hline \verb+N+ & \xx & \xx & $\times$ & \xx & \xx \\
            \hline \verb+Z+ & $\times$ & $\times$ & $\times$ & $\times$ & $\times$ \\
            \hline \verb+D+ & $\times$ & $\times$ & $\times$ & $\times$ & $\times$ \\
            \hline \verb+Q+ & $\times$ & $\times$ & $\times$ & $\times$ & $\times$ \\
            \hline \verb+R+ & $\times$ & $\times$ & $\times$ & $\times$ & $\times$ \\
            \hline \verb+C+ & \xx & \xx & $\times$ & \xx & \xx \\
            \hline \verb+H+ & \xx & \xx & $\times$ & \xx & \xx \\
            \hline \verb+O+ & \xx & \xx & $\times$ & \xx & \xx \\
        \end{tabular}
    \end{center}
    \label{table:suffixes-sets}
\end{table}

% == Table of suffixes - END == %



        \subsubsection{Des suffixes à la carte}

            \paragraph{Exemple d'utilisation}

\begin{tcblisting}{}
Il est tout de même possible d'écrire $\specialset{\CC}{n}$ ou $\specialset{\HH}{sp}$.
Il y a aussi $\specialset*{\probaset{P}}{n}$ avec une autre mise en forme.
\end{tcblisting}


            \paragraph{Fiches techniques}

\IDmacro*{specialset}{2}

\IDmacro*{specialset*}{2}

\IDarg{1} l'ensemble à "suffixer".

\IDarg{2} l'un des suffixes \verb+n+, \verb+p+, \verb+s+, \verb+sn+ ou \verb+sp+.




% \section{Sets}

    \subsection{Intervalles}
    
        \subsubsection{Intervalles réels - Notation française (?\,)}

            \paragraph{Exemple d'utilisation 1}

Dans cet exemple, la syntaxe fait référence à \textbf{O}-pened et \textbf{C}-losed pour "ouvert" et "fermé" en anglais.
Nous verrons que \textbf{CC} et \textbf{OO} sont contractés en \textbf{C} et \textbf{O}.

\begin{tcblisting}{}
Dans $I = ]a ; b] = \intervalOC{a}{b}$, vous constatez que la macro utilisée résout
un problème d'espacement vis à vis du signe $=$ .
\end{tcblisting}


            \paragraph{Exemple d'utilisation 2}

Les crochets s'étendent verticalement automatiquement. Pour empêcher cela, il suffit d'utiliser la version étoilée de la macro.
Dans ce cas, les crochets restent tout de même un peu plus grands que des crochets utilisés directement. Voici un exemple.

\begin{tcblisting}{}
$\displaystyle \intervalC{ \frac{1}{2} }{ 1^{2^{3}} }
             = [ \frac{1}{2} ; 1^{2^{3}} ]
             = \intervalC*{ \frac{1}{2} }{ 1^{2^{3}} }$
\end{tcblisting}


            \paragraph{Fiches techniques}

Pour toutes les macros ci-dessous, la version non étoilée produit des délimiteurs qui s'étirent si besoin verticalement, tandis que la version étoilée ne le fait pas.


\bigskip


% Docs for french real intervals - START

\IDmacro*{intervalCO}{2}

\IDmacro*{intervalCO*}{2}

\IDarg{1} borne inférieure $a$ de l'intervalle $\intervalCO{a}{b}$.

\IDarg{2} borne supérieure $b$ de l'intervalle $\intervalCO{a}{b}$.


\bigskip


\IDmacro*{intervalC}{2}

\IDmacro*{intervalC*}{2}

\IDarg{1} borne inférieure $a$ de l'intervalle $\intervalC{a}{b}$.

\IDarg{2} borne supérieure $b$ de l'intervalle $\intervalC{a}{b}$.


\bigskip


\IDmacro*{intervalO}{2}

\IDmacro*{intervalO*}{2}

\IDarg{1} borne inférieure $a$ de l'intervalle $\intervalO{a}{b}$.

\IDarg{2} borne supérieure $b$ de l'intervalle $\intervalO{a}{b}$.


\bigskip


\IDmacro*{intervalOC}{2}

\IDmacro*{intervalOC*}{2}

\IDarg{1} borne inférieure $a$ de l'intervalle $\intervalOC{a}{b}$.

\IDarg{2} borne supérieure $b$ de l'intervalle $\intervalOC{a}{b}$.

% Docs for french real intervals - END



        \subsubsection{Intervalles réels - Notation américaine}

            \paragraph{Exemple d'utilisation}

Dans cet exemple, la syntaxe fait référence à \textbf{P}-arenthèse.

\begin{tcblisting}{}
Aux États Unis, un intervalle semi-fermé s'écrit $\intervalPC{a}{b} = (a ; b]$ et
un intervalle ouvert se tape $\intervalP{a}{b} = (a ; b)$.
\end{tcblisting}


            \paragraph{Fiches techniques}

Pour toutes les macros ci-dessous, la version non étoilée produit des délimiteurs qui s'étirent si besoin verticalement, tandis que la version étoilée ne le fait pas.


\bigskip


% Docs for american real intervals - START

\IDmacro*{intervalCP}{2}

\IDmacro*{intervalCP*}{2}

\IDarg{1} borne inférieure $a$ de l'intervalle $\intervalCP{a}{b}$.

\IDarg{2} borne supérieure $b$ de l'intervalle $\intervalCP{a}{b}$.


\bigskip


\IDmacro*{intervalP}{2}

\IDmacro*{intervalP*}{2}

\IDarg{1} borne inférieure $a$ de l'intervalle $\intervalP{a}{b}$.

\IDarg{2} borne supérieure $b$ de l'intervalle $\intervalP{a}{b}$.


\bigskip


\IDmacro*{intervalPC}{2}

\IDmacro*{intervalPC*}{2}

\IDarg{1} borne inférieure $a$ de l'intervalle $\intervalPC{a}{b}$.

\IDarg{2} borne supérieure $b$ de l'intervalle $\intervalPC{a}{b}$.

% Docs for american real intervals - END



        \subsubsection{Intervalles discrets d'entiers}

            \paragraph{Exemple d'utilisation}

Dans l'exemple, la syntaxe fait référence à $\ZZ$ l'ensemble des entiers relatifs.

\begin{tcblisting}{}
Par définition, $\ZintervalC{-1}{4} = \{ -1 ; 0 ; 1 ; 2 ; 3 ; 4 \}$. Donc nous avons
$\ZintervalC{-1}{4} = \ZintervalO{-2}{5}$.
\end{tcblisting}


            \paragraph{Fiches techniques}

Pour toutes les macros ci-dessous, la version non étoilée produit des délimiteurs qui s'étirent si besoin verticalement, tandis que la version étoilée ne le fait pas.


\bigskip


% Docs for discrete intervals - START

\IDmacro*{ZintervalCO}{2}

\IDmacro*{ZintervalCO*}{2}

\IDarg{1} borne inférieure $a$ de l'intervalle $\ZintervalCO{a}{b}$.

\IDarg{2} borne supérieure $b$ de l'intervalle $\ZintervalCO{a}{b}$.


\bigskip


\IDmacro*{ZintervalC}{2}

\IDmacro*{ZintervalC*}{2}

\IDarg{1} borne inférieure $a$ de l'intervalle $\ZintervalC{a}{b}$.

\IDarg{2} borne supérieure $b$ de l'intervalle $\ZintervalC{a}{b}$.


\bigskip


\IDmacro*{ZintervalO}{2}

\IDmacro*{ZintervalO*}{2}

\IDarg{1} borne inférieure $a$ de l'intervalle $\ZintervalO{a}{b}$.

\IDarg{2} borne supérieure $b$ de l'intervalle $\ZintervalO{a}{b}$.


\bigskip


\IDmacro*{ZintervalOC}{2}

\IDmacro*{ZintervalOC*}{2}

\IDarg{1} borne inférieure $a$ de l'intervalle $\ZintervalOC{a}{b}$.

\IDarg{2} borne supérieure $b$ de l'intervalle $\ZintervalOC{a}{b}$.

% Docs for discrete intervals - END




\section{Analyse}

    \subsection{Constantes}

        \subsubsection{Constantes classiques}

            \paragraph{La liste complète}
       
% List of classical constants - START

\foreach \k in {ggamma, ppi, ttau, ee, ii, jj, kk}{\IDconstant{\k}}

\begin{tcblisting}{}
Voici la liste des constantes classiques où $\ttau = 2 \ppi$ est la benjamine :
$\ggamma$, $\ppi$, $\ttau$, $\ee$, $\ii$, $\jj$ and $\kk$.
\end{tcblisting}

% List of classical constants - END


\begin{remark}
	Faites attention car \verb+{\Large $\ppi \neq \pi$}+ produit {\Large $\ppi \neq \pi$}. Comme vous le constatez, les symboles ne sont pas identiques. Ceci est vraie pour toutes les constantes grecques.
\end{remark}



        \subsubsection{Constantes latines personnelles}

            \paragraph{Exemple d'utilisation}

\begin{tcblisting}{}
Il est aisé d'écrire $\ct{a} x^2 + \ct{b} x + \ct{c}$ au lieu de $a x^2 + b x + c$
afin de souligner que $\ct{a}$, $\ct{b}$ et $\ct{c}$ sont des constantes.
\end{tcblisting}


            \paragraph{Fiche technique}

\IDmacro*{ct}{1}

\IDarg{} un texte utilisant l'alphabet latin.




% \section{Analysis}

    \subsection{La fonction valeur absolue}

            \paragraph{Un exemple d'utilisation}

\begin{tcblisting}{}
Il est facile d'écrire $\abs{2}$ ou $\displaystyle \abs{\frac{3}{5}}$ voire aussi
$\displaystyle \abs*{\frac{3}{5}}$ si vous préférez des petits traits verticaux.
\end{tcblisting}


\begin{remark}
	Le code \LaTeX{} vient directement de ce poste : \url{https://tex.stackexchange.com/a/43009/6880}.
\end{remark}


            \paragraph{Fiches techniques}

\IDmacro*{abs}{1}

\IDmacro*{abs*}{1}

\IDarg{} l'expression à laquelle on applique la fonction valeur absolue.




% \section{Analysis}

    \subsection{Fonctions nommées spéciales}

        \subsubsection{Un exemple d'utilisation}

\begin{tcblisting}{}
Quelques fonctions nommées supplémentaires (voir la liste ecomplète ci-dessous) :
$\ch x \neq ch x$ , $\ppcm(x;y)$ , $\lg x =\logb{2} x$ and $\expb{6} y$.
\end{tcblisting}


        \subsubsection{Fonctions nommées sans paramètre}

Toutes les macros suivantes n'ont aucun paramètre.

\medskip

% List of functions without parameter - START

\foreach \k in {ch, sh, th, ach, ash, ath, arccosh, arcsinh, arctanh, acos, asin, atan, pgcd, ppcm}{\IDconstant{\k}}

\begin{tabular*}{\textwidth}{@{\extracolsep{\fill}}*{4}{l}}
    \verb+\ch+ & \verb+\sh+ & \verb+\th+\\
    \verb+\ach+ & \verb+\ash+ & \verb+\ath+\\
    \verb+\arccosh+ & \verb+\arcsinh+ & \verb+\arctanh+\\
    \verb+\acos+ & \verb+\asin+ & \verb+\atan+\\
    \verb+\pgcd+ & \verb+\ppcm+ & \\
\end{tabular*}

% List of functions without parameter - END



        \subsubsection{Fonctions nommées avec un paramètre}

            \paragraph{La liste complète}

Toutes les macros suivantes ont un seul paramètre.

\medskip

% List of functions with parameters - START

\begin{tabular*}{\textwidth}{@{\extracolsep{\fill}}*{4}{l}}
    \verb+\expb+ \, (1 paramètre) & \verb+\logb+ \, (1 paramètre) &  & \\
\end{tabular*}

% List of functions with parameters - END



            \paragraph{Fiches techniques}

\IDmacro*{expb}{1}

\IDmacro*{logb}{1}

\IDarg{} la base de l'exponentielle ou du logarithme.




% \section{Analysis}

    \subsection{Des notations complémentaires pour des suites spéciales}

            \paragraph{Exemple d'utilisation}

\begin{tcblisting}{}
Parfois nous avons besoin d'écrire $\seqplus{F}{1}{2}$ ou $\hypergeo{F}{1}{2}$ et
le fou (?\,) aime vraiment $\suprageo{F}{1}{2}{3}{4}$.
\end{tcblisting}


            \paragraph{Fiches techniques}

\IDmacro*{seqplus}{2}

\IDarg{1} l'exposant à droite.

\IDarg{2} l'indice à droite.


\bigskip


\IDmacro*{hypergeo}{2}

\IDarg{1} l'indice à gauche.

\IDarg{2} l'indice à droite.


\bigskip


\IDmacro*{suprageo}{4}

\IDarg{1} l'indice à gauche.

\IDarg{2} l'indice à droite.

\IDarg{3} l'exposant à droite.

\IDarg{4} l'exposant à gauche.




% \section{Analysis}

    \subsection{Calcul différentiel}

        \subsubsection{\texorpdfstring{Les opérateurs $\pp{}$ et $\dd{}$}%
                               {Les opérateurs "d rond" et "d droit"}}

           \paragraph{Exemple d'utilisation}

\begin{tcblisting}{}
Vous pouvez écrire $\dd{x}$ et $\pp{t}$ et aussi $ \dd[5]{x}$ ou $\pp[n]{x}$.
\end{tcblisting}


           \paragraph{Fiches techniques}

\IDmacro{dd}{1}{1}

\IDmacro{pp}{1}{1}

\IDoption{} utilisée, cette option sera mise en exposant du symbole $\pp{}$ ou $\dd{}$.

\IDarg{} la variable de différentiation à droite du symbole $\pp{}$ ou $\dd{}$.



        \subsubsection{Dérivation totale}

           \paragraph{Exemple d'utilisation 1}

\begin{tcblisting}{}
$\displaystyle \derpow*{f} (a) = \derpow{f} (a)
                               = \derfrac{f}{x} (a)
                               = \dersub{f}{x} (a)$
\end{tcblisting}


           \paragraph{Exemple d'utilisation 2}

\begin{tcblisting}{}
$\displaystyle \derpow*[3]{f}(a) = \derpow[3]{f} (a)
                                 = \derfrac[3]{f}{x} (a)
                                 = \dersub[3]{f}{x} (a)$
et
$\displaystyle \derpow*[10]{\cos} a = \derfrac[10]{\cos}{x} (a)$ avec beaucoup trop
de primes à gauche.
\end{tcblisting}


           \paragraph{Exemple d'utilisation 3}

\begin{tcblisting}{}
Si $\displaystyle f(x) = \frac{1}{x^2+3}$ alors nous avons :
   $\displaystyle \derpow[3]{f} (a)
                = \derfrac*[3]{\left( \frac{1}{x^2+3} \right)}{x} (a)$.
\end{tcblisting}


           \paragraph{Fiches techniques}

\IDmacro{derpow}{1}{1}

\IDmacro{derpow*}{1}{1}

\IDoption{} utilisée, cette option sera l'exposant de dérivation mis entre des parenthèses pour la version non étoilée, et le nombre de primes pour la version étoilée.

\IDarg{} la fonction à différencier.


\bigskip


\IDmacro{derfrac}{1}{2}

\IDmacro{derfrac*}{1}{2}

\IDmacro{dersub}{1}{2}

\IDoption{} utilisée, cette option sera l'exposant de dérivation.

\IDarg{1} la fonction à dériver.

\IDarg{2} la variable.



        \subsubsection{Dérivation partielle}

           \paragraph{Exemple d'utilisation 1}

\begin{tcblisting}{}
$\displaystyle \partialfrac{f}{x} (a;b)
             = \partialsub{f}{x} (a;b)
             = \partialprime{f}{x} (a;b)$
\end{tcblisting}


           \paragraph{Exemple d'utilisation 2}

\begin{tcblisting}{}
$\displaystyle \partialfrac[3]{G}{f^2 | v} (a;b)
             = \partialfrac{G}{f^2 | v} (a;b)
             = \partialsub{G}{f^2 | v} (a;b)
             = \partialprime{G}{f^2 | v} (a;b)$
\end{tcblisting}


           \paragraph{Exemple d'utilisation 3}

\begin{tcblisting}{}
Si $\displaystyle f(x;y) = \frac{cos(x y)}{x^2+y^2}$ alors nous avons
   $\displaystyle \partialfrac[2]{f}{x | y}
                = \partialfrac*[2]{\left( \frac{cos(x y)}{x^2 + y^2} \right)}{x | y}$.
\end{tcblisting}


           \paragraph{Fiches techniques}

\IDmacro{partialfrac}{1}{2}

\IDmacro{partialfrac*}{1}{2}

\IDoption{} utilisée, cette option sera l'exposant total de dérivation mis en exposant de $\pp$.

\IDarg{1} la fonction à dériver partiellement.

\IDarg{2} les variables utilisées pour la dérivation partielle en utilisant la syntaxe suivante : par exemple, \verb+x | y^3 | ...+ indique de dériver suivant $x$ une fois, puis suivant $y$ trois fois... etc.


\bigskip


\IDmacro*{partialsub}{2}

\IDmacro*{partialprime}{2}

\IDarg{1} la fonction à dériver partiellement.

\IDarg{2} les variables utilisées pour la dérivation partielle en utilisant la syntaxe suivante : par exemple, \verb+x | y^3 | ...+ indique de dériver suivant $x$ une fois, puis suivant $y$ trois fois... etc.




% \section{Analysis}

    \subsection{Calcul intégral}

        \subsubsection{L'opérateur crochet -- 1\textsuperscript{ère} version}

            \paragraph{Exemple d'utilisation 1}

\begin{tcblisting}{}
Par définition, $\displaystyle \int_{a}^{b} f(x) \dd{x} = \hook{F(x)}{a}{b}$ où
$\hook{F(x)}{a}{b} = F(b) - F(a)$.
\end{tcblisting}


            \paragraph{Exemple d'utilisation 2}

Par défaut, les crochets s'étirent verticalement si besoin, mais si cela vous dérange, vous pouvez faire appel à la version étoilée de la macro comme dans l'exemple suivant

\begin{tcblisting}{}
$\displaystyle \hook{\frac{x - 1}{5 + x^2}}{a}{b}
             = \hook*{\frac{x - 1}{5 + x^2}}{a}{b}$.
\end{tcblisting}


            \paragraph{Fiches techniques}

\IDmacro*{hook}{3}

\IDmacro*{hook*}{3}

\IDarg{1} le contenu entre les crochets.

\IDarg{2} la borne inférieure affichée en indice.

\IDarg{3} la borne supérieure affichée en exposant.



        \subsubsection{L'opérateur crochet -- 2\textsuperscript{nde} version}

            \paragraph{Exemple d'utilisation 1}

\begin{tcblisting}{}
Vous pouvez utiliser $\vhook{F(x)}{a}{b}$ au lieu de $\hook{F(x)}{a}{b}$.
\end{tcblisting}


            \paragraph{Exemple d'utilisation 2}

Tout comme avec la première version de l'opérateur crochet, vous pouvez utiliser une version étoilée pour empêcher l'étirement verticalement du trait vertical. Voici un exemple.

\begin{tcblisting}{}
$\displaystyle \vhook{\frac{x - 1}{5 + x^2}}{a}{b}
             = \vhook*{\frac{x - 1}{5 + x^2}}{a}{b}$.
\end{tcblisting}


            \paragraph{Fiches techniques}

\IDmacro*{vhook}{3}

\IDmacro*{vhook*}{3}

\IDarg{1} le contenu avant le trait vertical.

\IDarg{2} la borne inférieure affichée en indice.

\IDarg{3} la borne supérieure affichée en exposant.



        \subsubsection{Intégrales multiples}

Le package réduit les espacements entres des symboles $\int$ successifs. Voici un exemple.

\begin{tcblisting}{}
$\displaystyle
  \int \int \int F(x;y;z) \dd{x} \dd{y} \dd{z}
= \int_{a}^{b} \int_{c}^{d} \int_{e}^{f} F(x;y;z) \dd{x} \dd{y} \dd{z}$
\end{tcblisting}


\begin{remark}
	Par défaut, \LaTeX{} affiche
	\makeatletter
    	$\displaystyle \original@int \original@int \original@int F(x;y;z) \dd{x} \dd{y} \dd{z}
    	= \original@int_{a}^{b} \original@int_{c}^{d} \original@int_{e}^{f} F(x;y;z) \dd{x} \dd{y} \dd{z}$.
	\makeatother
\end{remark}




% \section{Analysis}

    \subsection{Comparaison asymptotique de suites et de fonctions}

        \subsubsection{\texorpdfstring{Les notations $\bigO{}$ et $\smallO{}$}%
                               {Les notations "grand O" et "petit O"}}

            \paragraph{Exemple d'utilisation 1}

\begin{tcblisting}{}
Vous pouvez utiliser les symboles $\bigO{}$ et $\smallO{}$ créés par Landau.
\end{tcblisting}


            \paragraph{Exemple d'utilisation 2}

\begin{tcblisting}{}
Vous pouvez écrire $\bigO{x} \neq \smallO{x}$ et $e^{t + \smallO{t}} = e^{\bigO{t}}$.
\end{tcblisting}


            \paragraph{Fiches techniques}

\IDmacro*{bigO}{1}

\IDmacro*{smallO}{1}

\IDarg{} non vide, cet argument sera mis entre des parenthèses après $\bigO{}$ ou $\smallO{}$.



        \subsubsection{\texorpdfstring{La notation $\bigomega{}$}%
                               {La notation "grand Omega"}}

            \paragraph{Exemple d'utilisation 1}

\begin{tcblisting}{}
Vous pouvez utiliser le symbole $\bigomega{}$ créé par Hardy et Littlewood.
\end{tcblisting}


            \paragraph{Exemple d'utilisation 2}

\begin{tcblisting}{}
$f(n) = \bigomega{g(n)}$ signifie :
$\exists (m, n_0)$ tel que $n \geqslant n_0$ implique $f(n) \geqslant m g(n)$.
\end{tcblisting}


            \paragraph{Fiche technique}

\IDmacro*{bigomega}{1}

\IDarg{} non vide, cet argument sera mis entre des parenthèses après $\bigomega{}$.



        \subsubsection{\texorpdfstring{La notation $\bigtheta{}$}%
                               {La notation "grand Theta"}}

            \paragraph{Exemple d'utilisation 1}

\begin{tcblisting}{}
Voici le dernier symbole $\bigtheta{}$ qui peut rendre service.
\end{tcblisting}


            \paragraph{Exemple d'utilisation 2}

\begin{tcblisting}{}
$f(n) = \bigtheta{g(n)}$ signifie : $\exists (m, M, n_0)$ tel que $n \geqslant n_0$
implique $m g(n) \leqslant f(n) \leqslant M g(n)$.
\end{tcblisting}


            \paragraph{Fiche technique}

\IDmacro*{bigtheta}{1}

\IDarg{} non vide, cet argument sera mis entre des parenthèses après $\bigtheta{}$.




\section{Géométrie}

    \subsection{Points}

            \paragraph{Exemple d'utilisation 1}

\begin{tcblisting}{}
$\pt{I}$ indique un point nommé "I".
\end{tcblisting}


            \paragraph{Exemple d'utilisation 2}

\begin{tcblisting}{}
Une liste de points : $\pt*{I}{1}$, $\pt*{I}{2}$ \dots
\end{tcblisting}


            \paragraph{Fiches techniques}

\IDmacro*{pt}{1}

\IDarg{} un texte donnant le nom d'un point.


\bigskip


\IDmacro*{pt*}{2}

\IDarg{1} un texte indiquant $\pt{UP}$ dans le nom $\pt*{UP}{down}$ d'un point.

\IDarg{2} un texte indiquant $down$ dans le nom $\pt*{UP}{down}$ d'un point.




% \section{Géométrie}

    \subsection{Vecteurs}

		\subsubsection{Les écrire}

            \paragraph{Exemple d'utilisation 1}

\begin{tcblisting}{}
Voici un vecteur $\vect{ABCDEF}$ avec beaucoup de lettres et vous pouvez écrire
$\vect*{e}{rot}$ au lieu de $\vect{e_{rot}}$.
\end{tcblisting}


            \paragraph{Exemple d'utilisation 2}

\begin{tcblisting}{}
Vous pouvez écrire $\vect{i}$ and $\vect*{j}{2}$ sans point.
\end{tcblisting}



            \paragraph{Fiches techniques}

\IDmacro*{vect}{1}

\IDarg{} un texte donnant le nom d'un vecteur.


\bigskip


\IDmacro*{vect*}{2}

\IDarg{1} un texte indiquant $up$ dans le nom $\vect*{up}{down}$ d'un vecteur.

\IDarg{2} un texte indiquant $down$ dans le nom $\vect*{up}{down}$ d'un vecteur.




% \section{Géométrie}

%    \subsection{Vecteurs}

		\subsubsection{Norme}

            \paragraph{Exemple d'utilisation}

\begin{tcblisting}{}
Nous pouvons écrire $\norm{\vect{i}}$, $\displaystyle \norm{\frac{2}{7} \vect*{e}{k}}$, 
ou $\displaystyle \norm*{\frac{2}{7} \vect*{e}{k}}$ avec de petites barres verticales.
\end{tcblisting}


\begin{remark}
	Le code \LaTeX{} vient directement de ce message : \url{https://tex.stackexchange.com/a/43009/6880}.
\end{remark}


            \paragraph{Fiches techniques}

\IDmacro*{norm}{1}

\IDmacro*{norm*}{1}

\IDarg{} le vecteur sur lequel appliquer la norme.




% \section{Géométrie}

%    \subsection{Vecteurs}

		\subsubsection{Produit scalaire}

            \paragraph{Exemple d'utilisation 1}

\begin{tcblisting}{}
En mathématique, il est usage d'écrire un produit scalaire avec un point via
$\dotprod{\vect{i}}{\vect{j}}$ .
\end{tcblisting}


            \paragraph{Exemple d'utilisation 2}

Dans l'exemple suivant, le préfixe \verb+a+ est pour \textbf{a}-ngle.

\begin{tcblisting}{}
Les physiciens pourront utiliser
$\displaystyle \adotprod{\frac{1}{2} \vect{i}}{\vect{j}}$
ou
$\displaystyle \adotprod*{\frac{1}{2} \vect{i}}{\vect{j}}$ .
\end{tcblisting}


            \paragraph{Fiches techniques}

\IDmacro*{dotprod}{2}

\IDarg{1} le premier vecteur.

\IDarg{2} le second vecteur.


\bigskip

\IDmacro*{adotprod}{2} où \quad \verb&a = a-ngle&

\IDmacro*{adotprod*}{2}

\IDarg{1} le premier vecteur.

\IDarg{2} le second vecteur.





		\subsubsection{Produit vectoriel}

            \paragraph{Exemple d'utilisation}

\begin{tcblisting}{}
Un produit vectoriel peut s'écrire via $\crossprod{\vect{i}}{\vect{j}}$ .
\end{tcblisting}


            \paragraph{Fiche technique}

\IDmacro*{crossprod}{2}

\IDarg{1} le premier vecteur.

\IDarg{2} le second vecteur.




%\section{Géométrie}

    \subsection{Coordonnées}

            \paragraph{Exemple d'utilisation 1}

\begin{tcblisting}{}
On peut choisir d'écrire
$\displaystyle \pt{I} \coord{\frac{1}{3} | -4 | 0}$
ou bien
$\displaystyle \pt{I} \coord*{\frac{1}{3} | -4 | 0}$ .
\end{tcblisting}


            \paragraph{Exemple d'utilisation 2}

Dans l'exemple suivant, le préfixe \verb+v+ est pour \textbf{v}-ertical.

\begin{tcblisting}{}
Pour les vecteurs, on peut préférer $\vect{i} \vcoord{3 | -4 | 0}$, voire
$\vect{i} \vcoord*{3 | -4 | 0}$ , à la place de $\vect{i} \coord{3 | -4 | 0}$ 
afin de bien différencier les coordonnées de points de celles de vecteurs.
\end{tcblisting}


            \paragraph{Fiches techniques}

\IDmacro*{coord}{1}

\IDmacro*{coord*}{1}

\IDarg{} l'argument est une suite de "morceaux" séparés par des barres \verb+|+ , chaque morceau étant une coordonnée. Il peut n'y avoir qu'un seul morceau.



\bigskip

\IDmacro*{vcoord}{1} où \quad \verb&v = v-ertical&

\IDmacro*{vcoord*}{1} pour des crochets à la place de parenthèses

\IDarg{} l'argument est une suite de "morceaux" séparés par des barres \verb+|+ , chaque morceau étant une coordonnée. Il peut n'y avoir qu'un seul morceau.




% \section{Géométrie}

    \subsection{Nommer un repère}

            \paragraph{Exemple d'utilisation 1 -- La méthode basique}

Commençons par la manière la plus basique d'écrire un repère \textit{(nous verrons d'autres méthodes qui peuvent être plus efficaces)}.

\begin{tcblisting}{}
Dans le plan, trois points $\pt{O}$, $\pt{I}$ and $\pt{J}$ non alignés définissent
un repère cartésien $\axes{\pt{O} | \pt{I} | \pt{J}}$.
\end{tcblisting}


            \paragraph{Exemple d'utilisation 2 -- La méthode basique en version étoilée}

Dans l'exemple ci-dessous, on voit que la version étoilée produit des petites parenthèses.
\begin{tcblisting}{}
$\displaystyle \axes{\pt{O} | \frac{7}{3} \vect{i} | \vect{j}}$
ou
$\displaystyle \axes*{\pt{O} | \frac{7}{3} \vect{i} | \vect{j}}$
\end{tcblisting}


            \paragraph{Exemple d'utilisation 3 -- La méthode basique en dimension quelconque}

Il faut au minimum deux "morceaux" séparés par des barres \verb+|+, cas de la dimension $1$, mais il n'y a pas de maximum, cas d'une dimension quelconque $n > 0$.

\begin{tcblisting}{}
$\axes{\pt{O} | \vect*{i}{1} | \vect*{i}{2} | \vect*{i}{3} | \dots |
 \vect*{i}{9} | \vect*{i}{10} | \vect*{i}{11} | \vect*{i}{12}}$
\end{tcblisting}


            \paragraph{Exemple d'utilisation 4 -- Repère affine}

Dans l'exemple suivant, le préfixe \verb+p+ est pour \textbf{p}-oint.

\begin{tcblisting}{}
$\paxes{O | I | J | K}$ évite de taper $\axes{\pt{O} | \pt{I} | \pt{J} | \pt{K}}$.
\end{tcblisting}


            \paragraph{Exemple d'utilisation 5 -- Repère vectoriel (méthode 1)}

Dans l'exemple suivant, le préfixe \verb+v+ est pour \textbf{v}-ecteur.

\begin{tcblisting}{}
$\vaxes{\pt{O} | i | j}$ est un raccourci de $\axes{\pt{O} | \vect{i} | \vect{j}}$.
\end{tcblisting}


            \paragraph{Exemple d'utilisation 6 -- Repère vectoriel (méthode 2)}

Dans l'exemple suivant, le préfixe \verb+gpv+ permet de combiner à la fois les fonctionnalités proposés par les préfixes \verb+gp+ et \verb+v+.

\begin{tcblisting}{}
$\pvaxes{O | i | j}$ donne rapidement $\axes{\pt{O} | \vect{i} | \vect{j}}$.
\end{tcblisting}



            \paragraph{Fiches techniques}

\IDmacro*{axes}{1}

\IDmacro*{axes*}{1}

\IDarg{} l'argument est une suite de "morceaux" séparés par des barres \verb+|+.

\begin{itemize}[topsep=0pt]
	\item Le premier morceau est l'origine du repère.
	
	\item Les morceaux suivants sont des points ou des vecteurs qui "définissent" chaque axe.
\end{itemize}


\bigskip

\IDmacro*{paxes}{1} où \quad \verb+p = p-oint+

\IDarg{} l'argument est une suite de "morceaux" séparés par des barres \verb+|+.

\begin{itemize}[topsep=0pt]
	\item Le premier morceau est le nom de l'origine du repère sur laquelle la macro-commande \verb+\pt+ sera automatiquement appliquée.
	
	\item Viennent ensuite les noms des points "définissant" chaque axe. Pour chacun de ces points la macro-commande \verb+\pt+ sera automatiquement appliquée.
\end{itemize}


\bigskip

\IDmacro*{vaxes}{1} où \quad \verb+v = v-ector+

\IDarg{} l'argument est une suite de "morceaux" séparés par des barres \verb+|+.

\begin{itemize}[topsep=0pt]
	\item Le premier morceau est l'origine du repère.
	
	\item Viennent ensuite les noms des vecteurs "définissant" chaque axe. Pour chacun de ces vecteurs la macro-commande \verb+\vect+ sera automatiquement appliquée.
\end{itemize}


\bigskip

\IDmacro*{pvaxes}{3} où \quad \verb&pv = p + v&

\IDarg{} l'argument est une suite de "morceaux" séparés par des barres \verb+|+.

\begin{itemize}[topsep=0pt]
	\item Le premier morceau est le nom de l'origine du repère sur laquelle la macro-commande \verb+\pt+ sera automatiquement appliquée.
	
	\item Viennent ensuite les noms des vecteurs "définissant" chaque axe. Pour chacun de ces vecteurs la macro-commande \verb+\vect+ sera automatiquement appliquée.
\end{itemize}




% \section{Géométrie}

    \subsection{Arcs circulaires}

            \paragraph{Exemple d'utilisation 1}

\begin{tcblisting}{}
Voici un arc $\arc{ABCDEF}$ utilisant beaucoup de lettres, et vous pouvez écrire
$\arc*{A}{rot}$ à la place de $\arc{A_{rot}}$.
\end{tcblisting}


            \paragraph{Exemple d'utilisation 2}

\begin{tcblisting}{}
$\arc{i}$ et $\arc*{j}{2}$ n'affiche pas de point sous l'arc.
\end{tcblisting}



            \paragraph{Fiches techniques}

\IDmacro*{arc}{1}

\IDarg{} un texte donnant le nom d'un arc circulaire.


\bigskip


\IDmacro*{arc*}{2}

\IDarg{1} un texte indiquant $up$ dans le nom $\arc*{up}{down}$ d'un arc circulaire.

\IDarg{2} un texte indiquant $down$ dans le nom $\arc*{up}{down}$ d'un arc circulaire.




% \section{Géométrie}

    \subsection{Angles}
    
		\subsubsection{Angles géométriques intérieurs}

            \paragraph{Exemple d'utilisation 1}

\begin{tcblisting}{}
Voici un angle géométrique intérieur $\anglein{ABCDEF}$ avec un long nom, et vous 
pouvez écrire $\anglein*{A}{rot}$ au lieu de $\anglein{A_{rot}}$.
\end{tcblisting}


            \paragraph{Exemple d'utilisation 2}

\begin{tcblisting}{}
Vous pouvez aussi écrire $\anglein{i}$ and $\anglein*{j}{2}$ sans point.
\end{tcblisting}



            \paragraph{Fiches techniques}

\IDmacro*{anglein}{1}

\IDarg{} un texte donnant le nom d'un angle intérieur.


\bigskip


\IDmacro*{anglein*}{2}

\IDarg{1} un texte indiquant $up$ dans le nom $\anglein*{up}{down}$ d'un angle intérieur.

\IDarg{2} un texte indiquant $down$ dans le nom $\anglein*{up}{down}$ d'un angle intérieur.




% \section{Géométrie}

%    \subsection{Angles}
    
		\subsubsection{Angles orientés de vecteurs}

            \paragraph{Exemple d'utilisation}

\begin{tcblisting}{}
En mathématique, il est d'usage de noter les angles orientés via
$\displaystyle \angleorient{\frac{1}{2} \vect{i}}{\vect{j}}$
ou
$\displaystyle \angleorient*{\frac{1}{2} \vect{i}}{\vect{j}}$ .
\end{tcblisting}



            \paragraph{Fiche technique}

\IDmacro*{angleorient}{2}

\IDarg{1} le premier vecteur.

\IDarg{2} le second vecteur.




\section{Fractions continuées}

	\subsection{Fractions continuées standard}

            \paragraph{Exemple d'utilisation}

Dans l'exemple suivant, la notation en ligne semble avoir être due à Alfred Pringsheim. La notation à gauche utilise toujours le maximum d'espace pour améliorer la lisibilité.

\begin{tcblisting}{}
$ \contfrac{u_0 | u_1 | u_2 | \dots | u_n}
= \contfrac*{u_0 | u_1 | u_2 | \dots | u_n}$
\end{tcblisting}


            \paragraph{Fiches techniques}

\IDmacro*{contfrac}{1}

\IDmacro*{contfrac*}{1}

\IDarg{} tous les éléments de la fraction continuée séparés par des \verb+|+.




	\subsection{Fractions continuées généralisées}

            \paragraph{Exemple d'utilisation}

Voici comment écrire une fraction continuée généralisée.

\begin{tcblisting}{}
$\displaystyle \contfracgene{a | b | c | d | e | f | \dots | y | z}
             = \contfracgene*{a | b | c | d | e | f | \dots | y | z}$
\end{tcblisting}


            \paragraph{Fiches techniques}

\IDmacro*{contfracgene}{1}

\IDmacro*{contfracgene*}{1}

\IDarg{} tous les éléments de la fraction continuée généralisée séparés par des \verb+|+.



	\subsection{Comme une fraction continuée isolée}

            \paragraph{Exemple d'utilisation}

La raison d'être de la macro ci-dessous vient juste de son usage en interne.

\begin{tcblisting}{}
Les fous (?\,) adorent vraiment écrire des choses comme $\singlecontfrac{a}{b}$.
\end{tcblisting}


            \paragraph{Fiche technique}

\IDmacro*{singlecontfrac}{2}

\IDarg{1} le pseudo numérateur.

\IDarg{2} le pseudo dénominateur.



    \subsection{\texorpdfstring{L'opérateur $\contfracope$}%
                               {L'opérateur K}}

            \paragraph{Exemple d'utilisation 1}

\IDconstant{contfracope}

La notation suivante est proche de celle qu'utilisait Carl Friedrich Gauss.

\begin{tcblisting}{}
$\displaystyle
  \contfracope_{k=1}^{n} (b_k:c_k)
= \cfrac{b_1}{\contfracgene{c_1 | b_2 | c_2 | b_3 | \dots | b_n | c_n}}$
\end{tcblisting}


\begin{remark}
	La lettre $\contfracope$ vient de "kettenbruch" qui signifie "fraction continuée" en allemand.
\end{remark}


            \paragraph{Exemple d'utilisation 2}

\begin{tcblisting}{}
$\displaystyle
  u_0 + \contfracope_{k=1}^{n} (1:u_k)
= \contfrac{u_0 | u_1 | u_2 | \dots | u_n}$
\end{tcblisting}




\section{Algèbre}

	\subsection{Polynômes, séries formelles et compagnie}

        \subsubsection{Polynômes}

            \paragraph{Exemple d'utilisation}

\begin{tcblisting}{}
$\polyset{\RR}{X}$ est l'ensemble des polynômes à coefficients réels en la variable
$X$, et $\polyset{\RR}{X | Y | Z}$ est l'ensemble des polynômes à coefficients réels
en les variables $X$ , $Y$ et $Z$.
\end{tcblisting}



        \subsubsection{Fractions polynômiales}

            \paragraph{Exemple d'utilisation}

\begin{tcblisting}{}
$\polyfracset{\QQ}{T}$ et $\polyfracset{\QQ}{S_1 | S_2 | \dots | S_k}$ permettent 
d'indiquer des ensemble de fractions polynomiales à coefficients rationnels.
\end{tcblisting}



        \subsubsection{Séries formelles}

            \paragraph{Exemple d'utilisation}

\begin{tcblisting}{}
$\serieset{\CC}{X}$ et $\serieset{\CC}{T | O | P}$ permettent de travailler avec des
séries formelles à coefficients complexes.
\end{tcblisting}



        \subsubsection{Corps des fractions de séries formelles}

            \paragraph{Exemple d'utilisation}

\begin{tcblisting}{}
$\seriefracset{\ZZ}{X}$ et $\seriefracset{\ZZ}{Z | T | O | P}$ permettent de travailler
avec des fractions de séries formelles à coefficients entiers.
\end{tcblisting}



        \subsubsection{Toutes les fiches techniques}

\IDmacro*{polyset}{2}

\IDmacro*{polyfracset}{2}

\IDmacro*{serieset}{2}

\IDmacro*{seriefracset}{2}

\IDarg{1} l'ensemble auquel les coefficients appartiennent.

\IDarg{2} cet argument est une suite de "morceaux" séparés par des barres \verb+|+, chaque morceau étant une variable formelle.




\newpage

\section{Historique}

Tous les changements sont décrits en anglais uniquement dans le dossier \verb+change-log+ : voir le code source de \verb+lymath+ sur \verb+github+. Nous ne donnons ici qu'un très bref historique de \verb+lymath+ côté utilisateur principalement.

\begin{description}[leftmargin=1em]
	\setlength\itemsep{1em}

	\item[2019-02-21] Nouvelle version mineure \verb+0.2.0-beta+ dont voici les principaux changements.
	\begin{itemize}
		\item L'usage de \verb+//+ pour les macros-commandes avec un nombre quelconque d'arguments a été remplacé par celui de \verb+|+.

		\item En géométrie, il y a diverses nouveautés.
		\begin{itemize}
			\item Ajout de l'écriture de coordonnées, de produits scalaires et de produits vectoriels.

			\item \verb+\axis+ a été correctement traduit en \verb+\axes+.

			\item Les macros \verb+\gpaxis+ et \verb+\gpvaxis+ deviennent \verb+\paxes+ et \verb+\pvaxes+ pour être cohérent avec \verb+\pt+ qui a remplacé l'ancien \verb+\gpt+.
		\end{itemize}

		\item En analyse, ajout de la macro commande étoilée \verb+\derpow*+ pour la gestion automatique des primes d'une dérivée.

		\item Une nouvelle section "algèbre" propose des macros pour écrire des ensembles de polynômes, de fractions polynomiales, de séries fromelles et aussi de fractions de séries formelles.

		\item Redéfinition de \verb+\frac+ et \verb+\dfrac+ pour obtenir des traits de fraction un peu plus longs.

		\item Ajout de \verb+\lymathsep+ qui définit globalement le séparateur d'arguments à l'impression.
	\end{itemize}


	\item[2017-11-01] Nouvelle version mineure \verb+0.1.0-beta+ : pour les ensembles, les fonctions et la géométrie, il y a eu des changements et l'ajout de nouveaux outils.


	\item[2017-10-21] Historique court de \verb+lymath+ ajouté au présent document.


	\item[2017-10-18] Nouvelle version "patchée" \verb+0.0.2-beta+ : de nouveaux outils pour le calcul différentiel.


	\item[2017-10-06] Nouvelle version "patchée" \verb+0.0.1-beta+ : de nouveaux outils pour l'arithmétique, la géométrie, le calcul intégral et le calcul différentiel.


	\item[2017-10-02] Première version \verb+0.0.0-beta+ du package.
\end{description}



\end{document}
